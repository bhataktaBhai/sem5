\lecture{2024-10-02}{}
% \begin{definition}[relation] \label{def:relation}
%     A \emph{relation} $S$ from $A$ to $B$ is a subset of $A \times B$.
%     We define the \emph{domain} of $S$ to be
%     $D(S) = \sset{x \in A}{(x, y) \in S \text{ for some } y \in B}$
%     and the \emph{range} of $S$ to be
%     $R(S) = \sset{y \in B}{(x, y) \in S \text{ for some } x \in A}$.
%
%     A relation $S$ from $A$ to $B$ is called a \emph{function} if $D(S) = A$
%     and whenever $(a, b) \in S$ and $(a, b') \in S$, $b = b'$.
% \end{definition}
% \begin{theorem}[implicit function theorem] \label{thm:imft}
%     Let $f\colon \R^n \times \R^m \to \R^m$ be differentiable, and let
%     $S =\sset{(x, y) \in \R^n \times \R^m}{f(x, y) = 0}$.
% \end{theorem}
% \begin{examples}
%     \item Consider $f\colon \R \times \R \to \R$ given by
%     $f(x, y) = x + y - 1$.
%     Then $S = \sset{(x, y) \in \R^2}{x + y = 1}$ is a function.
%     \item Consider $f\colon \R \times \R \to \R$ given by
%     $f(x, y) = y^5 + y^3 + y + x$.
%     Then $S = \sset{(x, y) \in \R^2}{y^5 + y^3 + y + x = 0}$ \emph{is} a
%     function.
%     For any $x \in \R$, $y^5 + y^3 + y + x$ is strictly increasing in $y$,
%     and equals $0$ exactly once.
%     \begin{center}
%         \begin{tikzpicture}
%             \begin{axis}[
%                 axis lines = middle,
%                 xlabel = $y$,
%                 ylabel = $f(x, y)$,
%                 xmin = -1.5,
%                 xmax = 1.5,
%                 ymin = -1.5,
%                 ymax = 1.5,
%             ]
%             \addplot[domain=-1.5:1.5, samples=100] {x + x^3 + x^5};
%             \end{axis}
%         \end{tikzpicture}
%     \end{center}
% \end{examples}
