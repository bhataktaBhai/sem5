\lecture[9]{2024-09-05}{}

\textbf{Recap:}
\begin{itemize}
    \item Pseudorandomness and pseudorandom generators
    \item An ind-secure secret key encryption and proof
\end{itemize}

Today, we will try to reduce our key size, and see if a key can be reused.

\section{Multi-message security} \label{sec:sec:multi}

We will prove that multi-message ind-security is stronger than
single-message ind-security.

The multi-message game is as follows:
\begin{definition*}
    An adversary $\mcA$ picks a vector $M_0$ of messages
    $(m_0^{(1)}, \dots, m_0^{(q)})$ and a vector $M_1$ of messages
    $(m_1^{(1)}, \dots, m_1^{(q)})$ of the same length,
    where each $m_b^{(i)}$ has the same length.
    The challenger chooses a $b \in \set{0, 1}$ uniformly at random,
    encrypts each message in $M_b$ using the same key $k$ under the scheme
    $\Pi$ and gives the resulting ciphertexts to $\mcA$.
    $\mcA$ outputs a guess $b'$.

    $\Pi$ is multi-ind-secure if for all PPT adversaries $\mcA$,
    \[
        \Pr[b = b'] \leq \frac{1}{2} + \negl(n),
    \]
\end{definition*}
\begin{proposition}
    Multi-ind-security is \emph{strictly} stronger than ind-security.
\end{proposition}
\begin{proof}
    We know that the pseudo one-time pad is ind-secure.
    However, the encryption algorithm is deterministic, so it fails
    multi-ind-security.

    Construct an adversary $\mcA$ that picks $M_0 = (a, a)$ and
    $M_1 = (a, b)$ where $a \ne b \in \M$.
    $\mcA$ will output $b'$ as $0$ if the two ciphertexts are the same,
    and $1$ otherwise.
    Then \[
        \Pr[b = b'] = 1. \qedhere
    \] 
\end{proof}

\section{Onion routing} \label{sec:onion}

\section{PRG with poly expansion factor} \label{sec:prg-poly-exp}
\begin{theorem}
    If there exists a PRG with expansion factor $\ell(n) = n + 1$, then for
    each polynomial $p(n)$, there exists a PRG with expansion factor
    $\ell'(n) = n + p(n)$.
\end{theorem}
\begin{proof}
    Let $G\colon 2^n \to 2^{n+1}$ be a PRG.
    For any $s \in 2^n$, let $G(s) = (I(s), L(s))$ where
    $I(s) \in 2^n$ and $L(s) \in 2$.

    Starting with a seed $s \in 2^n$, \begin{align*}
        s_1 &= (Is, Ls), \\
        s_2 &= (IIs, LIs, Ls), \\
        s_3 &= (IIIs, LIIs, LIs, Ls),
    \end{align*} and so on,
    where we apply $G$ to the first $n$ bits of the previous output.
\end{proof}
