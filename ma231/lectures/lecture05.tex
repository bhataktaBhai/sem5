\lecture{2024-08-12}{}

\subsection{Topological proof of the infinitude of primes} \label{sec:primes}

\begin{lemma}
    Arithmetic progressions form a basis of a topology on $\Z$.
\end{lemma}
\begin{proof}
    Let $S(a, b) = \set{an + b \mid n \in \Z}$.

    $S(1, 0) = \Z$.
    Suppose $x \in S(a_1, x) \cap (a_2, x)$.
    Then $x \in S(a_1a_2, x) \subseteq S(a_1, x) \cap S(a_2, x)$.
\end{proof}

\begin{theorem}
    There are infinitely many primes.
\end{theorem}
\begin{proof}
    Notice that $\Z \setminus S(a, b) = \bigcup_{i=1}^{a-1} S(a, b + i)$.
    Thus $S(a, b)$ is clopen.

    Now notice that
    $\bigcup_{p \text{ prime}} S(p, 0) = \Z \setminus \set{\pm 1}$.

    The union of finitely many closed sets is closed, so if there were
    a finite number of primes, then $\Z \setminus \set{\pm 1}$ would be
    closed.
    Thus $\set{\pm 1}$ would be open, which is impossible, since each
    basis is infinite.
\end{proof}

From \cref{thm:closure}, we can say that a set is closed iff it contains all
its limit points.
\begin{example}
    In $\R_L$, the lower-limit topology on $\R$, $\wbar{\Q} = \R$.
    Let $x \in \R \setminus \Q$.
    We need to show that any basis element containing $x$ intersects $\Q$.
    But basis elements are of the form $[a, b)$, so every basis element
    intersects $\Q$.
    Thus $x \in \wbar{\Q}$.
\end{example}
