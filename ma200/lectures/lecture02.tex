\lecture{2024-08-05}{One norm to rule them all}
\begin{proposition}
    Equivalent norms induce the same topology.
    That is, let ${\norm\cdot}_a \sim {\norm\cdot}_b$.
    Then a set is open (resp. compact) under ${\norm\cdot}_a$ iff it is
    open (resp. compact) under ${\norm\cdot}_b$.
\end{proposition}
\begin{proof}
    Suppose $c_1 {\norm x}_a \le {\norm x}_b \le c_2 {\norm x}_a$.

    Let $U \subseteq V$ be open under ${\norm\cdot}_a$.
    Let $x \in U$.
    There exists $\eps > 0$ such that
    ${\norm{y - x}}_a < \eps \implies y \in U$.
    But then ${\norm{y - x}}_b < c_1 \eps \implies y \in U$.
    Thus $U$ is open under ${\norm\cdot}_b$.

    Compactness follows from openness.
\end{proof}

\begin{proposition} \label{thm:norm:eq-lp}
    Every $\ell^p$ norm is equivalent to $\ell^\infty$.
\end{proposition}
\begin{proof}
    Let $x \in \R^n$.
    Then ${\norm x}_\infty \le {\norm x}_p \le n^{\frac1p} {\norm x}_\infty$.
\end{proof}

The usual topology on $\R^n$ is the one induced by the Euclidean norm.
This norm itself is induced by the inner product
$\innerp{x}{y} = \sum_{i=1}^n x_i y_i$. % \ip
Using Cauchy-Schwarz, we can define the angle between two vectors
$x, y \in \R^n$ to be \[
    \cos^{-1} \ab(\frac{\innerp{x}{y}}{\norm{x} \norm{y}}).
\]

\begin{lemma*} \label{thm:norm:cont}
    Let $\norm\cdot$ be any norm on $\R^n$.
    Then the function $x \mapsto \norm{x}$ is Lipschitz continuous
    with respect to the Euclidean topology.
\end{lemma*}
\begin{proof}
    \begin{align*}
        \norm{x} &= \norm*{\sum x_i e_i} \\
            &\le \sum \abs{x_i} \norm{e_i} \\
            &\le M {\norm x}_2
    \end{align*} where $M = \sum \norm{e_i}$.

    The reverse triangle inequality gives
    \begin{align*}
        \abs[\big]{\norm{x} - \norm{y}} &\le \norm{x - y} \\
            &\le M {\norm{x - y}}_2. \qedhere
    \end{align*}
\end{proof}

\begin{theorem*} \label{thm:norm:eq}
    Any two norms on $\R^n$ are equivalent.
\end{theorem*}
\begin{proof}
    Let $\norm\cdot$ be any norm on $\R^n$.
    Then $x \mapsto \norm x$ is continuous with respect to ${\norm\cdot}_2$.
    Let \[
        S(0, 1)_{{\norm\cdot}_2} = \set{x \in \R^n : {\norm x}_2 = 1}
                      = S^{n-1}.
    \] $\norm\cdot$ attains a minimum and a maximum on $S^{n-1}$
    by compactness.
    Thus there exist positive constants $c_1, c_2$ such that \[
        c_1 \le \norm{x} \le c_2
    \] for all $x \in S^{n-1}$.

    Now for any $x \in \R^n \setminus \set{0}$,
    dividing by $\norm{x}_2$ gives a point that lies on $S^{n-1}$.
    Thus \[
        c_1 \le \norm*{\frac{x}{{\norm x}_2}} \le c_2.
    \] By homogeneity \labelcref{def:norm:hom}, \[
        c_1 {\norm x}_2 \le \norm{x} \le c_2 {\norm x}_2.
    \] This is also trivially true for $x = 0$.

    Thus $\norm\cdot \sim {\norm\cdot}_2$.
\end{proof}
\begin{remark}
    The idea of the proof is as follows.

    Any homogenous function is determined by its value on the unit sphere.
    A homogenous function of degree \emph{zero} is essentially nothing but
    a function on the unit sphere ($f(v) = f(\what{v})$).

    The function $x \mapsto \frac{\norm x}{{\norm x}_2}$ is a
    continuous homogenous function on degree $0$.
    The unit sphere is known to be compact under the Euclidean norm
    (and every other, but not before we complete the proof).
    Thus \[
        c_1 \le \frac{\norm x}{{\norm x}_2} \le c_2
    \] for some positive constants $c_1, c_2$.

    Definiteness and $\triangle$ are required for the ratio to be
    continuous.
    Homogeneity is required for it to be homogenous.
    Is positivity required? % TODO
\end{remark}
\begin{remark}
    We technically only need to show $c_1 \norm{x}_2 \le \norm{x}$,
    since the other inequality is proven in the previous proof.
    It is nonetheless clearer to show both inequalities.
\end{remark}

\begin{exercise*}[self]
    Show that \labelcref{def:norm:pos} follows from
    \labelcref{def:norm:hom,def:norm:tri}.
\end{exercise*}
\begin{solution}
    Let $v \in V$.
    By triangle inequality,
    $\norm v = \norm{-v + 2v} \le \norm{-v} + \norm{2v}$.
    By homogeneity, this is $3\norm v$.
    Thus $\norm v \le 3\norm v$, so $\norm v \ge 0$.
\end{solution}

\begin{remarks}[Finite-dimensional vector spaces]
    \item Let $V$ be a vector space over $\R$ with dimension $n < \infty$.
    Using a basis for $V$, any norm on $V$ induces a norm on $\R^n$,
    and vice versa.
    Norms on $V$ are in a one-to-one correspondence with norms on $\R^n$.
    \item Thus any two norms on $V$ are equivalent.
    \item Any two inner products on $V$ will also be equivalent due to this.
    \item Any finite-dimensional vector space over $\R$ is complete.
\end{remarks}

\begin{exercise*} \label{thm:reciprocal-cont}
    Let $f\colon \R \setminus \set{0} \to \R$ be given by $f(x) = \frac1x$.
    Show that $f$ is continuous.
    What is the key idea of your proof?
\end{exercise*}
\begin{solution}
    Let $x_0 \in \R \setminus \set{0}$ and $\eps > 0$.
    Choose $\delta
    = \min\set{\eps \cdot \frac12\abs{x_0}^2, \frac12 \abs{x_0}}$.
    Then for any $x$ in the $\delta$-neighbourhood of $x_0$, \begin{align*}
        \abs{f(x) - f(x_0)}
            &= \abs*{\frac1x - \frac1{x_0}} \\
            &= \frac{\abs{x_0 - x}}{\abs{x} \abs{x_0}} \\
            &< \frac{\delta}{\abs{x} \abs{x_0}} \\
            &< \frac{2\delta}{\abs{x_0}^2} \\
            &\le \eps. \qedhere
    \end{align*}
\end{solution}
\begin{remark}
    The proof works by bounding $\frac1{\abs{x}}$.
    The rest goes to zero as $x \to a$.
    We will do a similar proof in \cref{thm:inv-cont-rudin}.
\end{remark}
