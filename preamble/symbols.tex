\newcommand\ie{\textit{i.e.}}
\newcommand\eg{\textit{e.g.}}

\usepackage{amsmath}
\usepackage{amssymb}
\usepackage{mathrsfs} % for \mathscr
\usepackage{bm} % for \bm
\usepackage{booktabs}

% undefine \abs and \norm
\let\abs\relax
\let\norm\relax

\usepackage{mathtools} % for delimiters and \coloneqq
\DeclarePairedDelimiter{\paren}{(}{)}
\DeclarePairedDelimiter{\brk}{[}{]}
\DeclarePairedDelimiter{\set}{\{}{\}}
\DeclarePairedDelimiter{\abs}{\lvert}{\rvert}
\DeclarePairedDelimiter{\norm}{\lVert}{\rVert}
\DeclarePairedDelimiter{\floor}{\lfloor}{\rfloor}
\DeclarePairedDelimiter{\ceil}{\lceil}{\rceil}
\DeclarePairedDelimiter{\angled}{\langle}{\rangle}
% \DeclarePairedDelimiterX{\innerp}[2]{\langle}{\rangle}{#1,\,#2}
% \DeclarePairedDelimiterX{\outerp}[2]{\langle}{\rangle}{#1\otimes#2}
% \DeclarePairedDelimiterX{\braket}[3]{\langle}{\rangle}%
% {#1\,\delimsize\vert\,\mathopen{}#2\,\delimsize\vert\,\mathopen{}#3}
\DeclarePairedDelimiterX{\innerp}[2]{\langle}{\rangle}{#1,\,#2}
% \DeclarePairedDelimiterX{\outerp}[2]{\langle}{\rangle}{#1\otimes#2}
\DeclarePairedDelimiterX{\outerp}[2]{\vert}{\vert}%
{#1\delimsize\rangle\delimsize\langle\mathopen{}#2}
\let\braket\relax
\DeclarePairedDelimiterX{\braket}[3]{\langle}{\rangle}%
{#1\,\delimsize\vert\,\mathopen{}#2\,\delimsize\vert\,\mathopen{}#3}

\renewcommand\O{\ensuremath{\varnothing}}
\newcommand\N{\ensuremath{\mathbb{N}}}
\newcommand\Z{\ensuremath{\mathbb{Z}}}
\newcommand\Q{\ensuremath{\mathbb{Q}}}
\newcommand\R{\ensuremath{\mathbb{R}}}
\newcommand\C{\ensuremath{\mathbb{C}}}
\newcommand\eps{\ensuremath{\varepsilon}}
% \renewcommand\P{\ensuremath{\mathbb{P}}}

% fix spacing for \forall and \exists
% \let\oldforall\forall
% \renewcommand{\forall}{\oldforall \, }
% \let\oldexist\exists
% \renewcommand{\exists}{\oldexist \: }
\newcommand\unique{\exists!}
\newcommand\lxor{\oplus}

\providecommand{\dd}{\,\mathrm{d}}

\newcommand\mcA{\ensuremath{\mathcal{A}}}
\newcommand\mcB{\ensuremath{\mathcal{B}}}
\newcommand\mcC{\ensuremath{\mathcal{C}}}
\newcommand\mcD{\ensuremath{\mathcal{D}}}
\newcommand\mcE{\ensuremath{\mathcal{E}}}
\newcommand\mcF{\ensuremath{\mathcal{F}}}
\newcommand\mcG{\ensuremath{\mathcal{G}}}
\newcommand\mcH{\ensuremath{\mathcal{H}}}
\newcommand\mcI{\ensuremath{\mathcal{I}}}
\newcommand\mcJ{\ensuremath{\mathcal{J}}}
\newcommand\mcK{\ensuremath{\mathcal{K}}}
\newcommand\mcL{\ensuremath{\mathcal{L}}}
\newcommand\mcM{\ensuremath{\mathcal{M}}}
\newcommand\mcN{\ensuremath{\mathcal{N}}}
\newcommand\mcO{\ensuremath{\mathcal{O}}}
\newcommand\mcP{\ensuremath{\mathcal{P}}}
\newcommand\mcQ{\ensuremath{\mathcal{Q}}}
\newcommand\mcR{\ensuremath{\mathcal{R}}}
\newcommand\mcS{\ensuremath{\mathcal{S}}}
\newcommand\mcT{\ensuremath{\mathcal{T}}}
\newcommand\mcU{\ensuremath{\mathcal{U}}}
\newcommand\mcV{\ensuremath{\mathcal{V}}}
\newcommand\mcW{\ensuremath{\mathcal{W}}}
\newcommand\mcX{\ensuremath{\mathcal{X}}}
\newcommand\mcY{\ensuremath{\mathcal{Y}}}
\newcommand\mcZ{\ensuremath{\mathcal{Z}}}

%%%% WIDE BAR THAT IS JUST THE RIGHT LENGTH %%%%
%% FROM https://tex.stackexchange.com/a/60253 %%
\makeatletter
\let\save@mathaccent\mathaccent
\newcommand*\if@single[3]{%
  \setbox0\hbox{${\mathaccent"0362{#1}}^H$}%
  \setbox2\hbox{${\mathaccent"0362{\kern0pt#1}}^H$}%
  \ifdim\ht0=\ht2 #3\else #2\fi
  }
%The bar will be moved to the right by a half of \macc@kerna, which is computed by amsmath:
\newcommand*\rel@kern[1]{\kern#1\dimexpr\macc@kerna}
%If there's a superscript following the bar, then no negative kern may follow the bar;
%an additional {} makes sure that the superscript is high enough in this case:
\newcommand*\widebar[1]{\@ifnextchar^{{\wide@bar{#1}{0}}}{\wide@bar{#1}{1}}}
%Use a separate algorithm for single symbols:
\newcommand*\wide@bar[2]{\if@single{#1}{\wide@bar@{#1}{#2}{1}}{\wide@bar@{#1}{#2}{2}}}
\newcommand*\wide@bar@[3]{%
  \begingroup
  \def\mathaccent##1##2{%
%Enable nesting of accents:
    \let\mathaccent\save@mathaccent
%If there's more than a single symbol, use the first character instead (see below):
    \if#32 \let\macc@nucleus\first@char \fi
%Determine the italic correction:
    \setbox\z@\hbox{$\macc@style{\macc@nucleus}_{}$}%
    \setbox\tw@\hbox{$\macc@style{\macc@nucleus}{}_{}$}%
    \dimen@\wd\tw@
    \advance\dimen@-\wd\z@
%Now \dimen@ is the italic correction of the symbol.
    \divide\dimen@ 3
    \@tempdima\wd\tw@
    \advance\@tempdima-\scriptspace
%Now \@tempdima is the width of the symbol.
    \divide\@tempdima 10
    \advance\dimen@-\@tempdima
%Now \dimen@ = (italic correction / 3) - (Breite / 10)
    \ifdim\dimen@>\z@ \dimen@0pt\fi
%The bar will be shortened in the case \dimen@<0 !
    \rel@kern{0.6}\kern-\dimen@
    \if#31
      \overline{\rel@kern{-0.6}\kern\dimen@\macc@nucleus\rel@kern{0.4}\kern\dimen@}%
      \advance\dimen@0.4\dimexpr\macc@kerna
%Place the combined final kern (-\dimen@) if it is >0 or if a superscript follows:
      \let\final@kern#2%
      \ifdim\dimen@<\z@ \let\final@kern1\fi
      \if\final@kern1 \kern-\dimen@\fi
    \else
      \overline{\rel@kern{-0.6}\kern\dimen@#1}%
    \fi
  }%
  \macc@depth\@ne
  \let\math@bgroup\@empty \let\math@egroup\macc@set@skewchar
  \mathsurround\z@ \frozen@everymath{\mathgroup\macc@group\relax}%
  \macc@set@skewchar\relax
  \let\mathaccentV\macc@nested@a
%The following initialises \macc@kerna and calls \mathaccent:
  \if#31
    \macc@nested@a\relax111{#1}%
  \else
%If the argument consists of more than one symbol, and if the first token is
%a letter, use that letter for the computations:
    \def\gobble@till@marker##1\endmarker{}%
    \futurelet\first@char\gobble@till@marker#1\endmarker
    \ifcat\noexpand\first@char A\else
      \def\first@char{}%
    \fi
    \macc@nested@a\relax111{\first@char}%
  \fi
  \endgroup
}
\makeatother
\let\what\widehat
\let\wtld\widetilde
\let\wbar\widebar
\let\ubar\underline

\DeclareMathOperator\sgn{sgn}

\let\oldleft\left
\let\oldright\right
\renewcommand{\left}{\mathopen{}\mathclose\bgroup\oldleft}
\renewcommand{\right}{\aftergroup\egroup\oldright}
