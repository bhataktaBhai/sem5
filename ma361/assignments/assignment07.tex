\documentclass[12pt]{article}
% \usepackage{parskip}
\usepackage{lmodern} % https://tex.stackexchange.com/a/58088/295755
\renewcommand\bfdefault{b}
\usepackage{microtype}

\usepackage{amsmath}
\newcommand\yesnumber{\stepcounter{equation}\tag{\theequation}}

\ifundef{\chapter}{}{\providecommand\thechapter{\Roman{chapter}}}

\usepackage{marginnote}
\usepackage[en-GB,calc]{datetime2}
\usepackage{calc}
\usepackage{xifthen}
\usepackage{tocloft}

% https://tex.stackexchange.com/a/454168
\newcommand\monthday[1]{\DTMmonthname{\DTMfetchmonth{#1}}~\number\DTMfetchday{#1}}

\newlistof{lecture}{lec}{Lectures} % what is this file extension business?
\makeatletter
\setlength\marginparwidth{1in}
\newcommand*{\lecture}[3][]{
    \ifthenelse{\isempty{#1}}{%
        \refstepcounter{lecture}%
    }{%
        \setcounter{lecture}{#1}%
    }%
    \DTMsavedate{lecdate}{#2}%
    \def\lecdow{\DTMweekdayname{\DTMfetchdow{lecdate}}}
    \def\lecshortdow{\DTMshortweekdayname{\DTMfetchdow{lecdate}}}
    \def\lecmonth{\DTMmonthname{\DTMfetchmonth{lecdate}}}
    \def\lecday{\number\DTMfetchday{lecdate}}
    % \marginpar{\raggedright\small \textsf{\textbf{Lecture \thelecture.}%
    %             \footnotesize\DTMusedate{lecdate}}}%
    \marginnote{\raggedright\small%
        \textsf{{\textbf{Lecture \thelecture.}} \\
        \footnotesize\lecdow\\\lecmonth\ \lecday}}%
    \ifthenelse{\isempty{#3}}{%
        \addcontentsline{lec}{lecture}{\protect\numberline{\thelecture}%
        \lecshortdow, \lecmonth\ \lecday}%
        \def\@lecture{Lecture \thelecture}%
    }{%
        \addcontentsline{lec}{lecture}{\protect\numberline{\thelecture}%
        \makebox[\widthof{Mon,}][l]{\lecshortdow,}\ \makebox[\widthof{September 00}][l]{\lecmonth\ \lecday} #3}%
        \def\@lecture{Lecture \thelecture: #3}%
    }%
    \par%
}
\g@addto@macro\normalsize{%
  \setlength\abovedisplayskip{7pt}%
  \setlength\belowdisplayskip{7pt}%
  \setlength\abovedisplayshortskip{1pt}%
  \setlength\belowdisplayshortskip{1pt}%
}
\makeatother

\usepackage[twoside]{fancyhdr}
\setlength{\headheight}{15pt}
\pagestyle{fancy}
\fancyhf{}
% \fancyhead[r]{\thepage}
\makeatletter
\fancyhead[LE,RO]{\thepage}
\fancyhead[RE]{\textbf{\nouppercase\leftmark}}
\fancyhead[LO]{\nouppercase\rightmark}
\providecommand\@lecture{}
\fancyfoot[R]{\small\@lecture}
\makeatother

% homeworks
\newlistof{hw}{hw}{Assignments} % counter `assignment' already defined
\makeatletter
\newcommand*{\assignment}[5][]{%[number]{file}{date posted}{date due}{date quiz}
    \ifthenelse{\isempty{#1}}{%
        \refstepcounter{hw}%
        \stepcounter{assignment}%
    }{%
        \setcounter{hw}{#1}%
        \setcounter{assignment}{#1}%
    }%
    \pagebreak
    \ifthenelse{\isempty{#3}}{}{\DTMsavedate{posted}{#3}}%
    \ifthenelse{\isempty{#4}}{}{\DTMsavedate{due}{#4}}%
    \ifthenelse{\isempty{#5}}{}{\DTMsavedate{quiz}{#5}}%
    \section*{Assignment \thehw}
    \ifthenelse{\isempty{#4}}{
        \ifthenelse{\isempty{#5}}{
            \ifthenelse{\isempty{#3}}{
                \def\hw@toc{}
            }{
                \def\hw@toc{posted \monthday{up}}
            }
        }{
            \def\hw@toc{quiz \monthday{quiz}}
        }
    }{
        \def\hw@toc{due \monthday{due}}
    }
    \addcontentsline{hw}{hw}{\protect\numberline{\thehw}\hw@toc}\par
    \marginpar{\raggedright\footnotesize\textsf{%
        \ifthenelse{\isempty{#3}}{}{\makebox[\widthof{quiz}][l]{up} \monthday{posted} \\}%
        \ifthenelse{\isempty{#4}}{}{\makebox[\widthof{quiz}][l]{due} \monthday{due} \\}%
        \ifthenelse{\isempty{#5}}{}{quiz \monthday{quiz}}%
    }}
    \def\@lecture{Assignment \thehw\ifx\hw@toc\empty{}\else\ --- \hw@toc\fi}
    \input{#2}
    \newpage
}
\makeatother

\usepackage{amsmath}
\usepackage{amsthm}
\usepackage[dvipsnames]{xcolor}
\colorlet{exercise}{cyan!70!black}
\colorlet{solved}{green!40!black}
\colorlet{self_proof}{blue!70!black}
\colorlet{Red}{red!80!black}

% Gilles Castel's theorems
\newtheoremstyle{mddefinition}% <name>
  {-.25\topsep}%                 <space above>
  {-.25\topsep}%                         <space below>
  {\normalfont}%              <body font>
  {}%                         <indent amount>
  {\bfseries}%                <theorem head font>
  {.}%                        <punctuation after theorem head>
  {.5em}%                     <space after theorem head>
  {}%                         <theorem head spec>
\newtheoremstyle{mdplain}% <name>
  {-.25\topsep}%                 <space above>
  {-.25\topsep}%                         <space below>
  {\itshape}%                 <body font>
  {}%                         <indent amount>
  {\bfseries}%                <theorem head font>
  {.}%                        <punctuation after theorem head>
  {.5em}%                     <space after theorem head>
  {}%                         <theorem head spec>

\usepackage[framemethod=Tikz]{mdframed}
\mdfsetup{skipbelow=0pt}
\mdfdefinestyle{axiomstyle}{
    outerlinewidth = 1.5,
    roundcorner = 10,
    leftmargin = 15,
    rightmargin = 15,
    backgroundcolor = yellow!7
}
\mdfdefinestyle{defstyle}{
    outerlinewidth = 1,
    roundcorner = 2,
    leftmargin = 7,
    rightmargin = 7,
    backgroundcolor = green!5
}
\mdfdefinestyle{thmstyle}{
    outerlinewidth = 1,
    roundcorner = 8,
    leftmargin = 7,
    rightmargin = 7,
    backgroundcolor = cyan!5
}
\mdfdefinestyle{lemmastyle}{
    outerlinewidth = 1.5,
    roundcorner = 10,
    leftmargin = 7,
    rightmargin = 7,
    backgroundcolor = yellow!10
}
\ifundef\chapter{%
    \providecommand\theoremnumberwithin{section}
    \theoremstyle{mddefinition}
    \newmdtheoremenv[nobreak=true, style=axiomstyle]{axiom}{Axiom}[section]
    \theoremstyle{plain}
    \newtheorem{theorem}{Theorem}[\theoremnumberwithin]

    \newcounter{assignment}
    \theoremstyle{plain}
    \newtheorem{problem}{Problem}
    \theoremstyle{mddefinition}
    \newmdtheoremenv[nobreak=true, outerlinewidth=0.7]{problem*}[problem]{Problem}
}{
    \providecommand\theoremnumberwithin{chapter}
    \theoremstyle{mddefinition}
    \newmdtheoremenv[nobreak=true, style=axiomstyle]{axiom}{Axiom}[chapter]
    \theoremstyle{plain}
    \newtheorem{theorem}{Theorem}[\theoremnumberwithin]

    \newcounter{assignment}
    \theoremstyle{plain}
    \newtheorem{problem}{Problem}[assignment]
    \theoremstyle{mddefinition}
    \newmdtheoremenv[nobreak=true, outerlinewidth=0.7]{problem*}[problem]{Problem}
}
\theoremstyle{mddefinition}
\newmdtheoremenv[nobreak=true, style=defstyle]{definition*}[theorem]{Definition}

\theoremstyle{mdplain}
\newmdtheoremenv[nobreak=true, style=thmstyle]{theorem*}[theorem]{Theorem}
\newmdtheoremenv[nobreak=true]{proposition*}[theorem]{Proposition}
\newmdtheoremenv[nobreak=true]{lemma*}[theorem]{Lemma}
\newmdtheoremenv[nobreak=true]{corollary*}[theorem]{Corollary}
\newmdtheoremenv[nobreak=true, style=thmstyle]{fact*}[theorem]{Fact}
\newmdtheoremenv[nobreak=true]{exercise*}[theorem]{Exercise}
\newmdtheoremenv[nobreak=true]{question*}[theorem]{Question}

\theoremstyle{definition}
\newtheorem{definition}[theorem]{Definition}

\theoremstyle{plain}
\newtheorem{proposition}[theorem]{Proposition}
\newtheorem{lemma}[theorem]{Lemma}
\newtheorem{corollary}[theorem]{Corollary}
\newtheorem{fact}[theorem]{Fact}
\newtheorem{exercise}[theorem]{Exercise}
\newtheorem{question}[theorem]{Question}

\theoremstyle{remark}
\newtheorem*{remark}{Remark}
\newtheorem*{remarkx}{Remarks}
\newtheorem*{example}{Example}
\newtheorem*{examplex}{Examples}
\newtheorem*{idea}{Idea}
%%%% HAXXXXXX %%%%
% \def\innerqed{\qedsymbol}
% \def\outerqed{$\blacksquare$}
\let\oldproof\proof
\let\endoldproof\endproof
\newenvironment{solution}[1][]
  {\renewcommand\qedsymbol{$\blacksquare$}%
  \begin{oldproof}[Solution\ifx&#1&\else\ (#1)\fi]}
  {\end{oldproof}}
\newenvironment{answer}
  {\renewcommand\qedsymbol{$\blacksquare$}\begin{oldproof}[Answer]}
  {\end{oldproof}}
\renewenvironment{proof}
  {\renewcommand\qedsymbol{$\blacksquare$}\begin{oldproof}}
  {\end{oldproof}}
\newenvironment{subproof}[1][Subproof]{%
  \renewcommand{\qedsymbol}{$\square$}\begin{oldproof}[#1]}
  {\end{oldproof}}
\newtheorem*{notation}{Notation}
\newtheorem*{claim}{Claim}

\usepackage{hyperref}
\usepackage[noabbrev]{cleveref}

% <cref>
\crefname{theorem}{theorem}{theorems}
\crefname{proposition}{proposition}{propositions}
\crefname{lemma}{lemma}{lemmas}
\crefname{corollary}{corollary}{corollaries}
\crefname{axiom}{axiom}{axioms}
\crefname{definition}{definition}{definitions}
\crefname{problem}{problem}{problems}
\crefname{exercise}{exercise}{exercises}
\crefname{fact}{fact}{facts}
\crefname{question}{question}{questions}
\crefname{remark}{remark}{remarks}
\crefname{example}{example}{example}
\crefname{notation}{notation}{notations}
\crefname{claim}{claim}{claims}
% \crefname{section}{\S}{\S\S}
\crefname{theorem*}{theorem}{theorems}
\crefname{proposition*}{proposition}{propositions}
\crefname{lemma*}{lemma}{lemmas}
\crefname{corollary*}{corollary}{corollaries}
\crefname{definition*}{definition}{definitions}
\crefname{problem*}{problem}{problems}
\crefname{exercise*}{exercise}{exercises}
\crefname{fact*}{fact}{facts}
\crefname{question*}{question}{questions}
% </cref>

% <hyperlinks>
\hypersetup{colorlinks,
    linkcolor={blue},
    citecolor={blue!50!black},
    urlcolor={blue!80!black}}
% </hyperlinks>

\usepackage[shortlabels]{enumitem}
% change default label for enumerate, and fix long labels popping out
\setenumerate{label*=(\roman*),ref=(\roman*),leftmargin=*}
% casework list using https://tex.stackexchange.com/a/30035
\newcounter{casecount}
\newlist{casework}{description}{1}
\setlist[casework]{%
  before={\setcounter{casecount}{0}%
      \renewcommand*\thecasecount{\arabic{casecount}}}%
  ,font=\bfseries Case \stepcounter{casecount}\thecasecount:
}

\newenvironment{examples}[1][]
{\begin{examplex}[#1]\leavevmode\begin{itemize}}{\end{itemize}\end{examplex}}
\newenvironment{remarks}[1][]
{\begin{remarkx}[#1]\leavevmode\begin{itemize}}{\end{itemize}\end{remarkx}}

% omg this is so HaXy
% \renewenvironment{proof}[1][\proofname]{{\it\bfseries #1. }}{\qed}
% \providecommand{\qedsymbol}{\openbox}
% \makeatletter
% \renewenvironment{proof}[1][\proofname]{\par
%   \pushQED{\qed}%
%   \normalfont \topsep6\p@\@plus6\p@\relax
%   \trivlist
%   \item[\hskip\labelsep
%         \itshape\bfseries%this is the change (boldface instead of italics)
%         % \fontseries{bx}\selectfont
%     #1\@addpunct{.}]\ignorespaces
% }{%
%   \popQED\endtrivlist\@endpefalse
% }
\makeatother
\crefname{enumi}{part}{parts}
\crefname{enumii}{part}{parts}
\crefname{enumiii}{part}{parts}
% \setlist[itemize]{itemsep=2pt}
\newcounter{dummy}
\makeatletter
\newcommand\myitem[1][]{\item[#1]\refstepcounter{dummy}\def\@currentlabel{#1}}
\makeatother

\makeatletter
\newcommand*{\refifdef}[3]{%label,command,fallback
    \@ifundefined{r@#1}{#3}{#2{#1}}%
}
\makeatother

\newcommand\ie{\textit{i.e.}}
\newcommand\eg{\textit{e.g.}}
\usepackage{physics}

\usepackage{amsmath}
\usepackage{amssymb}
\usepackage{mathrsfs} % for \mathscr
\usepackage{bm} % for \bm
\usepackage{booktabs}

% undefine \abs and \norm
\let\abs\relax
\let\norm\relax

\usepackage{mathtools} % for delimiters and \coloneqq
\DeclarePairedDelimiter{\paren}{(}{)}
\DeclarePairedDelimiter{\brk}{[}{]}
\DeclarePairedDelimiter{\set}{\{}{\}}
\DeclarePairedDelimiter{\abs}{\lvert}{\rvert}
\DeclarePairedDelimiter{\norm}{\lVert}{\rVert}
\DeclarePairedDelimiter{\floor}{\lfloor}{\rfloor}
\DeclarePairedDelimiter{\ceil}{\lceil}{\rceil}
\DeclarePairedDelimiter{\angled}{\langle}{\rangle}
% \DeclarePairedDelimiterX{\innerp}[2]{\langle}{\rangle}{#1,\,#2}
% \DeclarePairedDelimiterX{\outerp}[2]{\langle}{\rangle}{#1\otimes#2}
% \DeclarePairedDelimiterX{\braket}[3]{\langle}{\rangle}%
% {#1\,\delimsize\vert\,\mathopen{}#2\,\delimsize\vert\,\mathopen{}#3}
\DeclarePairedDelimiterX{\innerp}[2]{\langle}{\rangle}{#1,\,#2}
% \DeclarePairedDelimiterX{\outerp}[2]{\langle}{\rangle}{#1\otimes#2}
\DeclarePairedDelimiterX{\outerp}[2]{\vert}{\vert}%
{#1\delimsize\rangle\delimsize\langle\mathopen{}#2}
\let\braket\relax
\DeclarePairedDelimiterX{\braket}[3]{\langle}{\rangle}%
{#1\,\delimsize\vert\,\mathopen{}#2\,\delimsize\vert\,\mathopen{}#3}

\renewcommand\O{\ensuremath{\varnothing}}
\newcommand\N{\ensuremath{\mathbb{N}}}
\newcommand\Z{\ensuremath{\mathbb{Z}}}
\newcommand\Q{\ensuremath{\mathbb{Q}}}
\newcommand\R{\ensuremath{\mathbb{R}}}
\newcommand\C{\ensuremath{\mathbb{C}}}
% \renewcommand\P{\ensuremath{\mathbb{P}}}

% fix spacing for \forall and \exists
% \let\oldforall\forall
% \renewcommand{\forall}{\oldforall \, }
% \let\oldexist\exists
% \renewcommand{\exists}{\oldexist \: }
\newcommand\unique{\exists!}
\newcommand\lxor{\oplus}

\providecommand{\dd}{\,\mathrm{d}}

\newcommand\mcA{\ensuremath{\mathcal{A}}}
\newcommand\mcB{\ensuremath{\mathcal{B}}}
\newcommand\mcC{\ensuremath{\mathcal{C}}}
\newcommand\mcD{\ensuremath{\mathcal{D}}}
\newcommand\mcE{\ensuremath{\mathcal{E}}}
\newcommand\mcF{\ensuremath{\mathcal{F}}}
\newcommand\mcG{\ensuremath{\mathcal{G}}}
\newcommand\mcH{\ensuremath{\mathcal{H}}}
\newcommand\mcI{\ensuremath{\mathcal{I}}}
\newcommand\mcJ{\ensuremath{\mathcal{J}}}
\newcommand\mcK{\ensuremath{\mathcal{K}}}
\newcommand\mcL{\ensuremath{\mathcal{L}}}
\newcommand\mcM{\ensuremath{\mathcal{M}}}
\newcommand\mcN{\ensuremath{\mathcal{N}}}
\newcommand\mcO{\ensuremath{\mathcal{O}}}
\newcommand\mcP{\ensuremath{\mathcal{P}}}
\newcommand\mcQ{\ensuremath{\mathcal{Q}}}
\newcommand\mcR{\ensuremath{\mathcal{R}}}
\newcommand\mcS{\ensuremath{\mathcal{S}}}
\newcommand\mcT{\ensuremath{\mathcal{T}}}
\newcommand\mcU{\ensuremath{\mathcal{U}}}
\newcommand\mcV{\ensuremath{\mathcal{V}}}
\newcommand\mcW{\ensuremath{\mathcal{W}}}
\newcommand\mcX{\ensuremath{\mathcal{X}}}
\newcommand\mcY{\ensuremath{\mathcal{Y}}}
\newcommand\mcZ{\ensuremath{\mathcal{Z}}}

%%%% WIDE BAR THAT IS JUST THE RIGHT LENGTH %%%%
%% FROM https://tex.stackexchange.com/a/60253 %%
\makeatletter
\let\save@mathaccent\mathaccent
\newcommand*\if@single[3]{%
  \setbox0\hbox{${\mathaccent"0362{#1}}^H$}%
  \setbox2\hbox{${\mathaccent"0362{\kern0pt#1}}^H$}%
  \ifdim\ht0=\ht2 #3\else #2\fi
  }
%The bar will be moved to the right by a half of \macc@kerna, which is computed by amsmath:
\newcommand*\rel@kern[1]{\kern#1\dimexpr\macc@kerna}
%If there's a superscript following the bar, then no negative kern may follow the bar;
%an additional {} makes sure that the superscript is high enough in this case:
\newcommand*\widebar[1]{\@ifnextchar^{{\wide@bar{#1}{0}}}{\wide@bar{#1}{1}}}
%Use a separate algorithm for single symbols:
\newcommand*\wide@bar[2]{\if@single{#1}{\wide@bar@{#1}{#2}{1}}{\wide@bar@{#1}{#2}{2}}}
\newcommand*\wide@bar@[3]{%
  \begingroup
  \def\mathaccent##1##2{%
%Enable nesting of accents:
    \let\mathaccent\save@mathaccent
%If there's more than a single symbol, use the first character instead (see below):
    \if#32 \let\macc@nucleus\first@char \fi
%Determine the italic correction:
    \setbox\z@\hbox{$\macc@style{\macc@nucleus}_{}$}%
    \setbox\tw@\hbox{$\macc@style{\macc@nucleus}{}_{}$}%
    \dimen@\wd\tw@
    \advance\dimen@-\wd\z@
%Now \dimen@ is the italic correction of the symbol.
    \divide\dimen@ 3
    \@tempdima\wd\tw@
    \advance\@tempdima-\scriptspace
%Now \@tempdima is the width of the symbol.
    \divide\@tempdima 10
    \advance\dimen@-\@tempdima
%Now \dimen@ = (italic correction / 3) - (Breite / 10)
    \ifdim\dimen@>\z@ \dimen@0pt\fi
%The bar will be shortened in the case \dimen@<0 !
    \rel@kern{0.6}\kern-\dimen@
    \if#31
      \overline{\rel@kern{-0.6}\kern\dimen@\macc@nucleus\rel@kern{0.4}\kern\dimen@}%
      \advance\dimen@0.4\dimexpr\macc@kerna
%Place the combined final kern (-\dimen@) if it is >0 or if a superscript follows:
      \let\final@kern#2%
      \ifdim\dimen@<\z@ \let\final@kern1\fi
      \if\final@kern1 \kern-\dimen@\fi
    \else
      \overline{\rel@kern{-0.6}\kern\dimen@#1}%
    \fi
  }%
  \macc@depth\@ne
  \let\math@bgroup\@empty \let\math@egroup\macc@set@skewchar
  \mathsurround\z@ \frozen@everymath{\mathgroup\macc@group\relax}%
  \macc@set@skewchar\relax
  \let\mathaccentV\macc@nested@a
%The following initialises \macc@kerna and calls \mathaccent:
  \if#31
    \macc@nested@a\relax111{#1}%
  \else
%If the argument consists of more than one symbol, and if the first token is
%a letter, use that letter for the computations:
    \def\gobble@till@marker##1\endmarker{}%
    \futurelet\first@char\gobble@till@marker#1\endmarker
    \ifcat\noexpand\first@char A\else
      \def\first@char{}%
    \fi
    \macc@nested@a\relax111{\first@char}%
  \fi
  \endgroup
}
\makeatother
\let\what\widehat
\let\wtld\widetilde
\let\wbar\widebar
\let\ubar\underline

\DeclareMathOperator\sgn{sgn}

\let\oldleft\left
\let\oldright\right
\renewcommand{\left}{\mathopen{}\mathclose\bgroup\oldleft}
\renewcommand{\right}{\aftergroup\egroup\oldright}


\newcommand\highlight[1]{\textcolor{blue}{#1}}

\title{Homework 7}
\author{Naman Mishra (22223)}
\date{1 October, 2024}

\begin{document}
\maketitle

% Problem 1
\begin{problem*}
    Let $\mu$ and $\nu$ be Borel probability measures on $\R$.
    Suppose there exists a probability measure $\theta$ on $\R^2$ having
    marginals $\theta \circ \Pi_1^{-1} = \mu$
    and $\theta \circ \Pi_2^{-1} = \nu$
    such that $\theta\sset{(x, x)}{x \in \R} > 0$.
    Then show that $\mu$ and $\nu$ cannot be singular.
\end{problem*}
\begin{solution}
    Assume there exists a measurable set $A$ such that
    $\mu(A) = \nu(A^c) = 1$.
    Let $U = A \times \R$ and $V = \R \times A^c$.
    Then $\theta(U) = (\theta \circ \Pi_1^{-1})(A) = 1$ and
    $\theta(V) = (\theta \circ \Pi_2^{-1})(A^c) = 1$.
    Thus $\theta(U^c \cup V^c) \le \theta(U^c) + \theta(V^c) = 0$,
    so $\theta(U \cap V) = 1$ (de Morgan).
    But $U \cap V = A \times A^c$ lies entirely outside the diagonal,
    since $(x, x) \in A \times A^c \implies A \cap A^c \ne \O$.
    Thus $\theta\sset{(x, x)}{x \in \R} \le \theta((U \cap V)^c) = 0$,
    a contradiction.

    (More directly, the diagonal lies inside $U^c \cup V^c$, but my brain
    finds it easier to see that it lies outside $U \cap V$.)
\end{solution}

% Problem 2
\begin{problem*}
    Place $r_m$ balls in $m$ bins at random and count the number of empty
    bins $Z_m$.
    Fix $\delta > 0$.
    If $r_m > (1 + \delta) m \log m$, show that $\P(Z_m > 0) \to 0$ while
    if $r_m < (1 - \delta) m \log m$, show that $\P(Z_m > 0) \to 1$.
\end{problem*}
\begin{solution}
    This is simply by the coupon collector problem.

    Let $X_n \sim \Unif([m])$ be iid, and note that \[
        Z_m \sim m - \#\set{X_1, \dots, X_{r_m}}.
    \] Write \[
        Z_m = \sum_{k=1}^m Z_{m,k},
            \quad \text{where } Z_{m,k} = \prod_{i=1}^{r_m} \1{X_i \ne k}
    \] indicates that bin $k$ is empty.
    Now by independence of $X_i$s,
    \begin{align*}
        \E[Z_{m,k}] &= \P\set{X_1 \ne k}^{r_m}
            = \ab(1 - \frac1m)^{r_m}, \\
        \E[Z_{m,k}Z_{m,\ell}] &= \P\set{X_1 \ne k, \ell}^{r_m}
            = \ab(1 - \frac2m)^{r_m}
        \intertext{for each $k \ne \ell$. Thus}
        \E[Z_m] &= \sum_{k=1}^m \E[Z_{m,k}]
            = m \ab(1 - \frac1m)^{r_m} \\
        \E[Z_m^2] &= \E\ab[\sum_{k=1}^m Z_{m,k}
            + \sum_{k \ne \ell} Z_{m,k}Z_{m,\ell}] \\
            &= m \ab(1 - \frac1m)^{r_m}
                + m(m-1) \ab(1 - \frac2m)^{r_m}.
    \end{align*}
    For $r_m > (1 + \delta) m \log m$, we have \[
        \E[Z_m] \le m\ab(1 - \frac1m)^{(1 + \delta) m \log m}
        \le m{(e^{-1/m})}^{(1 + \delta) m \log m}
            = \frac1{m^\delta} \to 0.
    \]
    By Markov's inequality, \begin{align*}
        \highlight{\P(Z_m > 0) = \P(Z_m \ge 1) \le \E[Z_m] \to 0}.
    \end{align*}

    For $r_m < (1 - \delta) m \log m$, we have
    \begin{gather*}
        e^{(-\frac1m-\frac1{m^2})r_m}
            \le \ab(1 - \frac1m)^{r_m}
            \le e^{-\frac1m r_m}, \\
        \ab(1 - \frac2m)^{r_m}
            \le e^{-\frac2m r_m}.
    \end{gather*}
    This gives \begin{align*}
        \frac{\E[Z_m]^2}{\E[Z_m^2]}
            &\ge \frac{m^2 e^{2(-\frac1m-\frac1{m^2})r_m}}
                {m e^{-\frac1m r_m} + m(m-1) e^{-\frac2m r_m}} \\
            &= \frac{m e^{-\frac 2{m^2} r_m}}{e^{\frac1m r_m} + (m-1)} \\
            &\ge \frac{m e^{-\frac 2{m^2} (1 - \delta) m \log m}}
                {e^{\frac1m (1 - \delta) m \log m} + m - 1} \\
            &= \frac{m \cdot m^{-2(1 - \delta)\frac1m}}
                    {m^{1 - \delta} + m - 1} \\
            &= \frac{{\ab(m^{-\frac1m})}^{2(1 - \delta)}}
                {m^{-\delta} + 1 - m^{-1}} \to 1
    \end{align*} since $\lim\limits_{x \to \infty} x^{\frac1x} = 1$.

    By the Paley-Zygmund inequality, \[
        \highlight{\P(Z_m > 0) \ge \frac{\E[Z_m]^2}{\E[Z_m^2]} \to 1}.
        \qedhere
    \]
\end{solution}

\section*{Recap of product measures}
If $X_1, X_2, \dots\colon (\Omega, \F, \P) \to \R$ are measurable functions,
so is $X = (X_1, X_2, \dots)$.
The push-forward $\mu = {\P} \circ X^{-1}$ is then a probability measure on
$\R^\N$.
Obviously, $\mu_k = {\P} \circ X_k^{-1}$ is a probability measure on $\R$.
But $X_k = \Pi_k \circ X$, so $\mu_k = \mu \circ \Pi_k^{-1}$.
\begin{quote}
    Why? Change of variables gives that if a random variable $X$ has
    distribution $\mu$, then $Y = g(X)$ has distribution $\mu \circ g^{-1}$,
    because \[
        \P(Y^{-1}(A)) = \P((g \circ X)^{-1}(A))
            = \P(X^{-1}(g^{-1}(A)))
            = \mu(g^{-1}(A)).
    \]
\end{quote}
In other words, $\mu_k(A) = \mu(\R^{k-1} \times A \times \R^\N)$.
If $X_k$'s are known to be independent, then \[
    \P\set{X_1 \in A_1, \dots, X_n \in A_n}
        = \prod_{k=1}^n \mu_k(A_k)
\] for any $A_k \in \mcB(\R)$.
Since cylinder sets are a $\pi$-system generating $\mcB(\R^\N)$, this
uniquely determines $\mu$.

\begin{definition*}[product measure] \label{def:product}
% Let (Ωi, Fi, μi), i ∈ I, be probability spaces indexed by an arbitrary set I. Let Ω = ×i∈IΩi and let F (usually denoted ⊗i∈IFi) be the sigma-algebra generated by all finite dimensional cylinders (equivalently, the smallest sigma-algebra on Ω for which all the projections Πi : Ω → Ωi are measurable). If μ is a probability measure on (Ω, F) such that for any cylinder
% setA=Π−1(A )∩...∩Π−1(A )forsomeA ∈F ,
% then we say that μ is the product of μi, i ∈ I, and write μ = ⊗i∈Iμi.
    Let $(\Omega_i, \mcF_i, \mu_i)$, $i \in I$, be probability spaces indexed
    by an arbitrary set $I$.
    Let $\Omega = \prod_{i \in I} \Omega_i$ and let $\mcF$ (usually denoted
    $\bigotimes_{i \in I} \mcF_i$) be the $\sigma$-algebra generated by all
    finite-dimensional cylinders (equivalently, the smallest
    $\sigma$-algebra on $\Omega$ for which all the projections
    $\Pi_i \colon \Omega \to \Omega_i$ are measurable).
    If $\mu$ is a probability measure on $(\Omega, \mcF)$ such that for any
    cylinder set
    $A = \Pi_{i_1}^{-1}(A_{i_1}) \cap \cdots \cap \Pi_{i_n}^{-1}(A_{i_n})$
    for some $A_{i_r} \in \mcF_{i_r}$, \[
        \mu(A) = \prod_{r=1}^k \mu_{i_r}(A_{i_r}),
    \] then we say that $\mu$ is the product of $\mu_i$, $i \in I$,
    and write $\mu = \bigotimes\limits_{i \in I} \mu_i$.
\end{definition*}

\begin{theorem}
    Let $\mu_n \in \mcP(\R)$.
    Then $\mu = \bigotimes_{n \in \N} \mu_n$ exists and is unique.
\end{theorem}
\begin{proof}
    Let $X_n$ be independent random variables with distribution $\mu_n$.
    Then $X = (X_n)_n$ is a random variable with distribution $\mu$.
\end{proof}

\subsection{Fubini's theorem} \label{sec:fubini}
\begin{theorem*}[Fubini-Tonelli theorem] \label{thm:fubini}
% Let(Ω,F,P),i = 1,2,beprobabilityspacesandletΩ = Ω ×Ω,F = F ⊗F and iii 1212
% P = P ⊗ P . Let Y : Ω → R be a random variable that is either positive or integrable w.r.t. P. 12
% Then,Y(ω ,·):Ω →RisarandomvariableonΩ foreachω ∈Ω ,andiseitherpositive 12211
% R
% or integrable (w.r.t. P ) for a.e. ω [P ]. Further, the function ω 7→ Ω Y(ω , ω )dP (ω ) 2 11 1 1222
% is a random variable on Ω , and is either positive or integrable. Finally, 1
% Two remarks:
% ZZ Z Y(ω,ω)dP(ω) dP(ω)= YdP.
    Let $(\Omega_i, \mcF_i, \P_i)$, $i = 1, 2$, be probability spaces and
    let $\Omega = \Omega_1 \times \Omega_2$, $\mcF = \mcF_1 \otimes \mcF_2$,
    and $\P = \P_1 \otimes \P_2$.
    Let $Y\colon \Omega \to \R$ be a random variable that is either positive
    or integrable with respect to $\P$.
    Then $Y(\omega_1, \cdot)\colon \Omega_2 \to \R$ is a random variable on
    $\Omega_2$ for each $\omega_1 \in \Omega_1$, and is either positive or
    integrable (with respect to $\P_2$) for almost every $\omega_1$
    $[\P_1]$.
    Further, the function
    $\omega_1 \mapsto \int_{\Omega_2} Y(\omega_1, \omega_2) \dd \P_2(\omega_2)$
    is a random variable on $\Omega_1$, and is either positive or integrable.
    Finally, \[
        \int_{\Omega_1} \ab[\int_{\Omega_2} Y(\omega_1, \omega_2) \dd \P_2(\omega_2)]
                                    \dd \P_1(\omega_1)
            = \int_\Omega Y \dd \P.
    \]
\end{theorem*}

\section{The Radon-Nikodym theorem and conditional probability} \label{sec:radon}
Consider a probability space $(\Omega, \mcF, \P)$.
Let $X\colon \Omega \to \R$ be a non-negative random variable with
expectation $1$.
Define $\Qr(A) \coloneq \E[X\1A]$ for $A \in \mcF$.
Then $\Qr$ is a probability measure on $(\Omega, \mcF)$.
\begin{itemize}
    \item $\Qr(\O) = \E[X\1\O] = 0$.
    \item $\Qr(A \sqcup B) = \E[X\1{A \sqcup B}] = \E[X\1A + X\1B]
        = \Qr(A) + \Qr(B)$.
    \item If $A_n \upto A$, then $\1{A_n} \upto \1A$, so by MCT
        $\Qr(A_n) \upto \Qr(A)$.
\end{itemize}

\begin{definition*}
% Two measures μ and ν on the same (Ω, F) are said to be mutually singular and write μ ⊥ ν if there is a set A ∈ F such that μ(A) = 0 and ν(Ac) = 0. We say that μ is absolutely continuous to ν and write μ ≪ ν if μ(A) = 0 whenever ν(A) = 0.
    Two measures $\mu$ and $\nu$ on the same $(\Omega, \F)$ are said to be
    \emph{mutually singular} and write $\mu \sing \nu$, if these is a set
    $A \in \F$ such that $\mu(A) = \nu(A^c) = 0$.
    We say that $\mu$ is \emph{absolutely continuous} to $\nu$ and write
    $\mu \ll \nu$ if $\mu(A) = 0$ whenever $\nu(A) = 0$.
\end{definition*}

\begin{theorem*}[Radon-Nikodym theorem] \label{thm:radon}
% Suppose μ and ν are two finite measures on (Ω, F). If ν ≪ μ, then dν = fdμ for some f ∈ L1(μ).
    Suppose $\mu$ and $\nu$ are two finite measures on $(\Omega, \F)$.
    If $\nu \ll \mu$, then $\dd \nu = f \dd \mu$ for some $f \in L^1(\mu)$.
\end{theorem*}

\begin{lemma*}
    Let $\mu_1$ and $\mu_2$ be two probability measures on a common
    measurable space $(\Omega, \F)$, and let $\mu_p = p\mu_1 + (1-p)\mu_2$
    for $p \in [0, 1]$.
    Let $X$ be any random variable on $(\Omega, \F)$.
    Then $\int X \dd \mu_p = p \int X \dd \mu_1 + (1-p) \int X \dd \mu_2$.
\end{lemma*}
\begin{proof}
    From the construction of expectation, \[
        \int X \dd \P = \lim_{n \to \infty} \sum_{k=1}^{n2^n-1}
            \frac{k}{2^n} \P\set*{\frac{k}{2^n} \le X < \frac{k+1}{2^n}},
    \] we have \begin{align*}
        \int X \dd \mu_p &= \lim_{n \to \infty} \sum_{k=1}^{n2^n-1}
            \frac{k}{2^n} \mu_p\set*{\frac{k}{2^n} \le X < \frac{k+1}{2^n}} \\
        &= \lim_{n \to \infty} \sum_{k=1}^{n2^n-1}
            \frac{k}{2^n} (p\mu_1 + (1-p)\mu_2)\set*{\frac{k}{2^n} \le X < \frac{k+1}{2^n}} \\
        &= p \int X \dd \mu_1 + (1-p) \int X \dd \mu_2.
    \end{align*}
\end{proof}

% Problem 3
\begin{problem*}
% Letμ,ν∈P(R)andletθ=12μ+21ν. (1) Showthatμ≪θandν≪θ.
% (2) If μ ⊥ ν, describe the Radon Nikodym derivative of μ w.r.t. θ.
    Let $\mu, \nu \in \mcP(\R)$ and let $\theta = \frac12 \mu + \frac12 \nu$.
    \begin{enumerate}
        \item Show that $\mu \ll \theta$ and $\nu \ll \theta$.
        \item If $\mu \sing \nu$, describe the Radon-Nikodym derivative of
            $\mu$ with respect to $\theta$.
    \end{enumerate}
\end{problem*}
\begin{solution} \leavevmode
    \begin{enumerate}
        \item If $\theta(A) = 0$, then $\mu(A) = \nu(A) = 0$.
        Thus $\mu \ll \theta$ and $\nu \ll \theta$.
        \item Let $S \in \mcB(\R)$ be such that $\mu(S) = \nu(S^c) = 1$.
        Choose random variable $X = 2 \times \1S$.
        Then \begin{align*}
            \int X \1A \dd \theta
                &= \int \1A \1S \dd \mu + \int \1A \1S \dd \nu \\
                &= \mu(A \cap S) + \nu(A \cap S) \\
                &= \mu(A).
        \end{align*}
        Thus $\dd \mu = 2 \times \1S \dd \theta$.
        Similarly $\dd \nu = 2 \times \1{S^c} \dd \theta$.
    \end{enumerate}
\end{solution}

% Problem 4
\begin{problem*}
% Decide true or false and justify. Take μi,νi to be proba- bility measures on (Ωi, Fi).
% (1) Ifμ1⊗μ2 ≪ν1⊗ν2,thenμ1 ≪ν1 andμ2 ≪ν2. (2) Ifμ1 ≪ν1 andμ2 ≪ν2,thenμ1⊗μ2 ≪ν1⊗ν2.
    Decide true or false and justify.
    Take $\mu_i, \nu_i$ to be probability measures on $(\Omega_i, \mcF_i)$.
    \begin{enumerate}
        \item If $\mu_1 \otimes \mu_2 \ll \nu_1 \otimes \nu_2$, then
            $\mu_1 \ll \nu_1$ and $\mu_2 \ll \nu_2$.
        \item If $\mu_1 \ll \nu_1$ and $\mu_2 \ll \nu_2$, then
            $\mu_1 \otimes \mu_2 \ll \nu_1 \otimes \nu_2$.
    \end{enumerate}
\end{problem*}
\begin{solution}
    Let $\mathfrak X = \mathfrak X_1 \otimes \mathfrak X_2$ for
    $\mathfrak X = \Omega, \mcF, \mu, \nu$.
    \begin{enumerate}
        \item True.
        $\mu_1(A) = \mu(A \times \R)$ and $\nu_1(A) = \nu(A \times \R)$.
        Then \[
            \nu_1(A) = 0 \iff \nu(A \times \R) = 0
                \implies \mu(A \times \R) = 0
                \iff \mu_1(A) = 0.
        \]
        \item True?
        If $A = A_1 \times A_2$ is a cylinder set, then \[
            \nu(A) = 0 \implies \nu_1(A_1) \nu_2(A_2) = 0
                \implies \mu_1(A_1) \mu_2(A_2) = 0
                \implies \mu(A) = 0.
        \]
    \end{enumerate}
\end{solution}

\begin{problem*} \leavevmode
% (1)Ifμn≪νforeachnandμn→d μ,thenisitnecessarilytruethatμ≪ν?If μn ⊥ ν for each n and μn →d μ, then is it necessarily true that μ ⊥ ν? In either case, justify or give a counterexample.
% (2) Suppose X,Y are independent (real-valued) random variables with distribution μ and ν respectively. If μ and ν are absolutely continuous w.r.t Lebesgue measure, show that the distribution of X + Y is also absolutely continuous w.r.t Lebesgue measure.
    \begin{enumerate}
        \item If $\mu_n \ll \nu$ for each $n$ and $\mu_n \dto \mu$, then is
        it necessarily true that $\mu \ll \nu$?
        If $\mu_n \sing \nu$ for each $n$ and $\mu_n \dto \mu$, then is it
        necessarily true that $\mu \sing \nu$?
        In either case, justify or give a counterexample.
        \item Suppose $X$, $Y$ are independent (real-valued) random
        variables with distribution $\mu$ and $\nu$ respectively.
        If $\mu$ and $\nu$ are absolutely continuous with respect to
        Lebesgue measure, show that the distribution of $X + Y$ is also
        absolutely continuous with respect to Lebesgue measure.
    \end{enumerate}
\end{problem*}
\begin{solution} \leavevmode
    \begin{enumerate}
        \item It may be that $\mu \not\ll \nu$ even when $\mu_n \ll \nu$ for
        each $n$.
        Take $\mu_n = \delta_{\frac1n}$ and
        $\nu = \sum_{n=1}^\infty \frac1{2^n} \mu_n$.
        Then $\nu \gg \mu_n \dto \delta_0 \not\ll \nu$.

        The same is a counterexample for singularity,
        only with $\nu = \delta_0$.
    \end{enumerate}
\end{solution}

\end{document}
