\lecture[7]{2024-08-22}{}

% ind-security
Security parameter $n$ determines how secure the scheme is.

The parties and the adversary are assumed to be computationally bounded by
a polynomial in $n$.

We want the probability of breaking the scheme to be \emph{negligible} in
$n$.

\section{Choosing the parameter} \label{sec:choosing-n}
Suppose there is a sheme which an adversary can break with probability
$2^{40} 2^{-n}$ after running for $n^3$ minutes.
\begin{itemize}
    \item $n \le 40$ is dumb.
    Within $2^{40}$ minutes, approximatly $6$ weeks, the adversary can break
    the scheme with probability $1$.
    \item $n = 50$ sounds nice, but $2^{10}$ is only $1024$.
    An adversary can break the scheme with probability $1/1000$ within
    $6$ weeks.
    \item $n = 128$ works. %👍
\end{itemize}

Finally, we are allowed to leak the \emph{length} of the message.
We will say that an SKE that leaks the length of the message is
secure.
Why?
\begin{exercise}
    Show that there is no scheme which realises the above definition of
    security without leaking the length of the message.
\end{exercise}

\begin{definition}[Parity secure] \label{def:pp-secure}
    The experiment $\privk^{pp}_{A, \Pi}(n)$ returns $1$ if the adversary
    $A$ can predict the parity of the
    message, and $0$ otherwise.
    We say that $\Pi$ is pp-secure if for every PPT adversary $A$,
    \[
        \Pr[\privk^{pp}_{A, \Pi}(n) = 1] \le \frac{1}{2} + \negl(n).
    \]
\end{definition}

\section{Proofs by reduction} \label{sec:reduction}
Proofs from this point on will be highly conditional.
\begin{itemize}
    \item If $A$ holds then $\Pi$ is $x$-secure.
    \item If $\Pi$ is $x$-secure then $A$ holds.
    \item If $A_1$ holds then $A_2$ holds.
    \item If $\Pi$ that is $x$-secure, then $\Pi'$ is $y$-secure.
\end{itemize}
We keep the last one in mind as we try to look at general techniques to
prove such statements.
\begin{itemize}
    \item \textbf{Proof by contradiction/contrapositive:}
    Suppose $\Pi'$ is not $y$-secure.
    Show that $\Pi$ is not $x$-secure.

    That is, there exists a PPT adversary $A'$ that breaks $\Pi'$ with
    non-negligible probability $f(n)$.
    We need to construct a PPT adversary $A$ that breaks $\Pi$ with
    non-negligible probability $g(n)$.
    This is \emph{reduction} of breaking $\Pi$ to breaking $\Pi'$.

    Given a challenge for $\Pi$, we have to simulate a challenge for $\Pi'$
    (in polynomial time).
    Using the broken challenge for $\Pi'$ (with substantial probability),
    we need to break the challenge for $\Pi$ (with substantial probability).

    The product of two substantial probabilities is substantial.
\end{itemize}

\begin{proposition}
    If $\Pi$ is ind-secure, then $\Pi$ is pp-secure.
\end{proposition}
\begin{proof}
    Suppose $\Pi$ is not pp-secure.
    Then there exists a PPT adversary $A$ that predicts the parity of the
    message with probability $1/2 + f(n)$, where $f(n)$ is non-negligible.

    That is, there is some polynomial $p(n)$ such that $f(n) \ge 1/p(n)$
    for infinitely many $n$.
    To summarize, \[
        \Pr[\privk^{pp}_{A, \Pi}(n) = 1] \ge \frac12 + \frac{1}{p(n)}.
    \]

    Let us construct a PPT adversary $A'$ that breaks the ind-security of
    $\Pi$.
    Select two messages $m_0$ and $m_1$ of the same length,
    but with $\parity(m_0) = 0$ and $\parity(m_1) = 1$.
    Pass the challenge to $A$.
    Let $p$ be the reply from $A$.
    Output $p$.
    Then \[
        \privk^{pp}_{A, \Pi}(n) = 1
        \iff
        \privk^{ind}_{A', \Pi}(n) = 1.
    \] Thus \[
        \Pr[\privk^{ind}_{A', \Pi}(n) = 1]
        = \Pr[\privk^{pp}_{A, \Pi}(n) = 1]
        \ge \frac12 + \frac{1}{p(n)}
    \] is substantial.
\end{proof}
