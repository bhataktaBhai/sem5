\lecture{2024-08-23}{The chain rule; directional derivatives}
\begin{exercise}[sum]
    Let $U \subseteq \R^n$ be open and $a \in U$.
    Let $f, g\colon U \to \R^m$ both be differentiable at $U$.
    Then $f+g$ is differentiable at $a$ with
    $(f+g)'(a) = f'(a) + g'(a)$.
\end{exercise}

\begin{lemma} \label{thm:oO}
    Let $f, g, h\colon \R^n \to \R^m$.
    Supppose $g = O(f)$ and $h = o(g)$.
    Then $h = o(f)$.
\end{lemma}
\begin{proof}
    \[
        \frac{\norm{h(x)}}{\norm{f(x)}}
        = \frac{\norm{h(x)}}{\norm{g(x)}} \frac{\norm{g(x)}}{\norm{f(x)}}
    \] The first term goes to $0$ and the second term is bounded.
    Thus $\frac{\norm{h(x)}}{\norm{f(x)}} \to 0$.
    \TODO[What if $f$ vanishes a lot?] % TODO
\end{proof}

\begin{proposition*}[chain rule] \label{thm:chain}
    Let $U \subseteq \R^n$ be open and $a \in U$.
    Let $f\colon U \to \R^m$ be differentiable at $a$.
    Let $V \subseteq \R^m$ be an open set containing $f(a)$
    and $g\colon V \to \R^k$ be differentiable at $f(a)$.
    Then $g \circ f$ is differentiable at $a$ with \[
        (g \circ f)'(a) = g'(f(a)) \circ f'(a).
    \]
\end{proposition*}
Perhaps more intuitive notation is \[
    D_a(g \circ f) = D_{f(a)} g \circ D_a f,
\] so that \[
    D_a(g \circ f)(h) = D_{f(a)} g(D_a f(h)).
\] This is exactly what the chain rule says.
\[
    (g \circ f)'(a)(h) = g'(f(a))(f'(a)(h)).
\]
\begin{proof}
    For small enough $h$, \begin{align*}
        g(f(a + h)) - g(f(a))
            &= g(f(a) + f'(a)h + u(h)) - g(f(a)) \\
            &= g'(f(a))[f'(a)h + u(h)] + v(f'(a)h + u(h))
    \end{align*}
    where $\frac{\norm{u(h)}}{\norm h} \to 0$ and
    $\frac{\norm{v(f'(a)h + u(h))}}{\norm{f'(a)h + u(h)}} \to 0$.

    Call $f(a) = b$ and $f'(a)h + u(h) = k(h)$ for convenience.
    We have \[
        \frac{\norm{u(h)}}{\norm h} \to 0
        \quad \text{and} \quad
        \frac{\norm{v(k(h))}}{\norm{k(h)}} \to 0
    \] and \[
        g(f(a + h)) - g(f(a)) = g'(b)f'(a)h + g'(b)u(h) + v(k(h)).
    \] We need to show \[
        g'(b) u(h) + v(k(h)) = o(h).
    \] The first term is easy, since
    $\norm{g'(b)u(h)} \le \norm{g'(b)} \norm{u(h)}$.

    For the second term, we write \begin{align*}
        \frac{\norm{v(k(h))}}{\norm{h}}
            &= \frac{\norm{v(k(h))}}{\norm{k(h)}}
                \frac{\norm{k(h)}}{\norm h} \\
            &\le \frac{\norm{v(k(h))}}{\norm{k(h)}}
                \ab(\norm{f'(a)} + \frac{\norm{u(h)}}{\norm h})
    \end{align*} which goes to $0$ as the second term is bounded.
    This is problematic if $k(h)$ vanishes at some points
    arbitrarily close to $0$.

    Let $\eps > 0$ be given.
    Then there exists a neighbourhood $W_1$ of $0$ in $V$ on which \[
        \norm{v(x)} \le \eps \norm x.
    \] Since $f$ is continuous, there exists a neighbourhood $W_2$ of $0$
    in $U$ such that \[
        k(h) = f(a + h) - f(a) \in W_1
    \] for each $h \in W_2$.
    Thus on this neighbourhood $W_2$, we have \[
        \frac{\norm{v(k(h))}}{\norm{h}}
        \le \frac{\eps \norm{k(h)}}{\norm h}
        = \eps \frac{\norm{f'(a)h + u(h)}}{\norm h}
        \le \eps \ab(\norm{f'(a)} + \frac{\norm{u(h)}}{\norm h}).
    \] Thus \[
        \lim_{h \to 0} \frac{v(k(h))}{h} = 0. \qedhere
    \]
\end{proof}
How does Rudin deal with the vanishing of $k(h)$?
% \begin{proof}[Proof (Rudin).]
%     Define \[
%         \eta(k) = \begin{cases}
%             \frac{\norm{v(k)}}{\norm k} & \text{if } k \ne 0, \\
%             0 & \text{if } k = 0.
%         \end{cases}
%     \] Then we can write \[
%         \frac{\norm{v(k(h))}{\norm h}
%             = \frac{\eta(k(h)) \norm{k(h)}}{\norm h}
%             \le \eta(k(h)) \ab(\norm{f'(a)} + \frac{\norm{u(h)}}{\norm h}).
%             \qedhere
%     \]
% \end{proof}
\begin{examples}
    \item Consider $f\colon \GL_n(\R) \to M_n(\R)$ and
    $g\colon M_n(\R) \to M_n(\R)$ given by \[
        f(X) = X^{-1} \quad \text{and} \quad g(X) = X^2.
    \] These are differentiable with $f'(X)(H) = -X^{-1}HX^{-1}$ and
    $g'(X)(H) = XH + HX$.
    By the chain rule, we have \begin{align*}
        (g \circ f)'(X)(H) &= g'(f(X))(f'(X)(H)) \\
        &= X^{-1}(-X^{-1}HX^{-1}) + (-X^{-1}HX^{-1})X^{-1} \\
        &= -X^{-1}(X^{-1}H + HX^{-1})X^{-1}.
    \end{align*} 
\end{examples}

\section{Directions and partials} \label{sec:pdv}

\begin{definition}[directional derivative] \label{def:ddv}
    Let $U \subseteq \R^n$ be open and $a \in U$.
    Let $f\colon U \to \R$.
    For $v \in \R^n$, we define \[
        (D_v f)(a) \coloneq \lim_{t \to 0} \frac{f(a + tv) - f(a)}{t}
    \] to be the \emph{directional derivative} of $f$ at $a$
    in the direction of $v$.
\end{definition}

\begin{exercise}
    Show that $D_{\lambda v} = \lambda D_v$.
\end{exercise}
\begin{solution}
    If $\lambda = 0$, both sides are $0$.
    Otherwise, \begin{align*}
        (D_{\lambda v} f)(a)
            &= \lim_{t \to 0} \frac{f(a + t\lambda v) - f(a)}{t} \\
            &= \lambda \lim_{t \to 0} \frac{f(a + (\lambda t)v) - f(a)}
                                    {\lambda t} \\
            &= \lambda \lim_{t \to 0} \frac{f(a + tv) - f(a)}{t} \\
            &= \lambda (D_v f)(a) = (\lambda D_v f)(a). \qedhere
    \end{align*}
\end{solution}

\begin{definition*}[partial derivative] \label{def:pdv}
    Let $U \subseteq \R^n$ be open and $a \in U$.
    Let $f\colon U \to \R$.
    We define \[
        \pdv{f}{x_i}(a) = D_i f(a) = \partial_i f(a) \coloneq (D_{e_i} f)(a)
    \] to be the \emph{partial derivative} of $f$ at $a$
    with respect to the $i$-th coordinate.
\end{definition*}
Observe that if $g = x \mapsto f(a_1,\dots,a_{i-1},x,a_{i+1},\dots,a_n)$,
then $D_i f(a) = g'(a_i).$
