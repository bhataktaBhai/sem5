\chapter{Orthogonal range searching} \label{chp:ors}
\lecture{2024-08-14}{Orthogonal range searching}

\begin{question*}
    Given a set of points in $\R^2$ and a rectangle
    $[x_1, x_2] \times [y_1, y_2]$, report all points in the rectangle.
\end{question*}

\begin{figure}
    \centering
    \begin{tikzpicture}[scale=0.5]
        \tikzmath{
            \x1 = 3; \x2 = 7; \y1 = 2; \y2 = 5; \maxX = 10; \maxY = 9;
        }
        \draw[->] (-1, 0) -- (\maxX + 1, 0) node[right] {$x$};
        \draw[->] (0, -1) -- (0, \maxY + 1) node[above] {$y$};
        \foreach \i in {1,...,32}{
            \node[draw,Red,circle,inner sep=1pt,fill]
                at (\maxX * rnd, \maxY * rnd) {};
        }
        \draw[dashed] (\x1, 0) -- (\x1, \y1);
        \draw[dashed] (\x2, 0) -- (\x2, \y1);
        \draw[dashed] (0, \y1) -- (\x1, \y1);
        \draw[dashed] (0, \y2) -- (\x1, \y2);
        \draw (\x1, \y1) rectangle (\x2, \y2);
    \end{tikzpicture}
    \caption{Orthogonal range searching in $\R^2$.}
    \label{fig:ors}
\end{figure}

Without preprocessing, we have
\begin{itemize}
    \item $O(n)$ space,
    \item $O(n)$ time.
\end{itemize}

In the worst case, we can have $O(n)$ points in the rectangle,
so we cannot do better than $O(n)$ time.

Let $k$ be the number of points in the rectangle.
Usually, $k \ll n$.
Thus we will design a data structure that allows us to report
the $k$ points in $O(f(n) + k)$ time, where $f(n)$ is as small as possible.

\section{The $1$-dimensional case} \label{sec:ors:1d}
\begin{question*}
    Given a set of points in $\R$ and an interval $[x_1, x_2]$,
    report all points in the interval.
\end{question*}
\begin{solution}
    Sort the points in $O(n \log n)$ time using only $O(n)$ space.
    For each query, binary search for $x_1$ and keep going until $x_2$.
    This takes $O(\log n + k)$ time.
\end{solution}

\section{The $2$-dimensional unbounded case} \label{sec:ors:2d-unbounded}
\begin{question*}
    Given a set of points in $\R^2$ and an unbounded rectangle
    $[x_1, x_2] \times [y_1, \infty)$, report all points in the rectangle.
\end{question*}
\begin{solution}[Naive]
    Project onto the $x$-axis and solve the $1$-dimensional case.
    Discard all points with $y$-coordinate less than $y_1$.

    This takes $O(n \log n)$ preprocessing time.
    For each query, it takes $O(\log n + \#[x_1, x_2])$ time,
    where $\#[x_1, x_2]$ is the number of points with $x$-coordinate
    in $[x_1, x_2]$.
\end{solution}
This can be much larger than $k$, in fact, as large as $n$.

\begin{solution}
    Create a \emph{priority search tree} (PST) in $O(n \log n)$ time
    using $O(n)$ space, and sorted by $x$-coordinate with $y$-coordinate
    as priority.

    Each query can be answered as follows.
    \begin{itemize}
        \item Locate the predecessor of $x_1$ and successor of $x_2$.
        To ensure they exist, add sentinels $-\infty$ and $\infty$ during
        preprocessing.
        \item Query each subtree in between with a DFS, searching down until
        the $y$-coordinate is less than $y_1$. \qedhere
    \end{itemize}
\end{solution}
\begin{proof}[Analysis]
    Querying the subtree $\mcT$ takes $O(1 + k_\mcT)$ time, where
    $k_\mcT$ is the number of rectangled points in the subtree.
    This is because each rectangled point is visited exactly once,
    and any non-rectangled point that is checked is preceded by a
    rectangled point.

    Thus the total time is \begin{align*}
        \sum_\mcT O(1 + k_\mcT) &= O\ab(\sum_\mcT 1) + O(k) \\
        &= O(\log n + k).
    \end{align*}
    This is because the each subtree arises from a point on the search path
    for the predecessor or successor, which has length at most $\log n$.
    \begin{algo}
        \Fn{PST-Unbound}{$P$}
            \State $\mcP \gets \text{PST}(P \cup \set{(-\infty, \bot), (\infty, \bot)})$
            \State $u \gets \text{root}(\mcP)$, $v \gets \text{root}(\mcP)$
            \While{$u = v$}
                \State TODO % TODO
            \EndWhile
        \EndFn
    \end{algo}
\end{proof}

\subsection{Priority search trees} \label{sec:ors:pst}
A priority search tree is a binary search tree where each node has an
associated priority.

