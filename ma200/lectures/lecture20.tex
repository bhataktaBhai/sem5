\lecture{2024-09-27}{}
\begin{example}
    $f\colon x \in \R \mapsto x^2 \in \R$.
    Set-theoretically, \[
        f = \sset{(x, x^2)}{x \in \R}
    \] (ignoring the codomain discussion).
    The graph of $f$ is \[
        G(f) = \sset{(x, f(x))}{x \in \R}.
    \]
\end{example}

\section{The implicit function theorem} \label{sec:imft}
Let $f\colon \R^n \times \R^m \to \R^k$ be a differentiable function.
Define the relation \[
    S \coloneq \sset{(x, y)}{f(x, y) = 0}
\] from $\R^n$ to $\R^m$.

The impicit function theorem gives sufficient conditions on $f$ such that
$S$ is a function (graph of a function) on $D(S)$ (domain of $S$).
In other words, $y$ can be solved in terms of $x$.
In other other words, $y = \phi(x)$ for some function
$\phi\colon D(S) \to \R^m$.

\begin{examples}
    \item $f\colon \R \times \R \to \R$ defined by $f(x, y) = x + y - 1$.
    Then $S = \sset{(x, y)}{x + y = 0}$ is the straight line
    \begin{center}
        \begin{tikzpicture}
            \draw[->] (-1, 0) -- (2, 0) node[right] {$x$};
            \draw[->] (0, -1) -- (0, 2) node[above] {$y$};
            \draw[Blue] (-1, 2) -- (2, -1);
        \end{tikzpicture}
    \end{center}
    $D(S) = \R$, and $S$ is a function ($y = 1 - x$).
    \item Let $f\colon \R \times \R \to \R$ be given by \[
        f(x, y) = y^5 + y^3 + y + x.
    \] Fix an $x_0 \in \R$ and let $p = f(x_0)$.
    Then $p$ is a strictly increasing odd degree polynomial, so there is
    exactly one $y_0 \in \R$ such that $f(x_0, y_0) = 0$.
    \item Let $f\colon \R \times \R \to \R$ be given by \[
        f(x, y) = x^2 + y^2 - 1.
    \] The graph of $f$ is the unit circle in $\R^2$.
    \begin{center}
        \begin{tikzpicture}
            \draw[->] (-1.5, 0) -- (1.5, 0) node[right] {$x$};
            \draw[->] (0, -1.5) -- (0, 1.5) node[above] {$y$};
            \draw[Blue] (0, 0) circle (1);
        \end{tikzpicture}
    \end{center}
    Let $(a, b) \in S$.
    The implicit function theorem tells us when $S$ can be \emph{locally} a
    function around $(a, b)$.
    That is, there is a neighbourhood $V$ of $(a, b)$ such that
    $S_1 \coloneq V \cap S$ is a function on $D(S_1)$.

    In this case, $D(S) = [-1, 1]$, and $S$ is locally a function around
    each point except $(-1, 0)$ and $(1, 0)$.
    For $b > 0$, $y = \sqrt{1 - x^2}$ locally, and for $b < 0$,
    $y = -\sqrt{1 - x^2}$ locally.
\end{examples}

% \begin{theorem}[implicit function theorem] \label{thm:imft}
%     Let $k \ge 1$, $U \subopeneq \R \times \R$ and $f \in C^k(U, \R)$.
%     Suppose $(a, b) \in U$ is such that $f(a, b) = 0$ and
%     $\pdv{f}{y}(a, b) \ne 0$.
%     Let $S_1 = \sset{(x, y)}{f(x, y) = 0}$.
%     Then there exists a neighbourhood $V$ of $(a, b)$ and a function
%     $\phi\colon D(S_1) \to \R$ such that
%     \begin{enumerate}
%         \item $(x, \phi(x)) \in V$ for all $x \in D(S_1)$,
%         \item $\phi(a) = b$,
%         \item $f(x, \phi(x)) = 0$ for all $x \in D(S_1)$.
%         \item If $f(x, y) = 0$ for $x \in D(S_1)$ and $(x, y) \in V$,
%             then $y = \phi(x)$.
%     \end{enumerate}
%     Then there exists an open neighbourhood $V$ of $(a, b)$ and an open
%     neighbourhood $W$ of $a$ such that
% \end{theorem}
\begin{theorem*}[implicit function theorem] \label{thm:imft}
    Let $k \ge 1$, $U \subopeneq \R^m \times \R^n$ and $f \in C^k(U, \R^m)$.
    Suppose $(a, b) \in U$ is such that $f(a, b) = 0$ and
    $\pdv{f}{y}(a, b)$ is invertible.
    Then there exists a neighbourhood $V$ of $(a, b)$ and $W$ of $a$,
    and a function $\phi\colon W \to \R^n$ such that
    \begin{enumerate}
        \item $(x, \phi(x)) \in V$ for all $x \in W$,
        \item $\phi(a) = b$,
        \item $f(x, \phi(x)) = 0$ for all $x \in W$.
        \item If $f(x, y) = 0$ for $x \in W$ and $(x, y) \in V$,
            then $y = \phi(x)$.
        \item $\phi'(a) = -\ab({\pdv fy(a, b)})^{-1} \pdv fx(a, b)$.
    \end{enumerate}
\end{theorem*}
\begin{proof}
    Define $F\colon U \to \R^{n+m}$ by $F(x, y) = (x, f(x, y))$.
    Obviously $F$ is $C^k$ on $U$.
    Then $F'(x, y)(h, k) = (h, f'(x, y)(h, k))$ is zero iff $h = 0$ and
    $f'(x, y)(0, k) = \pdv fy(x, y)(k) = 0$.
    Since $\pdv fy(a, b)$ is invertible, $F'(a, b)(h, k) = 0$ iff $h = 0$
    and $k = 0$.
    Thus $F'(a, b)$ is invertible.

    By the inverse function theorem, there exist open sets
    $V_1 \ni (a, b)$ and $W_1 \ni (a, 0)$ such that $F\colon V_1 \to W_1$
    is a $C^k$-diffeomorphism.
    Let $W \ni a$ be an open set such that $(x, 0) \in W_1$
    for all $x \in W$.
    Then there exists a function $\phi\colon W \to \R^n$ given by
    $\phi(x) = \pi_2(F^{-1}(x, 0))$ such that
    \begin{enumerate}
        \item $(x, \phi(x)) \in V_1$ for all $x \in W$
            (since $\pi_1(F^{-1}(x, 0)) = x$),
        \item $\phi(a) = \pi_2(F^{-1}(a, 0)) = \pi_2(a, b) = b$,
        \item $F(x, \phi(x)) = F(x, \pi_2(F^{-1}(x, 0))) = (x, 0)$, so that
            $f(x, \phi(x)) = 0$ for all $x \in W$.
        \item yada yada \qedhere
    \end{enumerate}
\end{proof}
