\lecture[7]{2024-08-28}{Orthogonal range reporting -- optimal space}

\textbf{Only one midterm, in the middle of September.} \\

In the pointer machine model, memory is organised in a graph.
Chazelle (1990) showed that in a pointer machine model, any algorithm
for range reporting that achieves $O(\log^c n + k)$ time must use
$\Omega\ab(n \frac{\log n}{\log \log n})$ space.

We will achieve this bound in today's lecture.

\begin{question*} \label{que:ors:special}
    Given a set of points in $\R^2$ and an $x_0 \in \R$,
    process queries of the form
    $[x_1, x_2] \times [y_1, y_2]$ where $x_1 < x_0 < x_2$.
\end{question*}
\begin{solution}
    Partition the set of points $P$ into
    $P_L = P \cap ([-\infty, x_0) \times \R)$
    and $P_R = P \cap ([x_0, \infty) \times \R)$.
    Use the 3-sided range reporting algorithm to report the points in
    $[x_1, \infty) \times [y_1, y_2]$ from $P_L$ and
    $(-\infty, x_2] \times [y_1, y_2]$ from $P_R$.

    This takes $O(\log n + k)$ time and consumes $O(n)$ space.
\end{solution}

We can generalize this to work for any query, by using a balanced binary
search tree on the $x$-coordinates of the points.
\begin{solution}
    Let $x_m$ be the median $x$-coordinate in $P$.

    Partition $P$ into $P_L$ and $P_R$ as before.
    Recurse on $P_L$ and $P_R$.
    For each node, construct the data structure described above:
    two priority search trees containing all descendants of the node.
    The space requirement is given by the recursion \[
        S(n) \le O(n) + 2S(n/2)
    \] which has solution $S(n) = O(n \log n)$.
    This can be thought of as $O(n)$ space per level of the tree.
    The construction time is given by the recursion \[
        T(n) \le O(n \log n) + 2T(n/2)
    \] which has solution $T(n) = O(n \log^2 n)$.

    For each query, locate the highest node in the tree that intersects
    the query rectangle.
    This takes $O(\log n)$ time.
    Use the data structure at that node to report the entire rectangle in
    $O(\log n + k)$ time.
\end{solution}

The key idea is to use a search tree with larger fanout to reduce
the height.
A tree with fanout $f$ has height $O(\log_f n)$.
For a fanout of $\log n$, this is $O\ab(\frac{\log n}{\log \log n})$.

Let us look at the special case \cref{que:ors:special} first.
\begin{question*}
    Given a set of $n$ points in $\R^2$ and
    $\ell_1, \dots, \ell_{F-1} \in \R$,
    process queries of the form $[x_1, x_2] \times [y_1, y_2]$
    where $(x_1, x_2)$ contains at least one $\ell_i$.
\end{question*}
\begin{solution}[Solution (naive)]
    Split the points into $F$ slabs $L_1, \dots, L_F$
    by the lines $x = \ell_1, \dots, \ell_{F-1}$.

    For each slab, build a priority search tree.
    Any query $[x_1, x_2] \times [y_1, y_2]$ will overlap with
    $L_i, \dots, L_j$ for some $i < j$.
    Run the 3-sided range reporting algorithm on each of these slabs.
    This takes $O(n)$ space and $O(F \log n + k)$.
    Oops!
\end{solution}
(diagram) The range is three-sided only in the slabs $L_i$ and $L_j$.
The middle slabs are essentially $1$-dimensional queries.
Thus we can do better.

\begin{solution}
    For each slab, build a PST.
    In addition, maintain a fractional cascading structure on the slabs,
    where each point is projected onto the $y$-axis.
    The total space occupied is still $O(n)$.

    Any query $[x_1, x_2] \times [y_1, y_2]$ will overlap with
    $L_i, \dots, L_j$ for some $i < j$.
    The intersection will be 3-sided only with $L_i$ and $L_j$.
    Solve these using the PSTs in $O(\log n + k)$ time.

    Use fractional cascading to obtain the successor of $y_1$ in each
    slab in the middle in $O(\log n + F)$ time.
    Keep going until $y_2$, in $O(k)$ time.
    The total time is $O(\log n + F + k)$.
\end{solution}
If $F = O(\log n)$, we have $O(\log n + k)$ time using $O(n)$ space.

Finally, we can generalize this to any query.
\begin{solution}
    Let $F = \ceil{\log n}$.
    Partition $P$ into equal $P_1, \dots, P_F$ by $x$-coordinate.
    Recurse on each $P_i$,
    storing $F$ PSTs and a fractional cascade of $F$ slabs at each node.

    This consumes $O(n)$ space at each level, for a total of
    $O\ab(\frac{n \log n}{\log \log n})$.

    At query time, we have two cases:
    \begin{enumerate}[label=(Case \arabic*)]
        \item The query rectangle intersects at least one slab boundary.
            Solve the special case in $O(\log n + k)$ time.
        \item The query rectangle is contained within a slab.
            Recurse into the slab.
    \end{enumerate}
    Figuring out which case we are in takes $O(\log F)$ time.
    The tree has height $O\ab(\frac{\log n}{\log \log n})$,
    so getting to case $1$ takes at most
    $O\ab(\log F \cdot \frac{\log n}{\log \log n}) = O(\log n)$ time.
    Once we are in case $1$, reporting requires $O(\log n + k)$ time.
    Thus we still have $O(\log n + k)$ time using the lower bound space.
\end{solution}
