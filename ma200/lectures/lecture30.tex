\lecture{2024-10-23}{}
\section{The method of Lagrange multipliers} \label{sec:lm}
We will first consider the case of two variables.
\begin{proposition*} \label{thm:lm:2}
    Let $f, g \in C^1(\R^2, \R)$.
    Let $(x_0, y_0)$ be an extremum of $f$ subject to the constraint
    $g(x, y) = 0$.
    Assume also that $\nabla g \ne 0$.
    Then there exists $\lambda \in \R$ such that \[
        \nabla(f + \lambda g)(x_0, y_0) = 0.
    \]
\end{proposition*}
\begin{remark}
    This is NOT a sufficient condition.
\end{remark}
The method is then to find $x, y$ that satisfy \[
    \left\{\begin{aligned}f_x + \lambda g_x &= 0, \\
    f_y + \lambda g_y &= 0, \\
    g &= 0.
    \end{aligned}\right.
\] These are $3$ equations in $3$ unknowns.

\begin{example}
    Find the extrema of $f(x, y) = x^2 + y^2 - x - y - xy$ in the domain
    $D = \sset{(x, y)}{x + y \le 3, x \ge 0, y \ge 0}$.
    \begin{description}
        \item[Step 1] Find the critical points in the interior of $D$.
        \begin{align*}
            f_x(x, y) &= 2x - y - 1 \overset{!}{=} 0, \\
            f_y(x, y) &= 2y - x - 1 \overset{!}{=} 0.
        \end{align*}
        This has unique solution $x = y = 1$.
        \item[Step 2] Find the critical points on the boundary of $D$.
        \begin{description}
            \item[($x = 0$)] $f(0, y) = y^2 - y$, with critical point
            $y = \frac12$.
            \item[($y = 0$)] $f(x, 0) = x^2 - x$, with critical point
            $x = \frac12$.
            \item[($x + y = 3$)] Here we will use Lagrange multipliers.
            (I think this is overkill.)
            \begin{align*}
                (2x - y - 1) + \lambda(1) &= 0, \\
                (2y - x - 1) + \lambda(1) &= 0, \\
                x + y - 3 &= 0.
            \end{align*}
            Again by symmetry, $x = y$, so $x = y = \frac 32$ and
            $\lambda = -\frac12$ is the unique solution.
            \item[endpoints] $(0, 0)$, $(0, 3)$, $(3, 0)$.
        \end{description}
        \item[Step 3] Compute all critical values.
        \begin{align*}
            f(1, 1) &= -1 \\
            f(0, \frac12) &= -\frac14 \\
            f(\frac12, 0) &= -\frac14 \\
            f(\frac32, \frac32) &= -\frac 34 \\
            f(0, 0) &= 0 \\
            f(0, 3) &= 6 \\
            f(3, 0) &= 6
        \end{align*}
    \end{description}
    Thus $f$ attains a global minimum at $(1, 1)$, and global maxima at
    $(0, 3)$ and $(3, 0)$.
\end{example}

\begin{example}
    Find the highest and lowest point of the ellipse that lies in the
    intersection of the cylinder $x^2 + y^2 = 1$ and the plane
    $x + y + 2z = 2$.

    We wish to extremize $z$ over the given constraints.
    Let \begin{align*}
        f(x, y, z) &= z, \\
        g(x, y, z) &= \begin{pmatrix}
            x^2 + y^2 - 1 \\
            x + y + 2z - 2
        \end{pmatrix}.
    \end{align*}
    Computing the derivatives, \begin{align*}
        f'(x, y, z) &= \begin{pmatrix}
            0 & 0 & 1
        \end{pmatrix}, \\
        g'(x, y, z) &= \begin{pmatrix}
            2x & 2y & 0 \\
            1 & 1 & 2
        \end{pmatrix}.
    \end{align*}
    Clearly $g'$ has full rank everywhere (except the origin, which we are
    not concerned with).
    We wish to solve \begin{align*}
        f'(x, y, z) &= \begin{pmatrix}
            \lambda_1 & \lambda_2
        \end{pmatrix} g'(x, y, z) \\
        \begin{pmatrix}
            0 & 0 & 1
        \end{pmatrix} &= \begin{pmatrix}
            2x \lambda_1 + \lambda_2 & 2y \lambda_1 + \lambda_2 & 2 \lambda_2
        \end{pmatrix} \\
        \implies \lambda_2 &= \frac12 \\
        \implies x = y &= - \frac1{4\lambda_1}.
    \end{align*}
    Since $(x, y, z)$ must satisfy the constraints, we have \[
        \frac1{16\lambda_1^2} + \frac1{16\lambda_1^2} = 1,
        \text{ so } \lambda_1 = \pm\frac1{2\sqrt 2}.
    \] The corresponding $x$ and $y$ are $\mp \frac1{\sqrt 2}$.
    The highest point is
    $(-\frac1{\sqrt 2}, -\frac1{\sqrt 2}, 1 + \frac1{\sqrt 2})$
    and the lowest is
    $(\frac1{\sqrt 2}, \frac1{\sqrt 2}, 1 - \frac1{\sqrt 2})$.
\end{example}

\begin{example}
    Find the operator norm of $A = \begin{pmatrix}
        1 & 2 \\
        0 & 1
    \end{pmatrix}$.

    We wish to maximize $\norm{Ax}^2$ subject to $\norm x^2 = 1$.
    Let \begin{align*}
        f(x, y) &= \norm*{\begin{pmatrix}
            1 & 2 \\
            0 & 1
        \end{pmatrix} \begin{pmatrix}
            x
            \\ y
        \end{pmatrix}}^2 = \norm*{\begin{pmatrix}
            x + 2y
            \\ y
        \end{pmatrix}}^2 = x^2 + 4xy + 5y^2, \\
        g(x, y) &= x^2 + y^2 - 1.
    \end{align*}
    Following the method, \begin{align*}
        f'(x, y) &= \begin{pmatrix}
            2x + 4y &
            4x + 10y
        \end{pmatrix}, \\
        g'(x, y) &= \begin{pmatrix}
            2x & 2y
        \end{pmatrix} \ne 0 \\
        2x + 4y &\overset{!}{=} 2\lambda x, \\
        4x + 10y &\overset{!}{=} 2\lambda y,
        \shortintertext{Rearranging gives}
        (1 - \lambda)x + 2y &= 0, \\
        2x + (5 - \lambda)y &= 0.
    \end{align*}
    For these to have non-trivial solutions, we must have \[
        (1 - \lambda)(5 - \lambda) - 4 = 0,
    \] which gives $\lambda = 3 \pm 2\sqrt 2$.
    Then the equations simplify to $x = -(1 \pm \sqrt 2)y$.
    Let $u = -(1 \pm \sqrt2)$.
    The corresponding $x$ and $y$ on the unit circle are given by \begin{align*}
        y^2 + (1 \pm \sqrt 2)^2 y^2 &= 1, \\
        \iff (4 \pm 2\sqrt 2)y^2 &= 1, \\
        \iff y &= \pm \frac1{\sqrt{4 \pm 2\sqrt 2}} \\
        \iff x &= \mp \frac{1 \pm \sqrt 2}{\sqrt{4 \pm 2\sqrt 2}}.
    \end{align*}
    Then \begin{align*}
        f(x, y) &= x^2 + 4xy + 5y^2 \\
        &= (u^2 + 4u + 5)y^2 \\
        &= \frac{3 \pm 2\sqrt 2 - 4 \mp 4 \sqrt 2 + 5}{4 \pm 2\sqrt 2} \\
        &= \frac{4 \mp 2\sqrt 2}{4 \pm 2\sqrt 2} \\
        &= \frac{2 \mp \sqrt 2}{2 \pm \sqrt 2} \\
        &= \frac12 (6 \mp 4\sqrt 2) \\
        &= 3 \mp 2\sqrt 2.
    \end{align*}
    Thus the maximum is $3 + 2\sqrt 2$, and so the operator norm is \[
        \sqrt{3 + 2\sqrt 2}
        = \sqrt{1 + 2 \sqrt 2 + \sqrt 2^2}
        = 1 + \sqrt 2.
    \]
\end{example}
