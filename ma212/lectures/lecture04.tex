\lecture{2024-08-09}{}
    \item $\GL_n(F)$ is not finitely generated for any infinite field $F$.
    \begin{itemize}
        \item There is an isomorphic copy of $F^\times$ in $\GL_n(F)$.
            If $\GL_n(F)$ were finitely generated, so would $F^\times$.
        \item $\det\colon \GL_n(F) \to F^\times$ is a surjective
            homomorphism.
            If $\GL_n(F)$ were finitely generated, so would $F^\times$.
    \end{itemize}
    However, $\F^\times$ is not finitely generated since it contains
    $\Q^\times$.
    \item In the non-abelian setting, a subgroup of a finitely generated
        group is not necessarily finitely generated.
        Let \[
            G = \angled*{\begin{pmatrix}
                1 & 1 \\
                0 & 1
            \end{pmatrix}, \begin{pmatrix}
                2 & 0 \\
                0 & 1
            \end{pmatrix}} \le \GL_2(\R).
        \] Let \[
            H = \sset*{g \in G}{g = \begin{pmatrix}
                1 & * \\
                0 & 1
            \end{pmatrix}} \le G.
        \]
        Check that \[
            H = \sset*{\begin{pmatrix}
                1 & n/2^m \\
                0 & 1
            \end{pmatrix}}{n, m \in \Z}.
        \] This is not finitely generated.
        $H$ is isomorphic to the additive group of rationals with power-of-2
        denominators.
        The span of any finite set \[
            S = \set*{\frac{n_1}{2^{m_1}}, \dots, \frac{n_k}{2^{m_k}}},
        \]
        cannot contain any rational with a denominator larger than
        $2^{\max m_i}$.
\end{examples}

\begin{exercise}
    Can any non-empty finite set $S$ be given the structure of a group?
    What if $S$ is countable?
    What if it is any set?
\end{exercise}
\begin{solution}
    In the case $\abs{S} = n$, there is an obvious isomorphism to $\Z/n\Z$.
    If $\abs{S} = \aleph_0$, there is an obvious isomorphism to $\Z$.

    If $S$ is a set of sets, the symmetric difference
    $A \Delta B = (A \setminus B) \cup (B \setminus A)$
    gives a group structure.
    Thus in pure set theory, any set can be given the structure of a group.

    What if the elements of $S$ are not sets? % TODO
\end{solution}

\section{Orders of Elements} \label{sec:order}

\begin{lemma} \label{thm:order:bezout}
    Let $G$ be a group.
    If $x^m = x^n = 1$, then $x^{(m, n)} = 1$.
\end{lemma}
\begin{proof}
    Bezout's identity.
\end{proof}

\begin{corollary} \label{thm:order:divides}
    If $x^\alpha = 1$, then $\ord x \mid \alpha$.
\end{corollary}
\begin{proof}
    $(\ord x, \alpha) \le \ord x$ by elementary number theory.
    But $x^{(\ord x, \alpha)} = 1$ (by the previous lemma)
    gives $(\ord x, \alpha) \ge \ord x$ by minimality of $\ord x$.
    Thus $(\ord x, \alpha) = \ord x$ so $\ord x \mid \alpha$.
\end{proof}
Alternate proof in \cref{thm:group:order}.

\begin{lemma*} \label{thm:order:power}
    Let $G$ be a group.
    \begin{enumerate}
        \item If $\ord x = \infty$, then $\ord x^k = \infty$ for every
            $k \in \Z^\times$.
        \item If $\ord x = n < \infty$, then $\ord x^k = n / (n, k)$.
    \end{enumerate}
\end{lemma*}
\begin{proof}
    It suffices to prove the second statement. \TODO[How?] % TODO: how?
    Let $y = x^k$ and $d = (n, k)$.
    Write $n = \wtld{n} d$ and $k = \wtld{k} d$.
    Suppose $y^m = 1$.
    Then by the previous corollary, $n \mid mk$ and so
    $\wtld{n} \mid m \wtld{k} \implies \wtld{n} \mid m$.

    Thus $m \ge \wtld{n}$.
    But $y^{\wtld{n}} = x^{k \wtld{n}} = x^{n \wtld{k}} = 1$.
    Thus $\ord y = \wtld{n}$.
\end{proof}

\begin{lemma*}
    Let $H = \angled{x}$.
    \begin{enumerate}
        \item If $\ord x = \infty$, then $H$ is generated by $x^a$ iff
            $a = \pm 1$.
        \item If $\ord x = n$, then $H$ is generated by $x^a$ iff
            $(a, n) = 1$.
    \end{enumerate}
\end{lemma*}
\begin{proof}
    For the first case, assume $H = \Z$ by isomorphism.
    $\Z = a\Z \implies \exists n \in \Z \st an = 1$.
    Then $\abs{a} = 1$.
    The converse is by inspection.

    For the second, assume $H = \Z/n\Z$ by isomorphism.
    Let $\wbar{a} \in \Z/n\Z$ be a generator.
    Then $\ord \wbar{a} = n$.
    By the previous lemma, $\ord \wbar{a} = n / (n, a)$
    (since $\ord \wbar{1} = n$).
    Thus $(n, a) = 1$.
\end{proof}
