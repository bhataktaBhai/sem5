\lecture{2024-08-21}{}

We will continue our examples.
\begin{examples}
    \item Let $f\colon \R^2 \to \R$ be given by $f(x, y) = xy$.
    Write $a = (a_1, a_2)$, $h = (h_1, h_2)$.
    Then \begin{align*}
        f(a + h) &= f(a_1 + h_1, a_2 + h_2) \\
            &= (a_1 + h_1)(a_2 + h_2) \\
            &= a_1 a_2 + a_1 h_2 + a_2 h_1 + h_1 h_2 \\
            &= f(a) + \begin{pmatrix}
                a_2 & a_1
            \end{pmatrix} \begin{pmatrix}
                h_1 \\ h_2
            \end{pmatrix} + o(h).
    \end{align*}
    Let us show $h_1 h_2 = o(h)$.
    \[
        \frac{\abs{h_1 h_2}}{\norm h}
            = \abs{h_1} \frac{\abs{h_2}}{\norm h}
            \le \abs{h_1} \to 0.
    \]
    Thus $f'(a)$ is the map $(h_1, h_2) \mapsto a_2 h_1 + a_1 h_2$.
    As a matrix, this is $\begin{pmatrix}
        a_2 & a_1
    \end{pmatrix}$.
    \item Let $f\colon M_n(\R) \to M_n(\R)$ be given by $f(X) = X$.

    We could identify $M_n(\R)$ with $\R^{n^2}$ and construct
    a linear map from $\R^{n^2}$ to $\R^{n^2}$.
    It is however advisable to construct a linear map from $M_n(\R)$ to
    $M_n(\R)$.
    This is again a specical case of the second example.
    Thus $f'(A) = f = \id$.
    \item Let $f\colon M_n(\R) \to M_n(\R)$ be given by $f(X) = X^2$.
    Then \begin{align*}
        f(A + H) &= (A + H)^2 \\
            &= A^2 + AH + HA + H^2 \\
            &= f(A) + AH + HA + o(H)
    \end{align*} since \[
        \frac{\norm{H^2}}{\norm H} \le \frac{\norm H^2}{\norm H}
            = \norm H \to 0.
    \]
    Thus $f'(A)(H) = AH + HA$.
    \item Let $f\colon M_n(\R) \to M_n(\R)$ be given by $f(X) = X^3$.
    Then \begin{align*}
        f(A + H) &= (A + H)^3 \\
            &= A^3 + A^2 H + A H A + H A^2 \\
            &\qquad+ A H^2 + H A H + H^2 A + H^3 \\
            &= A^3 + (A^2 H + A H A + H A^2) + o(H).
    \end{align*}
    \item Let $f\colon \GL_n(\R) \to M_n(\R)$ be given by $f(X) = X^{-1}$.
    Recall that if $\norm Z < 1$, then \[
        (I - Z)^{-1} = I + Z + O(Z^2) = I + Z + o(Z).
    \] Thus for small enough $\norm H$, \begin{align*}
        (A + H)^{-1} &= \ab(A(I + A^{-1}H))^{-1} \\
            &= \ab(I - A^{-1}H + o(-A^{-1}H)) A^{-1} \\
            &= A^{-1} - A^{-1}HA^{-1} + o(H). \yesnumber \label{eq:inv-diff}
    \end{align*}
    Let us do the $o(H)$ term more carefully.
    Let $(I + A^{-1}H)^{-1} = I - A^{-1}H + u(H)$ where \[
        \lim_{H \to 0} \frac{\norm{u(H)}}{\norm {-A^{-1}H}} = 0.
    \] Then
    \begin{align*}
        (A + H)^{-1} &= (I + A^{-1}H)^{-1} A^{-1} \\
            &= (I - A^{-1}H + u(H)) A^{-1} \\
            &= A^{-1} - A^{-1}HA^{-1} + u(H)A^{-1}.
    \end{align*}
    But \begin{align*}
        \frac{\norm{u(H) A^{-1}}}{\norm H}
            &\le \frac{\norm{u(H)}}{\norm H} \norm{A^{-1}} \\
            &\le \frac{\norm{u(H)}}{\norm{-A^{-1}H}}
                \frac{\norm {-A^{-1}H}}{\norm H} \norm{A^{-1}} \\
            &\le \frac{\norm{u(H)}}{\norm{-A^{-1}H}} \norm{A^{-1}}^2 \\
            &\to 0.
    \end{align*}
    Thus from \cref{eq:inv-diff}, $f'(A)(H) = -A^{-1}HA^{-1}$.

    However, we have can simply use \cref{TODO} as follows: \[
        u(H) = o(-A^{-1}H) = o(O(H)) = o(H).
    \]
    \item Let $f\colon M_n(\R) \to M_n(\R)$ be given by $f(X) = X^{-2}$.
    Then \begin{align*}
        f(A + H) &= (A + H)^{-2} \\
            &= (A^{-1} - A^{-1}HA^{-1} + o(H))
                (A^{-1} - A^{-1}HA^{-1} + o(H)) \\
            &= A^{-2} - A^{-2}HA^{-1} - A^{-1}HA^{-2} + o(H).
    \end{align*}
    Thus $f'(A)(H) = -A^{-1} (A^{-1} H + H A^{-1}) A^{-1}$.
\end{examples}
\begin{remarks}
    \item \[
        \frac1{x+h} - \frac1x = \frac1{x+h} (x - (x + h)) \frac1x
        \to -\frac1x h \frac1x
    \]
    \item \[
        \frac1{(x+h)^2} - \frac1{x^2}
            = \frac1{(x+h)^2} (x^2 - (x + h)^2) \frac1{x^2}
    \]
\end{remarks}
