\lecture{2024-08-12}{Continuity of the inverse; differentiation}

\begin{exercise}
    Let $Z$ be as in 
    $(I - Z)^{-1} = I + Z + O(Z^2)$ and also
    $(I - Z)^{-1} = I + Z + o(Z^2)$.
\end{exercise}

The proof of \cref{thm:inv-cont} is nice and sweet.
However, the proof in Rudin generalises better to infinite dimensions.
We thus prove it again.
\begin{proposition*} \label{thm:inv-cont-rudin} \leavevmode
    \begin{enumerate}
        \item Let $A \in M_n(\R)$ be such that $\norm{I - A} < 1$.
        Then $A \in \GL_n(\R)$.
        \item Let $A \in \GL_n(\R)$ be fixed and let
        $B \in M_n(\R)$ be such that \[
            \norm{B - A} < \norm{A^{-1}}^{-1}.
        \] Then $B \in \GL_n(\R)$.
        \item $A \mapsto A^{-1}$ is continuous on $\GL_n(\R)$.
    \end{enumerate}
\end{proposition*}
\begin{remark}
    The second part shows that $\GL_n(\R)$ is open in $M_n(\R)$.
\end{remark}
\begin{proof}
    We proved the first part earlier in \cref{thm:gln-open-i} and again
    in \cref{thm:inv-series}
    (let $Z = I - A$, then $I - Z = A$).

    For the second part, let $A \in \GL_n(\R)$ be fixed and
    let $\norm{B - A} < {\norm{A^{-1}}}^{-1}$.
    We can write $B - A$ as $A(A^{-1}B - I)$.
    Now \begin{align*}
        \norm{A^{-1}B - I} &= \norm{A^{-1} (B - A)} \\
        &\leq \norm{A^{-1}} \norm{B - A} \\
        &< 1.
    \end{align*}
    Then by the first part, $A^{-1}B \in \GL_n(\R)$, so that
    $B \in \GL_n(\R)$.

    For the last part, we want $B^{-1} \to A^{-1}$ as $B \to A$.
    \begin{equation}
        B^{-1} - A^{-1} = A^{-1}(A - B)B^{-1} \label{eq:com-denom}
    \end{equation}
    We need to bound $\norm{B^{-1}}$.
    Let $W$ be an open neighbourhood of $A$ of radius
    $\frac12 \norm{A^{-1}}^{-1}$.
    Then $W \subseteq \GL_n(\R)$.

    For any $B \in W$, $\norm{A - B} \norm{A^{-1}} < \frac12$
    and \begin{align*}
        \norm{B^{-1}} - \norm{A^{-1}}
            &\le \norm{B^{-1} - A^{-1}} \\
            &\le \norm{B^{-1}} \norm{A - B} \norm{A^{-1}}
                \tag{by \cref{eq:com-denom}} \\
            &\le \frac12 \norm{B^{-1}}.
    \end{align*}
    This bounds $\norm{B^{-1}}$ above by $2\norm{A^{-1}}$.
    Using \cref{eq:com-denom} again, we have \begin{align*}
        \norm{B^{-1} - A^{-1}}
            &\le \norm{A^{-1}} \norm{A - B} \norm{B^{-1}} \\
            &\le 2 {\norm{A^{-1}}}^2 \cdot \norm{A - B}.
    \end{align*}
    As $B \to A$, $B^{-1} \to A^{-1}$.
\end{proof}
\begin{idea}
    This is similar in spirit to \cref{thm:reciprocal-cont}.
    \begin{itemize}
        \item \Cref{eq:com-denom} is similar to taking the common
        denominator in $\frac1x - \frac1a$.
        \item The choice of $W$ is similar to choosing
        $\delta \le \frac12\abs{a}$, and leads to an identical bound.
    \end{itemize}
\end{idea}

\chapter{Differentiation} \label{chp:diff}
\begin{definition}
    Let $f\colon \R \to \R$ be a function.
    We say that $f$ is \emph{differentiable} at $a \in \R$ if \[
        \lim_{h \to 0} \frac{f(a + h) - f(a)}{h}
    \] exists.
    We denote this limit by $f'(a)$ and call it the \emph{derivative}
    of $f$ at $a$.
\end{definition}
This doesn't make sense for $f\colon \R^n \to \R^m$ when $n > 2$
(for $n = 2$ we can identify $\R^2$ with $\C$).

\begin{theorem}[Hurwitz' theorem] \label{thm:diff:hurwitz}
    $\R^n$ is a 
\end{theorem}

We will redefine differentiability for real functions.
\begin{proposition*}
    Let $U$ be an open subset of $\R$ and $f\colon U \to \R$.
    Let $a \in U$.
    Then $f$ is differentiable at $a$ if and only if there exists
    a linear map $T \in L(\R, \R)$ such that \[
        f(a + h) - f(a) = Th + o(h).
    \]
\end{proposition*}
\begin{proof}
    Suppose $f$ is differentiable at $a \in U$. \[
        \lim_{h \to 0} \frac{f(a + h) - f(a)}h = f'(a)
    \]
    We can rewrite this as \begin{align*}
        \lim_{h \to 0} \frac{f(a + h) - f(a) - f'(a)h}h &= 0 \\
        \implies \lim_{h \to 0} \frac{\abs{f(a + h) - f(a) - T_{f'(a)}h}}{\abs h} &= 0
    \end{align*} where $T_\alpha \in L(\R, \R)$ is the linear map
    $x \mapsto \alpha x$.

    Conversely, suppose there exists a linear map $T$ such that
    $f(a + h) - f(a) - Th = o(h)$.
    Then \begin{align*}
        \lim_{h \to 0} \frac{\abs{f(a + h) - f(a) - Th}}{\abs h} &= 0 \\
        \implies \lim_{h \to 0}\;\abs*{\frac{f(a+h) - f(a)}h - T(1)} &= 0 \\
        \implies \lim_{h \to 0} \frac{f(a + h) - f(a)}h &= T(1). \qedhere
    \end{align*}
\end{proof}
