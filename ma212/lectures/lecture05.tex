\lecture{2024-08-12}{}

\section{Presentations and $D_{2n}$} \label{sec:groups:pres}

\begin{lemma}
    Let $G$ be a group and let $a, b \in G$ commute.
    Let $\ord a = m$, $\ord b = n$, $\lcm(m, n) = \ell$.
    Then $\ord(ab) \mid \ell$.
    If $(m, n) = 1$, then $\ord(ab) = \ell$.
\end{lemma}
\begin{proof}
    $(ab)^\ell = a^\ell b^\ell = 1$.

    Now suppose that $(m, n) = 1$.
    Let $d = \ord(ab) \implies d \mid \ell$.
    Now \begin{align*}
        (ab)^d = 1 &\implies a^d b^d = 1 \\
        &\implies a^d = b^{-d}.
    \end{align*}
    Raising to the power $m$ gives $a^{dm} = 1 = b^{-dm}$.
    Thus $n \mid m d \implies n \mid d$ (coprime).
    Similarly $m \mid d$.
    Thus $nm = \ell \mid d$.
    Together with $d \mid \ell$, we get $d = \ell$.
\end{proof}

\begin{examples}
    \item If $(a, b) \ne 1$, we can't say anything.
        For example, $b = a^{-1}$ gives $\ord(ab) = 1$.
    \item If $ab \ne ba$, things can go crazy.
        For example, $a = \begin{pmatrix}
            0 & 1 \\
            1 & 0
        \end{pmatrix}$, $b = \begin{pmatrix}
            0 & 1 / 2 \\
            2 & 0
        \end{pmatrix}$.
        Then $a^2 = b^2 = 1$ but $ab = \begin{pmatrix}
            1 / 2 & 0 \\
            0 & 2
        \end{pmatrix}$ has infinite order.
\end{examples}

\begin{definition}[presentation] \label{def:groups:pres}
    
\end{definition}

\begin{definition}[the dihedral group] \label{def:groups:dihedral}
    For $n \ge 3$, the dihedral group $D_{2n}$ is the group of rigid
    motions of a regular $n$-gon $R_n$ in $\R^2$.
\end{definition}
\begin{remark}
    A ``rigid motion'' is an isoemtry: a distance preserving bijection.
    For example, reflections and rotations.
    Note how rigid motions being a bijection (when restricted to the
    $n$-gon) implies that only those isometries that preserve the
    $n$-gon are allowed.
\end{remark}
\begin{center}
    \begin{tikzpicture}
        \node[draw,minimum size=2cm,regular polygon,regular polygon sides=6] (a) {};
    \end{tikzpicture}
\end{center}

Rigid motions in $\R^n$ are given by $x \mapsto Ax + b$ where $A \in O_n$,
the set of orthogonal matrices in $M_n$.
\[
    (A_1, b_1) \circ (A_2, b_2) = (A_1 A_2, A_1 b_2 + b_1).
\] $A_1 A_2 \in O_n$ so the product is closed.
Associativity is inherited from function composition.
The identity is $(1, 0)$ and the inverse of $(A, b)$ is $(A^\T, -A^\T b)$.

\begin{lemma}
    Every point $P$ on $R_n$ is determined, among all other points on $R_n$,
    by its distance from \emph{any} two fixed adjacent vertices of $R_n$.

    That is, let $A$ and $B$ be adjacent vertices of $R_n$.
    Then for any $d_A, d_B \in \R^+$, there is at most one point $P$
    on $R_n$ such that $d(P, A) = d_A$ and $d(P, B) = d_B$.
\end{lemma}
\begin{proof}
    Look at the edge $\wbar{AB}$.
    \begin{center}
        \begin{tikzpicture}
            \node[draw,minimum size=3cm,regular polygon,regular polygon sides=5] (a) {};
            \draw[fill] (a.corner 5) circle (1pt) node[above right] {$A$};
            \draw[fill] (a.corner 1) circle (1pt) node[above] {$B$};
        \end{tikzpicture}
    \end{center}
    Draw a circle of radius $d_A$ around $A$ and a circle of radius $d_B$
    around $B$.
    They intersect in at most two points, but they are on opposite sides
    of $\wbar{AB}$.
    $R_n$ is convex, so every point on $R_n$ lies on one of only one
    side of $\wbar{AB}$.
    Thus only one of these two points can lie on $R_n$.
    \begin{center}
        \begin{tikzpicture}
            \node[draw,minimum size=3cm,regular polygon,regular polygon sides=5] (a) {};
            \draw[fill] (a.corner 5) circle (1pt) node[above right] {$A$};
            \draw[fill] (a.corner 1) circle (1pt) node[above] {$B$};
            \draw (a.corner 5) circle (1cm);

            % \draw[fill] (a.corner 5) ++(:1) circle (1pt) node[above right] {$P$};
            \coordinate (P) at ($(a.corner 5)!1cm!(a.corner 4)$);
            \coordinate (Q) at ($(a.corner 5) + (36:1cm)$);
            \node[draw] at (a.corner 1) [circle through={(P)}] {};
            \draw[fill] (P) circle (1pt) node[below right] {$P$};
            \draw[fill] (Q) circle (1pt) node[right] {$Q$};
        \end{tikzpicture}
    \end{center}
\end{proof}

\begin{proposition}
    $\abs{D_{2n}} = 2n$.
\end{proposition}
\begin{proof}
    We first show that $\abs{D_{2n}} \le 2n$.
    Start with any two vertices $A$ and $B$ of $R_n$.
    Let $g \in D_{2n}$.

    \begin{claim}
        $g$ takes vertices to vertices.
    \end{claim}
    To see this, note that the vertices are special in that they are
    distinguised from all other points on $R_n$ as follows:
    \begin{quotation}
        Let $P \in \R_n$ and $r > 0$ be small.
        We can find two points $P'_r$ and $P''_r$ on $R_n$ such that
        $d(P, P'_r) = d(P, P''_r) = r$.
        If $P$ is \emph{not} a vertex, then $d(P'_r, P''_r) = 2r$.
        If $P$ \emph{is} a vertex, then $d(P'_r, P''_r) < 2r$.
    \end{quotation}
    Thus we can distinguish between $P$ being a vertex or not solely by the
    distance function.
    Since $g$ is an isometry (even Lipschitz), this property is preserved.
    % even some particular $r$ will do
    Thus $g$ takes vertices to vertices.

    \begin{claim}
        $g$ preserves adjacency of vertices.
    \end{claim}
    Fix a vertex $A$ on $R_n$.
    Then $d(P, A)$ for a vertex distinct from $A$ is minimized when $P$
    is adjacent to $A$.
    Since $g$ preserves distances, $g$ must take adjacent vertices to
    adjacent vertices.

    Combining these two claims, we have proven that for any $P \in R_n$,
    $g(P)$ is uniquely determined by its distance from $g(A)$ and $g(B)$,
    where $A$ and $B$ are any two adjacent vertices.
    Thus $g$ is determined by $g(A)$ and $g(B)$.

    By the first claim, there are $n$ possible choices for $g(A)$.
    By the second claim, there are $2$ possible choices for $g(B)$.
    Thus there are at most $2n$ possible $g$'s.

    Finally, we can produce $2n$ distinct elements as follows.
    \begin{itemize}
        \item Consider the $n$ rotations: rotate by $2\pi k / n$ for
            $k \in n$.
        \item The $n$ reflections:
        \begin{itemize}
            \item For odd $n$, reflect over the line through a vertex and
                the midpoint of the opposite edge.
            \item For even $n$, reflect over the line through two opposite
                vertices or through two opposite midpoints.
        \end{itemize}
    \end{itemize}
    Each reflection fixes exactly two points.
    Any non-trivial rotation fixes no points.
    Thus the $2n$ elements are distinct.
\end{proof}

\begin{notation}
    Let $r$ denote the counter-clockwise rotation by $2\pi / n$ and
    let $s$ denote the reflection over the line through some fixed vertex
    $V_0$.

    Then $r^n = s^2 = 1$.

    Observe that $\set{1, r, r^2, \dots, r^{n-1}}$ gives all the rotations
    in $D_{2n}$.
\end{notation}

\begin{lemma} \label{thm:d2n:reflections}
    All reflections in $D_{2n}$ are given by
    $\set{s, rs, r^2s, \dots, r^{n-1}s}$.
\end{lemma}
\begin{proof}
    All of these elements are distinct, since $r^k \ne 1$ for $0 < k < n$.
    None of these elements are rotations, since if $r^k s = r^m$ for some
    $k, m \in n$, then $s = r^{m-k}$, which is a contradiction.
\end{proof}

\begin{theorem}
    $\abs{D_{2n}} = 2n$ and
    $D_{2n} = \set{1, r, \dots, r^{n-1}, s, rs, \dots, r^{n-1}s}$.
\end{theorem}

\begin{proposition}
    In $D_{2n}$, $sr = r^{-1}s$.
\end{proposition}
\begin{proof}
    From \cref{thm:d2n:reflections}, we know that $rs$ is a reflection.
    Thus $(rs)(rs) = 1$, which immediately gives $sr = r^{-1}s$.
\end{proof}

\textbf{Next lecture:}
$D_{2n} = \angled{r, s \mid r^n = s^2 = 1, sr = r^{-1}s}$.
