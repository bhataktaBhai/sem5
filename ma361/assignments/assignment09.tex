\documentclass[12pt]{article}
% \usepackage{parskip}
\usepackage{lmodern} % https://tex.stackexchange.com/a/58088/295755
\renewcommand\bfdefault{b}
\usepackage{microtype}

\usepackage{amsmath}
\newcommand\yesnumber{\stepcounter{equation}\tag{\theequation}}

\ifundef{\chapter}{}{\providecommand\thechapter{\Roman{chapter}}}

\usepackage{marginnote}
\usepackage[en-GB,calc]{datetime2}
\usepackage{calc}
\usepackage{xifthen}
\usepackage{tocloft}

% https://tex.stackexchange.com/a/454168
\newcommand\monthday[1]{\DTMmonthname{\DTMfetchmonth{#1}}~\number\DTMfetchday{#1}}

\newlistof{lecture}{lec}{Lectures} % what is this file extension business?
\makeatletter
\setlength\marginparwidth{1in}
\newcommand*{\lecture}[3][]{
    \ifthenelse{\isempty{#1}}{%
        \refstepcounter{lecture}%
    }{%
        \setcounter{lecture}{#1}%
    }%
    \DTMsavedate{lecdate}{#2}%
    \def\lecdow{\DTMweekdayname{\DTMfetchdow{lecdate}}}
    \def\lecshortdow{\DTMshortweekdayname{\DTMfetchdow{lecdate}}}
    \def\lecmonth{\DTMmonthname{\DTMfetchmonth{lecdate}}}
    \def\lecday{\number\DTMfetchday{lecdate}}
    % \marginpar{\raggedright\small \textsf{\textbf{Lecture \thelecture.}%
    %             \footnotesize\DTMusedate{lecdate}}}%
    \marginnote{\raggedright\small%
        \textsf{{\textbf{Lecture \thelecture.}} \\
        \footnotesize\lecdow\\\lecmonth\ \lecday}}%
    \ifthenelse{\isempty{#3}}{%
        \addcontentsline{lec}{lecture}{\protect\numberline{\thelecture}%
        \lecshortdow, \lecmonth\ \lecday}%
        \def\@lecture{Lecture \thelecture}%
    }{%
        \addcontentsline{lec}{lecture}{\protect\numberline{\thelecture}%
        \makebox[\widthof{Mon,}][l]{\lecshortdow,}\ \makebox[\widthof{September 00}][l]{\lecmonth\ \lecday} #3}%
        \def\@lecture{Lecture \thelecture: #3}%
    }%
    \par%
}
\g@addto@macro\normalsize{%
  \setlength\abovedisplayskip{7pt}%
  \setlength\belowdisplayskip{7pt}%
  \setlength\abovedisplayshortskip{1pt}%
  \setlength\belowdisplayshortskip{1pt}%
}
\makeatother

\usepackage[twoside]{fancyhdr}
\setlength{\headheight}{15pt}
\pagestyle{fancy}
\fancyhf{}
% \fancyhead[r]{\thepage}
\makeatletter
\fancyhead[LE,RO]{\thepage}
\fancyhead[RE]{\textbf{\nouppercase\leftmark}}
\fancyhead[LO]{\nouppercase\rightmark}
\providecommand\@lecture{}
\fancyfoot[R]{\small\@lecture}
\makeatother

% homeworks
\newlistof{hw}{hw}{Assignments} % counter `assignment' already defined
\makeatletter
\newcommand*{\assignment}[5][]{%[number]{file}{date posted}{date due}{date quiz}
    \ifthenelse{\isempty{#1}}{%
        \refstepcounter{hw}%
        \stepcounter{assignment}%
    }{%
        \setcounter{hw}{#1}%
        \setcounter{assignment}{#1}%
    }%
    \pagebreak
    \ifthenelse{\isempty{#3}}{}{\DTMsavedate{posted}{#3}}%
    \ifthenelse{\isempty{#4}}{}{\DTMsavedate{due}{#4}}%
    \ifthenelse{\isempty{#5}}{}{\DTMsavedate{quiz}{#5}}%
    \section*{Assignment \thehw}
    \ifthenelse{\isempty{#4}}{
        \ifthenelse{\isempty{#5}}{
            \ifthenelse{\isempty{#3}}{
                \def\hw@toc{}
            }{
                \def\hw@toc{posted \monthday{up}}
            }
        }{
            \def\hw@toc{quiz \monthday{quiz}}
        }
    }{
        \def\hw@toc{due \monthday{due}}
    }
    \addcontentsline{hw}{hw}{\protect\numberline{\thehw}\hw@toc}\par
    \marginpar{\raggedright\footnotesize\textsf{%
        \ifthenelse{\isempty{#3}}{}{\makebox[\widthof{quiz}][l]{up} \monthday{posted} \\}%
        \ifthenelse{\isempty{#4}}{}{\makebox[\widthof{quiz}][l]{due} \monthday{due} \\}%
        \ifthenelse{\isempty{#5}}{}{quiz \monthday{quiz}}%
    }}
    \def\@lecture{Assignment \thehw\ifx\hw@toc\empty{}\else\ --- \hw@toc\fi}
    \input{#2}
    \newpage
}
\makeatother

\usepackage{amsmath}
\usepackage{amsthm}
\usepackage[dvipsnames]{xcolor}
\colorlet{exercise}{cyan!70!black}
\colorlet{solved}{green!40!black}
\colorlet{self_proof}{blue!70!black}
\colorlet{Red}{red!80!black}

% Gilles Castel's theorems
\newtheoremstyle{mddefinition}% <name>
  {-.25\topsep}%                 <space above>
  {-.25\topsep}%                         <space below>
  {\normalfont}%              <body font>
  {}%                         <indent amount>
  {\bfseries}%                <theorem head font>
  {.}%                        <punctuation after theorem head>
  {.5em}%                     <space after theorem head>
  {}%                         <theorem head spec>
\newtheoremstyle{mdplain}% <name>
  {-.25\topsep}%                 <space above>
  {-.25\topsep}%                         <space below>
  {\itshape}%                 <body font>
  {}%                         <indent amount>
  {\bfseries}%                <theorem head font>
  {.}%                        <punctuation after theorem head>
  {.5em}%                     <space after theorem head>
  {}%                         <theorem head spec>

\usepackage[framemethod=Tikz]{mdframed}
\mdfsetup{skipbelow=0pt}
\mdfdefinestyle{axiomstyle}{
    outerlinewidth = 1.5,
    roundcorner = 10,
    leftmargin = 15,
    rightmargin = 15,
    backgroundcolor = yellow!7
}
\mdfdefinestyle{defstyle}{
    outerlinewidth = 1,
    roundcorner = 2,
    leftmargin = 7,
    rightmargin = 7,
    backgroundcolor = green!5
}
\mdfdefinestyle{thmstyle}{
    outerlinewidth = 1,
    roundcorner = 8,
    leftmargin = 7,
    rightmargin = 7,
    backgroundcolor = cyan!5
}
\mdfdefinestyle{lemmastyle}{
    outerlinewidth = 1.5,
    roundcorner = 10,
    leftmargin = 7,
    rightmargin = 7,
    backgroundcolor = yellow!10
}
\ifundef\chapter{%
    \providecommand\theoremnumberwithin{section}
    \theoremstyle{mddefinition}
    \newmdtheoremenv[nobreak=true, style=axiomstyle]{axiom}{Axiom}[section]
    \theoremstyle{plain}
    \newtheorem{theorem}{Theorem}[\theoremnumberwithin]

    \newcounter{assignment}
    \theoremstyle{plain}
    \newtheorem{problem}{Problem}
    \theoremstyle{mddefinition}
    \newmdtheoremenv[nobreak=true, outerlinewidth=0.7]{problem*}[problem]{Problem}
}{
    \providecommand\theoremnumberwithin{chapter}
    \theoremstyle{mddefinition}
    \newmdtheoremenv[nobreak=true, style=axiomstyle]{axiom}{Axiom}[chapter]
    \theoremstyle{plain}
    \newtheorem{theorem}{Theorem}[\theoremnumberwithin]

    \newcounter{assignment}
    \theoremstyle{plain}
    \newtheorem{problem}{Problem}[assignment]
    \theoremstyle{mddefinition}
    \newmdtheoremenv[nobreak=true, outerlinewidth=0.7]{problem*}[problem]{Problem}
}
\theoremstyle{mddefinition}
\newmdtheoremenv[nobreak=true, style=defstyle]{definition*}[theorem]{Definition}

\theoremstyle{mdplain}
\newmdtheoremenv[nobreak=true, style=thmstyle]{theorem*}[theorem]{Theorem}
\newmdtheoremenv[nobreak=true]{proposition*}[theorem]{Proposition}
\newmdtheoremenv[nobreak=true]{lemma*}[theorem]{Lemma}
\newmdtheoremenv[nobreak=true]{corollary*}[theorem]{Corollary}
\newmdtheoremenv[nobreak=true, style=thmstyle]{fact*}[theorem]{Fact}
\newmdtheoremenv[nobreak=true]{exercise*}[theorem]{Exercise}
\newmdtheoremenv[nobreak=true]{question*}[theorem]{Question}

\theoremstyle{definition}
\newtheorem{definition}[theorem]{Definition}

\theoremstyle{plain}
\newtheorem{proposition}[theorem]{Proposition}
\newtheorem{lemma}[theorem]{Lemma}
\newtheorem{corollary}[theorem]{Corollary}
\newtheorem{fact}[theorem]{Fact}
\newtheorem{exercise}[theorem]{Exercise}
\newtheorem{question}[theorem]{Question}

\theoremstyle{remark}
\newtheorem*{remark}{Remark}
\newtheorem*{remarkx}{Remarks}
\newtheorem*{example}{Example}
\newtheorem*{examplex}{Examples}
\newtheorem*{idea}{Idea}
%%%% HAXXXXXX %%%%
% \def\innerqed{\qedsymbol}
% \def\outerqed{$\blacksquare$}
\let\oldproof\proof
\let\endoldproof\endproof
\newenvironment{solution}[1][]
  {\renewcommand\qedsymbol{$\blacksquare$}%
  \begin{oldproof}[Solution\ifx&#1&\else\ (#1)\fi]}
  {\end{oldproof}}
\newenvironment{answer}
  {\renewcommand\qedsymbol{$\blacksquare$}\begin{oldproof}[Answer]}
  {\end{oldproof}}
\renewenvironment{proof}
  {\renewcommand\qedsymbol{$\blacksquare$}\begin{oldproof}}
  {\end{oldproof}}
\newenvironment{subproof}[1][Subproof]{%
  \renewcommand{\qedsymbol}{$\square$}\begin{oldproof}[#1]}
  {\end{oldproof}}
\newtheorem*{notation}{Notation}
\newtheorem*{claim}{Claim}

\usepackage{hyperref}
\usepackage[noabbrev]{cleveref}

% <cref>
\crefname{theorem}{theorem}{theorems}
\crefname{proposition}{proposition}{propositions}
\crefname{lemma}{lemma}{lemmas}
\crefname{corollary}{corollary}{corollaries}
\crefname{axiom}{axiom}{axioms}
\crefname{definition}{definition}{definitions}
\crefname{problem}{problem}{problems}
\crefname{exercise}{exercise}{exercises}
\crefname{fact}{fact}{facts}
\crefname{question}{question}{questions}
\crefname{remark}{remark}{remarks}
\crefname{example}{example}{example}
\crefname{notation}{notation}{notations}
\crefname{claim}{claim}{claims}
% \crefname{section}{\S}{\S\S}
\crefname{theorem*}{theorem}{theorems}
\crefname{proposition*}{proposition}{propositions}
\crefname{lemma*}{lemma}{lemmas}
\crefname{corollary*}{corollary}{corollaries}
\crefname{definition*}{definition}{definitions}
\crefname{problem*}{problem}{problems}
\crefname{exercise*}{exercise}{exercises}
\crefname{fact*}{fact}{facts}
\crefname{question*}{question}{questions}
% </cref>

% <hyperlinks>
\hypersetup{colorlinks,
    linkcolor={blue},
    citecolor={blue!50!black},
    urlcolor={blue!80!black}}
% </hyperlinks>

\usepackage[shortlabels]{enumitem}
% change default label for enumerate, and fix long labels popping out
\setenumerate{label*=(\roman*),ref=(\roman*),leftmargin=*}
% casework list using https://tex.stackexchange.com/a/30035
\newcounter{casecount}
\newlist{casework}{description}{1}
\setlist[casework]{%
  before={\setcounter{casecount}{0}%
      \renewcommand*\thecasecount{\arabic{casecount}}}%
  ,font=\bfseries Case \stepcounter{casecount}\thecasecount:
}

\newenvironment{examples}[1][]
{\begin{examplex}[#1]\leavevmode\begin{itemize}}{\end{itemize}\end{examplex}}
\newenvironment{remarks}[1][]
{\begin{remarkx}[#1]\leavevmode\begin{itemize}}{\end{itemize}\end{remarkx}}

% omg this is so HaXy
% \renewenvironment{proof}[1][\proofname]{{\it\bfseries #1. }}{\qed}
% \providecommand{\qedsymbol}{\openbox}
% \makeatletter
% \renewenvironment{proof}[1][\proofname]{\par
%   \pushQED{\qed}%
%   \normalfont \topsep6\p@\@plus6\p@\relax
%   \trivlist
%   \item[\hskip\labelsep
%         \itshape\bfseries%this is the change (boldface instead of italics)
%         % \fontseries{bx}\selectfont
%     #1\@addpunct{.}]\ignorespaces
% }{%
%   \popQED\endtrivlist\@endpefalse
% }
\makeatother
\crefname{enumi}{part}{parts}
\crefname{enumii}{part}{parts}
\crefname{enumiii}{part}{parts}
% \setlist[itemize]{itemsep=2pt}
\newcounter{dummy}
\makeatletter
\newcommand\myitem[1][]{\item[#1]\refstepcounter{dummy}\def\@currentlabel{#1}}
\makeatother

\makeatletter
\newcommand*{\refifdef}[3]{%label,command,fallback
    \@ifundefined{r@#1}{#3}{#2{#1}}%
}
\makeatother

\newcommand\ie{\textit{i.e.}}
\newcommand\eg{\textit{e.g.}}
\usepackage{physics}

\usepackage{amsmath}
\usepackage{amssymb}
\usepackage{mathrsfs} % for \mathscr
\usepackage{bm} % for \bm
\usepackage{booktabs}

% undefine \abs and \norm
\let\abs\relax
\let\norm\relax

\usepackage{mathtools} % for delimiters and \coloneqq
\DeclarePairedDelimiter{\paren}{(}{)}
\DeclarePairedDelimiter{\brk}{[}{]}
\DeclarePairedDelimiter{\set}{\{}{\}}
\DeclarePairedDelimiter{\abs}{\lvert}{\rvert}
\DeclarePairedDelimiter{\norm}{\lVert}{\rVert}
\DeclarePairedDelimiter{\floor}{\lfloor}{\rfloor}
\DeclarePairedDelimiter{\ceil}{\lceil}{\rceil}
\DeclarePairedDelimiter{\angled}{\langle}{\rangle}
% \DeclarePairedDelimiterX{\innerp}[2]{\langle}{\rangle}{#1,\,#2}
% \DeclarePairedDelimiterX{\outerp}[2]{\langle}{\rangle}{#1\otimes#2}
% \DeclarePairedDelimiterX{\braket}[3]{\langle}{\rangle}%
% {#1\,\delimsize\vert\,\mathopen{}#2\,\delimsize\vert\,\mathopen{}#3}
\DeclarePairedDelimiterX{\innerp}[2]{\langle}{\rangle}{#1,\,#2}
% \DeclarePairedDelimiterX{\outerp}[2]{\langle}{\rangle}{#1\otimes#2}
\DeclarePairedDelimiterX{\outerp}[2]{\vert}{\vert}%
{#1\delimsize\rangle\delimsize\langle\mathopen{}#2}
\let\braket\relax
\DeclarePairedDelimiterX{\braket}[3]{\langle}{\rangle}%
{#1\,\delimsize\vert\,\mathopen{}#2\,\delimsize\vert\,\mathopen{}#3}

\renewcommand\O{\ensuremath{\varnothing}}
\newcommand\N{\ensuremath{\mathbb{N}}}
\newcommand\Z{\ensuremath{\mathbb{Z}}}
\newcommand\Q{\ensuremath{\mathbb{Q}}}
\newcommand\R{\ensuremath{\mathbb{R}}}
\newcommand\C{\ensuremath{\mathbb{C}}}
% \renewcommand\P{\ensuremath{\mathbb{P}}}

% fix spacing for \forall and \exists
% \let\oldforall\forall
% \renewcommand{\forall}{\oldforall \, }
% \let\oldexist\exists
% \renewcommand{\exists}{\oldexist \: }
\newcommand\unique{\exists!}
\newcommand\lxor{\oplus}

\providecommand{\dd}{\,\mathrm{d}}

\newcommand\mcA{\ensuremath{\mathcal{A}}}
\newcommand\mcB{\ensuremath{\mathcal{B}}}
\newcommand\mcC{\ensuremath{\mathcal{C}}}
\newcommand\mcD{\ensuremath{\mathcal{D}}}
\newcommand\mcE{\ensuremath{\mathcal{E}}}
\newcommand\mcF{\ensuremath{\mathcal{F}}}
\newcommand\mcG{\ensuremath{\mathcal{G}}}
\newcommand\mcH{\ensuremath{\mathcal{H}}}
\newcommand\mcI{\ensuremath{\mathcal{I}}}
\newcommand\mcJ{\ensuremath{\mathcal{J}}}
\newcommand\mcK{\ensuremath{\mathcal{K}}}
\newcommand\mcL{\ensuremath{\mathcal{L}}}
\newcommand\mcM{\ensuremath{\mathcal{M}}}
\newcommand\mcN{\ensuremath{\mathcal{N}}}
\newcommand\mcO{\ensuremath{\mathcal{O}}}
\newcommand\mcP{\ensuremath{\mathcal{P}}}
\newcommand\mcQ{\ensuremath{\mathcal{Q}}}
\newcommand\mcR{\ensuremath{\mathcal{R}}}
\newcommand\mcS{\ensuremath{\mathcal{S}}}
\newcommand\mcT{\ensuremath{\mathcal{T}}}
\newcommand\mcU{\ensuremath{\mathcal{U}}}
\newcommand\mcV{\ensuremath{\mathcal{V}}}
\newcommand\mcW{\ensuremath{\mathcal{W}}}
\newcommand\mcX{\ensuremath{\mathcal{X}}}
\newcommand\mcY{\ensuremath{\mathcal{Y}}}
\newcommand\mcZ{\ensuremath{\mathcal{Z}}}

%%%% WIDE BAR THAT IS JUST THE RIGHT LENGTH %%%%
%% FROM https://tex.stackexchange.com/a/60253 %%
\makeatletter
\let\save@mathaccent\mathaccent
\newcommand*\if@single[3]{%
  \setbox0\hbox{${\mathaccent"0362{#1}}^H$}%
  \setbox2\hbox{${\mathaccent"0362{\kern0pt#1}}^H$}%
  \ifdim\ht0=\ht2 #3\else #2\fi
  }
%The bar will be moved to the right by a half of \macc@kerna, which is computed by amsmath:
\newcommand*\rel@kern[1]{\kern#1\dimexpr\macc@kerna}
%If there's a superscript following the bar, then no negative kern may follow the bar;
%an additional {} makes sure that the superscript is high enough in this case:
\newcommand*\widebar[1]{\@ifnextchar^{{\wide@bar{#1}{0}}}{\wide@bar{#1}{1}}}
%Use a separate algorithm for single symbols:
\newcommand*\wide@bar[2]{\if@single{#1}{\wide@bar@{#1}{#2}{1}}{\wide@bar@{#1}{#2}{2}}}
\newcommand*\wide@bar@[3]{%
  \begingroup
  \def\mathaccent##1##2{%
%Enable nesting of accents:
    \let\mathaccent\save@mathaccent
%If there's more than a single symbol, use the first character instead (see below):
    \if#32 \let\macc@nucleus\first@char \fi
%Determine the italic correction:
    \setbox\z@\hbox{$\macc@style{\macc@nucleus}_{}$}%
    \setbox\tw@\hbox{$\macc@style{\macc@nucleus}{}_{}$}%
    \dimen@\wd\tw@
    \advance\dimen@-\wd\z@
%Now \dimen@ is the italic correction of the symbol.
    \divide\dimen@ 3
    \@tempdima\wd\tw@
    \advance\@tempdima-\scriptspace
%Now \@tempdima is the width of the symbol.
    \divide\@tempdima 10
    \advance\dimen@-\@tempdima
%Now \dimen@ = (italic correction / 3) - (Breite / 10)
    \ifdim\dimen@>\z@ \dimen@0pt\fi
%The bar will be shortened in the case \dimen@<0 !
    \rel@kern{0.6}\kern-\dimen@
    \if#31
      \overline{\rel@kern{-0.6}\kern\dimen@\macc@nucleus\rel@kern{0.4}\kern\dimen@}%
      \advance\dimen@0.4\dimexpr\macc@kerna
%Place the combined final kern (-\dimen@) if it is >0 or if a superscript follows:
      \let\final@kern#2%
      \ifdim\dimen@<\z@ \let\final@kern1\fi
      \if\final@kern1 \kern-\dimen@\fi
    \else
      \overline{\rel@kern{-0.6}\kern\dimen@#1}%
    \fi
  }%
  \macc@depth\@ne
  \let\math@bgroup\@empty \let\math@egroup\macc@set@skewchar
  \mathsurround\z@ \frozen@everymath{\mathgroup\macc@group\relax}%
  \macc@set@skewchar\relax
  \let\mathaccentV\macc@nested@a
%The following initialises \macc@kerna and calls \mathaccent:
  \if#31
    \macc@nested@a\relax111{#1}%
  \else
%If the argument consists of more than one symbol, and if the first token is
%a letter, use that letter for the computations:
    \def\gobble@till@marker##1\endmarker{}%
    \futurelet\first@char\gobble@till@marker#1\endmarker
    \ifcat\noexpand\first@char A\else
      \def\first@char{}%
    \fi
    \macc@nested@a\relax111{\first@char}%
  \fi
  \endgroup
}
\makeatother
\let\what\widehat
\let\wtld\widetilde
\let\wbar\widebar
\let\ubar\underline

\DeclareMathOperator\sgn{sgn}

\let\oldleft\left
\let\oldright\right
\renewcommand{\left}{\mathopen{}\mathclose\bgroup\oldleft}
\renewcommand{\right}{\aftergroup\egroup\oldright}


\newcommand\highlight[1]{\textcolor{blue}{#1}}

\title{Homework 9}
\author{Naman Mishra (22223)}
\date{15 October, 2024}

\begin{document}
\maketitle

% Problem 1
\begin{problem*}
    Determine whether the following statements are true or false with proper
    justification.
    \begin{enumerate}
        \item Let $X_1, X_2, \dots$ be a sequence of \iid random variables
        taking values in $\R$ and defined on the same probability space.
        Then $\frac{X_n}{n} \pto 0$.
        \item Let $X_1, X_2, \dots$ be a sequence of \iid random
        variables, taking values in $\R$ and are defined on same probability
        space.
        Then $\frac{X_n}{n} \to 0$ almost surely.
        \item Let $X$ be a random variable which is finite almost surely.
        Then $\frac Xn \pto 0$.
        \item Let $X$ be a random variable which is finite almost surely.
        Then $\frac Xn \to 0$ almost surely.
    \end{enumerate}
\end{problem*}
\begin{solution} \leavevmode
    \begin{enumerate}
        \item \textbf{True.}
        Fix a $\delta > 0$.
        Then $\P\set{\abs{X_n/n} > \delta} = \P\set{\abs{X_1} > n\delta}
        \to 0$ as $n \to \infty$.
        \item \textbf{False.} Not necessarily.
        Let $X_1 = k$ with probability $B \frac1{k^2}$ for $k \ge 1$,
        where $B = \ab(\sum_{k=1}^\infty \frac1{k^2})^{-1}$ is the
        normalizing constant.
        Then \begin{align*}
            \P\set*{\frac{X_n}{n} \ge 1} &= \P\set{X_1 \ge n} \\
                &= B \sum_{k \ge n} \frac1{k^2} \\
                &\ge B \frac1n
        \end{align*} using that
        $\sum_{k=N}^\infty \frac1{k^2} \ge \int_N^\infty \frac1{x^2} \dd x$.
        Thus $\set{X_n/n \ge 1}$ are independent events with
        probabilities summing to infinity.
        By the second Borel-Cantelli lemma,
        $\P\set{X_n/n > 1 \text{ i.o.}} = 1$, so
        $\P\set{X_n/n \to 0} = 0$.
        \item \textbf{True.}
        \item \textbf{True.}
        If $X < \infty$, then $X/n \to 0$.
        Thus $X/n \asto 0$. \qedhere
    \end{enumerate}
\end{solution}

% Problem 2
\begin{problem*}
    Let $X_n$ and $X$ be random variables on a common probability space.
    Show that if $X_n \pto X$ then there is a subsequence $n_k$ such that
    $X_{n_k} \to X$ almost surely.
\end{problem*}
\begin{solution}
    Let $n_1 = 1$ and for each $k \ge 2$, let $n_k \ge n_{k-1}$ be such that
    \[
        \P\set*{\abs{X_{n_k} - X} > \frac 1k} \le \frac 1{k^2}.
    \] (Since $\P\set{\abs{X_n - X} > 1/k} \to 0$ as $n \to \infty$.)
    Fix an $M \in \N \setminus \set 0$ and observe \begin{align*}
        \sum_{k=1}^\infty \P\set*{\abs{X_{n_k} - X} > \frac 1M}
            &\le \sum_{k=1}^M \P\set*{\abs{X_{n_k} - X} > \frac 1M} \\
            &\qquad+ \sum_{k > M} \P\set*{\abs{X_{n_k} - X} > \frac 1k} \\
            &\le M + \sum_{k > M} \frac 1{k^2} \\
            &< \infty.
    \end{align*}
    By the Borel-Cantelli lemma,
    $\P\set*{\abs{X_{n_k} - X} > \frac 1M \text{ i.o.}} = 0$.
    In other words, \[
        \P\ab(\bigcup_{K=1}^\infty \bigcap_{k \ge K}
            \set*{\abs{X_{n_k} - X} \le \frac 1M})
                = 1.
    \]
    Since the intersection of countably many almost sure events is almost
    sure, we have \[
        \P\set{\lim_{k \to \infty} X_{n_k} = X}
        = \P\ab(\bigcap_{M=1}^\infty \bigcup_{K=1}^\infty \bigcap_{k \ge K}
            \set*{\abs{X_{n_k} - X} \le \frac 1M})
        = 1. \qedhere
    \]
\end{solution}

% Problem 3
\begin{problem*}
% Let X1, X2, . . . be i.i.d. with distribution μ ∈ P(R). Recall that the support of μ is the smallest closed set K with μ(K) = 1. Show that {X1, X2, . . .} = K a.s. (the left side is the closure of the set {Xn})
    Let $X_1, X_2, \dots$ be \iid random variables with distribution
    $\mu \in \mcP(\R)$.
    Recall that the support of $\mu$ is the smallest closed set $K$ with
    $\mu(K) = 1$.
    Show that $\wbar{\set{X_1, X_2, \dots}} = K$ almost surely.
\end{problem*}

% Problem 4
\begin{problem*}
% Let Xn be independent and P(Xn = n−a) = 21 = P(Xn = −n−a) where a > 0 is fixed. For what values of a does the series PXn converge a.s.? For which values of a does the series converge absolutely, a.s.?
    Let $X_n$ be independent and $\P\set{X_n = n-a}
    = \P\set{X_n = -n-a} = 1/2$ where $a > 0$ is fixed.
    For what values of $a$ does the series $\sum_{n=1}^\infty X_n$ converge
    absolutely, a.s.?
\end{problem*}

% Problem 5
\begin{problem*}
% Suppose Xn are i.i.d random variables with finite mean. Which of the following assumptions guarantee that P Xn converges a.s.?
% (1) (i)E[Xn]=0forallnand(ii)PE[Xn2∧1]<∞. (2) (i)E[Xn]=0forallnand(ii)PE[Xn2∧|Xn|]<∞.
    Suppose $X_n$ are \iid random variables with finite mean.
    Which of the following assumptions guarantee that $\sum_{n=1}^\infty X_n$
    converges a.s.?
    \begin{enumerate}
        \item (i) $\E[X_n] = 0$ for all $n$ and
            (ii) $\sum \E[X_n^2 \wedge 1] < \infty$.
        \item (i) $\E[X_n] = 0$ for all $n$ and
            (ii) $\sum \E[X_n^2 \wedge \abs{X_n}] < \infty$.
    \end{enumerate}
\end{problem*}

% Problem 6
\begin{problem*}[Large deviation for Bernoullis]
% Let Xn be i.i.d Ber(1/2). Fix p > 21 .
% (1) ShowthatP(Sn >np)≤e 2 foranyλ>0.
% (2) Optimize over λ to get P(Sn > np) ≤ e−nI(p) where I(p) = −plogp − (1 − p)log(1 − p). (Observe that this is the entropy of the Ber(p) measure).
    Let $X_n$ be \iid $\Ber(1/2)$.
    Fix $p > 1/2$.
    \begin{enumerate}
        \item Show that
        $\P\set{S_n > np} \le e^{-np\lambda}\ab(\frac{e^\lambda+1}{2})^n$
        for any $\lambda > 0$.
        \item Optimize over $\lambda$ to get $\P\set{S_n > np} \le e^{-nI(p)}$
        where $I(p) = -p\log p - (1-p)\log(1-p)$.
        (Observe that this is the entropy of the $\Ber(p)$ measure.)
    \end{enumerate}
\end{problem*}
\begin{solution} \leavevmode
    \begin{enumerate}
        \item Let $Y_n = X_n - p$.
        Then $\E[Y_n] = 0$ and $\abs{Y_n} \le p$.
        By Hoeffding's inequality, \begin{align*}
            \P\set{S_n^X > np} &= \P\set{S_n^Y > 0} \\
                &= 
        \end{align*}
    \end{enumerate}
\end{solution}

% Problem 7
\begin{problem*}
% Carry out the same program for i.i.d exponential(1) random variables and deduce thatP(Sn >nt)≤e−nI(t) fort>1andP(Sn <nt)≤e−nI(t) fort<1whereI(t):=t−1−logt.
    Carry out the same program for \iid $\Exp(1)$ random variables and
    deduce that $\P\set{S_n > nt} \le e^{-nI(t)}$ for $t > 1$ and
    $\P\set{S_n < nt} \le e^{-nI(t)}$ for $t < 1$ where
    $I(t) := t-1-\log t$.
\end{problem*}
\begin{solution}
    \[
        \E[e^{\lambda S_n}] = \prod_{k=1}^n \E[e^{\lambda X_k}]
        = \frac1{(1-\lambda)^n}.
    \] Thus for $t > 1$, \begin{align*}
        \P\set{S_n > nt} &= \P\set{e^{\lambda S_n} > e^{n\lambda t}} \\
            &\le e^{-n\lambda t} \E[e^{\lambda S_n}] \\
            &= \ab(\frac{e^{-\lambda t}}{1-\lambda})^n.
    \end{align*}
    Optimizing over $\lambda$, \begin{align*}
        0 &= \odv{}{\lambda} \frac{e^{-\lambda t}}{1 - \lambda} \\
            &= \frac{-t e^{-\lambda t}(1 - \lambda) + e^{-\lambda t}}{(1-\lambda)^2} \\
        \implies 1 &= t(1-\lambda) \\
        \implies \lambda &= 1 - \frac1t.
    \end{align*}
    So $e^{-\lambda t} = e^{-t+1}$ and $\frac1{1-\lambda} = t$.
    Thus \[
        \P\set{S_n > nt} \le (e^{-t+1+\log t})^n = e^{-nI(t)}.
    \] Similarly for $t < 1$.
\end{solution}

% Problem 8
\begin{problem*}
% Let (Ω,F,P) be a probability space.
% (1) LetX,Y berandomvariablesonΩ.Defineafunctiond(X,Y)=E |X−Y| .Showthat
% 1+|X−Y |
% d is a metric on the set of all random variables and d(Xn, X) → 0 iff Xn →P X.
% (2) Show that if Xn,X are random variables such that any subsequence of Xn has a further subsequence that converge almost surely to X then Xn →P X.
    Let $(\Omega, \mcF, \P)$ be a probability space.
    \begin{enumerate}
        \item Let $X, Y$ be random variables on $\Omega$.
        Define a function $d(X, Y) = \E\ab[\frac{\abs{X-Y}}{1+\abs{X-Y}}]$.
        Show that $d$ is a metric on the set of all random variables and
        $d(X_n, X) \to 0$ if and only if $X_n \pto X$.
        \item Show that if $X_n, X$ are random variables such that any
        subsequence of $X_n$ has a further subsequence that converges almost
        surely to $X$ then $X_n \pto X$.
    \end{enumerate}
\end{problem*}
\begin{solution} \leavevmode
    \begin{enumerate}
        \item Let $f(x, y) = \frac{\abs{x-y}}{1+\abs{x-y}}. $\begin{align*}
            \E[f(X, Y)] &= \E[f(X, Y) \1{\abs{X - Y} < t}]
                + \E[f(X, Y) \1{\abs{X - Y} \ge t}] \\
                &\le t + \P\set{\abs{X - Y} \ge t}.
        \end{align*}
        Thus for any $\delta > 0$, \[
            \E[f(X_n, X)]
                \le \delta + \P\set{\abs{X_n - X} \ge \delta} \to \delta.
        \]
        \item Let $p_n = \P\set{\abs{X_n - X} > \delta}$.
        Let $(p_{n_k})_k$ be a convergent subsequence.
        Then $(p_{n_{k_j}})_j$ is a further subsequence that converges to
        $0$, since $X_{n_{k_j}} \asto X$ implies $X_{n_{k_j}} \pto X$.
        Thus $p_{n_k} \to 0$, so $\limsup p_n = 0$. \qedhere
    \end{enumerate}
\end{solution}

% Problem 9
\begin{problem*}
% Let Xn, Yn, X, Y be random variables on a common probability space. If Xn →P X and Yn →P Y (all r.v.s on the same probability space), show that f(Xn,Yn) → f(X,Y) for any continuous f : R2 → R. In particular, this implies if Xn →P X,Yn →P Y then for any a,b ∈ R,aXn+bYn→P aX+bY.
    Let $X_n, Y_n, X, Y$ be random variables on a common probability space.
    If $X_n \pto X$ and $Y_n \pto Y$ (all random variables on the same
    probability space), show that $f(X_n, Y_n) \pto f(X, Y)$ for any
    continuous $f\colon \R^2 \to \R$.
    In particular, this implies if $X_n \pto X$ and $Y_n \pto Y$ then for
    any $a, b \in \R$, $aX_n + bY_n \pto aX + bY$.
\end{problem*}
\begin{solution}
    Let $(n_k)_k$ be any subsequence of $\N$.
    Then $X_{n_k} \pto X$ and $Y_{n_k} \pto Y$.
    So there is a subsequence $(n_{k_j})_j$ such that
    $X_{n_{k_j}} \asto X$ and $Y_{n_{k_j}} \asto Y$.
    Then $f(X_{n_{k_j}}, Y_{n_{k_j}}) \asto f(X, Y)$ by continuity of $f$.
    Thus every subsequence of $f(X_n, Y_n)$ has a further subsequence that
    converges almost surely to $f(X, Y)$.
    By the previous problem, $f(X_n, Y_n) \pto f(X, Y)$.
\end{solution}

% Problem 10
\begin{problem*}
% Give examples of two sequence of random variables Xn and Yn such that Xn →d X and Yn →d Y but Xn + Yn does not converge in distribution to X + Y.
    Give examples of two sequences of random variables $X_n$ and $Y_n$ such
    that $X_n \dto X$ and $Y_n \dto $ but $X_n + Y_n$ does not converge in
    distribution to $X + Y$.
\end{problem*}
\begin{solution}
    Let $X_n \sim \Ber(1/2)$ and $Y_n = 1 - X_n$.
    Choose $X$, $Y$ \iid $\Ber(1/2)$.
    Then $X_n \dto X$ and $Y_n \dto Y$, but $X_n + Y_n \sim \delta_1$.
\end{solution}

\end{document}
