\lecture{2024-10-07}{}
\begin{definition} \label{def:quadform}
    Let $A \in M_n(\R)$ be symmetric.
    We define $B_A\colon \R^n \times \R^n \to \R$ by \[
        B_A(x, y) = \innerp{Ax}{y} = x^T A y
    \] and $q_A\colon \R^n \to \R$ by \[
        q_A(x) = B_A(x, x) = \innerp{Ax}{x} = x^T A x.
    \]
\end{definition}

\begin{theorem}[Taylor's theorem] \label{thm:taylor}
    Let $U \subopeneq \R^n$ be convex and $f \in C^m(U; \R)$.
    Then for any $a, a + h \in U$, \[
        f(a + h) = \sum_{k=0}^m \sum_{i_1, \dots, i_k} \frac1{k!}
            (D_{i_1 \dots i_k} f(a)) h_{i_1} \dots h_{i_k}
            + r(h)
    \] where \[
        r(h) = \frac1{m!} \sum_{i_1, \dots, i_m} \ab(
                D_{i_1 \dots i_m} f(c) - D_{i_1 \dots i_m} f(a)
            ) h_{i_1} \dots h_{i_m}
    \] for some $c$ in the segment $a$--$(a + h)$ is $o(\norm h^m)$.
\end{theorem}
\begin{proof}
    Let $\gamma(t) = a + th$ and $g = f \circ \gamma$.
    If $f$ is $C^m$, then so is $g$.
    By the $1$-dimensional Taylor's theorem, \[
        g(1) = \sum_{k=0}^{m-1} \frac1{k!} g^{(k)}(0)
            + \frac1{m!} g^{(m)}(\xi)
    \] for some $\xi \in (0, 1)$.
    By the composition rule, \begin{align*}
        g'(t) &= f'(\gamma(t)) \gamma'(t)
            = \sum_{i=1}^n D_i f(\gamma(t)) h_i \\
        g''(t) &= \sum_{j=1}^n \sum_{i=1}^n D_{ji} f(\gamma(t)) h_i h_j \\
            &\mathrel{\makebox[\widthof{=}]{\vdots}} \\
        g^{(m)}(t) &= \sum_{i_1, \dots, i_m} D_{i_1 \dots i_m} f(\gamma(t))
            h_{i_1} \dots h_{i_m} \qedhere
    \end{align*}
\end{proof}
