\lecture{2024-09-13}{Higher derivatives}
\begin{theorem} \label{thm:pdvs}
    Let $U \subopeneq \R^n$ and $f = (f_1, \dots, f_m)\colon U \to \R^m$.
    Then
    \begin{enumerate}
        \item for $k \ge 1$, $f \in C^k(U, \R^m)$ iff $f'$ exists and
        $f' \in C^{k-1}(U, L(\R^n, \R^m))$.
        \item $f \in C^2(U, \R^m)$ iff all partial derivatives of $f_i$
        upto second order exist and are continuous.
        \item $f \in C^k(U, \R^m)$ iff all partial derivatives of $f_i$
        upto $k$th order exist and are continuous.
    \end{enumerate}
\end{theorem}

\begin{definition*}[smoothness] \label{def:smooth}
    $f\colon U \subopeneq \R^n \to \R^m$ is \emph{smooth} if
    $f \in C^\infty(U, \R^m)$, where \[
        C^\infty(U, \R^m) = \bigcap_{k=1}^\infty C^k(U, \R^m).
    \]
\end{definition*}

\begin{corollary}
    From \cref{thm:pdvs}, $f$ is smooth iff all partial derivatives of $f_i$
    of all orders exist and are continuous.
\end{corollary}

\begin{example}
    The function \begin{align*}
        i\colon \GL_n(\R) &\to X^{-1} \in \M_n(\R) \\
        X &\mapsto X^{-1}
    \end{align*}
    is smooth, since \[
        i(X) = \frac{1}{\det X} \adj X
    \] is rational in the entries of $X$.
    That is, each entry of $i(X)$ is a rational function of the entries of
    $X$.
    Thus the partial derivatives of each coordinate function of $i$ exist
    and are continuous.
\end{example}

\begin{definition*}[diffeomorphism] \label{def:diffeo}
    Let $U, V \subopeneq \R^n$.
    A bijective differentiable map $f\colon U \to V$ is a
    \emph{diffeomorphism} if $f^{-1}\colon V \to U$ is also differentiable.

    More generally, a bijective $C^k$ map $f\colon U \to V$ is a
    $C^k$-diffeomorphism if $f^{-1}$ is also $C^k$.
\end{definition*}

We will now state and prove a generalised version of the
inverse function theorem.
\begin{theorem*}[generalized inverse function theorem] \label{thm:gen-ift}
    Let $U \subopeneq \R^n$ and $f\colon U \to \R^n$ be $C^k$ with
    $k \ge 1$.
    Suppose $f'(a)$ is invertible for some $a \in U$.
    Then there exist open sets $V \ni a, W \subopeneq \R^n$ such that
    $f\colon V \to W$ is a $C^k$-diffeomorphism.
\end{theorem*}
The case $k = 1$ is the usual inverse function theorem.

\begin{exercise*}
    Let $U \subopeneq \R^n$ and $V \subopeneq \R^m$.
    Suppose $f\colon U \to V$ is a differentiable bijection, whose inverse
    is also differentiable.
    Then $m = n$.
\end{exercise*}
\begin{solution}
    $f^{-1} \circ f = \id_V$ and $f \circ f^{-1} = \id_U$.
    Both of these are differentiable, with the derivative being the
    identity map everywhere.
    By the composition rule, \begin{align*}
        \id_U'(x) &= (f^{-1} \circ f)'(x) = {f^{-1}}'(f(x)) \circ f'(x) \\
        \id_V'(y) &= (f \circ f^{-1})'(y) = f'(f^{-1}(y)) \circ {f^{-1}}'(y)
    \end{align*} requires $f'(x) \in L(\R^n, \R^m)$
    and ${f^{-1}}'(y) \in L(\R^m, \R^n)$ to be injective.
    This requires $m = n$.
\end{solution}

\begin{fact}
    Let $U \subopeneq \R^n$ and $V \subopeneq \R^m$.
    Suppose $f\colon U \to V$ is a homeomorphism.
    Then $m = n$.
\end{fact}
The proof is hard.

For the midterm, you should be comfortable with finding derivatives without
any computation.
You will not be required to prove the derivatives.
For example, recall the product rule:
If $f(x) = u(x) v(x)$, then $f'(a)(h) = u'(a)(h) v(a) + u(a) v'(a)(h)$.
\begin{examples}
    \item $f(X) = X^2$ has derivative $f'(A)(H) = HA + AH$.
    \item $f(X) = XBX$ has derivative $f'(A)(H) = HBA + ABH$.
    \item $f(X) = X^{-1}BX$ has derivative
        $f'(A)(H) = -A^{-1}HA^{-1}BA + A^{-1}BH$.
    \item $f(X) = X^{-1}$ has derivative $f'(A)(K) = -A^{-1}KA^{-1}$.
        What is $f''$?
        \begin{align*}
            f&\colon \GL_n \to \M_n \\
            f'&\colon \GL_n \to L(\M_n, \M_n) \\
            f''&\colon \GL_n \to L(\M_n, L(\M_n, \M_n)) \\
            f''(A)&\colon \M_n \to L(\M_n, \M_n) \\
            f''(A)(H)&\colon \M_n \to \M_n \\
            f''(A)(H)(K)&\colon \M_n
        \end{align*}
        We know $f'(A)(H) = A^{-1}HA^{-1}$.
        We want $f'(A + K) - f'(A)$.
        We can evaluate this pointwise.
        \TODO[WHY???] % TODO
        \begin{align*}
            f'(A+K)(H) - f'(A)(H)
                &= (A+K)^{-1}H(A+K)^{-1} - A^{-1}HA^{-1} \\
            \shortintertext{and we use
            $(A + K)^{-1} = A^{-1} - A^{-1}KA^{-1} + o(K)$ to write}
            f'(A+K)(H) - f'(A)(H)
                &= A^{-1} H A^{-1} - A^{-1}KA^{-1}HA^{-1} \\
                    &\qquad- A^{-1}HA^{-1}KA^{-1} - A^{-1}HA^{-1} \\[.4em]
                &= -A^{-1}KA^{-1}HA^{-1} -A^{-1}HA^{-1}KA^{-1} \\[.4em]
                &= -\frac1AK\frac1AH\frac1A -\frac1AH\frac1AK\frac1A.
        \end{align*}
        This is the second derivative of $X \mapsto X^{-1}$. \begin{align*}
            T\colon \GL_n &\to L(\M_n, L(\M_n, \M_n)) \\
            T(A)(K)(H) &= -A^{-1}KA^{-1}HA^{-1} - A^{-1}HA^{-1}KA^{-1}.
        \end{align*}
        What we have shown is that \begin{align*}
            \lim_{K \to 0} \frac{\norm{f'(A+K)(H) - f'(A)(H) - T(A)(K)(H)}}
                {\norm K} &= 0 \\
            \shortintertext{for each $H \in \M_n$.
            Taking the supremum over all $\norm H = 1$ gives}
            \lim_{K \to 0} \frac{\norm{f'(A+K) - f'(A) - T(A)(K)}}
                {\norm K} &= 0.
        \end{align*}
\end{examples}
