\documentclass[12pt]{article}
% \usepackage{parskip}
\usepackage{lmodern} % https://tex.stackexchange.com/a/58088/295755
\renewcommand\bfdefault{b}
\usepackage{microtype}

\usepackage{amsmath}
\newcommand\yesnumber{\stepcounter{equation}\tag{\theequation}}

\ifundef{\chapter}{}{\providecommand\thechapter{\Roman{chapter}}}

\usepackage{marginnote}
\usepackage[en-GB,calc]{datetime2}
\usepackage{calc}
\usepackage{xifthen}
\usepackage{tocloft}

% https://tex.stackexchange.com/a/454168
\newcommand\monthday[1]{\DTMmonthname{\DTMfetchmonth{#1}}~\number\DTMfetchday{#1}}

\newlistof{lecture}{lec}{Lectures} % what is this file extension business?
\makeatletter
\setlength\marginparwidth{1in}
\newcommand*{\lecture}[3][]{
    \ifthenelse{\isempty{#1}}{%
        \refstepcounter{lecture}%
    }{%
        \setcounter{lecture}{#1}%
    }%
    \DTMsavedate{lecdate}{#2}%
    \def\lecdow{\DTMweekdayname{\DTMfetchdow{lecdate}}}
    \def\lecshortdow{\DTMshortweekdayname{\DTMfetchdow{lecdate}}}
    \def\lecmonth{\DTMmonthname{\DTMfetchmonth{lecdate}}}
    \def\lecday{\number\DTMfetchday{lecdate}}
    % \marginpar{\raggedright\small \textsf{\textbf{Lecture \thelecture.}%
    %             \footnotesize\DTMusedate{lecdate}}}%
    \marginnote{\raggedright\small%
        \textsf{{\textbf{Lecture \thelecture.}} \\
        \footnotesize\lecdow\\\lecmonth\ \lecday}}%
    \ifthenelse{\isempty{#3}}{%
        \addcontentsline{lec}{lecture}{\protect\numberline{\thelecture}%
        \lecshortdow, \lecmonth\ \lecday}%
        \def\@lecture{Lecture \thelecture}%
    }{%
        \addcontentsline{lec}{lecture}{\protect\numberline{\thelecture}%
        \makebox[\widthof{Mon,}][l]{\lecshortdow,}\ \makebox[\widthof{September 00}][l]{\lecmonth\ \lecday} #3}%
        \def\@lecture{Lecture \thelecture: #3}%
    }%
    \par%
}
\g@addto@macro\normalsize{%
  \setlength\abovedisplayskip{7pt}%
  \setlength\belowdisplayskip{7pt}%
  \setlength\abovedisplayshortskip{1pt}%
  \setlength\belowdisplayshortskip{1pt}%
}
\makeatother

\usepackage[twoside]{fancyhdr}
\setlength{\headheight}{15pt}
\pagestyle{fancy}
\fancyhf{}
% \fancyhead[r]{\thepage}
\makeatletter
\fancyhead[LE,RO]{\thepage}
\fancyhead[RE]{\textbf{\nouppercase\leftmark}}
\fancyhead[LO]{\nouppercase\rightmark}
\providecommand\@lecture{}
\fancyfoot[R]{\small\@lecture}
\makeatother

% homeworks
\newlistof{hw}{hw}{Assignments} % counter `assignment' already defined
\makeatletter
\newcommand*{\assignment}[5][]{%[number]{file}{date posted}{date due}{date quiz}
    \ifthenelse{\isempty{#1}}{%
        \refstepcounter{hw}%
        \stepcounter{assignment}%
    }{%
        \setcounter{hw}{#1}%
        \setcounter{assignment}{#1}%
    }%
    \pagebreak
    \ifthenelse{\isempty{#3}}{}{\DTMsavedate{posted}{#3}}%
    \ifthenelse{\isempty{#4}}{}{\DTMsavedate{due}{#4}}%
    \ifthenelse{\isempty{#5}}{}{\DTMsavedate{quiz}{#5}}%
    \section*{Assignment \thehw}
    \ifthenelse{\isempty{#4}}{
        \ifthenelse{\isempty{#5}}{
            \ifthenelse{\isempty{#3}}{
                \def\hw@toc{}
            }{
                \def\hw@toc{posted \monthday{up}}
            }
        }{
            \def\hw@toc{quiz \monthday{quiz}}
        }
    }{
        \def\hw@toc{due \monthday{due}}
    }
    \addcontentsline{hw}{hw}{\protect\numberline{\thehw}\hw@toc}\par
    \marginpar{\raggedright\footnotesize\textsf{%
        \ifthenelse{\isempty{#3}}{}{\makebox[\widthof{quiz}][l]{up} \monthday{posted} \\}%
        \ifthenelse{\isempty{#4}}{}{\makebox[\widthof{quiz}][l]{due} \monthday{due} \\}%
        \ifthenelse{\isempty{#5}}{}{quiz \monthday{quiz}}%
    }}
    \def\@lecture{Assignment \thehw\ifx\hw@toc\empty{}\else\ --- \hw@toc\fi}
    \input{#2}
    \newpage
}
\makeatother

\usepackage{amsmath}
\usepackage{amsthm}
\usepackage[dvipsnames]{xcolor}
\colorlet{exercise}{cyan!70!black}
\colorlet{solved}{green!40!black}
\colorlet{self_proof}{blue!70!black}
\colorlet{Red}{red!80!black}

% Gilles Castel's theorems
\newtheoremstyle{mddefinition}% <name>
  {-.25\topsep}%                 <space above>
  {-.25\topsep}%                         <space below>
  {\normalfont}%              <body font>
  {}%                         <indent amount>
  {\bfseries}%                <theorem head font>
  {.}%                        <punctuation after theorem head>
  {.5em}%                     <space after theorem head>
  {}%                         <theorem head spec>
\newtheoremstyle{mdplain}% <name>
  {-.25\topsep}%                 <space above>
  {-.25\topsep}%                         <space below>
  {\itshape}%                 <body font>
  {}%                         <indent amount>
  {\bfseries}%                <theorem head font>
  {.}%                        <punctuation after theorem head>
  {.5em}%                     <space after theorem head>
  {}%                         <theorem head spec>

\usepackage[framemethod=Tikz]{mdframed}
\mdfsetup{skipbelow=0pt}
\mdfdefinestyle{axiomstyle}{
    outerlinewidth = 1.5,
    roundcorner = 10,
    leftmargin = 15,
    rightmargin = 15,
    backgroundcolor = yellow!7
}
\mdfdefinestyle{defstyle}{
    outerlinewidth = 1,
    roundcorner = 2,
    leftmargin = 7,
    rightmargin = 7,
    backgroundcolor = green!5
}
\mdfdefinestyle{thmstyle}{
    outerlinewidth = 1,
    roundcorner = 8,
    leftmargin = 7,
    rightmargin = 7,
    backgroundcolor = cyan!5
}
\mdfdefinestyle{lemmastyle}{
    outerlinewidth = 1.5,
    roundcorner = 10,
    leftmargin = 7,
    rightmargin = 7,
    backgroundcolor = yellow!10
}
\ifundef\chapter{%
    \providecommand\theoremnumberwithin{section}
    \theoremstyle{mddefinition}
    \newmdtheoremenv[nobreak=true, style=axiomstyle]{axiom}{Axiom}[section]
    \theoremstyle{plain}
    \newtheorem{theorem}{Theorem}[\theoremnumberwithin]

    \newcounter{assignment}
    \theoremstyle{plain}
    \newtheorem{problem}{Problem}
    \theoremstyle{mddefinition}
    \newmdtheoremenv[nobreak=true, outerlinewidth=0.7]{problem*}[problem]{Problem}
}{
    \providecommand\theoremnumberwithin{chapter}
    \theoremstyle{mddefinition}
    \newmdtheoremenv[nobreak=true, style=axiomstyle]{axiom}{Axiom}[chapter]
    \theoremstyle{plain}
    \newtheorem{theorem}{Theorem}[\theoremnumberwithin]

    \newcounter{assignment}
    \theoremstyle{plain}
    \newtheorem{problem}{Problem}[assignment]
    \theoremstyle{mddefinition}
    \newmdtheoremenv[nobreak=true, outerlinewidth=0.7]{problem*}[problem]{Problem}
}
\theoremstyle{mddefinition}
\newmdtheoremenv[nobreak=true, style=defstyle]{definition*}[theorem]{Definition}

\theoremstyle{mdplain}
\newmdtheoremenv[nobreak=true, style=thmstyle]{theorem*}[theorem]{Theorem}
\newmdtheoremenv[nobreak=true]{proposition*}[theorem]{Proposition}
\newmdtheoremenv[nobreak=true]{lemma*}[theorem]{Lemma}
\newmdtheoremenv[nobreak=true]{corollary*}[theorem]{Corollary}
\newmdtheoremenv[nobreak=true, style=thmstyle]{fact*}[theorem]{Fact}
\newmdtheoremenv[nobreak=true]{exercise*}[theorem]{Exercise}
\newmdtheoremenv[nobreak=true]{question*}[theorem]{Question}

\theoremstyle{definition}
\newtheorem{definition}[theorem]{Definition}

\theoremstyle{plain}
\newtheorem{proposition}[theorem]{Proposition}
\newtheorem{lemma}[theorem]{Lemma}
\newtheorem{corollary}[theorem]{Corollary}
\newtheorem{fact}[theorem]{Fact}
\newtheorem{exercise}[theorem]{Exercise}
\newtheorem{question}[theorem]{Question}

\theoremstyle{remark}
\newtheorem*{remark}{Remark}
\newtheorem*{remarkx}{Remarks}
\newtheorem*{example}{Example}
\newtheorem*{examplex}{Examples}
\newtheorem*{idea}{Idea}
%%%% HAXXXXXX %%%%
% \def\innerqed{\qedsymbol}
% \def\outerqed{$\blacksquare$}
\let\oldproof\proof
\let\endoldproof\endproof
\newenvironment{solution}[1][]
  {\renewcommand\qedsymbol{$\blacksquare$}%
  \begin{oldproof}[Solution\ifx&#1&\else\ (#1)\fi]}
  {\end{oldproof}}
\newenvironment{answer}
  {\renewcommand\qedsymbol{$\blacksquare$}\begin{oldproof}[Answer]}
  {\end{oldproof}}
\renewenvironment{proof}
  {\renewcommand\qedsymbol{$\blacksquare$}\begin{oldproof}}
  {\end{oldproof}}
\newenvironment{subproof}[1][Subproof]{%
  \renewcommand{\qedsymbol}{$\square$}\begin{oldproof}[#1]}
  {\end{oldproof}}
\newtheorem*{notation}{Notation}
\newtheorem*{claim}{Claim}

\usepackage{hyperref}
\usepackage[noabbrev]{cleveref}

% <cref>
\crefname{theorem}{theorem}{theorems}
\crefname{proposition}{proposition}{propositions}
\crefname{lemma}{lemma}{lemmas}
\crefname{corollary}{corollary}{corollaries}
\crefname{axiom}{axiom}{axioms}
\crefname{definition}{definition}{definitions}
\crefname{problem}{problem}{problems}
\crefname{exercise}{exercise}{exercises}
\crefname{fact}{fact}{facts}
\crefname{question}{question}{questions}
\crefname{remark}{remark}{remarks}
\crefname{example}{example}{example}
\crefname{notation}{notation}{notations}
\crefname{claim}{claim}{claims}
% \crefname{section}{\S}{\S\S}
\crefname{theorem*}{theorem}{theorems}
\crefname{proposition*}{proposition}{propositions}
\crefname{lemma*}{lemma}{lemmas}
\crefname{corollary*}{corollary}{corollaries}
\crefname{definition*}{definition}{definitions}
\crefname{problem*}{problem}{problems}
\crefname{exercise*}{exercise}{exercises}
\crefname{fact*}{fact}{facts}
\crefname{question*}{question}{questions}
% </cref>

% <hyperlinks>
\hypersetup{colorlinks,
    linkcolor={blue},
    citecolor={blue!50!black},
    urlcolor={blue!80!black}}
% </hyperlinks>

\usepackage[shortlabels]{enumitem}
% change default label for enumerate, and fix long labels popping out
\setenumerate{label*=(\roman*),ref=(\roman*),leftmargin=*}
% casework list using https://tex.stackexchange.com/a/30035
\newcounter{casecount}
\newlist{casework}{description}{1}
\setlist[casework]{%
  before={\setcounter{casecount}{0}%
      \renewcommand*\thecasecount{\arabic{casecount}}}%
  ,font=\bfseries Case \stepcounter{casecount}\thecasecount:
}

\newenvironment{examples}[1][]
{\begin{examplex}[#1]\leavevmode\begin{itemize}}{\end{itemize}\end{examplex}}
\newenvironment{remarks}[1][]
{\begin{remarkx}[#1]\leavevmode\begin{itemize}}{\end{itemize}\end{remarkx}}

% omg this is so HaXy
% \renewenvironment{proof}[1][\proofname]{{\it\bfseries #1. }}{\qed}
% \providecommand{\qedsymbol}{\openbox}
% \makeatletter
% \renewenvironment{proof}[1][\proofname]{\par
%   \pushQED{\qed}%
%   \normalfont \topsep6\p@\@plus6\p@\relax
%   \trivlist
%   \item[\hskip\labelsep
%         \itshape\bfseries%this is the change (boldface instead of italics)
%         % \fontseries{bx}\selectfont
%     #1\@addpunct{.}]\ignorespaces
% }{%
%   \popQED\endtrivlist\@endpefalse
% }
\makeatother
\crefname{enumi}{part}{parts}
\crefname{enumii}{part}{parts}
\crefname{enumiii}{part}{parts}
% \setlist[itemize]{itemsep=2pt}
\newcounter{dummy}
\makeatletter
\newcommand\myitem[1][]{\item[#1]\refstepcounter{dummy}\def\@currentlabel{#1}}
\makeatother

\makeatletter
\newcommand*{\refifdef}[3]{%label,command,fallback
    \@ifundefined{r@#1}{#3}{#2{#1}}%
}
\makeatother

\newcommand\ie{\textit{i.e.}}
\newcommand\eg{\textit{e.g.}}
\usepackage{physics}

\usepackage{amsmath}
\usepackage{amssymb}
\usepackage{mathrsfs} % for \mathscr
\usepackage{bm} % for \bm
\usepackage{booktabs}

% undefine \abs and \norm
\let\abs\relax
\let\norm\relax

\usepackage{mathtools} % for delimiters and \coloneqq
\DeclarePairedDelimiter{\paren}{(}{)}
\DeclarePairedDelimiter{\brk}{[}{]}
\DeclarePairedDelimiter{\set}{\{}{\}}
\DeclarePairedDelimiter{\abs}{\lvert}{\rvert}
\DeclarePairedDelimiter{\norm}{\lVert}{\rVert}
\DeclarePairedDelimiter{\floor}{\lfloor}{\rfloor}
\DeclarePairedDelimiter{\ceil}{\lceil}{\rceil}
\DeclarePairedDelimiter{\angled}{\langle}{\rangle}
% \DeclarePairedDelimiterX{\innerp}[2]{\langle}{\rangle}{#1,\,#2}
% \DeclarePairedDelimiterX{\outerp}[2]{\langle}{\rangle}{#1\otimes#2}
% \DeclarePairedDelimiterX{\braket}[3]{\langle}{\rangle}%
% {#1\,\delimsize\vert\,\mathopen{}#2\,\delimsize\vert\,\mathopen{}#3}
\DeclarePairedDelimiterX{\innerp}[2]{\langle}{\rangle}{#1,\,#2}
% \DeclarePairedDelimiterX{\outerp}[2]{\langle}{\rangle}{#1\otimes#2}
\DeclarePairedDelimiterX{\outerp}[2]{\vert}{\vert}%
{#1\delimsize\rangle\delimsize\langle\mathopen{}#2}
\let\braket\relax
\DeclarePairedDelimiterX{\braket}[3]{\langle}{\rangle}%
{#1\,\delimsize\vert\,\mathopen{}#2\,\delimsize\vert\,\mathopen{}#3}

\renewcommand\O{\ensuremath{\varnothing}}
\newcommand\N{\ensuremath{\mathbb{N}}}
\newcommand\Z{\ensuremath{\mathbb{Z}}}
\newcommand\Q{\ensuremath{\mathbb{Q}}}
\newcommand\R{\ensuremath{\mathbb{R}}}
\newcommand\C{\ensuremath{\mathbb{C}}}
% \renewcommand\P{\ensuremath{\mathbb{P}}}

% fix spacing for \forall and \exists
% \let\oldforall\forall
% \renewcommand{\forall}{\oldforall \, }
% \let\oldexist\exists
% \renewcommand{\exists}{\oldexist \: }
\newcommand\unique{\exists!}
\newcommand\lxor{\oplus}

\providecommand{\dd}{\,\mathrm{d}}

\newcommand\mcA{\ensuremath{\mathcal{A}}}
\newcommand\mcB{\ensuremath{\mathcal{B}}}
\newcommand\mcC{\ensuremath{\mathcal{C}}}
\newcommand\mcD{\ensuremath{\mathcal{D}}}
\newcommand\mcE{\ensuremath{\mathcal{E}}}
\newcommand\mcF{\ensuremath{\mathcal{F}}}
\newcommand\mcG{\ensuremath{\mathcal{G}}}
\newcommand\mcH{\ensuremath{\mathcal{H}}}
\newcommand\mcI{\ensuremath{\mathcal{I}}}
\newcommand\mcJ{\ensuremath{\mathcal{J}}}
\newcommand\mcK{\ensuremath{\mathcal{K}}}
\newcommand\mcL{\ensuremath{\mathcal{L}}}
\newcommand\mcM{\ensuremath{\mathcal{M}}}
\newcommand\mcN{\ensuremath{\mathcal{N}}}
\newcommand\mcO{\ensuremath{\mathcal{O}}}
\newcommand\mcP{\ensuremath{\mathcal{P}}}
\newcommand\mcQ{\ensuremath{\mathcal{Q}}}
\newcommand\mcR{\ensuremath{\mathcal{R}}}
\newcommand\mcS{\ensuremath{\mathcal{S}}}
\newcommand\mcT{\ensuremath{\mathcal{T}}}
\newcommand\mcU{\ensuremath{\mathcal{U}}}
\newcommand\mcV{\ensuremath{\mathcal{V}}}
\newcommand\mcW{\ensuremath{\mathcal{W}}}
\newcommand\mcX{\ensuremath{\mathcal{X}}}
\newcommand\mcY{\ensuremath{\mathcal{Y}}}
\newcommand\mcZ{\ensuremath{\mathcal{Z}}}

%%%% WIDE BAR THAT IS JUST THE RIGHT LENGTH %%%%
%% FROM https://tex.stackexchange.com/a/60253 %%
\makeatletter
\let\save@mathaccent\mathaccent
\newcommand*\if@single[3]{%
  \setbox0\hbox{${\mathaccent"0362{#1}}^H$}%
  \setbox2\hbox{${\mathaccent"0362{\kern0pt#1}}^H$}%
  \ifdim\ht0=\ht2 #3\else #2\fi
  }
%The bar will be moved to the right by a half of \macc@kerna, which is computed by amsmath:
\newcommand*\rel@kern[1]{\kern#1\dimexpr\macc@kerna}
%If there's a superscript following the bar, then no negative kern may follow the bar;
%an additional {} makes sure that the superscript is high enough in this case:
\newcommand*\widebar[1]{\@ifnextchar^{{\wide@bar{#1}{0}}}{\wide@bar{#1}{1}}}
%Use a separate algorithm for single symbols:
\newcommand*\wide@bar[2]{\if@single{#1}{\wide@bar@{#1}{#2}{1}}{\wide@bar@{#1}{#2}{2}}}
\newcommand*\wide@bar@[3]{%
  \begingroup
  \def\mathaccent##1##2{%
%Enable nesting of accents:
    \let\mathaccent\save@mathaccent
%If there's more than a single symbol, use the first character instead (see below):
    \if#32 \let\macc@nucleus\first@char \fi
%Determine the italic correction:
    \setbox\z@\hbox{$\macc@style{\macc@nucleus}_{}$}%
    \setbox\tw@\hbox{$\macc@style{\macc@nucleus}{}_{}$}%
    \dimen@\wd\tw@
    \advance\dimen@-\wd\z@
%Now \dimen@ is the italic correction of the symbol.
    \divide\dimen@ 3
    \@tempdima\wd\tw@
    \advance\@tempdima-\scriptspace
%Now \@tempdima is the width of the symbol.
    \divide\@tempdima 10
    \advance\dimen@-\@tempdima
%Now \dimen@ = (italic correction / 3) - (Breite / 10)
    \ifdim\dimen@>\z@ \dimen@0pt\fi
%The bar will be shortened in the case \dimen@<0 !
    \rel@kern{0.6}\kern-\dimen@
    \if#31
      \overline{\rel@kern{-0.6}\kern\dimen@\macc@nucleus\rel@kern{0.4}\kern\dimen@}%
      \advance\dimen@0.4\dimexpr\macc@kerna
%Place the combined final kern (-\dimen@) if it is >0 or if a superscript follows:
      \let\final@kern#2%
      \ifdim\dimen@<\z@ \let\final@kern1\fi
      \if\final@kern1 \kern-\dimen@\fi
    \else
      \overline{\rel@kern{-0.6}\kern\dimen@#1}%
    \fi
  }%
  \macc@depth\@ne
  \let\math@bgroup\@empty \let\math@egroup\macc@set@skewchar
  \mathsurround\z@ \frozen@everymath{\mathgroup\macc@group\relax}%
  \macc@set@skewchar\relax
  \let\mathaccentV\macc@nested@a
%The following initialises \macc@kerna and calls \mathaccent:
  \if#31
    \macc@nested@a\relax111{#1}%
  \else
%If the argument consists of more than one symbol, and if the first token is
%a letter, use that letter for the computations:
    \def\gobble@till@marker##1\endmarker{}%
    \futurelet\first@char\gobble@till@marker#1\endmarker
    \ifcat\noexpand\first@char A\else
      \def\first@char{}%
    \fi
    \macc@nested@a\relax111{\first@char}%
  \fi
  \endgroup
}
\makeatother
\let\what\widehat
\let\wtld\widetilde
\let\wbar\widebar
\let\ubar\underline

\DeclareMathOperator\sgn{sgn}

\let\oldleft\left
\let\oldright\right
\renewcommand{\left}{\mathopen{}\mathclose\bgroup\oldleft}
\renewcommand{\right}{\aftergroup\egroup\oldright}


\newcommand\highlight[1]{\textcolor{blue}{#1}}

\title{Homework 8}
\author{Naman Mishra (22223)}
\date{8 October, 2024}

\begin{document}
\maketitle

% Problem 1
\begin{problem*}
    Let $X_n$ be independent random variables with $X_n \sim \Ber(p_n)$.
    For $k \ge 1$, find a sequence $(p_n)$ so that almost surely, the
    sequence $X_1, X_2, \dots$ has infinitely many segments of ones of
    length $k$ but only finitely many segments of ones of length $k + 1$.
    By a segment of length $k$ we mean a consecutive sequence
    $X_i, X_{i+1}, \dots, X_{i+k-1}$.
\end{problem*}
\begin{solution}
    Let $p_n = 0$ if $n$ is a multiple of $k + 1$ and $1$ otherwise.
    That is, $(p_1, p_2, \dots, p_k, p_{k+1}) = (1, 1, \dots, 1, 0)$
    and $(p_n)$ is periodic with period $k + 1$.
    Any $k+1$ consecutive indices $i, i+1, \dots, i+k$ must contain
    a multiple of $k+1$, so \[
        \P\set{X_i = X_{i+1} = \dots = X_{i+k} = 1} = 0.
    \] By the union bound, the probability that there is \emph{any} segment
    of ones of length $k+1$ is $0$.

    The event that there are infinitely many segments of ones of length
    $k$ is a subset of the event \[
        A \coloneq \set{X_n = 1 \text{ whenever } (k + 1) \nmid n}.
    \] This has probability $1$, since \[
        A = \bigcap_{(k + 1) \nmid n} \set{X_n = 1}
    \] is the intersection of almost sure events.\footnotemark
\end{solution}
\footnotetext{In case this cannot be assumed, $\P(A) = 1$ since
$A^c$ is a countable union of zero-probability events.}

% Problem 2
\begin{problem*}
    Let $A_1, A_2, \dots$ be a sequence of $M$ dependent events
    (i.e., $A_i$ and $A_j$ are independent iff $\abs{i - j} > M$).
    Prove that $\P(A_n \io) = 0$ or $1$.
\end{problem*}
\begin{solution}
    Let $N = M + 1$ for convenience.
    Consider the $N$ subsequences \begin{gather*}
        A_1, A_{1+N}, A_{1+2N}, \dots \\
        A_2, A_{2+N}, A_{2+2N}, \dots \\
        \vdots \\
        A_N, A_{N+N}, A_{N+2N}, \dots
    \end{gather*}
    Then $A_n \io$ iff infinitely many events from at least
    one of these subsequences occur.
    But $A_k, A_{k+N}, A_{k+2N}, \dots$ are independent for any $k$,
    so by Kolmogorov's zero-one law, $\P(A_{k+nN} \io) = 0$ or $1$
    (where $k$ is fixed and $n$ varies).
    \[
        \set{A_n \io} = \bigcup_{k=1}^N \set{A_{k+nN} \io}.
    \]
    \begin{itemize}
        \item If $\P(A_{k+nN} \io) = 0$ for all $k$, then $\P(A_n \io) = 0$
        by the union bound.
        \item If $\P(A_{k+nN} \io) = 1$ for some $k$,
        then $\P(A_n \io) \ge \P(A_{k+nN} \io) = 1$.
    \end{itemize}
    Thus $\P(A_n \io) = 0$ or $1$.
\end{solution}

% Problem 3
\begin{problem*}
% Let A1, A2, . . . be a sequence of events. Let nk be any sequence and Ck := Snk≤n<nk+1 An. Show that if Pn P(Cn) < ∞ then P(An i.o.) = 0.
    Let $A_1, A_2, \dots$ be a sequence of events.
    Let $(n_k)_k$ be any sequence and
    $C_k \coloneq \bigcup_{n_k \le n < n_{k+1}} A_n$.
    Show that if $\sum_n \P(C_n) < \infty$ then $\P(A_n \io) = 0$.
\end{problem*}
\begin{solution}
    It is assumed that $n_k$ are increasing.
    $\set{A_n \io} = \set{C_n \io}$, so by the first Borel-Cantelli lemma,
    $\P\set{A_n \io} = 0$.
\end{solution}

% Problem 4
\begin{problem*}
% Let A1, A2, . . . be a sequence of events in (Ω, F, P). Let pk be the probability that at least one of the events Ak, Ak+1, . . . occurs.
% (1) If inf pk > 0, then show that An occurs infinitely often, w.p.p (with positive probability). k
% (2) If pk → 0, then show that only finitely many An occur, w.p.1 (with probability 1).
    Let $A_1, A_2, \dots$ be a sequence of events in $(\Omega, \F, \P)$.
    Let $p_k$ be the probability that at least one of the events
    $A_k, A_{k+1}, \dots$ occcurs.
    \begin{enumerate}
        \item If $\inf p_k > 0$, then show that $A_n$ occurs infinitely
        often, w.p.p (with positive probability).
        \item If $p_k \to 0$, then show that only finitely many $A_n$
        occur, w.p.1 (with probability $1$).
    \end{enumerate}
\end{problem*}
\begin{solution} \leavevmode
    Let $B_k$ be the event that no $A_n$ occurs for $n \ge k$,
    and $B$ be the event that only finitely many $A_n$ occur.
    Then $B_1 \subseteq B_2 \subseteq \dots$ and $B = \bigcup_k B_k$.
    Thus $\P(B) = \lim_k \P(B_k) = \lim_k (1 - p_k)$.
    \begin{enumerate}
        \item $\P(B) < 1$.
        \item $\P(B) = 1$. \qedhere
    \end{enumerate}
\end{solution}

\begin{problem*}
% Let An be a sequence of independent events with P(An) < 1 for all n. Show that
% P(∪nAn) = 1 implies Pn P(An) = ∞.
    Let $A_n$ be a sequence of independent events with $\P(A_n) < 1$ for all
    $n$.
    Show that $\P\ab(\bigcup_n A_n) = 1$ implies $\sum_n \P(A_n) = \infty$.
\end{problem*}
\begin{solution}
    Let $B_n = \bigcup_{k \ge n} A_k$.
    We know that $A_1, \dots, A_{n-1}, B_n$ are independent.
    Thus \[
        \P(A_1^c \cap \dots \cap A_{n-1}^c \cap B_n^c)
            = \prod_{k=1}^{n-1} (1 - p_k) (1 - \P(B_n))
    \] If this were positive, then $\P\ab(\bigcup_n A_n) < 1$.
    Thus $\P(B_n) = 1$.
    Then $\set{A_n \io} = \cap_n B_n$ almost surely, so $\P(A_n \io) = 1$.
    By the first Borel-Cantelli lemma, $\sum_n p_n = \infty$.
\end{solution}

\begin{problem*}
% Let Xn and X are random variables and for all δ > 0,Pn P(|Xn − X| > δ) < ∞.
% Show that Xn → X almost surely.
    Let $X_n$ and $X$ be random variables such that for all $\delta > 0$,
    $\sum_n \P(\abs{X_n - X} > \delta) < \infty$.
    Show that $X_n \asto X$.
\end{problem*}
\begin{solution}
    Let $A_n = \set{\abs{X_n - X} > \frac1n \text{ finitely often}}$.
    By the first Borel-Cantelli lemma, $\P(A_n) = 1$.
    Then $\P\ab(\cap_n A_n) = 1$ gives that for each $\frac1n$, there exists
    $N$ such that $\abs{X_n - X} \le \frac1n$ for all $n \ge N$,
    almost surely.
    Thus $X_n \asto X$.
\end{solution}

\begin{problem*}
% Let Xn be any sequence of random variables such that Xn < ∞ almost surely. Show
% that there are constants cn → ∞ such that Xn/cn → 0 almost surely.
    Let $X_n$ be any sequence of random variables such that $X_n < \infty$
    almost surely.
    Show that there are constants $c_n \to \infty$ such that
    $X_n/c_n \to 0$ almost surely.
\end{problem*}
\begin{solution}
    Let $c_n$ be such that $\P\set{nX_n \ge c_n} < \frac1{2^n}$.
    Then $\sum_n \P\set{X_n/c_n \ge \frac1n} < \infty$,
    so $X_n/c_n \asto 0$.
\end{solution}

\begin{problem*}
% Let X1, X2, . . . be i.i.d. Exp(1) random variables and Mn = max1≤m≤n Xm. Show that
% (1) lim supn→∞ Xn/log n = 1 almost surely. (2) Mn/log n → 1 almost surely.
    Let $X_1, X_2, \dots$ be \iid $\Exp(1)$ random variables and
    $M_n = \max_{1 \le m \le n} X_m$.
    Show that
    \begin{enumerate}
        \item $\limsup_{n \to \infty} X_n/\log n = 1$ almost surely.
        \item $M_n/\log n \to 1$ almost surely.
    \end{enumerate}
\end{problem*}
\begin{solution} \leavevmode
    \begin{enumerate}
        \item $\P\set{X_i \ge t} = e^{-t}$.
        Thus $\P\set{X_n \ge k \log n} = \frac1{n^k}$
        is summable iff $k > 1$.
        Thus $X_n/\log n \ge 1$ infinitely often a.s.,
        so $\limsup X_n/\log n \ge 1$ almost surely.
        For any $\eps > 0$, $X_n/\log n \ge 1 + \eps$ only finitely often
        a.s., so $\limsup X_n/\log n \le 1$ almost surely.
        \item $M_n/\log n = \max_{1 \le m \le n} X_m/\log n$.
        If $X_m/\log m \ge 1 + \eps$ only finitely often, then
        $M_n/\log n < 1 + \eps$ for large enough $n$.
        (when $X_k/\log k$ is bounded by $1 + \eps$ and $\log n$ is large
        enough to make $X_m/\log n$ small even when $X_m/\log m$ is large).

        Now $M_n \le 1-\eps$ iff $X_k \le (1 - \eps) \log n$ for
        $k \in [n]$.
        \begin{align*}
            \P\set{M_n \le 1 - \eps}
                &= \prod_{k=1}^n \ab(1-e^{-(1-\eps)\log n}) \\
                &= \ab(1 - \frac1{n^{1-\eps}})^n \\
                &\le e^{-\frac1{n^{1-\eps}} n} \\
                &= e^{-n^\eps}.
        \end{align*}
        Let $\eps = \frac1m$.
        Then $\lim_{n \to \infty} \frac{n^2}{e^{n^\eps}}
        = \lim_{u \to \infty} \frac{u^{2m}}{e^2} = 0$.
        Thus $e^{-n^\eps}$ is summable.
        $M_n \le 1 - \eps$ finitely often a.s. for each $\eps$.
        Intersecting these events gives $M_n/\log n \to 1$ almost surely.
        \qedhere
    \end{enumerate}
\end{solution}

\begin{problem*}
% Let X1, X2, . . . be independent. Show that sup Xn < ∞ almost surely iff Pn P(Xn > A) < ∞ for some A.
    Let $X_1, X_2, \dots$ be independent.
    Show that $\sup X_n < \infty$ almost surely iff
    $\sum_n \P(X_n > A) < \infty$ for some $A$.
\end{problem*}
\begin{solution}
    If $X_n > A$ only finitely often, say only for some $n \le N$,
    then $\sup X_n \le \max\set{X_1, \dots, X_N} \vee A$.

    Conversely, suppose for all $A$, $X_n > A$ infinitely often.
    Then $\sup X_n = \infty$.

    Thus $\sup X_n < \infty$ iff there is some $A$ such that
    $X_n > A$ only finitely often.
    The two Borel-Cantelli lemmas give the result.
\end{solution}

\begin{problem*}
% Let ξ, ξn be i.i.d. random variables with E[log+ ξ] < ∞ and P(ξ = 0) < 1.
% (1) Show that the radius of convergence of the random power series P∞n=0 ξnzn is almost surely
% a constant.
% (2) Show that the radius of convergence is 1 almost surely by showing limsup|ξn|n = 1 a.s.
% n→∞
    Let $\xi, \xi_n$ be \iid random variables with $\E[\log_+ \xi] < \infty$
    and $\P(\xi = 0) < 1$.
    \begin{enumerate}
        \item Show that the radius of convergence of the random power series
        $\sum_{n=0}^\infty \xi_n z^n$ is almost surely a constant.
        \item Show that the radius of convergence is $1$ almost surely by
        showing $\limsup_{n \to \infty} \abs{\xi_n}^{\frac1n} = 1$ almost
        surely.
    \end{enumerate}
\end{problem*}
\begin{solution} \leavevmode
    \begin{enumerate}
        \item Kolmogorov.
        \item 
    \end{enumerate}
\end{solution}

\begin{problem*}
% Consider the Bernoulli bond percolation on Zd as defined in the class. Determine which of the following events are tail events: there exists a unique infinite cluster, 0 belongs to a infinite cluster, there are infinitely many infinite clusters, there are finitely many infinite clusters.
    Consider the Bernoulli bond percolation on $\Z^d$ as defined in class.
    Determine which of the following events are tail events:
        there exists a unique infinite cluster,
        $0$ belongs to a infinite cluster,
        there are infinitely many infinite clusters,
        there are finitely many infinite clusters.
\end{problem*}
\begin{solution} \leavevmode
    \begin{itemize}
        \item Whether there exists a unique infinite cluster is \emph{not}
        a tail event.
        Even in $\Z^1$, assume all edges except possibly $(-1, 0)$ and
        $(0, 1)$ are retained.
        Then there is either $1$ infinite cluster or $2$, depending on
        these $2$ edges.
        \item Whether $0$ belongs to an infinite cluster is \emph{not} a
        tail event.
        Same counterexample as before.
        \item Whether there are infinitely many infinite clusters \emph{is}
        a tail event.
        Finitely many edges can only affect finitely many clusters.
        \item Whether there are finitely many infinite clusters is the
        complement of the previous one, so also a tail event. \qedhere
    \end{itemize}
\end{solution}

\begin{problem*}
% Consider the Bernoulli bond percolation on Zd as defined in the class and consider the critical probability pc(Zd). Show that pc(Zd) > 0 for all d ≥ 1.
    Consider the Bernoulli bond percolation on $\Z^d$ as defined in class
    and consider the critical probability $p_c(\Z^d)$.
    Show that $p_c(\Z^d) > 0$ for all $d \ge 1$.
\end{problem*}
\begin{solution}
    We need to show that there is some $p > 0$ such that $G_p$ doesn't
    have an infinite cluster almost surely.

    % Fix a point $x \in \Z^d$ and let $B_n$ be the set of points at an
    % $L^\infty$ distance of $n$ from $x$.
    % Let $p_n$ be the probability that $x$ is connected to some point in
    % $B_n$.
    % Then \[
    %     p_n \le p_{n-1} \ab(1-(1-p)^{4k_n}), \quad p_0 = 1,
    % \] where $k_n = (2n+1)^{d-1}$.
    % Thus $p_n \le 4(2n+1)^{d-1} p p_{n-1}$,
    For $x \in \Z^d$, let $C_x$ be the event that $0$ is connected to $x$
    via a path contained entirely in the region
    $\sset{y \in \Z^d}{\norm y_1 \le \norm x_1}$.
    Let $p_n = \max_{\norm x_1 = n} \P(C_x)$.
    Then $p_n \le d p_{n-1} p$, so that $p_n \le d^n p^n$.
    If $A_n = \bigcup_{\norm x_1 = n} C_x$, then
    $\P(A_n) \le K n^d d^n p^n$ for some constant $K$.
    By the ratio test, this is summable for $p < \frac1d$.
    The Borel-Cantelli lemma gives $\P(A_n \io) = 0$.

    But for there to be an infinite cluster containing $0$,
    $0$ must be connected to some point in each $\set{\norm x_1 = n}$.
    For each of these sets, at least one point will be reachable from $0$
    without passing through any point with a larger Manhattan norm.
    Thus $\P(A_n \io) = 0$ gives that
    $\P(0 \text{ is in an infinite cluster}) = 0$.
    By the union bound, $\P(\text{there is an infinite cluster}) = 0$.
\end{solution}
\begin{figure}
    \centering
    \begin{tikzpicture}
        \draw[gray, thin] (-4.9, -4.9) grid (4.9, 4.9);
        \foreach \x in {-4, -3, ..., 4} {
        \foreach \y in {-4, -3, ..., 4} {
            \pgfmathsetmacro{\col}{ifthenelse(abs(\x)+abs(\y)==2,"red",
                ifthenelse(abs(\x)+abs(\y)==3,"blue",
                ifthenelse(abs(\x)+abs(\y)==4,"Green","none"))}
            \ifx\col{none}
            \else
                \node[circle,inner sep=1.2pt,fill=\col] at (\x,\y) {};
            \fi
        }}
        \draw[->, thick] (1, 2.4) -- (1, 2.8);
        \draw[->, thick] (.4, 3) -- (.8, 3);
    \end{tikzpicture}
    \caption{The green region must be reached first via the blue,
    from one of the $d$ possible directions.}
\end{figure}

\end{document}
