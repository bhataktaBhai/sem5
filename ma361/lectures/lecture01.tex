\chapter*{The course} \label{chp:course}
\lecture{2024-08-01}{Discrete probability and $\sigma$-algebras}

\section*{Grading} \label{sec:grading}
\begin{itemize}
    \item Homework: 20\%
    \item Two midterms: 15\% each
    \item Final: 50\%
\end{itemize}

\chapter{Review of discrete probability} \label{chp:review}

\begin{definition}[Discrete probability space] \label{def:discrete}
    A discrete probability space is a pair $(\Omega, p)$ where
    $\Omega$ is a finite or countable set called \emph{sample space} and
    $p: \Omega \to [0, 1]$ is a function giving the \emph{elementary probabilities}
    of each $\omega \in \Omega$ such that \[
        \sum_{\omega \in \Omega} p(\omega) = 1.
    \]
\end{definition}
\begin{examples}
    \item ``Toss a fair $n$ times'' is modeled as \[
        \Omega = \set{0, 1}^n
    \] with \[
        p(\omega) \equiv \frac{1}{2^n}.
    \]
    \item ``Throw $r$ balls randomly into $m$ bins'' is modeled as \[
        \Omega = [m]^r
    \] with $p$ given by the multinomial distribution (assuming
    uniformity).
    \item ``A box has $N$ coupons, draw one of them.''
    \begin{align*}
        \Omega &= [N] \\
        p &= \omega \mapsto \frac{1}{N}.
    \end{align*}
    \item ``Toss a fair coin countably many times.''
    The set of outcomes is clear: $\Omega = \set{0, 1}^\N$.
    What about the elementary probabilities?

    Probabilities of some events are also fairly intuitive.
    For example, the event \[
        A = \set{\ubar{\omega} \in \Omega
                \mid \omega_1 = 1, \omega_2 = 1, \omega_3 = 0}
    \] has probability $1/8$.
    Similarly $B = \set{\ubar{\omega} \in \Omega \mid
                    \omega_1 = 1, \omega_2 = 0}$ has probability $1/4$.
    Where does this come from?

    What about this event: \[
        C = \set{\ubar{\omega} \in \Omega
                \Bigm\vert \frac1n \sum_{i=1}^n \omega_i \to 0.6}
    \]
    What about: \[
        D = \set{\ubar{\omega} \in \Omega
                \Bigm\vert \sum_{i=1}^n \omega_i = \frac{n}{2}
                \text{ for infinitely many } n}\footnotemark
    \] \footnotetext{$\P(C) = 0$ and $\P(D) = 1$.}
    \item ``Draw a number uniformly at random from $[0, 1]$.''
    $\Omega$ is obviously $[0, 1]$.
    Again some events have obvious probabilities.
    \[
        A = [0.1, 0.3] \implies \P(A) = 0.2
    \] Similarly \[
        B = [0.1, 0.2] \cup (0.7, 1) \implies \P(B) = 0.4
    \] What about $C = \Q \cap [0, 1]$?
    What about $D$, the $\frac13$-Cantor set?

    The $\frac13$-Cantor set is given by the limit of the following
    sequence of sets. \begin{align*}
        K_0 &= [0, 1] \\
        K_1 &= [0, 1/3] \cup [2/3, 1] \\
        K_2 &= [0, 1/9] \cup [2/9, 1/3] \cup [2/3, 7/9] \cup [8/9, 1] \\
        &\vdotswithin{=}
    \end{align*}
    where each $K_{n+1}$ is obtained by removing the middle third of each
    interval in $K_n$.\footnote{$\P(C) = \P(D) = 0$.}
\end{examples}

The resolution for the above examples is achieved by taking the `obvious'
cases as definitions.\\[2ex]
\begin{minipage}{0.4\textwidth}
    \textbf{What we agree on:}
    \begin{enumerate}[($*$)]
        \item $\P([a, b]) = b - a$ for all $0 \le a \le b \le 1$.
            \label{def:discrete:interval}
    \end{enumerate}
\end{minipage}%
\qquad%
\begin{minipage}{0.5\textwidth}
    \textbf{What we wish for:}
    \begin{enumerate}[({\small\#}1)]
        \item If $A \cap B = \O$, then
            $\P(A \cup B) = \P(A) + \P(B)$.
            \makeatletter \def\@currentlabel{($*$)} \makeatother
            \label{def:discrete:disjoint}
        \item If $A_n \downarrow A$, then
            $\P(A_n) \downarrow \P(A)$.
            \label{def:discrete:decreasing}
    \end{enumerate}
\end{minipage}

\vspace{1em}%
\textbf{Question:} Does there exist a $\P\colon 2^{[0, 1]} \to [0, 1]$
that satisfies \labelcref{def:discrete:interval},
\labelcref{def:discrete:disjoint} and \labelcref{def:discrete:decreasing}?
\textbf{No.}

\textbf{Question:} Does there exist a $\P\colon 2^{[0, 1]} \to [0, 1]$
that satisfies \labelcref{def:discrete:interval},
\labelcref{def:discrete:disjoint} and even \emph{translational invariance}?
\textbf{Yes!} \\
However, it is not unique.

What about the same for a probability measure on $[0, 1]^2$ that is
translation and rotation invariant? \\
What about $[0, 1]^3$?%
\footnote{The Banach-Tarski paradox gives a ``no'' for the 3D case.}

Lack of uniqueness is a disturbing issue.
The way out is the following: restrict the class of sets on which $\P$ is
defined to a $\sigma$-algebra.

\chapter{Measure-theoretic probability} \label{chp:measure}

\section{$\sigma$-algebras} \label{sec:sigma}
\begin{definition*}[$\sigma$-algebra] \label{def:sigma}
    Given a set $\Omega$, a collection $\F \subseteq 2^\Omega$ is called a
    $\sigma$-algebra if
    \begin{enumerate}[\small($\varsigma$1)]
        \item $\O \in \F$. \label{def:sigma:o}
        \item $A \in \F \implies A^c \in \F$. \label{def:sigma:c}
        \item If $A_1, A_2, \ldots \in \F$, then
            $\bigcup_{n=1}^\infty A_n \in \F$. \label{def:sigma:u}
    \end{enumerate}
\end{definition*}

This gives us a modified question. \\
\textbf{Question:} Does there exist \emph{any} $\sigma$-algebra $\F$ on
$[0, 1]$ and a function $\P\colon \F \to [0, 1]$ that satisfies
(\#), ($*$) and ($**$)?%

\textbf{Answer:} Yes, and it is sort-of unique.

\begin{exercise}
    Prove that ($**$) is equivalent to the following:
    if $(B_n)_\N$ are pairwise disjoint, then \[
        \P\ab(\bigcup B_n) = \sum \P(B_n).
    \]
\end{exercise}
\begin{solution}
    % TODO: don't we need some Pr(A u B) = Pr(A) + Pr(B) here?
\end{solution}

A $\sigma$-algebra that works for our case is the \emph{smallest}
one that contains all intervals.

\begin{exercise}
    If $\set{\F_i}_{i \in I}$ are $\sigma$-algebras on $\Omega$, then
    $\bigcap_{i \in I} \F_i$ is also a $\sigma$-algebra.
\end{exercise}
\begin{proof}
    $\O$ is in each $\F_i$ and hence in the intersection.
    If $A$ is in each $\F_i$, then so is $A^c$.
    If $A_1, A_2, \ldots$ are in each $\F_i$, then so is
    $\bigcup_{n=1}^\infty A_n$.
\end{proof}

This allows us to make sense of the word `smallest' above.
\begin{definition} \label{def:sigma:gen}
    Let $\mcS \subseteq 2^\Omega$.
    The \emph{smallest} $\sigma$-algebra containing $\mcS$ is given by the
    intersection of all $\sigma$-algebras on $\Omega$ that contain $\mcS$.
    We denote this by $\sigma(\mcS)$.
\end{definition}
This will contain $\mcS$ since $2^\Omega$ itself is a $\sigma$-algebra.

\begin{example}[Borel $\sigma$-algebra]
    The \emph{Borel $\sigma$-algebra} on $[0, 1]$ is the smallest
    $\sigma$-algebra containing all intervals in $[0, 1]$.
    It is denoted by $\mcB_{[0, 1]}$.
\end{example}
