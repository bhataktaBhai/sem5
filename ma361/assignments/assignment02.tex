\documentclass[12pt]{article}
\input{~/IISc/sem5/preamble}

% \makeatletter
% \newmathcommand{\l}{\@ifstar\@l\@@l}
% \makeatother

\DeclareMathOperator\id{id}
\DeclareMathOperator\GL{GL}
\DeclareMathOperator\M{M}
\DeclareMathOperator\Tr{Tr}
\DeclareMathOperator\adj{adj}
\newcommand\T{\top}
\DeclareMathOperator\argmax{argmax}

\makeatletter
\newcommand\@hsn[1]{{\norm{#1}}_{HS}}
\newcommand\@@hsn[1]{{\norm*{#1}}_{HS}}
\newcommand\hsn{\@ifstar\@@hsn\@hsn}
\makeatother

% \pdv
\usepackage{derivative}
\newcommand*\grad{\nabla}

% https://tex.stackexchange.com/a/235120
\makeatletter
\newcommand*\dotp{\mathpalette\bigcdot@{.5}}
\newcommand*\bigcdot@[2]{\mathbin{\vcenter{\hbox{\scalebox{#2}{$\m@th#1\bullet$}}}}}
\makeatother

\newcommand*\subopeneq{\subseteq_{\mathrm{op}}}


% \makeatletter
% \newmathcommand{\l}{\@ifstar\@l\@@l}
% \makeatother

\DeclareMathOperator\id{id}
\DeclareMathOperator\GL{GL}
\DeclareMathOperator\M{M}
\DeclareMathOperator\Tr{Tr}
\DeclareMathOperator\adj{adj}
\newcommand\T{\top}
\DeclareMathOperator\argmax{argmax}

\makeatletter
\newcommand\@hsn[1]{{\norm{#1}}_{HS}}
\newcommand\@@hsn[1]{{\norm*{#1}}_{HS}}
\newcommand\hsn{\@ifstar\@@hsn\@hsn}
\makeatother

% \pdv
\usepackage{derivative}
\newcommand*\grad{\nabla}

% https://tex.stackexchange.com/a/235120
\makeatletter
\newcommand*\dotp{\mathpalette\bigcdot@{.5}}
\newcommand*\bigcdot@[2]{\mathbin{\vcenter{\hbox{\scalebox{#2}{$\m@th#1\bullet$}}}}}
\makeatother

\newcommand*\subopeneq{\subseteq_{\mathrm{op}}}


% \makeatletter
% \newmathcommand{\l}{\@ifstar\@l\@@l}
% \makeatother

\DeclareMathOperator\id{id}
\DeclareMathOperator\GL{GL}
\DeclareMathOperator\M{M}
\DeclareMathOperator\Tr{Tr}
\DeclareMathOperator\adj{adj}
\newcommand\T{\top}
\DeclareMathOperator\argmax{argmax}

\makeatletter
\newcommand\@hsn[1]{{\norm{#1}}_{HS}}
\newcommand\@@hsn[1]{{\norm*{#1}}_{HS}}
\newcommand\hsn{\@ifstar\@@hsn\@hsn}
\makeatother

% \pdv
\usepackage{derivative}
\newcommand*\grad{\nabla}

% https://tex.stackexchange.com/a/235120
\makeatletter
\newcommand*\dotp{\mathpalette\bigcdot@{.5}}
\newcommand*\bigcdot@[2]{\mathbin{\vcenter{\hbox{\scalebox{#2}{$\m@th#1\bullet$}}}}}
\makeatother

\newcommand*\subopeneq{\subseteq_{\mathrm{op}}}

\setenumerate{label=(\arabic*)}

\title{Homework 2}
\author{Naman Mishra (22223)}
\date{20 August, 2024}

\begin{document}
\maketitle

% Problem 1
\begin{problem} \leavevmode
    \begin{enumerate}
        \item Let $\Omega$ be a set and $A \subseteq \Omega$.
        Define a function $\1A\colon \Omega \to \R$ as follows. \[
            \1A(\omega) = \begin{cases}
                1 & \text{if } \omega \in A \\
                0 & \text{if } \omega \notin A.
            \end{cases}
        \] What is the smallest $\sigma$-algebra on $\Omega$ with respect to
        which $\1A$ becomes a random variable?
        \item Assume that $A \in \F$.
        Give an explicit description of the push-forward measure
        $P \circ (\1A)^{-1}$ on \R.
    \end{enumerate}
\end{problem}
\begin{solution} \leavevmode
    \begin{enumerate}
        \item We need $\1A^{-1}(B) \in \F$ for $B \in \mcB(\R)$. \[
                (\1A)^{-1}(B) = \begin{cases}
                    \O & \text{if } 0, 1 \notin B, \\
                    A & \text{if } 1 \in B, 0 \notin B, \\
                    A^c & \text{if } 0 \in B, 1 \notin B, \\
                    \Omega & \text{if } 0, 1 \in B.
                \end{cases}
            \]
            Thus $\F$ must contain $\O, A, A^c, \Omega$.
            $\F = \set{\O, A, A^c, \Omega}$ is itself a $\sigma$-algebra,
            hence the smallest one that works.
        \item Let $B \in \mcB(\R)$. Then \[
        (P \circ (\1A)^{-1})(B) = \begin{cases}
                0 & \text{if } 0, 1 \notin B, \\
                P(A) & \text{if } 1 \in B, 0 \notin B, \\
                P(A^c) & \text{if } 0 \in B, 1 \notin B, \\
                1 & \text{if } 0, 1 \in B.
            \end{cases} \qedhere
        \]
    \end{enumerate}
\end{solution}

% Problem 2
\begin{problem}
    Recall the L\'evy metric $d$ defined in class.
    Show the following.
    \begin{enumerate}
        \item Let $a_n$ be a sequence of real numbers converging to $a$.
        For any $x \in \R$, $\delta_x$ is the measure define as follows:
        for $A \subseteq \R$, \[
            \delta_x(A) = \begin{cases}
                1 & \text{if } x \in A, \\
                0 & \text{if } x \notin A.
            \end{cases}
        \] Using the definition of the metric show that \[
            d(\delta_{a_n}, \delta_a) \to 0 \text{ as } n \to \infty.
        \]
        \item Consider the sequence of measures
        $\mu_n \coloneq \frac1n \sum_{i=1}^n \delta_{i/n}$
        and $\mu$ is the uniform measure on $[0, 1]$.
        Using the definition show that
        \[
            d(\mu_n, \mu) \to 0 \text{ as } n \to \infty.
        \]
    \end{enumerate}
\end{problem}
\begin{solution} \leavevmode
    \begin{enumerate}
        \item The CDF of $\delta_x$ is $F_x(\omega) = [\omega \ge x]$.
        If $\abs{x - y} = \eps$, then \[
            F_x(\omega + \eps) = [\omega + \eps \ge x]
            = [\omega \ge x - \eps]
            \ge [\omega \ge y]
            = F_y(\omega)
        \] since $\omega \ge y \implies \omega \ge x - \abs{x - y}$.
        Thus $d(\delta_x, \delta_y) \le \abs{x - y}$.
        As $a_n \to a$, $d(\delta_{a_n}, \delta_a) \to 0$.
        \item The CDF of $\mu$ is $F(x) = x$ for $x \in [0, 1]$.
        The CDF of $\mu_n$ is \[
            F_n(x) = \frac{\floor{nx}}n \quad \text{for } x \in [0, 1].
        \] ($\floor{nx}$ counts the number of points $i/n \le x$, and
        each of those has weight $1/n$.)
        We claim that $d(\mu_n, \mu) \le 1/n$.

        Let $x \in [0, 1]$. Then \begin{align*}
            F\ab(x + \frac1n) + \frac1n &= x + \frac{2}{n} \\
            &= \frac{nx + 2}{n} \\
            &> \frac{\floor{nx}}n = F_n(x).
            \shortintertext{and}
            F_n\ab(x + \frac1n) + \frac1n &= \frac{\floor{n(x + 1/n)} + 1}n \\
            &= \frac{\floor{nx} + 2}n \\
            &> \frac{nx}{n} \\
            &= x = F(x).
        \end{align*}
        Thus \begin{multline*}
            \qquad\frac1n \in \set{\eps > 0 : F_n(x + \eps) + \eps \ge F(x) \text{ and} \\
            F(x + \eps) + \eps \ge F_n(x) \text{ for all } x \in [0, 1]}\qquad
        \end{multline*} and so $d(\mu_n, \mu)$, which is the infimum of all
        such $\eps$, is at most $1/n$.
        It follows that $\lim\limits_{n \to \infty} d(\mu_n, \mu) = 0$
        by the squeeze theorem. \qedhere
    \end{enumerate}
\end{solution}

% Problem 3
\begin{problem}
    For $k \ge 0$, define the functions $r_k\colon [0, 1) \to \R$ by
    writing $[0, 1) = \bigsqcup\limits_{0 \le j < 2^k} I_j^{(k)}$ where
    $I_j^{(k)}$ is the dyadic interval
    $[j2^{-k}, (j + 1)2^{-k})$ and setting \[
        r_k(x) = \begin{cases}
            -1 & \text{if } x \in I_j^{(k)} \text{ for odd } j, \\
            1 & \text{if } x \in I_j^{(k)} \text{ for even } j.
        \end{cases}
    \]
    Fix $n \ge 1$ and define $T_n\colon [0, 1) \to \set{-1, 1}^n$ by
    $T_n(x) = (r_0(x), \dots, r_{n-1}(x))$.
    Find the push-forward of the Lebesgue measure on $[0, 1)$ under $T_n$.
\end{problem}
\begin{solution}
    
\end{solution}

\end{document}
