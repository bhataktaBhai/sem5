\documentclass[12pt]{article}
\input{~/IISc/sem5/preamble}

% \makeatletter
% \newmathcommand{\l}{\@ifstar\@l\@@l}
% \makeatother

\DeclareMathOperator\id{id}
\DeclareMathOperator\GL{GL}
\DeclareMathOperator\M{M}
\DeclareMathOperator\Tr{Tr}
\DeclareMathOperator\adj{adj}
\newcommand\T{\top}
\DeclareMathOperator\argmax{argmax}

\makeatletter
\newcommand\@hsn[1]{{\norm{#1}}_{HS}}
\newcommand\@@hsn[1]{{\norm*{#1}}_{HS}}
\newcommand\hsn{\@ifstar\@@hsn\@hsn}
\makeatother

% \pdv
\usepackage{derivative}
\newcommand*\grad{\nabla}

% https://tex.stackexchange.com/a/235120
\makeatletter
\newcommand*\dotp{\mathpalette\bigcdot@{.5}}
\newcommand*\bigcdot@[2]{\mathbin{\vcenter{\hbox{\scalebox{#2}{$\m@th#1\bullet$}}}}}
\makeatother

\newcommand*\subopeneq{\subseteq_{\mathrm{op}}}


% \makeatletter
% \newmathcommand{\l}{\@ifstar\@l\@@l}
% \makeatother

\DeclareMathOperator\id{id}
\DeclareMathOperator\GL{GL}
\DeclareMathOperator\M{M}
\DeclareMathOperator\Tr{Tr}
\DeclareMathOperator\adj{adj}
\newcommand\T{\top}
\DeclareMathOperator\argmax{argmax}

\makeatletter
\newcommand\@hsn[1]{{\norm{#1}}_{HS}}
\newcommand\@@hsn[1]{{\norm*{#1}}_{HS}}
\newcommand\hsn{\@ifstar\@@hsn\@hsn}
\makeatother

% \pdv
\usepackage{derivative}
\newcommand*\grad{\nabla}

% https://tex.stackexchange.com/a/235120
\makeatletter
\newcommand*\dotp{\mathpalette\bigcdot@{.5}}
\newcommand*\bigcdot@[2]{\mathbin{\vcenter{\hbox{\scalebox{#2}{$\m@th#1\bullet$}}}}}
\makeatother

\newcommand*\subopeneq{\subseteq_{\mathrm{op}}}


% \makeatletter
% \newmathcommand{\l}{\@ifstar\@l\@@l}
% \makeatother

\DeclareMathOperator\id{id}
\DeclareMathOperator\GL{GL}
\DeclareMathOperator\M{M}
\DeclareMathOperator\Tr{Tr}
\DeclareMathOperator\adj{adj}
\newcommand\T{\top}
\DeclareMathOperator\argmax{argmax}

\makeatletter
\newcommand\@hsn[1]{{\norm{#1}}_{HS}}
\newcommand\@@hsn[1]{{\norm*{#1}}_{HS}}
\newcommand\hsn{\@ifstar\@@hsn\@hsn}
\makeatother

% \pdv
\usepackage{derivative}
\newcommand*\grad{\nabla}

% https://tex.stackexchange.com/a/235120
\makeatletter
\newcommand*\dotp{\mathpalette\bigcdot@{.5}}
\newcommand*\bigcdot@[2]{\mathbin{\vcenter{\hbox{\scalebox{#2}{$\m@th#1\bullet$}}}}}
\makeatother

\newcommand*\subopeneq{\subseteq_{\mathrm{op}}}


\title{Homework 10}
\author{Naman Mishra (22223)}
\date{24 October, 2024}

\begin{document}
\maketitle

% Problem 1
\begin{problem*}
% Problem 1. Let X1,X2,... be i.i.d from μ. For each n, define the random probability measure μn = n1 (δX1 + . . . + δXn ). If Fn, F are the cumulative distribution functions of μn and μ, show that
% a.s.
% for any x ∈ R, we have Fn(x) → F(x).
    Let $X_1, X_2, \dots$ be i.i.d from $\mu$.
    For each $n$, define the random probability measure
    $\mu_n = \frac1n (\delta_{X_1} + \dots + \delta_{X_n})$.
    If $F_n, F$ are the cumulative distribution functions of $\mu_n$ and
    $\mu$, show that for any $x \in \R$, we have $F_n(x) \asto F(x)$.
\end{problem*}
\begin{solution}
    Fix $x \in \R$.
    Then $F_n(x) = \frac1n \sum_{i=1}^n \1{\set{X_i \le x}}$.
    $\P\set{X_i \le x} = F(x)$ and $X_i$'s are i.i.d, so
    $\1{\set{X_i \le x}}$ are i.i.d $\Ber(F(x))$ random variables.
    By the strong law,
    $F_n(x) \asto \E[\1{\set{X_1 \le x}}] = F(x)$.
\end{solution}

% Problem 2
\begin{problem*}
% Let Xn be a sequence of random variables with zero means, unit variances. Assume that|Cov(Xn,Xm)|≤δ(|n−m|)whereδ(k)→0ask→∞. Showthat n1Sn →P 0.
    Let $X_n$ be a sequence of random variables with zero means, unit
    variances.
    Assume that $\abs{\Cov(X_n, X_m)} \le \delta(\abs{n-m})$ where
    $\delta(k) \to 0$ as $k \to \infty$.
    Show that $\frac1n S_n \pto 0$.
\end{problem*}
\begin{solution}
    Compute the variance of $S_n$.
    \begin{align*}
        \Var(S_n)
            &= \sum_{i=1}^n \Var(X_i) + \sum_{i \ne j} \Cov(X_i, X_j) \\
            &\le n + 2 [(n-1) \delta(1) + (n-2) \delta(2) +
                                \dots + \delta (n-1)] \\
        \implies \Var\ab(\frac1n S_n)
            &\le \frac1n + \frac2n \sum_{k=1}^{n-1} \delta(k).
    \end{align*}
    \begin{claim}
        Let $T_n = \sum_{k=1}^n \delta(k)$.
        Then $\frac1n T_n \to 0$ as $n \to \infty$.
    \end{claim}
    \begin{subproof}[Proof of Claim]
        Let $\eps > 0$, and choose $N$ such that
        $\delta(n) < \frac{\eps}{2}$ for all $n \ge N$.
        Then for $n \ge N$, \[
            \frac1n T_n = \frac1n T_N + \frac1n \sum_{k=N+1}^n \delta(k)
            \le \frac1n T_N + \frac{\eps}{2}.
        \]
        For large enough $n$, we have $\frac1n T_N < \frac{\eps}{2}$,
        so $\frac1n T_n < \eps$.
    \end{subproof}
    Thus $\Var(\frac1n S_n) \to 0$ as $n \to \infty$.
    By Chebyshev's inequality, we have $\frac1n S_n \pto 0$.
\end{solution}

\end{document}
