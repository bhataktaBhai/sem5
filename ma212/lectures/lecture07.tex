\lecture[7]{2024-08-19}{Group actions and cosets}

\section{Actions and cosets} \label{sec:action}

\begin{definition*}[action] \label{def:action}
    We say that a group $(G, *)$ \emph{acts} on a set $X$ if there is a map
    $\cdot\colon G \times X \to X$ such that the following hold for every
    $g_1, g_2 \in G$ and $x \in X$:
    \begin{enumerate}
        \item $g_1 \cdot (g_2 \cdot x) = (g_1 * g_2) \cdot x$;
        \item $1_G \cdot x = x$.
    \end{enumerate}
    We say that $x \in X$ is \emph{moved} by $g \in G$ to $g \cdot x$.
\end{definition*}
\begin{examples}
    \item The trivial action $(g, x) \mapsto x$.
    \item The interesting action of $\GL_n(F)$ on $F^n$ (columns)
        by matrix-vector multiplication.
    \item The left regular action of $G$ on itself.
        $(g, x) \mapsto gx$
    \item The conjugating action of $G$ on itself.
        $(g, x) \mapsto gxg^{-1}$
\end{examples}
\begin{remark}
    Given a right action of $G$ on $X$, we can define a left action as \[
        g \cdot x = x \cdot g^{-1}.
    \] It thus suffices to study left actions.
\end{remark}

\begin{proposition}
    The set of actions of $G$ on $X$ is isomorphic to $\hom(G, S_X)$.
    The associated map is \[
        \cdot \mapsto (g \mapsto (x \mapsto g \cdot x)).
    \]
\end{proposition}
\begin{proof}
    For an action $\cdot$, define for each $g \in G$ the map
    $\sigma_g\colon X \to X$ by $\sigma_g(x) = g \cdot x$.

    Notice that $\sigma_g$ is a bijection, since \[
        (\sigma_{g^{-1}} \circ \sigma_g)(x) = g^{-1} \cdot (g \cdot x) = x.
    \] Thus $g \mapsto \sigma_g$ is a homomorphism from $G$ to $S_X$.

    $\hom(G, S_X)$ is a group under composition.

    For a homomorphism $\sigma\colon G \to S_X$, define the group action \[
        g \cdot x = \sigma_g(x). \qedhere
    \]
\end{proof}
\begin{remark}
    This is essentially a juggling between the set of functions
    $(A, B) \to C$ and the set of curried functions $A \to B \to C$.
\end{remark}

\begin{definition*}[orbits and stabilizers] \label{def:orbit}
    Let $G$ act on $X$.
    Then for $x \in X$, we define the \emph{orbit} of $x$ as \[
        \mcO_x = G x = \sset{g x}{g \in G} \subseteq X,
    \] and the \emph{stabilizer} of $x$ as \[
        \stab_x = G_x = \sset{g \in G}{g \cdot x = x}.
    \] We say that the action is \emph{faithful} if \[
        \bigcap_{x \in X} G_x = \set 1.
    \]
\end{definition*}
\begin{exercise}
    $G_x \le G$.
\end{exercise}
\begin{solution}
    $1$ stabilizes $x$, so $G_x \ne \O$.
    If $g, h \in G_x$, then \begin{align*}
        g \cdot h \cdot x &= g \cdot x = x \\
        g^{-1} \cdot x &= g^{-1} \cdot (g \cdot x) = x. \qedhere
    \end{align*}
\end{solution}
\begin{examples}
    \item For the trivial action, $\mcO_x = \set{x}$ and $G_x = G$.
    \item For the action of $\GL_n(\R)$ on $\R^n$, the orbit of $x$ is \[
        \mcO_x = \begin{cases}
            \set{0} & \text{if } x = 0, \\
            \R^n \setminus \set{0} & \text{if } x \ne 0.
        \end{cases}
    \] To see this, \textbf{challenge:} prove by induction.

    The stabilizer of $x$ is all matrices which have $x$ as an eigenvector
    with eigenvalue $1$.
    \item For the left regular action of $G$ on itself,
        $\mcO_x = G$ and $G_x = \set{1}$.
    \item For the conjugating action of $G$ on itself,
        $\mcO_x = \sset{gxg^{-1}}{g \in G}$ and
        $G_x = \sset{g \in G}{gxg^{-1} = x}$.
        There is no simple description of either.
\end{examples}

\begin{definition*}[centralizer and center] \label{def:center}
    For $x \in G$, the \emph{centralizer} of $x$ is \[
        C_G(x) = \sset{g \in G}{gx = xg}.
    \] The \emph{center} of $G$ is \[
        Z_G = \bigcap_{x \in G} C_G(x)
            = \sset{g \in G}{\forall x \in G, gx = xg}.
    \]
\end{definition*}
\begin{example}
    $Z_{\GL_n(\R)} = \sset{\lambda I}{\lambda \in \R^\times}$.
\end{example}

\begin{definition}
    Let $H \le G$.
    Define the relation $\sim$ on $G$ by $x \sim y$ if $x^{-1}y \in H$.
\end{definition}
\begin{lemma}
    $\sim$ is an equivalence relation.
\end{lemma}
\begin{proof}
    Reflexive since $1 \in H$.
    Symmetric since $H$ is closed under inverses, so \[
        x^{-1}y \in H \implies (x^{-1}y)^{-1} = y^{-1}x \in H.
    \] Transitive by closure under products, \[
        x^{-1}y, y^{-1}z \in H \implies x^{-1}y y^{-1}z = x^{-1}z \in H.
        \qedhere
    \]
\end{proof}
Notice that the equivalence class of $x \in G$ is \[
    [x] = \sset{y \in G}{x^{-1}y \in H} = xH.
\]
\begin{definition*}[coset] \label{def:coset}
    Let $H \le G$.
    The \emph{left coset} of $x \in G$ is \[
        xH = \sset{xh}{h \in H}.
    \] The \emph{right coset} of $x \in G$ is \[
        Hx = \sset{hx}{h \in H}.
    \]
\end{definition*}
We will only write left cosets, but the same results hold for right cosets.
\begin{corollary}
    $xH \cap yH \ne \O$ iff $x \sim y$.
\end{corollary}

\begin{definition*}
    The set of (left) cosets of $H$ in $G$ is denoted by $G/H$.
\end{definition*}

\begin{examples}
    \item $G$ acts on the power set $2^G$ by
        $g \cdot A = gA = \sset{ga}{a \in A}$.
    \item Restricting this is useful.
        Let $H \le G$.
        Then $G$ acts on the set of left cosets $G/H$ by
        $g \cdot xH = (gx)H$.
    \item If one wishes to define a left action on the set of \emph{right}
    cosets $G/H$, one can define \[
        g \cdot Hx = Hxg^{-1}.
    \] $g \cdot Hx = Hxg$ will not obey associativity.
    $g \cdot Hx = Hgx$ is not necessarily well-defined.
\end{examples}

\textbf{Question:} When and how can one make $G/H$ naturally a group?

We would want $(xH) (yH) = \sset{xh_1 yh_2}{h_1, h_2 \in H}$ to
be equal to $(xy)H$.
This would require for all $h_1, h_2 \in H$ that $xh_1yh_2 = xyh_3$
for some $h_3 \in H$.

If $hy = yh$ for all $y \in G$ and $h \in H$, then
$xh_1yh_2 = (xy)h_1h_2 \in (xy)H$.

\begin{example}
    Let $\varphi\colon G \to H$ be a group homomorphism.
    Let $K = \ker \varphi = \sset{g \in G}{\varphi(g) = 1}$.
    Then $G/K$ is a group under the operation \[
        (xK) \cdot (yK) = (xy)K.
    \] This is well-defined, since $xK = x'K$ iff
    $\varphi(x) = \varphi(x')$.
    Thus whenever $xK = x'K$ and $yK = y'K$, we have \[
        \varphi(x'y') = \varphi(x')\varphi(y')
            = \varphi(x)\varphi(y)
            = \varphi(xy),
    \]
    so that $(xy)K = (x'y')K$.

    Obviously this defines a group operation on $G/K$.
    Associativity, identity, inverses are borrowed from $G$.

    Note that for any $x \nsim y$, $xK \ne yK$.
    Thus $G = \bigsqcup\limits_{xH \in G/H} xH$.
\end{example}
