\lecture{2024-08-07}{}
If two groups are isomorphic, they are essentially the same group.
An isomorphism $\varphi\colon G \to H$ is only a ``re-parameterization''
of $G$ in terms of $H$.

\begin{lemma}
    $\abs{\angled{x}} = \ord x$.
\end{lemma}
\begin{proof}
    If $\ord x = \infty$, then $x^n \ne x^m$ for $n \ne m$.
    Thus $\abs{\angled{x}} = \infty$.

    If $\ord x = n < \infty$,
    then $x^0, x^1, \ldots, x^{n-1}$ are distinct.
    Let $x^m \in \angled{x}$.
    Write $x^m = x^{qn + r} = x^r$ with $0 \le r < n$.
    Thus these $n$ elements are the only ones in $\angled{x}$.
\end{proof}
\begin{proposition*}
    Let $G$ be a cyclic group.
    Then
    \begin{enumerate}
        \item if $\abs{G} = \infty$, then $G \cong \Z$;
        \item if $\abs{G} = n < \infty$, then $G \cong \Z/n\Z$.
    \end{enumerate}
\end{proposition*}
\begin{proof}
    Let $G = \angled{x}$.
    We want an isomorphism $\varphi\colon G \to \Z/n\Z$,
    where $n \in \N \cup \set{\infty}$.
    It suffices to define $\varphi(x)$ and extend it to all of $G$.

    If $\abs{G} = \infty$, define $\varphi(x) = 1$.
    Then $\varphi(x^n) = n$ for all $n \in \Z$.
    This is a bijection and $\varphi(ab) = \varphi(a) + \varphi(b)$ holds.

    If $\abs{G} = \set{1, x, \dots, x^{n-1}}$,
    define $\varphi(x) = \wbar{1} \in \Z/n\Z$.
    Then $\varphi(x^m) = \wbar{m}$ for all $m \in \Z$.
    It is clearly a surjection.
    The kernel is $\set{x^m \in G : n \mid m} = \set{1}$,
    so it is injective.
    Finally, $\varphi(x^m x^k) = \varphi(x^{m+k})
                      = \wbar{m+k}
                      = \wbar{m} + \wbar{k}$.
\end{proof}

Cyclic groups are generated by a single element.
What about groups generated by multiple elements?

Let $S \subseteq G$.
Define two sets \begin{align*}
  {\angled S}_1 &= \set{
        s_1^{\varepsilon_1} \dots s_k^{\varepsilon_k}
        \mid s_i \in S, \varepsilon_i \in \set{\pm 1}
    } \\
    &= \set{
        s_1^{\alpha_1} \dots s_k^{\alpha_k}
        \mid s_i \in S, \alpha_i \in \Z
    } \\
  {\angled S}_2 &= \bigcap_{S \subseteq H \le G} H.
\end{align*}

\begin{lemma}
    ${\angled S}_1 = {\angled S}_2 \eqcolon \angled S$.
\end{lemma}
\begin{proof}
    ${\angled S}_2 \le G$ since the intersection of subgroups is a subgroup.
    We first check that ${\angled S}_1 \le G$ under multiplication
    (which is essentially concatenation).
    Inverses are given by
    $s_1^{\varepsilon_1} \dots s_k^{\varepsilon_k} \mapsto
    s_k^{-\varepsilon_k} \dots s_1^{-\varepsilon_1}$.

    Moreover, $S \subseteq {\angled S}_1$.
    Thus ${\angled S}_2 \subseteq {\angled S}_1$.

    Since ${\angled S}_2$ is a group containing $S$,
    closure under products and inverses implies
    ${\angled S}_1 \subseteq {\angled S}_2$.
\end{proof}

\begin{examples}
    \item $S_n$ is generated by transpositions.
    \item $\GL_n(\R)$ is generated by the elementary matrices \[
        E_{ij}(\lambda) = I_n + \lambda e_{ij},
    \] where $e_{pq} = (\delta_{ip}\delta_{jq})_{i,j=1}^n$,
    taken together with the diagonal matrices.
    [swapping is done by $(a, b) \mapsto (a, a+b) \mapsto (-a, a+b)
    \mapsto (b, a+b) \mapsto (b, a)$]
    \item $\Q^\times$ is not finitely generated.
    Take any finite set $S \subseteq \Q^\times$ and look at the numerators.
    There are finitely many primes in the numerators of $S$,
    so any prime not in the numerators of $S$ is not in $\angled S$.
    \item $\SL_n(\R) = \set{M \in M_n(\R) \mid \det M = 1}$ is generated by
    \[
        E_{ij}(\lambda) = I_n + \lambda e_{ij}, \quad \text{with $i \ne j$}.
    \] % TODO: check
    \item Let $F$ be any infinite field.
    Then $(F^\times, \cdot)$ is not finitely generated.
    If $\chr F = p$, then $p$ is prime and %TODO

    Suppose $\chr F = 0$.
    Then $F$ contains (an isomorphic copy of) $\Q$.
    For $F^\times$ to be finitely generated,
    $Q^\times$ would have to be finitely generated.
    We will later see that subgroups of finitely generated groups
    are finitely generated. % TODO
    We will also see that $\Q^\times$ is not finitely generated. % TODO
    Thus $F^\times$ is not finitely generated.
