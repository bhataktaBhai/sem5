\chapter{Rings} \label{chp:ring}
\lecture[20]{2024-09-30}{}
\begin{definition*}[ring] \label{def:ring}
    A \emph{ring} is a set $R$ with two binary opeerations $(+, \cdot)$
    such that
    \begin{enumerate}[label=\small(R\arabic*)]
        \item $(R, +)$ is an abelian group;
        \item $\cdot$ is associative;
        \item $\cdot$ distributes over addition: \begin{align*}
            a \cdot (b + c) &= a \cdot b + a \cdot c, \\
            (a + b) \cdot c &= a \cdot c + b \cdot c;
        \end{align*}
        \item there is a multiplicative identity $1 \in R$ such that
        $1 \cdot a = a \cdot 1 = a$ for all $a \in R$,
        with $1 \ne 0$, the additive identity.
    \end{enumerate}
    $R$ is a \emph{commutative ring} if $\cdot$ is commutative.

    If for each $a \ne 0$, there exists a two-sided multiplicative
    inverse $b$ such that $a \cdot b = b \cdot a = 1$, then $R$ is a
    \emph{division ring}.

    A commutative division ring is a \emph{field}.
\end{definition*}

\begin{examples}
    \item $\Z$, $\Z/n\Z$, $\Q$, $\Q[\sqrt 2]$, $\R$, $\C$ are commutative
        rings.
        $\Q$, $\Q[\sqrt 2]$, $\R$, $\C$ among them are fields.
    \item $R[x]$ where $R$ is a ring is a ring.
        $R[x]$ is commutative iff $R$ is commutative.
    \item $M_n(R)$ is a non-commutative ring ($n \ge 2$).
    \item Let $M$ be an abelian group.
        Then $\End(M)$ is a ring with addition and composition.
    \item $2\Z$ is a rng: it satisfies all properties of a ring except
        the existence of a multiplicative identity.
    \item The Hamiltonians over $\R$ are a non-commutative division ring. \[
        H(\R) = \sset{a + bi + cj + dk}{a, b, c, d \in \R}
    \] The inverse of $a + bi + cj + dk$ is \[
        \frac{a - bi - cj - dk}{a^2 + b^2 + c^2 + d^2}.
    \]
    \item $\F_p = \Z/p\Z$ is a field for any prime $p$.
    \item $F(x) = $ \TODO[what]
\end{examples}

What is a polynomial ring?
More specifically, what is the ``$x$'' in $R[x]$?

Why is $\Z[x]$ not isomorphic to $\Z[\sqrt 2]$?
When we write $x$ in $\Z[x]$, we require $x$ to be transcendental over $\Z$.
But in $\Z[\sqrt 2]$, $\sqrt 2$ is a root of $x^2 - 2$.

If $x$ is transcendental over $R$, then \[
    a_0 + a_1 x + \dots + a_n x^n = b_0 + b_1 x + \dots + b_m x^m
\] iff $n = m$ and $a_i = b_i$ for all $i$.

We can construct such an element $x$ as follows:
\begin{align*}
    x^0 &= (1, 0, 0, 0, \dots) \\
    x^1 &= (0, 1, 0, 0, \dots) \\
    x^2 &= (0, 0, 1, 0, \dots) \\
    &\mathrel{\makebox[\widthof{=}]{\vdots}} \\
    x^n &= (\delta_{n, k})_{k \in \N}
\end{align*}
with addition and multiplication defined component-wise.
The product of two polynomials is given by convolution:
\begin{align*}
    (a_0, a_1, a_2, \dots) \cdot (b_0, b_1, b_2, \dots)
    &= (c_0, c_1, c_2, \dots) \\
    c_n &= \sum_{i = 0}^n a_i b_{n - i}.
\end{align*}

We can identify $R[x]$ with the set of all sequences
$(a_0, a_1, a_2, \dots)$ which are eventually zero, i.e.,
$a_n = 0$ for all $n \gg 0$.
Defining product by convolution gives that $x = e_1$ is transcendental over
$R$.

\begin{definition}[integral domain] \label{def:ring:int}
    A ring $R$ is an \emph{integral domain} if it is commutative and has no
    zero divisors.
    A \emph{zero divisor} is any non-zero element $a \in R$ for which there
    exists a non-zero element $b \in R$ with $ab = 0$.
\end{definition}
Otherwordly, a commutative ring $R$ is an integral domain iff
$a, b \ne 0 \implies ab \ne 0$.
\begin{examples}
    \item $\Z/n\Z$ ($n > 1$) is an integral domain iff $n$ is prime.
    \item $\Z$ is an integral domain.
    \item $M_n(\R)$ has a party of zero divisors for $n \ge 2$.
\end{examples}

\begin{definition}[subrings] \label{def:ring:sub}
    Let $R$ be a ring and $S \subseteq R$.
    $S$ is a \emph{subring} of $R$ if $(S, +) \le (R, +)$, $S \ni 1$ and
    $S$ is closed under multiplication.
\end{definition}
\begin{definition}[ideal] \label{def:ring:ideal}
    Let $R$ be a ring and $I \subseteq R$.
    $I$ is a \emph{left (resp. right) ideal} of $R$ if $(I, +) \le (R, +)$
    and $RI = \set{ri \mid r \in R, i \in I} = I$.
    An ideal is \emph{proper} if it is not $R$.
\end{definition}
\begin{examples}
    \item An ideal $I$ is a proper ideal iff it does not contain $1$.
\end{examples}
