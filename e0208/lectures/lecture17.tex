\lecture{2024-10-14}{}
\begin{center}
    \begin{tikzpicture}
        % \coordinate (p) at (3, 0);
        % \coordinate (q) at (0, -2);
        % \coordinate (r) at (-2, 1);
        % \coordinate (s) at (.5, 4);
        \node[draw, circle, inner sep=1pt, black, fill]  at (3, 0) {$p$};
        \node[point] (q) at (0, -2) {$q$};
        \node[point] (r) at (-2, 1) {$r$};
        \node[point] (s) at (.5, 4) {$s$};
    \end{tikzpicture}
\end{center}

\begin{definition}
    Let $T$ be a triangulation of a set of points $P$.
    We say that an edge $e = t_1 \cap t_2$ is \emph{locally Delaunay} if
    $C(t_1)^\circ \cap t_2 \cap P = \O$.
    We say that $T$ is \emph{locally Delaunay} if every edge in $T$
    is locally Delaunay.
\end{definition}

\begin{algo}
    \Fn{Delaunay}{$P$}
        \State $T \gets$ any triangulation of $P$
        \While{some edge $e$ is not locally Delaunay}
            \State flip $e$
        \EndWhile
    \EndFn
\end{algo}

\begin{proposition}[edge characterization] \label{thm:delaunay:edge}
    A triangulation $T$ is a Delaunay triangulation iff for every
    $p, q \in P$, there is some circle $C$ passing through $p$ and $q$ such
    that $C^\circ \cap P = \O$.
\end{proposition}

\begin{definition}[angle vector] \label{def:angle-vector}
    The angle vector of a triangulation $T$ is the vector of all angles of
    each triangle in $T$, listed in increasing order.
    We denote this by $\alpha(T)$.
\end{definition}
\begin{proposition} \label{thm:delaunay:angle-vector}
    Let $T_1 \xrightarrow{\textsc{Flip}} T_2$.
    Then $\alpha(T_1) < \alpha(T_2)$ under the lexicographic order.
\end{proposition}

\begin{corollary} \label{thm:delaunay:term}
    The algorithm \textsc{Delaunay} terminates.
\end{corollary}
\begin{proof}
    There are finitely many triangulations and the angle vector is strictly
    increasing.
\end{proof}

\begin{fact}
    An edge $pq$ once flipped by the \textsc{Delaunay} algorithm will never
    be flipped again.
\end{fact}
This is not easy to prove.

\begin{theorem}
    A locally Delaunay triangulation is a Delaunay triangulation.
\end{theorem}
\begin{proof}
    Let $T$ be a locally Delaunay triangulation that is \emph{not} Delaunay.
    That is, there is some $t \in T$ such that $C(t)^\circ \cap P$ contains
    some point $p$.

    Iteratively construct triangles $t_1, t_2, \ldots$ approaching $p$
    such that $p \in C(t_i)^\circ$ for all $i$.
    Each iteration, visits one new point, so eventually this is $p$.
    This violates local Delaunity.
\end{proof}
Read proof from David Mount.

\begin{corollary}
    The \textsc{Delaunay} algorithm computes the Delaunay triangulation.
\end{corollary}

\begin{corollary}
    The Delaunay triangulation is the unique triangulation which maximizes
    the angle vector.
\end{corollary}
\begin{proof}
    Start the \textsc{Delaunay} algorithm with any tiangulation.
    Then by \cref{thm:delaunay:angle-vector}, the Delaunay triangulation
    has a larger angle vector.
\end{proof}

\subsection{Incremental algorithm} \label{sec:delaunay:incremental}
Maintain a Delaunay triangulation.
For any new point $p_i$ inside the convex hull of the current set of points,
Let $abc$ be the triangle containing $p_i$.
Draw new edges $p_ia$, $p_ib$, $p_ic$.
The local Delaunay property can only be false for the edges $ab$, $bc$ and
$ca$.

\begin{algo}
    \Fn{Delaunay-Incremental}{$P$}
        \State shuffle $P$
        \State $T \gets \O$
        \For{$i \gets 1$ to $n$}
            \State find $\triangle^{abc} \in T$ containing $p_i$
            \State add the edges $p_ia$, $p_ib$, $p_ic$
            \For{$e \in \set{ab, bc, ca}$}
                \State \Call{Legalize-Edge}{$p_i$, $ab$}
            \EndFor
        \EndFor
        \State \Return $T$
    \EndFn
\end{algo}
\begin{algo}
    \Fn{Legalize-Edge}{$p$, $xy$}
        \State $z \gets$ vertex opposite to $p$
        \If{$xy$ isn't locally Delaunay}
            \State flip $xy$ to $pz$
            \State \Call{Legalize-Edge}{$p$, $xz$}
            \State \Call{Legalize-Edge}{$p$, $yz$}
        \EndIf
    \EndFn
\end{algo}
Each flip in \textsc{Legalize-Edge} connects a new point to $p$.
Thus \textsc{Legalize-Edge} runs in $O(n)$ time.
The total running time is thus $O(n^2)$.

\begin{fact}
    The expected degree of any vertex in the Delaunay triangulation is
    constant.
\end{fact}
Thus the expected running time of the incremental algorithm is $O(n \log n)$,
where the $\log n$ is required for searching $\triangle^{abc}$.

To ensure that points are only added inside the convex hull, add far away
dummy points before the incremental algorithm.
\TODO[Show that the edges from the dummy points in the Delaunay
triangulation of this only connect to the convex hull.]
