\chapter*{The course} \label{chp:course}
\lecture{2024-08-07}{The Skyline problem}

\href{https://www.csa.iisc.ac.in/~saladi/e0208/fa2024}{Course webpage}

\textbf{MS Teams:} 9ng333s

\section*{Grading}
Tentatively
\begin{itemize}
    \item[(30\%)] 4 assignments
    \item[(30\%)] 1 midterm
    \item[(40\%)] Final / project
\end{itemize}

\section*{Overview}
Applications to
\begin{itemize}
    \item Robotics
    \item VLSI
    \item Databases (spatial)
    \item Machine learning
\end{itemize}
\begin{examples}
    \item \textbf{Robotics:} Path planning
    \item \textbf{Databases:} Given a set of restaurants with ratings,
        find the top $k$ restaurants within a distance $r$ from a given
        location.
        \begin{itemize}
            \item Processing a single query takes $O(n)$,
            \item and preprocessing leads to space issues.
        \end{itemize}
        It seems we are fed.
    \item \textbf{Nearest-neighbour query:} Given a set of $n$ points and
        a query point $q$, return the point closest to $q$.
        (Voronoi diagram)
\end{examples}
The first half focuses on these data structure based problems.
The second half focuses on geometric optimization queries.
\begin{examples}
    \item \textbf{Travelling salesman} in the Euclidean setting.
    \item \textbf{Set cover:} Given a set of points and a set of disks,
        find the smallest set of disks that covers all points.
    \item \textbf{Convex hull/Skyline points/Arrangement of lines}
\end{examples}

\section*{Conferences} \label{sec:conferences}
\begin{itemize}
    \item \textbf{SoCG:} Symposium on Computational Geometry
    \item \textbf{CCCG:} Canadian Conference on Computational Geometry
\end{itemize}

\chapter{Skyline points} \label{chp:skyline}
% \begin{center}
%     \begin{tikzpicture}
%         \tikzmath{\maxX
%     \end{tikzpicture}
% \end{center}

\begin{definition*}[Domination and skyline] \label{def:skyline}
    A point $p = (p_1, p_2, \dots, p_d)$ \emph{dominates}
    another point $q = (q_1, q_2, \dots, q_d)$ if
    $p_i > q_i$ for all $i \in [d]$.

    A point $p$ in a set $S$ is a \emph{skyline point} if no other point in
    $S$ dominates it.
\end{definition*}

\begin{question*}
    Given a set of points in $\R^2$, find the set of skyline points.
\end{question*}
\begin{solution}
    Sort the points by decreasing $x$-coordinate.
    Iterate through the list keeping track of the maximum $y$-coordinate
    seen so far (sweep left).
    If the current point has a $y$-coordinate less than the maximum,
    it is a skyline point.
    \begin{algo}
        \Fn{Skyline}{$P$}
            \State sort $P$ by decreasing $x$-coordinate
            \State $S \gets \O$
            \State $y_{\max} \gets -\infty$
            \For{$p \in P$}
                \If{$p_2 > y_{\max}$}
                    \State $S \gets S \cup \{p\}$
                    \State $y_{\max} \gets p_2$
                \EndIf
            \EndFor
        \EndFn
    \end{algo} \vspace{-1.3em}
\end{solution}

\begin{question*}
    Given a set of points in $\R^3$, find the set of skyline points.
\end{question*}
\begin{solution}
    Sort $P$ by decreasing $z$-coordinate.
    Whether a point is a skyline point depends only on points before it.
    Sweep through $P$, while maintaining the solution set $S$, and the
    2D skyline for the projection of each point in $S$ onto the $xy$-plane.

    This 2D skyline is to be stored in a (balanced) binary search tree
    according to the $x$-coordinate.
    Each time a new point $p_i$ is seen, we query the tree for the successor
    $a_j$ of $p_i$.
    Then $p_i$ is a skyline point iff $(p_i)_y > (a_j)_y$.

    If $p_i$ is a skyline point, insert it into the tree, and delete all
    points dominated by $p_i$.
    $O((1 + k) \log n)$ time for deleting $k$ points.

    Time complexity is \begin{align*}
        \sum_{i=1}^n O(\log n) + k_i \cdot O(\log n)
            &= O\ab(\sum_{i=1}^n \log n) + \ab(\sum_{i=1}^n k_i) O(\log n)\\
            &= O(n \log n + n \log n) = O(n \log n)
    \end{align*}
    \begin{algo}
        \Fn{3DSkyline}{$P$}
            \State $P \gets P \cup \set{(-\infty, +\infty, -\infty),
                            (+\infty, -\infty, -\infty)}$
            \State sort $P$ by decreasing $z$-coordinate
            \State $S \gets \O$, $T \gets \Callp{create}{P}$
            \For{$p \in P$}
                \If{$\Callp{succ}{T, p}_y < p_y$}
                    \State $S \gets S \cup \{p\}$
                    \State $\Callp{add}{T, p}$
                \EndIf
                \While{$(q \coloneq \Callp{pred}{T, p})_y < p_y$}
                    \State $\Callp{remove}{T, q}$
                \EndWhile
            \EndFor
            \State \Return $S$
        \EndFn
    \end{algo}
    % \begin{minipage}{0.32\textwidth}
    %     \begin{algo}[1]
    %         \Fn{create}{$P$}
    %             \If{$P = \set p$}
    %                 \State \Return $(p, -\infty)$
    %             \EndIf
    %             \State $P_\ell, P_r \gets \Call{split}{P}$
    %             \State $T_\ell \gets \Call{create}{P_\ell}$
    %             \State $T_r \gets \Call{create}{P_r}$
    %             \State \Return $(T_\ell, T_r, -\infty)$
    %         \EndFn
    %     \end{algo}
    % \end{minipage} \begin{minipage}{0.33\textwidth}
    %     \begin{algo}[1]
    %         \Fn{add}{$T$, $p$}
    %             \State $T.\Varr{max} \gets T.\Varr{max} \vee p_x$
    %             \If{$T$ is a leaf}
    %                 \State \Return
    %             \EndIf
    %             \If{$p_x < T.k$}
    %                 \State $\Call{add}{T_\ell, p}$
    %             \Else
    %                 \State $\Call{add}{T_r, p}$
    %             \EndIf
    %         \EndFn
    %     \end{algo}
    % \end{minipage} \begin{minipage}{0.34\textwidth}
    %     \begin{algo}[1]
    %         \Fn{succ}{$T$, $p$}
    %             \If{$T$ is a leaf}
    %                 \State \Return $T$
    %             \EndIf
    %             \If{$p_x < T_\ell.\Varr{max}$}
    %                 \State \Return $\Call{succ}{T_\ell, p}$
    %             \ElsIf{$p_x < T_r.\Varr{max}$}
    %                 \State \Return $\Call{succ}{T_r, p}$
    %             \Else
    %                 \State \Return $\bot$
    %             \EndIf
    %         \EndFn
    %     \end{algo}
    % \end{minipage}
\end{solution}
