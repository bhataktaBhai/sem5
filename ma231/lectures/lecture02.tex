\lecture{2024-08-05}{}
We saw that $\abs{\R} = \abs{\R^2}$, even though both behave very
differently in a lot of ways.
We need a notion of ``proximity''.

\begin{definition*}[Topology] \label{def:topo}
    Let $X$ be a set.
    A \emph{topology} on $X$ is a collection $\T \subseteq 2^X$ that
    \begin{enumerate}[label=\small(T\arabic*)]
        \item contains $\O$ and $X$,
        \item is closed under finite intersections, and
        \item is closed under arbitrary unions.
    \end{enumerate}
    $(X, \T)$ is called a \emph{topological space}.
    Elements of $\T$ are called \emph{open sets}.
\end{definition*}

\begin{remark}
    A collection $\set{U_\alpha}_{\alpha \in J}$ is an \emph{indexed}
    collection of sets.
    We will sometimes not mention the index set $J$.
\end{remark}

\begin{examples}
    \item $\T = \set{\O, X}$ is called the trivial or \emph{indiscrete}
        topology.
    \item $\T = 2^X$ is called the \emph{discrete} topology.
    \item $X = \set{a, b}$ has topologies
    \begin{itemize}
        \item $\T = \set{\O, \set{a, b}}$,
        \item $\T = \set{\O, \set{a}, \set{a, b}}$ and
            $\T = \set{\O, \set{b}, \set{a, b}}$,
        \item $\T = \set{\O, \set{a}, \set{b}, \set{a, b}}$.
    \end{itemize}
    \item $X = \set{a, b, c}$ has topologies
    \begin{itemize}
        \item $\set{\O, \set{a, b, c}}$, \\
            $\set{\O, \set{a, b}, \set{a, b, c}}$,
            $\set{\O, \set{b, c}, \set{a, b, c}}$,
            $\set{\O, \set{c, a}, \set{a, b, c}}$,
        \item $\set{\O, \set{a}, \set{a, b, c}}$,
            $\set{\O, \set{b}, \set{a, b, c}}$,
            $\set{\O, \set{c}, \set{a, b, c}}$,
        \item $\set{\O, }$
    \end{itemize}
    \item For any $X$, the set \[
        \T = \set{A \in 2^X \mid X \setminus A \text{ is finite}}
                \cup \set{\O}
    \] is the \emph{cofinite} topology.
    This is a topology because
    \begin{enumerate}[label=\small(T\arabic*)]
        \item $\O$ is by definition, and the complement of $X$ is finite,
        \item Let $U, V \in \T$.
            If either is $\O$, then $U \cap V = \O$.
            Otherwise, $X \setminus (U \cap V) = (X \setminus U) \cup (X \setminus V)$
            is finite.
        \item Let $\set{U_\alpha} \subseteq \T$.
            Then $X \setminus (\bigcup_\alpha U_\alpha)
                = \bigcap_\alpha (X \setminus U_\alpha)$ is finite.
    \end{enumerate}
    \item The \emph{cocountable} topology on $X$ is what you would expect.
    \item \Cref{thm:topo:metric}.
\end{examples}

\begin{definition*}[Metric] \label{def:metric}
    A \emph{metric} on a set $X$ is a map $d\colon X \times X \to \R$
    satisfying
    \begin{enumerate}[label=\small(M\arabic*)]
        \item (positive definiteness) $d(x, y) \ge 0$ for all $x, y \in X$,
            and $d(x, y) = 0$ iff $x = y$,
        \item (symmetry) $d(x, y) = d(y, x)$ for all $x, y$,
        \item (triangle inequality)
            $d(x, y) + d(y, z) \ge d(x, z)$ for all $x, y, z$.
    \end{enumerate}
    $(X, d)$ is called a \emph{metric space}.
\end{definition*}
\begin{examples}
    \item $(\R, \abs{\ccdot})$.
    \item $(\R^n, {\norm{\ccdot}}_2)$ and in general
        $(\R^n, {\norm{\ccdot}}_p)$.
    \item $d(x, y) = \bm{1}_{x \ne y}$ on any set $X$.
\end{examples}

\begin{definition}[Open sets] \label{def:metric:open}
    An \emph{open ball} in a metric space $(X, d)$ is a subset \[
        B(x_0, r) = \set{x \in X \mid d(x, x_0) < r}
    \] where $x_0 \in X$ and $r > 0$.

    An \emph{open set} is a subset $U \subseteq X$ if for any $x \in U$,
    there is an open ball $B(x, r) \subseteq U$.
\end{definition}

\begin{theorem} \label{thm:topo:metric}
    Let $(X, d)$ be a metric space.
    The collection of open sets \[
        \T = \set{U \subseteq X \mid \text{$U$ is open}}
    \] is a topology on $X$.
\end{theorem}
\begin{proof}
    $\O$ is vacuously open, $X$ is trivially so.

    A ball here and a ball there, the smaller one is everywhere.

    Going to a party, without your friends? \\
    That attitude's not good. \\
    QED, leaving no dead ends, \\
    Take along your entire neighbourhood.
\end{proof}

\begin{theorem*}
    The co-finite topology on an infinite set is not metrizable.
\end{theorem*}
\begin{proof}
    A metric space is a \href{https://en.wikipedia.org/wiki/Hausdorff_space}
    {Hausdorff space}.
    But in the co-finite topology, any two (non-empty) open sets intersect.
\end{proof}

\textbf{Question:} When is a topology metrizable?
