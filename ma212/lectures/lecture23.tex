\lecture{2024-10-09}{}

\begin{lemma}
    Let $R$ be an integral domain.
    Then
    \begin{enumerate}
        \item $R[x]$ is an integral domain.
        \item For $f, g, \in R[x]$, $\deg fg = \deg f \deg g$.
        \item $(R[x])^\times = R^\times$.
    \end{enumerate}
\end{lemma}

\begin{theorem}[Chinese remainder theorem] \label{thm:crt}
    Let $I_1, \dots, I_n$ be pairwise maximal.
    Then $R \xrightarrow{f} R/I_1 \times \dots \times R/I_n$ induces an
    isomorphism \[
        f\colon R/(I_1 \dots I_n) \isoto \R/I_1 \times \dots \times R/I_n.
    \]
    Equivalently, given elements $x_1, \dots, x_n \in R$, there exists an
    $x$ such that $x \equiv x_j \pmod{I_j}$ for all $j \in [n]$.
\end{theorem}
\begin{proof}
    Consider the case $n = 2$.
    We can find $a_1 \in I_1$, $a_2 \in I_2$ such that $a_1 + a_2 = 1$.
    Then for $x_1, x_2 \in R$, we have that $x = x_1 a_2 + x_2 a_1$
    satisfies the required conditions. \[
        x = x_1 (1 - a_1) + x_2 a_1
          = x_1 + (x_2 - x_1) a_1
          \equiv x_1 \pmod{I_1} \\
    \]
    For $n \ge 2$, find for each $j > 1$ elements $a_j \in I_1$ and
    $b_j \in I_j$ such that $a_j + b_j = 1$.
    Now \begin{align*}
        1 = \prod_{j>1} (a_j + b_j) \in I_1 + \prod_{j>1} I_j,
    \end{align*} so $I_1$ and $\prod_{j>1} I_j$ are co-maximal.
    Similarly, $I_k$ and $\prod_{j \ne k} I_j$ are co-maximal for all $k$.

    By the case $n = 2$, we can find elements $y_k$ such that \begin{align*}
        y_1 &\equiv 1 \pmod{I_1} & y_1 &\equiv 0 \pmod{\prod_{j\ne1} I_j} \\
        y_2 &\equiv 1 \pmod{I_2} & y_2 &\equiv 0 \pmod{\prod_{j\ne2} I_j} \\[-1em]
        &\vdotswithin{\equiv} & &\vdotswithin{\equiv} \\
        y_n &\equiv 1 \pmod{I_n} & y_n &\equiv 0 \pmod{\prod_{j\ne n} I_j}.
    \end{align*}
    We can now construct $x = \sum_{j=1}^n x_j y_j$.
    \TODO[Finish this proof.]
\end{proof}

\begin{corollary}
    Let $m \in \Z \setminus \set 0$.
    Then \[
        \Z/m \cong \prod_{p \mid m} \Z/p^{v_p(m)}.
    \]
\end{corollary}

\begin{corollary}
    Let $m \in \Z \setminus \set 0$.
    Then \[
        (\Z/m)^\times \cong \prod_{p \mid m} (\Z/p^{v_p(m)})^\times.
    \]
\end{corollary}
\begin{remarks}
    \item If $A$ and $B$ are rings then
        \begin{itemize}
            \item If $f\colon A \isoto B$,
                then $f\vert_{A^\times}\colon A^\times \isoto B^\times$.
            \item $(A \times B)^\times = A^\times \times B^\times$.
        \end{itemize}
    \item $\abs{(\Z/p^\alpha)^\times} = p^\alpha \ab(1 - \frac1p)$,
        since the units are non-multiples of $p$.
    \item As a corollary, $\phi(m) \coloneq \abs{(\Z/m)^\times}
        = \prod\limits_{p \mid m} p^{v_p(m)} \ab(1 - \frac1p)$.
        This is multiplicative, i.e. $\phi(mn) = \phi(m) \phi(n)$ whenever
        $(m, n) = 1$.
\end{remarks}

\section{Ideals in commutative rings} \label{sec:comm-ideals}
\begin{definition*}[prime] \label{def:comm-ideals:prime}
    An ideal $P \subseteq R$ is \emph{prime} if whenever $ab \in P$,
    $P \ni a$ or $P \ni b$.
\end{definition*}
\begin{examples}
    \item $2\Z$ is a prime ideal of $\Z$, since $ab \in 2\Z$ implies that
        one of $a$ and $b$ is even.
    \item If $\set 0$ is prime, then $R$ is an integral domain, since
        $ab = 0 \implies a = 0$ or $b = 0$.
\end{examples}

\begin{definition*}[maximal ideal] \label{def:comm-ideals:maximal}
    An ideal $M \subseteq R$ is \emph{maximal} if it is a maximal proper
    ideal under inclusion.
\end{definition*}
\begin{examples}
    \item $n\Z$ is maximal in $\Z$ if and only if $n$ is prime.
        If $m\Z$ is an ideal containing $n\Z$, then $m \mid n$.
        \begin{itemize}
            \item If $n$ is prime, then $m = 1$ or $m = n$.
            \item If $n$ is composite with a prime factor $p$, then
                $m = p$ gives a proper ideal containing $n\Z$.
        \end{itemize}
    \item
\end{examples}

\begin{proposition}
    Let $I$ be an ideal of $R$.
    Then
    \begin{enumerate}
        \item $I$ is prime iff $R/I$ is an integral domain.
        \item $I$ is maximal iff $R/I$ is a field.
    \end{enumerate}
\end{proposition}

\begin{proposition}
    Every maximal ideal is prime.
\end{proposition}
\begin{proof}
    Every field is an integral domain.
\end{proof}
\begin{proof}[Direct proof]
    Let $M$ be a maximal ideal.
    Suppose $ab \in M$ but $a \notin M$.
    Then $M + \angled a = R$.
    Write $1 = m + ra$ with $m \in M$ and $r \in R$
    so that $b = mb + rab \in M$.
\end{proof}

\begin{remark}
    The converse is \emph{not} true.
    $\angled 0$ is prime in $\Z$, but not maximal.
    For a non-trivial example, consider the ring $R = \Z[x]$.
    \begin{itemize}
        \item $I = \angled x$ is prime, since $x \mid fg$ implies that
            at least one of $f$ and $g$ has constant term $0$.
            It is not maximal, since $I \subsetneq \angled{2, x}$.
        \item Let $I = \angled{p, x} \subseteq \Z[x]$ where $p$ is prime.
            Then $I = \sset{np + xf(x)}{n \in \Z, f \in \Z[x]}$.
            We show that $R/I \cong \Z/p$.
            The reduction mod $p$ \[
                \sum_{i=0}^n a_i x^i \mapsto \sum_{i=0}^n \wbar{a_i} x^i
            \] is a surjective ring homomorphism from $\Z[x]$ to
            $(\Z/p)[x]$.
            Compose this with the projection $(\Z/p)[x] \to \Z/p$
            (evaluation at $x = \wbar{0}$) to get a surjective ring
            homomorphism $\varphi\colon R \to \Z/p$ with kernel \[
                \ker \varphi = \sset{f \in R}{f(0) \equiv 0 \pmod p}
                    = \angled{p, x} = I.
            \] Thus $R/I \cong \Z/p$ is a field, so $I$ is maximal.
        \item $\angled p$ is prime.
            Note that $\angled p = (p\Z)[x]$.
            Suppose $f = a_0 + \dots a_nx^n$ and $g = b_0 + \dots + b_mx^m$
            have product in $\angled p$.
            Assume neither is in $\angled p$.
            Let $i$, $j$ be the smallest indices such that $a_i$, $b_j$
            are coprime to $p$.
            Then the coefficient of $x^{i+j}$ in $fg$ is $a_ib_j$ modulo
            $p$, which is not $0$.
            Contradiction.

            We can show that $\Z[x]/(p\Z)[x]$ is isomorphic to
            $(\Z/p)[x]$.
            Consider the reduction map $f \mapsto \wbar{f}$ from
            $\Z[x] \to (\Z/p)[x]$, which is a surjective homomorphism with
            kernel $\angled p$.
    \end{itemize}
\end{remark}

\begin{theorem}
    Every ideal is contained in a maximal ideal.
\end{theorem}
\begin{proof}
    Let $S$ be the set of proper ideals containing $I$, partially ordered
    by set inclusion.
    Let $\mcC$ be a chain in $S$, and $N$ be its union.
    \begin{itemize}
        \item If $x \in N$ then $x \in J$ for some $J \in \mcC$.
            So $x \cdot r \in J \subseteq N$ for all $r \in R$.
        \item If $x, y \in N$, then $x \in J_x$ and $y \in J_y$ for some
            $J_x, J_y \in \mcC$.
            Since $\mcC$ is a chain, one of these contains the other,
            and thus contains the sum.
        \item $1 \notin J$ for any $J \in \mcC$, so $1 \notin N$.
    \end{itemize}
    Thus $N$ is a proper ideal containing $I$, so $N \in S$ upper bounds
    $\mcC$.
    By \nameref{zorn}, $S$ has a maximal element.
\end{proof}

\begin{definition}[partial order] \label{def:partial-order}
    A relation $\le$ on a set $S$ is a \emph{partial order} if it is
    reflexive, transitive and antisymmetric.

    If for every $a, b \in S$, either $a \le b$ or $b \le a$ holds, then
    $\le$ is a \emph{total order}.

    A \emph{chain} is a totally ordered subset of $S$.
\end{definition}
\begin{example}
    The set of subsets of a set $X$ is partially ordered by inclusion.
    The regular order on $\R$ is a total order.
\end{example}

\begin{theorem*}[Zorn's lemma] \label{zorn}
    Let $(S, \le)$ be a poset whose every chain has an upper bound.
    Then $S$ has a maximal element.
\end{theorem*}
