\lecture{2024-08-12}{Skyline in $O(n \log k)$ (Chan's algorithm)}

If the number of skyline points is much smaller than the number of points,
we can do better than $O(n \log n)$.

In this lecture, we will figure out an $O(n \log k)$ algorithm,
where $k$ is the number of skyline points in the input.

\textbf{Warm-up:} If $k = 1$, is there a better algorithm than $O(n \log n)$? \\
\textbf{Answer:} Yes. Simply pick the maximum $x$-coordinate.

\section{A slow algorithm} \label{sec:slow}
\begin{question*}
    Can we find an $O(n k)$ algorithm for the skyline problem?
    (assuming $d$ to be constant)
\end{question*}
\begin{solution}
    Pick the largest $x$-coordinate,
    and remove all points that are dominated by it.
    Repeat until the set is exhausted.

    The next point chosen is not dominated by the previous one,
    hence it is not dominated by any other deleted point.
    Since it has the largest $x$-coordinate of the remaining points,
    it cannot be dominated by those either.
\end{solution}
We will try to make this faster later.

\section{Chan's algorithm} \label{sec:chan}
\begin{question*}
    Can we find an $O(n \log k)$ algorithm for the skyline problem in
    $\R^2$?
\end{question*}

We will first solve the following decision problem:
\begin{question*}
    Given a set of points $P$ having $k$ skyline points and an integer
    $\what{k}$, decide
    \begin{itemize}
        \item if $\what{k} \ge k$, output the skyline,
        \item if $\what{k} < k$, output \textbf{failure}.
    \end{itemize}
\end{question*}
\begin{solution}
    Partition $P$ into roughly $\what{k}$ slabs of roughly equal size
    using the median of medians algorithm.
    \begin{enumerate}
        \item Compute the median and partition the elements in $O(n)$ time.
        \item Repeat recursively on the two halves, upto a depth of
            $\log k$.
    \end{enumerate}
    In total, this takes $O(n \log k)$ time.
    (One remarked that it's like quicksort, but stopping once we have
    exhausted our time budget.)

    Visit each slab in order of decreasing $x$-coordinate.
    \begin{algo}
        \Fn{Chan-Decision-2D}{$P$, $\what{k}$}
            \State $P_1, \ldots, P_{\what{k}} \gets \Call{partition}{P}$
            \State $S \gets \O$
            \State $y(p^*) \gets -\infty$
            \For{$i = \what{k}$ downto $1$}
                \If{$\abs{S} > \what{k}$}
                    \State \Return Failure
                \Else
                    \State $P_i' \gets \set{p \in P_i \mid y(p) > y(p^*)}$
                    \State $S(P_i') \gets \Call{Skyline}{P_i'}$
                        \Comment{using the slow algorithm}
                    \State report $S(P_i')$
                    \State $S \gets S \cup S(P_i')$
                \EndIf
            \EndFor
            \State \Return $S$
        \EndFn
    \end{algo}
    Failure is not because of our inability to output the correct answer,
    but because the runtime would be too large.
\end{solution}
\begin{proof}[Analysis]
    Constructing the slabs took $O(n \log \what{k})$ time.

    Running the slow algorithm on $P_i$ takes $\frac{n}{\what{k}} \cdot k_i$
    time, where $k_i$ is the number of skyline points in $P_i$.
    In total, this step takes \[
        \sum_{i=1}^{\what{k}} O\ab(\frac{n}{\what{k}} \cdot k_i)
            = O\ab(\frac{n}{\what{k}} \cdot \sum_{i=1}^{\what{k}} k_i)
            = O(n)
    \] time.
    Thus the total runtime is $O(n \log \what{k})$.
\end{proof}

We are now prepared to solve the original problem.
\begin{solution}
    Guess $\what{k} = 1, 2, 4, \ldots, 2^{\ceil{\log k}}$.
    This will take \[
        \sum_{i=0}^{\ceil{\log k}} O(n \log 2^i)
            = \sum_{i=0}^{\ceil{\log k}} O(n i)
            = O(n (\log k)^2)
    \] time.
    We can be even more aggressive and guess
    $\what{k}_i = 2^{2^i}$ (square the previous guess).
    This takes \[
        \sum_{i=0}^{\ceil{\log \log k}} O(n \log 2^{2^i})
            = \sum_{i=0}^{\ceil{\log \log k}} O(n 2^i)
            = O\ab(n 2^{\ceil{\log \log k}})
            = O(n \log k)
    \] time.
    \begin{algo}
        \Fn{Chan-2D}{$P$}
            \State $\what{k} \gets 2$
            \For{ever}
                \State $S \gets \Call{Chan-Decision-2D}{P, \what{k}}$
                \If{$S \ne \text{Failure}$}
                    \State \Return $S$
                \EndIf
                \State square $\what{k}$
            \EndFor
        \EndFn
    \end{algo}
\end{solution}

\begin{question*}
    Can we generalize this to $\R^3$?
\end{question*}
It suffices to generalize the decision problem.

\begin{solution}
    We construct $\what{k}$ slabs based on the $x$-coordinate.
    The invariants are the following:
    \begin{itemize}
        \item $S_i$ will store the skyline of the first $i$ slabs.
        \item $S_i^{yz}$ will store the 2D skyline of the projection of
            $S_i$ onto the $yz$-plane.
        \item Each $S_i$ and $S_i^{yz}$ is sorted according to the
            $y$-coordinate.
    \end{itemize}
    The deletion step now takes $\abs{P_i} \log \abs{S_i^{yz}}$ time,
    (as opposed to $\abs{P_i}$ earlier)
    which is still fine.

    We now run the slow algorithm and report the elements added to $S_i$.
    This takes $O(n)$ time, and they are naturally reported in decreasing
    order of the $y$-coordinates.
    \begin{algo}
        \Fn{Chan-Decision}{$P$, $\what{k}$}
            \State $P_1, \ldots, P_{\what{k}} \gets
                \Call{partition}{P}$
            \State $S \gets \O$, $S^{yz} \gets \O$
            \State $y(p^*) \gets -\infty$
            \For{$i = \what{k}$ downto $1$}
                \If{$\abs{S} > \what{k}$}
                    \State \Return Failure
                \Else
                    \State $P_i' \gets \Call{filter}{P_i, S^{yz}}$
                    \State $S(P_i') \gets \Call{Skyline}{P_i'}$
                        \Comment{using the slow algorithm}
                    \State report $S_(P_i')$
                    \State $S \gets S \cup S(P_i')$
                \EndIf
            \EndFor
            \State \Return $S$
        \EndFn
    \end{algo}
\end{solution}
