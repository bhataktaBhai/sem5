\lecture{2024-08-13}{}
\section{$\eps$-nets} \label{sec:eps-nets}

\begin{definition}
    Let $(X, \mcR)$ be a range space.
    Let $A \subseteq X$ be finite.
    A set $N \subseteq A$ is a combinatorial $\eps$-net for $A$ if
    $N$ has non-empty intersection with every $R \in \mcR$ which satisfies
    $\abs{R \cap A} \ge \eps \abs{A}$.
\end{definition}

\begin{definition}
    Let $(X, \mcR)$ be a range space and let $\mcD$ be a probability
    distribution on $X$.
    A set $N \subseteq X$ is an $\eps$-net for $X$ wrt $\mcD$ if
    for every $R \in \mcR$ with $\Pr_\mcD(R) \ge \eps$,
    $N \cap R \ne \O$.
\end{definition}
\begin{example}
    Consider the uniform distribution on $[0, 1]^2$, with $\mcR$ being the
    collection of all axis-aligned rectangles.
    Then, for $\eps = 1/4$, the set \[
        N = \sset{(i/4, j/4)}{i, j \in [3]}
            \setminus \set{(1/2, 1/2)}
    \] is an $\eps$-net.
\end{example}

Last class: If there are $k$ sets of mass at least $\eps$,
then the probability that $m$ samples miss all given sets is at most
$k (1 - \eps)^m$.
For this to be bounded by $\delta$, we need \[
    m \ge \frac1{\eps} \ln \frac{k}{\delta}.
\] Using the Sauer-Shelah lemma, we can bound $k$ by $n^d$, which gives \[
    m \ge \frac{d}{\eps} \ln \frac{n}{\delta}
\] which still depends on $n$.

\begin{theorem*}[$\eps$-net theorem] \label{thm:eps-nets}
    Let $(X, \mcR)$ be a range space with VC-dimension $d$.
    Let $\mcD$ be a probability distribution on $X$.
    For any $0 < \delta, \eps \le \frac12$,
    there is an \[
        m = O\ab(\frac{d}{\eps} \ln \frac{d}{\eps} + \frac1{\eps} \ln \frac1{\delta})
    \] such that a random sample from $\mcD$ of size $m$ is an $\eps$-net
    for $X$ wrt $\mcD$ with probability at least $1 - \delta$.
\end{theorem*}
\begin{proof}
    Let $M$ be a set of $m$ independent samples from $X$ according to
    $\mcD$.
    Define \[
        E_1 \coloneq \sset{\exists S \in \mcR}{\Pr_\mcD(S) \ge \eps
                    \land S \cap M = \O}
    \] We wish to have $\Pr[E_1] \le \delta$.

    Choose a second set $T$ of $m$ independent samples from $X$ according to
    $\mcD$.
    Define \[
        E_2 \coloneq \set*{\exists S \in \mcR \mid \Pr_\mcD(S) \ge \eps
            \land S \cap M = \O \land \abs{S \cap T} \ge \frac12 \eps m}
    \]
    \begin{lemma}
        For $m \ge \frac{8}{\eps}$, $\Pr[E_2] \le \Pr[E_1] \le 2 \Pr[E_2]$.
    \end{lemma}
    \begin{proof}
        The left inequality is simply because $E_2 \subseteq E_1$.

        If $E_1$ occurs, then there exists a set $S'$ with probability
        at least $\eps$ such that $S' \cap M = \O$.
        Now \begin{align*}
            \frac{\Pr[E_2]}{\Pr[E_1]} &= \frac{\Pr[E_2 \cap E_1]}{\Pr[E_1]} \\
            &= \Pr[E_2 \mid E_1] \\
            &\ge \Pr(\abs{T \cap S'} \ge \frac12 \eps m)
        \end{align*}
        For a fixed $S'$ and a random sample $T$, $\abs{T \cap S'}$ is
        a $\Bin(m, \Pr_\mcD(S'))$ random variable.

        We will use the Chernoff bound
        \begin{align*}
            \Pr(\abs{T \cap S'} \ge (1 - \beta) E[\abs{T \cap S'}])
            &\le e^{-} \\
            &\vdotswithin{\le}
        \end{align*}
    \end{proof}
    Now define \[
        E_3 \coloneq \set{\exists S \in \mcR \mid
            S \cap M = \O \land \abs{S \cap T} \ge \frac12 \eps m}
            \supseteq E_2
    \] Thus we have \[
        \Pr[E_1] \le 2 \Pr[E_2] \le 2 \Pr[E_3].
    \]

    Let $k = \frac12 \eps m$.
    For a fixed $S \in \mcR$, define
    $E_S \coloneq \set{S \cap M = \O, \abs{S \cap T} \ge k}$.

    There are $\binom{2m}{m}$ possible partitions of $M \cup T$, but only
    in $\binom{2m-k}{m}$ of them does $S \cap M = \O$.
    \begin{lemma}
        For $m \ge \frac8{\eps}$,
        $\Pr[E_S] \le \Pr[E_1] \le 2 \Pr[E_S]$.
    \end{lemma}
    \begin{proof}
        \begin{align*}
            \Pr(E_S) &\le \Pr(M \cap S = \O \mid S \cap (M \cup T) \ge k) \\
            &\le \frac{\binom{2m-k}{m}}{\binom{2m}{m}} \\
            &= \frac{(2m-k)!m!}{2m!(m-k)!} \\
            &= \frac{m(m-1)\dots(m-k+1)}{2m(2m-1)\dots(2m-k+1)} \\
            &\le \frac12^k \\
            &\le 2^{-\eps m / 2} \qedhere
        \end{align*}

        By Sauer-Shelah, the projection of $\mcR$ on $M \cup T$ has at most
        $(2m)^d$ ranges and \[
            \Pr[E_S] \le (2m)^d 2^{-\frac12 \eps m} \le \delta.
        \]
    \end{proof}
\end{proof}

\section{Set cover} \label{sec:set-cover}
\begin{question*}[Set cover]
    Given $m$ sets $S_1, S_2, \dots, S_m$ and $n$ elements
    $e_1, e_2, \dots, e_n$,
    find the minimum number of sets such that all elements are covered.
\end{question*}

\begin{question*}[Hitting set]
    Given $m$ sets $S_1, S_2, \dots, S_m$ and $n$ elements
    $e_1, e_2, \dots, e_n$,
    find the minimum number of elements such that all sets are hit.
\end{question*}

These are equivalent in the general setting, by switching the roles of
sets and elements.
However, this may lose some nice properties of the problem.
For example, if the sets are rectangles, the dual hitting set problem
has no such rectangle structure.

We first rephrase the hitting set problem as an integer program.
\begin{question*}
    \textbf{Minimize:} $\sum_{u \in X} X_u$ \\
    \textbf{Subject to:} $\sum_{u \in S} X_u \ge 1$ for all $S \in \mcR$.
\end{question*}
Here $X_u$ indicates if element $u$ is chosen.

Define \begin{align*}
    \eps &\coloneq \frac1{\sum_{u \in X} X_u} \\
    \mu_u &\coloneq \eps X_u
\end{align*}
\begin{question*}
    \textbf{Maximize:} $\eps$ \\
    \textbf{Subject to:} $\sum_{i \in S} \mu_i \ge \eps$ for each
    $S \in \mcR$ and $\sum_{u \in X} \mu_u = 1$.
\end{question*}

\begin{solution} \leavevmode
    \begin{enumerate}
        \item Solve the LP for $(\eps^*, \mu^*)$.
        \item Find $\eps^*$-net in \[
            O\ab(\frac{d}{\eps} \ln \frac{d}{\eps} + \frac1{\eps} \ln \frac1{\delta})
        \] time.
        Since $\eps^* = \frac1{\OPT}$, we can rewrite this as \[
            O\ab(d \OPT \ln (d \OPT) + \OPT) % TODO: ???
        \]
    \end{enumerate}
    We claim that the $\eps^*$-net is a hitting set.
\end{solution}

\section{Art gallery} \label{sec:art-gallery}
\begin{question*}[Art gallery]
    Given a polygon $P$ with $n$ sides, find the minimum number of guards
    needed to see all of $P$.
\end{question*}
The range space is $(P, \mcR)$ where $P$ is the set of points inside the
polygon,
and $\mcR$ is the set of all visibility regions from a point in $P$.
The \emph{visibility region} from a point $p$ is \[
    V_P(p) = \sset{q \in P}{\wbar{pq} \subseteq P}.
\]
