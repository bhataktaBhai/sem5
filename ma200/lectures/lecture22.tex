\lecture[22]{2024-10-04}{}
\begin{example}
    Let $f(x, y) = x^3 e^y + 2x \cos(xy) - 3$.
    Can we write $y = \phi(x)$ in an open neighborhood $V$ of $1$ such that
    $f(x, \phi(x)) = 0$ for all $x \in V$?
    What is $\phi'(1)$?
    Then \begin{align*}
        \pdv{f}{y}(x, y) &= x^3 e^y - 2x^2 \sin(xy) \\
        \pdv{f}{x}(x, y) &= 3x^2 e^y + 2 \cos(xy) - 2xy \sin(xy)
    \end{align*}
    and $\pdv{f}{y}(1, 0) = 1 \ne 0$.
\end{example}

Let $U \subseteq \R^n$ be open and $f\colon U \to \R$
Then $f \in C^k(U)$ iff all partial derivatives of $f$ upto order $k$ exist
on $U$ and are continuous on $U$, which is iff $D_i f \in C^{k-1}(U)$ for
all $i \in [n]$.

\begin{notation} \leavevmode
    \begin{itemize}
        \item $D_if(a) = \pdv{f}{x_i}(a)$.
        \item $D_{ij}f(a) = \pdv{f}{x_i, x_j}(a) = (D_i(D_jf))(a)$.
        This only makes sense if $D_jf$ exists in a neighbourhood of $a$,
        and is called a \emph{mixed} partial derivative.
        \item In general,
        $D_{i_1, \dots, i_k}f(a) = \pdv{f}{x_{i_1}, \dots, x_{i_k}}(a)
        = D_{i_1}(D_{i_2}(\dots(D_{i_k}f)\dots))(a)$.
    \end{itemize}
\end{notation}

\begin{question}
    Suppose $D_{ij}f(a)$ and $D_{ji}f(a)$ exist.
    Can we say that they are equal?
\end{question}
\begin{solution}
    No, not necessarily.
    % Consider $f(x, y) = x(1 + y)$.
    % Then \begin{align*}
    %     \pdv{f}{x} &= 1 + 0 = 1 \\
    %     \pdv{f}{y} &= 
    % \end{align*}
    \TODO
\end{solution}

\begin{theorem}
    Let $U \subopeneq \R^2$ and $f\colon U \to \R$.
    Suppose
    \begin{enumerate}
        \item $D_1f$, $D_2f$ and $D_{21}f$ exist on $U$.
        \item $D_{21}f$ is continuous at $(a, b) \in U$.
    \end{enumerate}
    Then $D_{12}f(a, b)$ exists and \[
        D_{12}f(a, b) = D_{21}f(a, b).
    \]
\end{theorem}
