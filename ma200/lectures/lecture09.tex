\lecture{2024-08-26}{Gradient and curves}
\begin{definition*}[gradient] \label{def:gradient}
    Let $U \subseteq \R^n$ be open and $f\colon U \to \R$.
    We define the \emph{gradient} $\grad f\colon U \to \R^n$ of $f$ by \[
        (\grad f)(a) \coloneq
        \ab(\pdv{f}{x_1}(a), \ldots, \pdv{f}{x_n}(a))
    \] for all $a \in U$.
\end{definition*}
Throughout this lecture, $U \subseteq \R^n$ is open and $a \in U$,
and $f\colon U \to \R^m$.

\begin{proposition*}
    Suppose $f = (f_1, \dots, f_m)$ is differentiable at $a$.

    Then $\pdv{f_j}{x_i}(a)$ exists for each $i \in [n]$ and $j \in [m]$.
    Moreover, \[
        f'(a)(e_i) = \sum_{j=1}^m \pdv{f_j}{x_i}(a) e_j
    \] for all $i \in [n]$.
    Equivalently, \[
        f'(a) = \begin{pmatrix}
            \grad f_1(a) \\
            \vdots \\
            \grad f_m(a)
        \end{pmatrix}.
    \]
\end{proposition*}
We have abused notation in the above statement to let $e_1$ be in
$\R^n$ or $\R^m$ depending on the context.
\begin{proof}
    Let $f'(a) = T \in L(\R^n, \R^m)$.
    For $i \in [n]$, \begin{align*}
        \pdv{f_j}{x_i}(a)
            &= \lim_{t \to 0} \frac{f_j(a+te_i) - f_j(a)}{t} \\
        \implies \pdv{f_j}{x_i}(a) - T(e_i)_j
            &= \lim_{t \to 0} \frac{f_j(a+te_i) - f_j(a) - tT(e_i)_j}{t} \\
            &= \lim_{t \to 0} \ab(\frac{f(a+te_i) - T(te_i) - f(a)}t)_j \\
            &= 0.
    \end{align*}
    Thus \[
        T(e_i) = \sum_{j=1}^m \pdv{f_j}{x_i}(a) e_j \qedhere
    \]
\end{proof}

\begin{definition*}[curve] \label{def:curve}
    A map $\gamma\colon (a, b) \to \R^n$ is a \emph{curve} in $\R^n$.
    If $\gamma$ is differentiable, it is a \emph{differentiable curve}.
    % how shocking
\end{definition*}

\begin{remark}
    Chain rule remains valid for curves.

    Let $m = 1$ so that $f\colon U \to \R$, and let
    $\gamma\colon (a, b) \to U$.
    We will treat $f'(a)$ to be a scalar, instead of a map from
    $\R^n$ to $\R$.
    We will also treat $\gamma'(t)$ to be a vector in $\R^n$,
    not a map from $\R$ to $\R^n$.
    Another way to think about this is that we are identifying
    $\R^{1 \times 1}$ with $\R$ and $\R^{n \times 1}$ with $\R^n$.

    Then \begin{align*}
        (f \circ \gamma)'(t) &= f'(\gamma(t))(\gamma'(t)) \\
        &= f'(\gamma(t)) \ab(\sum \gamma_i'(t) e_i) \\
        &= \sum \gamma_i'(t) f'(\gamma(t))(e_i) \\
        &= \sum \gamma_i'(t) \pdv{f}{x_i}(\gamma(t)) \\
        % &= \innerp{\grad f(\gamma(t))}{\gamma'(t)}.
        &= \grad f(\gamma(t)) \dotp \gamma'(t).
    \end{align*}
    If we continue to treat $f'(a)$ and $\gamma'(t)$ as maps,
    we would write \begin{align*}
        (f \circ \gamma)'(t)(h)
            &= f'(\gamma(t))(\gamma'(t)(h)) \\
            &= \text{what now?} % TODO
    \end{align*}
\end{remark}

\begin{definition}
    Let $\gamma\colon (-\eps, \eps) \to U$ be such that $\gamma(0) = a$.
    $\gamma'(0)$ is the \emph{tangent vector} to $\gamma$ at $a$,
    and $(f \circ \gamma)'(0)$ is the
    \emph{derivative of $f$ along $\gamma$ at $a$}.
\end{definition}

\begin{proposition}
    If $f$ is differentiable at $a$, then for each $v \in \R^n$,
    $D_v f(a)$ exists and equals $\grad f(a) \dotp v$.
\end{proposition}
\begin{proof}
    Let $\gamma(t) = a + tv$.
    There exist some $\eps > 0$ such that $\gamma(-\eps, \eps) \subseteq U$
    and $\gamma'(0) = v$.
    Then \[
        D_v f(a) = \lim_{t \to 0} \frac{f(a + tv) - f(a)}{t}
        = (f \circ \gamma)'(0) = \grad f(a) \dotp v. \qedhere
    \]
\end{proof}
