\documentclass[12pt]{article}
\input{~/IISc/sem5/preamble}

% \makeatletter
% \newmathcommand{\l}{\@ifstar\@l\@@l}
% \makeatother

\DeclareMathOperator\id{id}
\DeclareMathOperator\GL{GL}
\DeclareMathOperator\M{M}
\DeclareMathOperator\Tr{Tr}
\DeclareMathOperator\adj{adj}
\newcommand\T{\top}
\DeclareMathOperator\argmax{argmax}

\makeatletter
\newcommand\@hsn[1]{{\norm{#1}}_{HS}}
\newcommand\@@hsn[1]{{\norm*{#1}}_{HS}}
\newcommand\hsn{\@ifstar\@@hsn\@hsn}
\makeatother

% \pdv
\usepackage{derivative}
\newcommand*\grad{\nabla}

% https://tex.stackexchange.com/a/235120
\makeatletter
\newcommand*\dotp{\mathpalette\bigcdot@{.5}}
\newcommand*\bigcdot@[2]{\mathbin{\vcenter{\hbox{\scalebox{#2}{$\m@th#1\bullet$}}}}}
\makeatother

\newcommand*\subopeneq{\subseteq_{\mathrm{op}}}


% \makeatletter
% \newmathcommand{\l}{\@ifstar\@l\@@l}
% \makeatother

\DeclareMathOperator\id{id}
\DeclareMathOperator\GL{GL}
\DeclareMathOperator\M{M}
\DeclareMathOperator\Tr{Tr}
\DeclareMathOperator\adj{adj}
\newcommand\T{\top}
\DeclareMathOperator\argmax{argmax}

\makeatletter
\newcommand\@hsn[1]{{\norm{#1}}_{HS}}
\newcommand\@@hsn[1]{{\norm*{#1}}_{HS}}
\newcommand\hsn{\@ifstar\@@hsn\@hsn}
\makeatother

% \pdv
\usepackage{derivative}
\newcommand*\grad{\nabla}

% https://tex.stackexchange.com/a/235120
\makeatletter
\newcommand*\dotp{\mathpalette\bigcdot@{.5}}
\newcommand*\bigcdot@[2]{\mathbin{\vcenter{\hbox{\scalebox{#2}{$\m@th#1\bullet$}}}}}
\makeatother

\newcommand*\subopeneq{\subseteq_{\mathrm{op}}}


% \makeatletter
% \newmathcommand{\l}{\@ifstar\@l\@@l}
% \makeatother

\DeclareMathOperator\id{id}
\DeclareMathOperator\GL{GL}
\DeclareMathOperator\M{M}
\DeclareMathOperator\Tr{Tr}
\DeclareMathOperator\adj{adj}
\newcommand\T{\top}
\DeclareMathOperator\argmax{argmax}

\makeatletter
\newcommand\@hsn[1]{{\norm{#1}}_{HS}}
\newcommand\@@hsn[1]{{\norm*{#1}}_{HS}}
\newcommand\hsn{\@ifstar\@@hsn\@hsn}
\makeatother

% \pdv
\usepackage{derivative}
\newcommand*\grad{\nabla}

% https://tex.stackexchange.com/a/235120
\makeatletter
\newcommand*\dotp{\mathpalette\bigcdot@{.5}}
\newcommand*\bigcdot@[2]{\mathbin{\vcenter{\hbox{\scalebox{#2}{$\m@th#1\bullet$}}}}}
\makeatother

\newcommand*\subopeneq{\subseteq_{\mathrm{op}}}


\title{Assignment 1}
\author{Naman Mishra}
\date{21 August, 2024}

\begin{document}
\maketitle

% Problem 1
\begin{problem}
    Let $n \ge 3$.
    Let $r, s$ be the usual generators of $D_{2n}$.
    Prove that the map $\varphi\colon D_{2n} \to \GL_2(\R)$ defined by
    \[
        \varphi(r) = \begin{pmatrix}
            \cos \theta & -\sin \theta \\
            \sin \theta & \cos \theta
        \end{pmatrix} \qquad \varphi(s) = \begin{pmatrix}
            1 & 0 \\
            0 & -1
        \end{pmatrix}
    \] where $\theta = \frac{2\pi}{n}$ extends to an injective
    homomorphism.
\end{problem}

\begin{problem}
    Show that $(\Z, +) \ncong (\Q, +)$,
    $S_m \ncong S_n$ for $m \ne n$ and $(\R, +) \ncong (\Q, +)$.
\end{problem}
\begin{solution}
    Let $\varphi\colon (\Z, +) \to (\Q, +)$ be a homomorphism.
    Let $\varphi(1) = a$.
    Then $\varphi(n) = na$ for $n \in \N$.
    $\varphi$ cannot be surjective, as no integer multiple of $a$ can
    equal $a/2$.

    For $m \ne n$, $S_m$ and $S_n$ have different cardinalities.
    So do $\R$ and \Q.
\end{solution}

\begin{problem}
    For a commutative ring $A$, let \[
        A^\times = \set{a \in A \mid ab = 1 \text{ for some } b \in A}
    \] be the unit group of $A$.
    Show that $A^\times$ is a group (under multiplication).

    Are $\Z[x]^\times$ and $\Q[x]^\times$ isomorphic?
    Are $(\Z[x], +)$ and $(\Q[x], +)$ isomorphic?
\end{problem}
\begin{solution}
    $1 \in A^\times$ is the identity.
    If $a_1, a_2 \in A^\times$, then $a_1 a_2 \in A^\times$,
    since there exist $b_1, b_2 \in A$ such that $(a_1 a_2) (b_2 b_1) =
    a_1 b_1 = 1$.
    Associativity is borrowed from $A$.
    Inverses exist by definition.

    $\Z[x]^\times = \set{1, -1}$,
    while $\Q[x]^\times = \Q \setminus \set{0}$.
    They are not isomorphic.

    Assume $\varphi\colon \Q[x] \to \Z[x]$ is an homomorphism.
    Then $\varphi(p) \ne 0$ for each non-zero polynomial $p$.

    Let $\varphi(1) = a_0 + a_1 x + \dots + a_n x^n$, with $a_n \ne 0$.
    Now \[
        \varphi(1) = \varphi\ab(k \cdot \frac1k)
            = \varphi(k) \varphi\ab(\frac1k)
            = k \varphi(1) \varphi\ab(\frac1k)
    \] for each $k \in \Z$.
    Thus $a_n$ is a multiple of $k$ for each $k \in \Z$.
    Absurd.
\end{solution}

\begin{problem}
    Prove that a finitely generated subgroup of $(\Q, +)$ is cyclic.
\end{problem}
\begin{solution}
    Let $H = \angled*{\frac{m_1}{n_1}, \dots, \frac{m_k}{n_k}}$.
    Write this as $\angled*{\frac{m'_1}{N}, \dots, \frac{m'_k}{N}}$,
    where $N = \lcm(n_1, \dots, n_k)$.
    Then $H = \sum_{i=1}^k a_i \frac{m'_i}{N}$ for some $a_i \in \Z$.
    Thus \[
        H = \frac1N \angled*{m'_1, \dots, m'_k}
        = \frac1N \angled*{\gcd(m'_1, \dots, m'_k)}. \qedhere
    \]
\end{solution}

\begin{problem}
    Show that $(\Z^m, +) \ncong (\Z^n, +)$ for $m \ne n$.
\end{problem}

\begin{problem}
    If $\tau_1 = (a_1\ b_1)$ and $\tau_2 = (a_2\ b_2)$ are transpositions
    in $S_n$ with $\set{a_1, b_1} \cap \set{a_2, b_2} = \O$,
    show that $\tau_1 \tau_2 = \tau_2 \tau_1$.
\end{problem}
\begin{solution}
    We have \[
        \tau_1(x) = \begin{cases}
            b_1 & \text{if } x = a_1, \\
            a_1 & \text{if } x = b_1, \\
            x & \text{otherwise}
        \end{cases}
        \qquad \text{and} \qquad
        \tau_2(x) = \begin{cases}
            b_2 & \text{if } x = a_2, \\
            a_2 & \text{if } x = b_2, \\
            x & \text{otherwise.}
        \end{cases}
    \] Now \[
        \tau_1 \tau_2(x) = \begin{cases}
            b_2 & \text{if } x = a_2, \\
            a_2 & \text{if } x = b_2, \\
            b_1 & \text{if } x = a_1, \\
            a_1 & \text{if } x = b_1, \\
            x & \text{otherwise}
        \end{cases}
        \qquad \text{and} \qquad
        \tau_2 \tau_1(x) = \begin{cases}
            b_1 & \text{if } x = a_1, \\
            a_1 & \text{if } x = b_1, \\
            b_2 & \text{if } x = a_2, \\
            a_2 & \text{if } x = b_2, \\
            x & \text{otherwise.}
        \end{cases}
    \] Thus $\tau_1 \tau_2 = \tau_2 \tau_1 = (a_1\ b_1) (a_2\ b_2)$.
\end{solution}

\begin{problem}
    Let $X_{2n} = \angled{x, y \mid x^n = y^2 = 1, xy = yx^2}$.
    What is $X_{2n}$?
\end{problem}
\begin{solution}
    \[
        x^4 = y^2 x^4 = y(yx^2)x^2 = yx(yx^2) = (yx x)y = xyy = x.
    \] Thus $x^3 = 1$.
    If $3 \nmid n$, then $x = 1$ and $X_{2n} = \set{1, y} \cong (\Z_2, +)$.

    If $3 \mid n$, then $X_{2n} = D_6$, since we can simplify the
    relations to \[
        X_{2n} = \angled{x, y \mid x^3 = y^2 = 1, xy = yx^{-1}}.
    \] This is the presentation of $D_6$ with
    $r \mapsto x$ and $s \mapsto y$.
\end{solution}

\begin{problem}
    Let $G = \set*{\begin{pmatrix}
        \pm 1 & c \\
        0 & 1
    \end{pmatrix} \mid c \in \Z/n\Z}$.
    Show that $G \cong D_{2n}$ ($n \ge 3$).
\end{problem}
\begin{solution}
    Put $R = \begin{pmatrix}
        1 & 1 \\
        0 & 1
    \end{pmatrix}$ and $S = \begin{pmatrix}
        -1 & 0 \\
        0 & 1
    \end{pmatrix}$.
    Then \[
        \begin{pmatrix}
            1 & c \\
            0 & 1
        \end{pmatrix} = R^c
        \qquad \text{and} \qquad
        \begin{pmatrix}
            -1 & c \\
            0 & 1
        \end{pmatrix} = S R^c.
    \] Note that $R^n = 1$, $S^2 = 1$ and \[
        SRS = \begin{pmatrix}
            -1 & 0 \\
            0 & 1
        \end{pmatrix} \begin{pmatrix}
            -1 & 1 \\
            0 & 1
        \end{pmatrix}
        = \begin{pmatrix}
            1 & -1 \\
            0 & 1
        \end{pmatrix}
        = R^{-1}.
    \] Thus $\varphi\colon D_{2n} \to G$ defined by \[
        \varphi(r) = R \quad \text{and} \quad \varphi(s) = S
    \] extends to an isomorphism.
\end{solution}

\begin{problem}
    Let $H_\R = \set*{\begin{pmatrix}
        1 & a & b \\
        0 & 1 & c \\
        0 & 0 & 1
    \end{pmatrix} \bigm\mid a, b, c \in \R}$.
    Show that $H_\R$ is a group and that every non-identity element
    has infinite order.
\end{problem}
\begin{solution}
    Let $X_1 = \begin{pmatrix}
        1 & a_1 & b_1 \\
        0 & 1 & c_1 \\
        0 & 0 & 1
    \end{pmatrix}$ and $X_2 = \begin{pmatrix}
        1 & a_2 & b_2 \\
        0 & 1 & c_2 \\
        0 & 0 & 1
    \end{pmatrix}$ be elements of $H_\R$.
    Then \[
        X_1 X_2 = \begin{pmatrix}
            1 & a_1 + a_2 & b_1 + b_2 + a_1 c_2 \\
            0 & 1 & c_1 + c_2 \\
            0 & 0 & 1
        \end{pmatrix} \in H_\R.
    \] Note that the identity matrix is in $H_\R$.
    Associativity is inherited from $M_3(\R)$.

    $a_2 = -a_1$, $c_2 = -c_1$ and $b_2 = -b_1 - a_1 c_2 = a_1 c_1 - b_1$
    gives $X_1 X_2 = I$.

    Let $X = \begin{pmatrix}
        1 & a & b \\
        0 & 1 & c \\
        0 & 0 & 1
    \end{pmatrix}$.
    It is easy to observe that \[
        X^n = \begin{pmatrix}
            1 & n a & * \\
            0 & 1 & n c \\
            0 & 0 & 1
        \end{pmatrix}
    \] If either of $a$ or $c$ is non-zero, then $X^n$ is not the
    identity matrix.
    If $a = c = 0$, then \[
        X^n = \begin{pmatrix}
            1 & 0 & n b \\
            0 & 1 & 0 \\
            0 & 0 & 1
        \end{pmatrix}.
    \] Thus $X^n = I$ iff $X = I$.
\end{solution}

\begin{problem}
    Define the quaternion group $Q_8 = \set{\pm 1, \pm i, \pm j, \pm k}$
    with the multiplication laws
    $i^2 = j^2 = k^2 = -1$, $ij = (-j)i = k$, $jk = (-k)j = i$
    and $ki = (-i)k = j$.
    Also assume that $-1$ commutes with and flips the sign of each element.
    Show that $Q_8$ is a group.
    Show that every subgroup of $Q_8$ is normal, yet it is not abelian.
\end{problem}
\begin{solution}
    % \begin{table}
    %     \centering
    %     $\begin{tblr}{
    %         colspec = {r *{8}{r}},
    %         vline{3,4,5,6,7,8,9} = {dotted,.2pt},
    %         hline{3,4,5,6,7,8,9} = {dotted,.2pt},
    %         vline{2} = {.5pt},
    %         hline{2} = {.5pt},
    %     }
    %            &  1 & -1 &  i & -i &  j & -j &  k & -k \\
    %          1 &  1 & -1 &  i & -i &  j & -j &  k & -k \\
    %         -1 & -1 &  1 & -i &  i & -j &  j & -k &  k \\
    %          i &  i & -i & -1 &    &  k &    &    &    \\
    %         -i & -i &  i &    &    &    &    &  j &    \\
    %          j &  j & -j &    &    & -1 &    &  i &    \\
    %         -j & -j &  j &  k &    &    &    &    &    \\
    %          k &  k & -k &  j &    &    &    & -1 &    \\
    %         -k & -k &  k &    &    &  i &    &    &    \\
    %     \end{tblr}$
    % \end{table}
    I refuse to show it is a group.
    Let's show that every subgroup is normal.
    A subgroup $N \le Q_8$ is normal iff
    $g n g^{-1} \in N$ for all $g \in Q_8$ and $n \in N$.

    $1$ and $-1$ commute with all elements,
    so we only need to check $n, g \in \set{\pm i, \pm j, \pm k}$.

    WLOG let $n = i$.
    $n$ commutes with $\pm i$, so $gng^{-1} = n \in N$ if $g = \pm i$.
    $n$ anti-commutes with $\pm j$ and $\pm k$.
    So if $g \in \set{\pm j, \pm k}$,
    then $g n g^{-1} = -n g g^{-1} = -n = n^{-1} \in N$.
\end{solution}

\end{document}
