\chapter{Convex hulls} \label{chp:hull}
\lecture{2024-10-07}{Convex hulls}

\begin{definition}[convex hull] \label{def:hull}
    Given a set $P$ of points in $\R^2$, the \emph{convex hull} of $P$
    is the smallest convex set containing all of $P$.
    We denote it by $\CH(P)$.
\end{definition}
\begin{figure}
    \centering
    \begin{tikzpicture}
        \drawrandompoints{5}{5}{10}
    \end{tikzpicture}
    \caption{Convex hull of a set of points}
    \label{fig:hull}
\end{figure}

\begin{theorem} \label{thm:hull}
    The convex hull of a finite set is a convex polygon.
\end{theorem}

We will represent a convex hull as a counter-clockwise ordering of vertices,
starting from the bottom-most vertex, and ignoring any degeneracies.

Assumptions:
\begin{itemize}
    \item No three points are collinear.
    \item All $x$- and $y$-coordinates are distinct.
\end{itemize}

\begin{question*}
    Given a set $P$ of $n$ points in $\R^2$, compute $\CH(P)$.
\end{question*}
\begin{solution}[Naivest]
    An edge of the convex hull can be characterized by the fact that all
    other points lie on one side of it.
    Check all line segments between pairs of points.
    This takes $O(n^3)$ time.
    Post-processing to order them in counter-clockwise order takes
    $O(n)$ time.
\end{solution}
For ease of use, we define the primitive operation
$\Callp{CCW}{p, q, r}$, which returns $\top$ if
$p \to q \to r$ is a counter-clockwise turn, and $\bot$ otherwise.

\begin{solution}[Naive -- Jarvis' march]
    The bottom-most point $p_0$ is on the convex hull.
    The next vertex can be characterized as the point which makes the
    smallest angle with the horizontal drawn through $p_0$.

    Once $p_{i-1}$ and $p_i$ have been computed, the next vertex $p_{i+1}$
    is the vertex $p_j$ such that $\angle p_{i-1} p_i p_j$ is maximized.
    Continue this until $p_i = p_0$.

    This takes $O(n^2)$ time.
    In terms of the size of the hull, the time taken is $O(nh)$.
    \begin{algo}
        \Fn{Hull-Angle}{$P$}
            \State $p_0 \gets$ bottom-most point in $P$
            \State $p \gets \Call{Next-Point}{p_0}$
            \State report $p_0$, $p$
            \While{$p \neq p_0$}
                \State $p \gets \Call{Next-Point}{P \setminus \set p, p}$
                \State report $p$
            \EndWhile
        \EndFn
    \end{algo}
\end{solution}
\begin{exercise}
    Write the $\textsc{Next-Point}$ function using only the \textsc{CCW}
    primitive (do not compute angles/inner products).
\end{exercise}
\begin{solution}
    \begin{algo}
        \Fn{Next-Point}{$P'$, $p$}
            \State $q \gets$ any point from $P'$
            \For{$q' \in P'$}
                \If{not $\Call{CCW}{p, q, q'}$}
                    \State $q \gets q'$
                \EndIf
            \EndFor
            \State \Return $q$
        \EndFn
    \end{algo}
\end{solution}

\begin{solution}[Graham's scan]
    Pick an arbitrary point $p_0$.
    Sort the points by angle with $p_0$.
    Traverse the sorted list, adding points to the hull if they make a
    counter-clockwise turn with the last two points on the hull.
    Otherwise, remove points from the hull until the turn is
    counter-clockwise.

    The sorting step takes $O(n \log n)$ time.
    The scan step takes $O(n)$ time.
\end{solution}
