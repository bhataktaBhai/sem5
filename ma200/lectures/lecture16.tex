\lecture{2024-09-11}{Inverse function theorem -- the conclusion}
    Similarly, \begin{align*}
        \norm{\phi_y(x_1) - \phi_y(x_2)}
        &= \norm{x_1 - x_2 - A^{-1}(f(x_1) - f(x_2))} \\
        &\ge \norm{x_1 - x_2} - \norm{A^{-1}(f(x_1) - f(x_2))} \\
        &\ge \norm{x_1 - x_2} - \norm{A^{-1}} \norm{f(x_1) - f(x_2)}.
    \end{align*}
    Combining this with \cref{eq:phi-lip} gives \begin{equation}
        \norm{x_1 - x_2} \le 2 \norm{A^{-1}} \norm{f(x_1) - f(x_2)}.
        \label{eq:g-lip}
    \end{equation}
    This shows that $g$ is (Lipschitz) continuous.

    Let $y_0 = f(x_0)$ and $y_0 + k = f(x_0 + h)$.
    Let $r(k) = g(y_0 + k) - g(y_0) - f'(g(y_0))^{-1}(k)$.
    Then \begin{align*}
        r(k) &= h - {f'(x_0)}^{-1}(k) \\
        &= -{f'(x_0)}^{-1}(k - f'(x_0)h) \\
        &= -{f'(x_0)}^{-1}(f(x_0 + h) - f(x_0) - f'(x_0)h) \\
        &= -{f'(x_0)}^{-1} o(h).
    \end{align*} Let $T = -{f'(x_0)}^{-1}$.
    Then \begin{align*}
        \frac{\norm{r(k)}}{\norm k} &= \frac{\norm{T (o(h))}}{\norm k} \\
        &\le \norm T \frac{\norm{o(h)}}{\norm k}
    \intertext{\Cref{eq:g-lip} gives $\norm h \le 2 \norm{A^{-1}} \norm k$,
    so} 
        \frac{\norm{r(k)}}{\norm k}
            &\le 2 \norm T \norm{A^{-1}} \frac{\norm{o(h)}}{\norm h} \to 0
    \end{align*}
    since $h \to 0$ as $k \to 0$ (by \labelcref{eq:g-lip}).
    Thus $r(k) = o(k)$, proving that $g$ is differentiable at $y_0$
    with \[
        g'(y_0) = f'(g(y_0))^{-1}.
    \] Since $g$ and $f'$ are continuous, $g$ is $C^1$. \[
        f(V) \xrightarrow{g} V \xrightarrow{f'} \GL_n(\R)
        \xrightarrow{(\cdot)^{-1}} \GL_n(\R). \qedhere
    \]
\end{proof}

\begin{corollary} \label{thm:ift:every}
    Let $f$ be as in \cref{thm:ift}, but suppose that $f'(x)$ is invertible
    for every $x \in U$.
    Then
    \begin{enumerate}
        \item $f$ is locally injective. \label{thm:ift:every:locally-11}
        \item $f$ is an open map. \label{thm:ift:every:open-map}
    \end{enumerate}
\end{corollary}
\begin{proof}
    \labelcref{thm:ift:every:locally-11} is only a restatement of
    \cref{thm:ift}~\labelcref{thm:ift:locally-11}.

    For \labelcref{thm:ift:every:open-map}, let $E \subseteq U$ be open.
    For each $a \in E$, there exists an open set $V_a \subseteq E$
    containing $a$ such that $f(V_a)$ is open.
    Then $f(E) = \bigcup_{a \in E} f(V_a)$ is open.
\end{proof}

Let $U \subopeneq \R^n$ and $f\colon U \to \R^m$ be differentiable.
Let $f = (f_1, \dots, f_m)$ and write \[
    {[f'(x)]}_{i,j} = {[D_j f_i(x)]}_{i,j}.
\] $f'\colon U \to L(\R^n, \R^m) \cong M_{m \times n}(\R)$.
This has a normed linear space structure itself.
Thus we can talk about differentiability of $f'$.
\begin{definition*}
    Let $U \subopeneq \R^n$.
    Write $f^{(k)}$ for the $k$th derivative of $f$, with $f^{(0)} = f$.
    Then \[
        C^k(U, \R^m) = \sset{g\colon U \to \R^m}{g, g', g'', \dots, g^{(k)}
        \text{ exist and are continuous}}.
    \]
\end{definition*}
Note that for $k \ge 1$, $g \in C^k(U, \R^m)$ iff
$g' \in C^{k-1}(U, L(\R^n, \R^m))$.

Also recall that $f \in C^1(U, \R^k)$ iff $D_j f_i \in C^1(U, \R)$ for all
$i, j$.

Thus $f' \in C^1(U, L(\R^n, \R^m))$ iff $D_k D_j f_i \in C^1(U, \R)$ for all
$i \in [m]$ and $j, k \in [n]$.

Thus $f \in C^2(U, \R^m)$ iff all second order partial derivatives of $f$
exist and are continuous.

\begin{exercise*}
    Compute the second derivative of $X \mapsto X^{-1}$ on $\GL_n(\R)$.
\end{exercise*}
\begin{solution}
    End of the next lecture.
\end{solution}
