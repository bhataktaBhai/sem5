\documentclass[12pt]{article}
% \usepackage{parskip}
\usepackage{lmodern} % https://tex.stackexchange.com/a/58088/295755
\renewcommand\bfdefault{b}
\usepackage{microtype}

\usepackage{amsmath}
\newcommand\yesnumber{\stepcounter{equation}\tag{\theequation}}

\ifundef{\chapter}{}{\providecommand\thechapter{\Roman{chapter}}}

\usepackage{marginnote}
\usepackage[en-GB,calc]{datetime2}
\usepackage{calc}
\usepackage{xifthen}
\usepackage{tocloft}

% https://tex.stackexchange.com/a/454168
\newcommand\monthday[1]{\DTMmonthname{\DTMfetchmonth{#1}}~\number\DTMfetchday{#1}}

\newlistof{lecture}{lec}{Lectures} % what is this file extension business?
\makeatletter
\setlength\marginparwidth{1in}
\newcommand*{\lecture}[3][]{
    \ifthenelse{\isempty{#1}}{%
        \refstepcounter{lecture}%
    }{%
        \setcounter{lecture}{#1}%
    }%
    \DTMsavedate{lecdate}{#2}%
    \def\lecdow{\DTMweekdayname{\DTMfetchdow{lecdate}}}
    \def\lecshortdow{\DTMshortweekdayname{\DTMfetchdow{lecdate}}}
    \def\lecmonth{\DTMmonthname{\DTMfetchmonth{lecdate}}}
    \def\lecday{\number\DTMfetchday{lecdate}}
    % \marginpar{\raggedright\small \textsf{\textbf{Lecture \thelecture.}%
    %             \footnotesize\DTMusedate{lecdate}}}%
    \marginnote{\raggedright\small%
        \textsf{{\textbf{Lecture \thelecture.}} \\
        \footnotesize\lecdow\\\lecmonth\ \lecday}}%
    \ifthenelse{\isempty{#3}}{%
        \addcontentsline{lec}{lecture}{\protect\numberline{\thelecture}%
        \lecshortdow, \lecmonth\ \lecday}%
        \def\@lecture{Lecture \thelecture}%
    }{%
        \addcontentsline{lec}{lecture}{\protect\numberline{\thelecture}%
        \makebox[\widthof{Mon,}][l]{\lecshortdow,}\ \makebox[\widthof{September 00}][l]{\lecmonth\ \lecday} #3}%
        \def\@lecture{Lecture \thelecture: #3}%
    }%
    \par%
}
\g@addto@macro\normalsize{%
  \setlength\abovedisplayskip{7pt}%
  \setlength\belowdisplayskip{7pt}%
  \setlength\abovedisplayshortskip{1pt}%
  \setlength\belowdisplayshortskip{1pt}%
}
\makeatother

\usepackage[twoside]{fancyhdr}
\setlength{\headheight}{15pt}
\pagestyle{fancy}
\fancyhf{}
% \fancyhead[r]{\thepage}
\makeatletter
\fancyhead[LE,RO]{\thepage}
\fancyhead[RE]{\textbf{\nouppercase\leftmark}}
\fancyhead[LO]{\nouppercase\rightmark}
\providecommand\@lecture{}
\fancyfoot[R]{\small\@lecture}
\makeatother

% homeworks
\newlistof{hw}{hw}{Assignments} % counter `assignment' already defined
\makeatletter
\newcommand*{\assignment}[5][]{%[number]{file}{date posted}{date due}{date quiz}
    \ifthenelse{\isempty{#1}}{%
        \refstepcounter{hw}%
        \stepcounter{assignment}%
    }{%
        \setcounter{hw}{#1}%
        \setcounter{assignment}{#1}%
    }%
    \pagebreak
    \ifthenelse{\isempty{#3}}{}{\DTMsavedate{posted}{#3}}%
    \ifthenelse{\isempty{#4}}{}{\DTMsavedate{due}{#4}}%
    \ifthenelse{\isempty{#5}}{}{\DTMsavedate{quiz}{#5}}%
    \section*{Assignment \thehw}
    \ifthenelse{\isempty{#4}}{
        \ifthenelse{\isempty{#5}}{
            \ifthenelse{\isempty{#3}}{
                \def\hw@toc{}
            }{
                \def\hw@toc{posted \monthday{up}}
            }
        }{
            \def\hw@toc{quiz \monthday{quiz}}
        }
    }{
        \def\hw@toc{due \monthday{due}}
    }
    \addcontentsline{hw}{hw}{\protect\numberline{\thehw}\hw@toc}\par
    \marginpar{\raggedright\footnotesize\textsf{%
        \ifthenelse{\isempty{#3}}{}{\makebox[\widthof{quiz}][l]{up} \monthday{posted} \\}%
        \ifthenelse{\isempty{#4}}{}{\makebox[\widthof{quiz}][l]{due} \monthday{due} \\}%
        \ifthenelse{\isempty{#5}}{}{quiz \monthday{quiz}}%
    }}
    \def\@lecture{Assignment \thehw\ifx\hw@toc\empty{}\else\ --- \hw@toc\fi}
    \input{#2}
    \newpage
}
\makeatother

\usepackage{amsmath}
\usepackage{amsthm}
\usepackage[dvipsnames]{xcolor}
\colorlet{exercise}{cyan!70!black}
\colorlet{solved}{green!40!black}
\colorlet{self_proof}{blue!70!black}
\colorlet{Red}{red!80!black}

% Gilles Castel's theorems
\newtheoremstyle{mddefinition}% <name>
  {-.25\topsep}%                 <space above>
  {-.25\topsep}%                         <space below>
  {\normalfont}%              <body font>
  {}%                         <indent amount>
  {\bfseries}%                <theorem head font>
  {.}%                        <punctuation after theorem head>
  {.5em}%                     <space after theorem head>
  {}%                         <theorem head spec>
\newtheoremstyle{mdplain}% <name>
  {-.25\topsep}%                 <space above>
  {-.25\topsep}%                         <space below>
  {\itshape}%                 <body font>
  {}%                         <indent amount>
  {\bfseries}%                <theorem head font>
  {.}%                        <punctuation after theorem head>
  {.5em}%                     <space after theorem head>
  {}%                         <theorem head spec>

\usepackage[framemethod=Tikz]{mdframed}
\mdfsetup{skipbelow=0pt}
\mdfdefinestyle{axiomstyle}{
    outerlinewidth = 1.5,
    roundcorner = 10,
    leftmargin = 15,
    rightmargin = 15,
    backgroundcolor = yellow!7
}
\mdfdefinestyle{defstyle}{
    outerlinewidth = 1,
    roundcorner = 2,
    leftmargin = 7,
    rightmargin = 7,
    backgroundcolor = green!5
}
\mdfdefinestyle{thmstyle}{
    outerlinewidth = 1,
    roundcorner = 8,
    leftmargin = 7,
    rightmargin = 7,
    backgroundcolor = cyan!5
}
\mdfdefinestyle{lemmastyle}{
    outerlinewidth = 1.5,
    roundcorner = 10,
    leftmargin = 7,
    rightmargin = 7,
    backgroundcolor = yellow!10
}
\ifundef\chapter{%
    \providecommand\theoremnumberwithin{section}
    \theoremstyle{mddefinition}
    \newmdtheoremenv[nobreak=true, style=axiomstyle]{axiom}{Axiom}[section]
    \theoremstyle{plain}
    \newtheorem{theorem}{Theorem}[\theoremnumberwithin]

    \newcounter{assignment}
    \theoremstyle{plain}
    \newtheorem{problem}{Problem}
    \theoremstyle{mddefinition}
    \newmdtheoremenv[nobreak=true, outerlinewidth=0.7]{problem*}[problem]{Problem}
}{
    \providecommand\theoremnumberwithin{chapter}
    \theoremstyle{mddefinition}
    \newmdtheoremenv[nobreak=true, style=axiomstyle]{axiom}{Axiom}[chapter]
    \theoremstyle{plain}
    \newtheorem{theorem}{Theorem}[\theoremnumberwithin]

    \newcounter{assignment}
    \theoremstyle{plain}
    \newtheorem{problem}{Problem}[assignment]
    \theoremstyle{mddefinition}
    \newmdtheoremenv[nobreak=true, outerlinewidth=0.7]{problem*}[problem]{Problem}
}
\theoremstyle{mddefinition}
\newmdtheoremenv[nobreak=true, style=defstyle]{definition*}[theorem]{Definition}

\theoremstyle{mdplain}
\newmdtheoremenv[nobreak=true, style=thmstyle]{theorem*}[theorem]{Theorem}
\newmdtheoremenv[nobreak=true]{proposition*}[theorem]{Proposition}
\newmdtheoremenv[nobreak=true]{lemma*}[theorem]{Lemma}
\newmdtheoremenv[nobreak=true]{corollary*}[theorem]{Corollary}
\newmdtheoremenv[nobreak=true, style=thmstyle]{fact*}[theorem]{Fact}
\newmdtheoremenv[nobreak=true]{exercise*}[theorem]{Exercise}
\newmdtheoremenv[nobreak=true]{question*}[theorem]{Question}

\theoremstyle{definition}
\newtheorem{definition}[theorem]{Definition}

\theoremstyle{plain}
\newtheorem{proposition}[theorem]{Proposition}
\newtheorem{lemma}[theorem]{Lemma}
\newtheorem{corollary}[theorem]{Corollary}
\newtheorem{fact}[theorem]{Fact}
\newtheorem{exercise}[theorem]{Exercise}
\newtheorem{question}[theorem]{Question}

\theoremstyle{remark}
\newtheorem*{remark}{Remark}
\newtheorem*{remarkx}{Remarks}
\newtheorem*{example}{Example}
\newtheorem*{examplex}{Examples}
\newtheorem*{idea}{Idea}
%%%% HAXXXXXX %%%%
% \def\innerqed{\qedsymbol}
% \def\outerqed{$\blacksquare$}
\let\oldproof\proof
\let\endoldproof\endproof
\newenvironment{solution}[1][]
  {\renewcommand\qedsymbol{$\blacksquare$}%
  \begin{oldproof}[Solution\ifx&#1&\else\ (#1)\fi]}
  {\end{oldproof}}
\newenvironment{answer}
  {\renewcommand\qedsymbol{$\blacksquare$}\begin{oldproof}[Answer]}
  {\end{oldproof}}
\renewenvironment{proof}
  {\renewcommand\qedsymbol{$\blacksquare$}\begin{oldproof}}
  {\end{oldproof}}
\newenvironment{subproof}[1][Subproof]{%
  \renewcommand{\qedsymbol}{$\square$}\begin{oldproof}[#1]}
  {\end{oldproof}}
\newtheorem*{notation}{Notation}
\newtheorem*{claim}{Claim}

\usepackage{hyperref}
\usepackage[noabbrev]{cleveref}

% <cref>
\crefname{theorem}{theorem}{theorems}
\crefname{proposition}{proposition}{propositions}
\crefname{lemma}{lemma}{lemmas}
\crefname{corollary}{corollary}{corollaries}
\crefname{axiom}{axiom}{axioms}
\crefname{definition}{definition}{definitions}
\crefname{problem}{problem}{problems}
\crefname{exercise}{exercise}{exercises}
\crefname{fact}{fact}{facts}
\crefname{question}{question}{questions}
\crefname{remark}{remark}{remarks}
\crefname{example}{example}{example}
\crefname{notation}{notation}{notations}
\crefname{claim}{claim}{claims}
% \crefname{section}{\S}{\S\S}
\crefname{theorem*}{theorem}{theorems}
\crefname{proposition*}{proposition}{propositions}
\crefname{lemma*}{lemma}{lemmas}
\crefname{corollary*}{corollary}{corollaries}
\crefname{definition*}{definition}{definitions}
\crefname{problem*}{problem}{problems}
\crefname{exercise*}{exercise}{exercises}
\crefname{fact*}{fact}{facts}
\crefname{question*}{question}{questions}
% </cref>

% <hyperlinks>
\hypersetup{colorlinks,
    linkcolor={blue},
    citecolor={blue!50!black},
    urlcolor={blue!80!black}}
% </hyperlinks>

\usepackage[shortlabels]{enumitem}
% change default label for enumerate, and fix long labels popping out
\setenumerate{label*=(\roman*),ref=(\roman*),leftmargin=*}
% casework list using https://tex.stackexchange.com/a/30035
\newcounter{casecount}
\newlist{casework}{description}{1}
\setlist[casework]{%
  before={\setcounter{casecount}{0}%
      \renewcommand*\thecasecount{\arabic{casecount}}}%
  ,font=\bfseries Case \stepcounter{casecount}\thecasecount:
}

\newenvironment{examples}[1][]
{\begin{examplex}[#1]\leavevmode\begin{itemize}}{\end{itemize}\end{examplex}}
\newenvironment{remarks}[1][]
{\begin{remarkx}[#1]\leavevmode\begin{itemize}}{\end{itemize}\end{remarkx}}

% omg this is so HaXy
% \renewenvironment{proof}[1][\proofname]{{\it\bfseries #1. }}{\qed}
% \providecommand{\qedsymbol}{\openbox}
% \makeatletter
% \renewenvironment{proof}[1][\proofname]{\par
%   \pushQED{\qed}%
%   \normalfont \topsep6\p@\@plus6\p@\relax
%   \trivlist
%   \item[\hskip\labelsep
%         \itshape\bfseries%this is the change (boldface instead of italics)
%         % \fontseries{bx}\selectfont
%     #1\@addpunct{.}]\ignorespaces
% }{%
%   \popQED\endtrivlist\@endpefalse
% }
\makeatother
\crefname{enumi}{part}{parts}
\crefname{enumii}{part}{parts}
\crefname{enumiii}{part}{parts}
% \setlist[itemize]{itemsep=2pt}
\newcounter{dummy}
\makeatletter
\newcommand\myitem[1][]{\item[#1]\refstepcounter{dummy}\def\@currentlabel{#1}}
\makeatother

\makeatletter
\newcommand*{\refifdef}[3]{%label,command,fallback
    \@ifundefined{r@#1}{#3}{#2{#1}}%
}
\makeatother

\newcommand\ie{\textit{i.e.}}
\newcommand\eg{\textit{e.g.}}
\usepackage{physics}

\usepackage{amsmath}
\usepackage{amssymb}
\usepackage{mathrsfs} % for \mathscr
\usepackage{bm} % for \bm
\usepackage{booktabs}

% undefine \abs and \norm
\let\abs\relax
\let\norm\relax

\usepackage{mathtools} % for delimiters and \coloneqq
\DeclarePairedDelimiter{\paren}{(}{)}
\DeclarePairedDelimiter{\brk}{[}{]}
\DeclarePairedDelimiter{\set}{\{}{\}}
\DeclarePairedDelimiter{\abs}{\lvert}{\rvert}
\DeclarePairedDelimiter{\norm}{\lVert}{\rVert}
\DeclarePairedDelimiter{\floor}{\lfloor}{\rfloor}
\DeclarePairedDelimiter{\ceil}{\lceil}{\rceil}
\DeclarePairedDelimiter{\angled}{\langle}{\rangle}
% \DeclarePairedDelimiterX{\innerp}[2]{\langle}{\rangle}{#1,\,#2}
% \DeclarePairedDelimiterX{\outerp}[2]{\langle}{\rangle}{#1\otimes#2}
% \DeclarePairedDelimiterX{\braket}[3]{\langle}{\rangle}%
% {#1\,\delimsize\vert\,\mathopen{}#2\,\delimsize\vert\,\mathopen{}#3}
\DeclarePairedDelimiterX{\innerp}[2]{\langle}{\rangle}{#1,\,#2}
% \DeclarePairedDelimiterX{\outerp}[2]{\langle}{\rangle}{#1\otimes#2}
\DeclarePairedDelimiterX{\outerp}[2]{\vert}{\vert}%
{#1\delimsize\rangle\delimsize\langle\mathopen{}#2}
\let\braket\relax
\DeclarePairedDelimiterX{\braket}[3]{\langle}{\rangle}%
{#1\,\delimsize\vert\,\mathopen{}#2\,\delimsize\vert\,\mathopen{}#3}

\renewcommand\O{\ensuremath{\varnothing}}
\newcommand\N{\ensuremath{\mathbb{N}}}
\newcommand\Z{\ensuremath{\mathbb{Z}}}
\newcommand\Q{\ensuremath{\mathbb{Q}}}
\newcommand\R{\ensuremath{\mathbb{R}}}
\newcommand\C{\ensuremath{\mathbb{C}}}
% \renewcommand\P{\ensuremath{\mathbb{P}}}

% fix spacing for \forall and \exists
% \let\oldforall\forall
% \renewcommand{\forall}{\oldforall \, }
% \let\oldexist\exists
% \renewcommand{\exists}{\oldexist \: }
\newcommand\unique{\exists!}
\newcommand\lxor{\oplus}

\providecommand{\dd}{\,\mathrm{d}}

\newcommand\mcA{\ensuremath{\mathcal{A}}}
\newcommand\mcB{\ensuremath{\mathcal{B}}}
\newcommand\mcC{\ensuremath{\mathcal{C}}}
\newcommand\mcD{\ensuremath{\mathcal{D}}}
\newcommand\mcE{\ensuremath{\mathcal{E}}}
\newcommand\mcF{\ensuremath{\mathcal{F}}}
\newcommand\mcG{\ensuremath{\mathcal{G}}}
\newcommand\mcH{\ensuremath{\mathcal{H}}}
\newcommand\mcI{\ensuremath{\mathcal{I}}}
\newcommand\mcJ{\ensuremath{\mathcal{J}}}
\newcommand\mcK{\ensuremath{\mathcal{K}}}
\newcommand\mcL{\ensuremath{\mathcal{L}}}
\newcommand\mcM{\ensuremath{\mathcal{M}}}
\newcommand\mcN{\ensuremath{\mathcal{N}}}
\newcommand\mcO{\ensuremath{\mathcal{O}}}
\newcommand\mcP{\ensuremath{\mathcal{P}}}
\newcommand\mcQ{\ensuremath{\mathcal{Q}}}
\newcommand\mcR{\ensuremath{\mathcal{R}}}
\newcommand\mcS{\ensuremath{\mathcal{S}}}
\newcommand\mcT{\ensuremath{\mathcal{T}}}
\newcommand\mcU{\ensuremath{\mathcal{U}}}
\newcommand\mcV{\ensuremath{\mathcal{V}}}
\newcommand\mcW{\ensuremath{\mathcal{W}}}
\newcommand\mcX{\ensuremath{\mathcal{X}}}
\newcommand\mcY{\ensuremath{\mathcal{Y}}}
\newcommand\mcZ{\ensuremath{\mathcal{Z}}}

%%%% WIDE BAR THAT IS JUST THE RIGHT LENGTH %%%%
%% FROM https://tex.stackexchange.com/a/60253 %%
\makeatletter
\let\save@mathaccent\mathaccent
\newcommand*\if@single[3]{%
  \setbox0\hbox{${\mathaccent"0362{#1}}^H$}%
  \setbox2\hbox{${\mathaccent"0362{\kern0pt#1}}^H$}%
  \ifdim\ht0=\ht2 #3\else #2\fi
  }
%The bar will be moved to the right by a half of \macc@kerna, which is computed by amsmath:
\newcommand*\rel@kern[1]{\kern#1\dimexpr\macc@kerna}
%If there's a superscript following the bar, then no negative kern may follow the bar;
%an additional {} makes sure that the superscript is high enough in this case:
\newcommand*\widebar[1]{\@ifnextchar^{{\wide@bar{#1}{0}}}{\wide@bar{#1}{1}}}
%Use a separate algorithm for single symbols:
\newcommand*\wide@bar[2]{\if@single{#1}{\wide@bar@{#1}{#2}{1}}{\wide@bar@{#1}{#2}{2}}}
\newcommand*\wide@bar@[3]{%
  \begingroup
  \def\mathaccent##1##2{%
%Enable nesting of accents:
    \let\mathaccent\save@mathaccent
%If there's more than a single symbol, use the first character instead (see below):
    \if#32 \let\macc@nucleus\first@char \fi
%Determine the italic correction:
    \setbox\z@\hbox{$\macc@style{\macc@nucleus}_{}$}%
    \setbox\tw@\hbox{$\macc@style{\macc@nucleus}{}_{}$}%
    \dimen@\wd\tw@
    \advance\dimen@-\wd\z@
%Now \dimen@ is the italic correction of the symbol.
    \divide\dimen@ 3
    \@tempdima\wd\tw@
    \advance\@tempdima-\scriptspace
%Now \@tempdima is the width of the symbol.
    \divide\@tempdima 10
    \advance\dimen@-\@tempdima
%Now \dimen@ = (italic correction / 3) - (Breite / 10)
    \ifdim\dimen@>\z@ \dimen@0pt\fi
%The bar will be shortened in the case \dimen@<0 !
    \rel@kern{0.6}\kern-\dimen@
    \if#31
      \overline{\rel@kern{-0.6}\kern\dimen@\macc@nucleus\rel@kern{0.4}\kern\dimen@}%
      \advance\dimen@0.4\dimexpr\macc@kerna
%Place the combined final kern (-\dimen@) if it is >0 or if a superscript follows:
      \let\final@kern#2%
      \ifdim\dimen@<\z@ \let\final@kern1\fi
      \if\final@kern1 \kern-\dimen@\fi
    \else
      \overline{\rel@kern{-0.6}\kern\dimen@#1}%
    \fi
  }%
  \macc@depth\@ne
  \let\math@bgroup\@empty \let\math@egroup\macc@set@skewchar
  \mathsurround\z@ \frozen@everymath{\mathgroup\macc@group\relax}%
  \macc@set@skewchar\relax
  \let\mathaccentV\macc@nested@a
%The following initialises \macc@kerna and calls \mathaccent:
  \if#31
    \macc@nested@a\relax111{#1}%
  \else
%If the argument consists of more than one symbol, and if the first token is
%a letter, use that letter for the computations:
    \def\gobble@till@marker##1\endmarker{}%
    \futurelet\first@char\gobble@till@marker#1\endmarker
    \ifcat\noexpand\first@char A\else
      \def\first@char{}%
    \fi
    \macc@nested@a\relax111{\first@char}%
  \fi
  \endgroup
}
\makeatother
\let\what\widehat
\let\wtld\widetilde
\let\wbar\widebar
\let\ubar\underline

\DeclareMathOperator\sgn{sgn}

\let\oldleft\left
\let\oldright\right
\renewcommand{\left}{\mathopen{}\mathclose\bgroup\oldleft}
\renewcommand{\right}{\aftergroup\egroup\oldright}


\setenumerate{label=(\alph*)}
\setenumerate[2]{label=(\roman*)}

\title{Assignment 2}
\author{Naman Mishra}
\date{21 August, 2024}

\begin{document}
\maketitle

% Problem 1
% For convenience, we will use the notation
% $f(x) \cdot [P(x)]$ to denote the function \[
%     f(x) \cdot [P(x)] = \begin{cases}
%         f(x) & \text{if } P(x) \text{ holds}, \\
%         0 & \text{otherwise}.
%     \end{cases}
% \]
\begin{problem} \label{prb:cont}
    Determine whether in each case the function $f\colon \R^2 \to \R$ is
    continuous or not.
    \begin{enumerate}
        \item $f(x, y) = [(x, y) \ne 0]\; \frac{x \sin^2 y}{x^2 + y^2}$
        \item $f(x, y) = [(x, y) \ne 0]\; \frac{\sin(x^2 + y^2)}{x^2 + y^2}
            + [(x, y) = 0]\; 1$
        \item $f(x, y) = [(x, y) \ne 0]\; \frac{xy}{\sqrt{x^2 + y^2}}$
        \item $f(x, y) = [(x, y) \ne 0]\; \frac{xy}{x^2 + y^2}$
            \label{prb:cont:sincos}
        \item $f(x, y) = [(x, y) \ne 0]\; \frac{xy^2}{x^2 + y^4}$
            \label{prb:cont:square-homo}
    \end{enumerate}
\end{problem}
\begin{solution}
    All of these are sums, products, quotients, and compositions of
    continuous functions in $\R^2 \setminus \set{(0, 0)}$.
    Thus we only need to check continuity at $(0, 0)$.
    \begin{enumerate}
        \item Continuous. \[
            \abs{f(x, y)} \le \abs x \frac{\sin^2 y}{y^2} \le \abs x \to 0
        \]
        \item Continuous. % TODO
        \item Continuous.
        \[
            \abs{f(x, y)} \le \frac{\abs{xy}}{\abs{y}} = \abs x \to 0
        \]
        \item Not continuous. This is homogenous of degree 0, but not
        constant on the unit circle.
        \item Not continuous.
        \[
            f(t^2, t) = \frac{t^4}{t^4 + t^4} \to \frac12 \qedhere
        \]
    \end{enumerate}
\end{solution}

% Problem 2
\begin{problem}
    Let $f\colon \R^2 \to \R$ be as in
    \cref{prb:cont}\labelcref{prb:cont:sincos}.
    Show that $D_1 f$ and $D_2 f$ exist at every point in $\R^2$,
    although the function is not continuous at $(0, 0)$.
\end{problem}
\begin{solution}
    They obviously exist in $\R^2 \setminus \set{(0, 0)}$.
    At $(0, 0)$, we have \[
        D_1 f(0, 0) = \lim_{h \to 0} \frac{f(h, 0) - f(0, 0)}{h} = 0.
    \] Similarly $D_2 f(0, 0) = 0$.
\end{solution}

% Problem 3
\begin{problem}
    Let $f\colon \R^2 \to \R$ be given by \[
        f(x, y) = \frac{xy(x^2 - y^2)}{x^4 + y^4} \cdot [(x, y) \ne 0].
    \]
    Show that $D_1 f$ and $D_2 f$ exist at every point in $\R^2$,
    although the function is not continuous at $(0, 0)$.
\end{problem}
\begin{solution}
    Obvious except at $(0, 0)$.
    Also obvious at $(0, 0)$.

    For continuity, notice again that this is homogenous of degree 0,
    but its value at $(1, 0)$ is different from at
    $(\cos 1, \sin 1)$.
\end{solution}

% Problem 4
\begin{problem}
    Let $f\colon \R^2 \to \R$ be as in
    \cref{prb:cont}\labelcref{prb:cont:square-homo}.
    Show that for every $v \in \R^2$, the directional derivative
    $D_vf$ exists at every point of $\R^2$, although the function is not
    continuous at $(0, 0)$.
\end{problem}
\begin{solution}
    Let $v = (a, b) \ne 0$.
    Obvious but at the origin.
    \begin{align*}
        D_vf(0, 0) &= \lim_{t \to 0} \frac{f(t)}{t} \\
        &= \lim_{t \to 0} \frac{ab^2}{a^2 + b^4 t^2} \\
        &= \frac{ab^2}{a^2 + b^4}. \qedhere
    \end{align*}
\end{solution}

% Problem 5
\begin{problem}
    Let $f\colon \R^2 \to \R$ be given by \[
        f(x, y) = \frac{xy^3}{x^2 + y^6} \cdot [(x, y) \ne 0].
    \]
    Show that for every $v \in \R^2$, the directional derivative
    $D_vf$ exists at every point of $\R^2$, although the function is not
    continuous at $(0, 0)$.
\end{problem}
\begin{solution}
    It's the same thing.
    \begin{align*}
        D_vf(0, 0) &= \lim_{t \to 0} \frac{ab^3 t}{a^2 + b^6 t^4} \\
        &= 0.
    \end{align*}
    Continuity is similarly disproven by $(t^3, t)$.
\end{solution}

% Problem 6
\begin{problem}
    Let $U$ be an open subset of $\R^n$ and $f\colon U \to \R^m$ be such
    that \[
        f(x) \coloneq (f_1(x), \dots, f_m(x)) \quad \text{for } x \in U.
    \]
    \begin{enumerate}
        \item Suppose that $f$ is differentiable at $a \in U$.
        Show that each $f_k$ is differentiable at $a$ for $k \in [m]$,
        with \[
            f_k'(a)(v) = \innerp{f'(a)(v)}{e_k}
            \quad \text{for } v \in \R^n.
        \]
        \item Suppose that each $f_k\colon U \to \R$ is differentiable at
        $a \in U$ for $k \in [m]$.
        Prove that $f$ is differentiable at $a$ with \[
            f'(a)(v) = (f_1'(a)(v), \dots, f_m'(a)(v))
            \quad \text{for } v \in \R^n. \qedhere
        \]
    \end{enumerate}
\end{problem}
\begin{solution} \leavevmode
    \begin{enumerate}
        \item Let $f'(a) = T \in L(\R^n, \R^m)$.
        Let $v = (v_1, \dots, v_n) \in \R^n$.
        Then \begin{align*}
            f_k'(a)(v) &= \lim_{t \to 0} \frac{f_k(a + tv) - f_k(a)}{t} \\
            \implies f_k'(a)(v) - \innerp{T(v)}{e_k} &= \lim_{t \to 0}
            \frac{f_k(a + tv) - f_k(a) - T(v)_k}{t} \\
            &= \lim_{t \to 0} \frac{f(a + tv) - f(a) - T(v)}{t} \dotp e_k \\
            &= 0.
        \end{align*}
        \item Let $f_k(a + v) = f_k(a) + f_k'(a)(v) + \norm v \eps_k(v)$
        where $\eps_k(v) \to 0$ as $v \to 0$.
        Then \begin{align*}
            f(a + v) - f(a) &= \sum_{k = 1}^m (f_k(a + v) - f_k(a)) e_k \\
                &= \sum_{k=1}^m f_k'(a)(v) e_k + \norm v \sum_{k=1}^m
                    \eps_k(v) e_k.
        \end{align*}
        Since $\eps_k(v) \to 0$ for each $k$,
        $(\eps_1(v), \dots, \eps_m(v)) \to 0$.
        Thus \[
            f'(a)(v) = \sum_{k=1}^m f_k'(a)(v) e_k. \qedhere
        \]
    \end{enumerate}
\end{solution}

% Problem 7
\begin{problem} \label{prb:vector}
    Let $U$ be an open subset of $\R^n$.
    Let $h\colon U \to \R^{k+m}$ be given by \[
        h(x) = (f(x), g(x)) \quad \text{for } x \in U,
    \] where $f\colon U \to \R^k$ and $g\colon U \to \R^m$.
    \begin{enumerate}
        \item Suppose that $h$ is differentiable at $a \in U$.
        Show that both $f$ and $g$ are differentiable at $a$, with \[
            h'(a)(v) = (f'(a)(v), g'(a)(v)) \quad \text{for } v \in \R^n.
        \]
        \item Suppose that both $f$ and $g$ are differentiable at $a \in U$.
        Prove that $h$ is differentiable at $a$ with \[
            h'(a)(v) = (f'(a)(v), g'(a)(v)) \quad \text{for } v \in \R^n.
        \]
    \end{enumerate}
\end{problem}
\begin{solution}
    Use the previous problem.
\end{solution}

% Problem 8
\begin{problem}
    Calculate the total derivative of the following maps.
    \begin{enumerate}
        \item Let $f\colon M_n(\R) \to M_n(\R)$ be given by \[
            f(X) \coloneq X^\T.
        \]
        \item Let $f\colon M_n(\R) \to \R$ be given by \[
            f(X) \coloneq \Tr(X).
        \]
        \item Let $f\colon M_n(\R) \to M_n(\R)$ be given by \[
            f(X) \coloneq XX^\T.
        \]
        \item Let $B \in M_n(\R)$ be fixed.
        Let $f\colon M_n(\R) \to M_n(\R)$ be given by \[
            f(X) \coloneq X^\T B X.
        \]
        \item Let $A \in M_n(\R)$ be fixed.
        Let $f\colon \R^n \to \R$ be given by \[
            f(x) = \innerp{Ax}{x}.
        \]
    \end{enumerate}
\end{problem}
\begin{solution} \leavevmode
    \begin{enumerate}
        \item $f(X + H) = X^\T + H^\T + 0$ where $H \mapsto H^\T$ is linear
        and $0 = o(H)$.
        Thus $f'(X)$ is given by \[
            f'(X)(H) = H^\T.
        \]
        \item $f(X + H) = \Tr(X) + \Tr(H) + 0$ where $H \mapsto \Tr(H)$ is
        linear and $0 = o(H)$.
        Thus $f'(X)$ is given by \[
            f'(X)(H) = \Tr(H).
        \]
        \item $f(X + H) = f(X) + XH^\T + HX^\T + HH^\T$
        where $H \mapsto XH^\T + HX^\T$ is linear and
        $HH^\T = o(H)$, since $\norm{HH^\T} \le \norm H \norm {H^\T}
        = \norm H^2$.
        Thus $f'(X)$ is given by \[
            f'(X)(H) = XH^\T + HX^\T.
        \]
        \item $f(X + H) = f(X) + X^\T BH + H^\T BX + H^\T BH$
        where $H \mapsto X^\T BH + H^\T BX$ is linear and
        $H^\T B H = o(H)$ since $\norm{H^\T B H} \le \norm B \norm H^2$.
        Thus $f'(X)$ is given by \[
            f'(X)(H) = X^\T BH + H^\T BX.
        \]
        \item
        $f(x + h) = f(x) + \innerp{Ax}{h} + \innerp{Ah}{x} + \innerp{Ah}{h}$
        where $h \mapsto \innerp{Ah}{x} + \innerp{Ax}{h}$ is linear and
        $\innerp{Ah}{h} = o(h)$ since $\abs{\innerp{Ah}{h}} \le \norm{Ah}
        \norm h \le \norm A \norm h^2$. \qedhere
    \end{enumerate}
\end{solution}

% Problem 9
\begin{problem} \label{prb:product}
    Let $U$ be an open subset of $M_n(\R)$.
    Let $f\colon U \to M_n(\R)$ and $g\colon U \to M_n(\R)$ be two maps
    that are both differentiable at $A \in U$.
    \begin{enumerate}
        \item \label{prb:product:prove}
        Show that the map $\psi\colon U \to M_n(\R)$ given by \[
            \psi(X) \coloneq f(X)g(X)
        \] is differentiable at $A$ and
        $\psi'\colon M_n(\R) \to M_n(\R)$ is given by \[
            \psi'(A)(H) = (f'(A)(H))g(A) + f(A)(g'(A)(H)).
        \]
        \item \label{prb:product:apply}
        Using \cref{prb:product}\labelcref{prb:product:prove}, answer the
        following questions (Don't use power series).
        \begin{enumerate}
            \item Let $f\colon M_n(\R) \to M_n(\R)$ be given by \[
                f(A) \coloneq A^2.
            \] Calculate $f'(A)$.
            \item Let $f\colon \GL_n(\R) \to M_n(\R)$ be given by \[
                f(A) \coloneq A^{-1}.
            \] Calculate $f'(A)$.
            \item Let $f\colon \GL_n(\R) \to M_n(\R)$ be given by \[
                f(A) \coloneq A^{-2}.
            \] Calculate $f'(A)$.
            \item Let $f\colon M_n(\R) \to M_n(\R)$ be given by \[
                f(A) \coloneq A^3.
            \] Calculate $f'(A)$.
            \item Let $f\colon \GL_n(\R) \to M_n(\R)$ be given by \[
                f(A) \coloneq A^{-3}.
            \] Calculate $f'(A)$.
        \end{enumerate}
    \end{enumerate}
\end{problem}
\begin{solution} \leavevmode
    \begin{enumerate}
        \item We have \begin{align*}
            \psi(A + H) &= f(A + H)g(A + H) \\
            &= (f(A) + f'(A)(H) + o(H))(g(A) + g'(A)(H) + o(H)) \\
            &= \psi(A) + f(A)g'(A)(H) + f'(A)(H)g(A) + o(H).
        \end{align*}
        The remaining terms are $o(H)$ because:
        \begin{itemize}
            \item $o(H) \cdot \text{constant} = o(H)$
            \item $o(H) \cdot \text{linear} = o(H)$ since
            the linear term is bounded in a neighbourhood.
            \item $o(H) \cdot o(H) = o(H)$ for the same reason
            (and because obviously).
            \item $\text{linear} \cdot \text{linear} = o(H)$ because \[
                \frac{\norm{T_1(H) T_2(H)}}{\norm H}
                \le \frac{\norm{T_1} \norm H \norm{T_2} \norm H}{\norm H}
                = \norm{T_1} \norm{T_2} \norm H \to 0
            \] for any linear maps $T_1, T_2$.
        \end{itemize}
        Thus $\psi$ is differentiable at $A$ with \begin{align*}
            \psi'(A)(H) = f(A)g'(A)(H) + f'(A)(H)g(A).
            \tag{$*$} \label{eq:product}
        \end{align*}
        \item
        \begin{enumerate}
            \item We know $X \mapsto X$ is differentiable everywhere, with
            the derivative being the identity map.
            Thus using \eqref{eq:product} yields that $f$ is
            differentiable everywhere, and \[
                f'(A)(H) = AH + HA.
            \]
            \item We know that $X \mapsto X^{-1}$ is continuous on
            $\GL_n(\R)$ (\crefifdef{thm:inv-cont-rudin}{last proposition
            regarding matrix norms}).
            Write $(A + H)^{-1} - A^{-1} = -(A + H)^{-1}H A^{-1}$ in
            \begin{align*}
                f(A + H) - f(A) + A^{-1}HA^{-1}
                &= -(A + H)^{-1}H A^{-1} + A^{-1}HA^{-1} \\
                &= (A^{-1} - (A + H)^{-1})HA^{-1} \\
                &= o(H),
            \end{align*}
            since \[
                \frac{\norm*{(A^{-1} - (A + H)^{-1})HA^{-1}}}{\norm H}
                \le \norm*{A^{-1}} \norm*{A^{-1} - (A + H)^{-1}} \to 0
            \] by the continuity of $X \mapsto X^{-1}$.
            Thus $f$ is differentiable everywhere, and \[
                f'(A)(H) = -A^{-1}HA^{-1}.
            \]
            \item Using the previous part and \eqref{eq:product}, we have
            that $f$ is differentiable everywhere with \begin{align*}
                f'(A)(H) &= A^{-1}(-A^{-1}HA^{-1})+(-A^{-1}HA^{-1})A^{-1} \\
                    &= -A^{-2}(HA + AH)A^{-2}.
            \end{align*}
            Alternatively, we can conclude that $f$ is differentiable
            as before, but compute the derivative using the product rule
            on $I = A^2 A^{-2}$.
            Any constant map has derivative $0$, so \begin{gather*}
                (AH + HA) A^{-2} + A^2 f'(A)(H) = 0 \\
                \implies f'(A)(H) = -A^{-2}(AH + HA)A^{-2}.
            \end{gather*}
            \item Since $X \mapsto X^2$ and $X \mapsto X$ are differentiable
            everywhere, \eqref{eq:product} gives that $f$ is differentiable
            everywhere with \[
                f'(A)(H) = A^2H + (AH + HA) A = A^2H + AHA + HA^2.
            \]
            \item Since $X \mapsto X^{-1}$ is differentiable everywhere,
            \eqref{eq:product} twice gives that $f$ is also differentiable
            everywhere.
            Then the product rule on $A^3 A^{-3} = I$ gives \begin{gather*}
                (A^2H + AHA + HA^2)A^{-3} + A^3 f'(A)(H) = 0 \\
                \implies f'(A)(H) = -A^{-3}(A^2H + AHA + HA^2)A^{-3}.
                \qedhere
            \end{gather*}
        \end{enumerate}
    \end{enumerate}
\end{solution}

% Problem 10
\begin{problem}
    Let $U$ be an open subset of $M_n(\R) \times M_n(\R)$.
    Let $F\colon U \to M_n(\R)$ be given by \[
        F(X, Y) \coloneq XY.
    \] Then for any $(A, B) \in U$, calculate $F'(A, B)$.
    Using \cref{prb:vector} and composition rule, give an alternate proof
    of \cref{prb:product}\labelcref{prb:product:prove}.
\end{problem}
\begin{solution}
    We have \begin{align*}
        F(X + H, Y + K) &= (X + H)(Y + K) \\
        &= XY + HY + XK + HK \\
        &= F(X, Y) + HY + XK + o(H, K),
    \end{align*} since $HK$ is the product of two linear maps
    from $(H, K)$ (projections).
    Thus $F$ is differentiable everywhere, and \[
        F'(X, Y)(H, K) = HY + XK.
    \]

    For \cref{prb:product}\labelcref{prb:product:prove}, consider \[
        \psi(X) = F(f(X), g(X)) = (F \circ h)(X),
    \] where $h(X) \coloneq (f(X), g(X))$.
    By \cref{prb:vector}, $h$ is differentiable at $A$ with \[
        h'(A)(H) = (f'(A)(H), g'(A)(H)).
    \] Thus by the composition rule, $\psi$ is differentiable at $A$ with
    \begin{align*}
        \psi'(A)(H) &= F'(h(A))(h'(A)(H)) \\
        &= F'(f(A), g(A))(f'(A)(H), g'(A)(H)) \\
        &= f'(A)(H)g(A) + f(A)g'(A)(H). \qedhere
    \end{align*}
\end{solution}

\end{document}
