\tutorial[2]{2024-08-27}{hmm}

\begin{proof}[Proof of \cref{thm:expectation} (existence)]
    Fix a probability space $(\Omega, \F, P)$.

    We will only consider non-negative random variables.
    We define a \emph{simple function} on $\Omega$ to be a random variable
    whose range is finite.
    For a simple function $X$ taking values $x_1, \dots, x_n$ on
    sets $A_1, \dots, A_n \in \F$, we define the expectation of $X$ to be \[
        \E[X] = \sum_{i=1}^n x_i P(A_i).
    \] For a general random variable $X$, we define the expectation of $X$
    to be \[
        \E[X] = \sup\set{\E[\phi] : 0 \le \phi \le X, \phi \text{ simple}}.
    \]
    We have to show
    \begin{itemize}
        \item $\E[X]$ is well-defined and agrees with the first definition
            when $X$ is simple.
        \item $\E[\1A] = P(A)$ for any $A \in \F$.
        \item $\E[X]$ is linear.
        \item $\E[X] \le \E[Y]$ if $X \le Y$.
    \end{itemize}
    \begin{equation}
        \E\ab[\phi \1{\bigsqcup_{i=1}^\infty A_i}]
            = \sum_{i=1}^\infty \E[\phi \1{A_i}]
            \label{eq:countable-additivity}
    \end{equation}

    Let $\eps > 0$ be arbitrary and \[
        E_n = \set{\omega \in \Omega : X_n(\omega) \ge
                (1 - \eps) \phi(\omega)}.
    \] Note that $E_n \subseteq E_{n+1}$ and
    $\bigcup_{n=1}^\infty E_n = \Omega$.
    That is, $E_n \upto \Omega$.
    Now \begin{align*}
        \E[X_n] &\ge \E[X_n \1{E_n}] \\
            &\ge \E[(1 - \eps) \phi \1{E_n}] \\
            &= (1 - \eps) \E[\phi \1{E_n}]
    \end{align*}
    Since $E_n \upto \Omega$, we have \[
        \lim_{n \to \infty} \E[X_n] \ge (1 - \eps) \E[\phi]
    \] by \cref{eq:countable-additivity}.
\end{proof}

\begin{proposition}[simple function approximation] \label{thm:simple-approx}
    Let $X\colon (\Omega, \F, \P) \to \R$ be a random variable,
    and let $f\colon \R \to \R$.
    Then \[
        \E[f(X)] = \int f(X) \dd \P = \int f(x) \dd \mu,
    \] where $\mu$ is the push-forward measure \begin{align*}
        \mu\colon \R &\to \R \\
        A &\mapsto \P(X^{-1}(A))
    \end{align*}
\end{proposition}
