\section{The inverse function theorem} \label{sec:ift}
\lecture{2024-09-04}{Contractions and the Banach fixed point}

\subsection{Contraction maps} \label{sec:contraction-maps}
\begin{definition*}[contraction map] \label{def:contract}
    Let $X, Y$ be two metric spaces.
    Then $f\colon X \to Y$ is a \emph{contraction map} if
    $d(f(x_1), f(x_2)) < d(x_1, x_2)$ for each $x_1 \ne x_2 \in X$.

    If there is a $k \in [0, 1)$ such that \[
        d(f(x_1), f(x_2)) \le k d(x_1, x_2) \text{ for all } x_1, x_2 \in X,
    \] $f$ is a \emph{strict contraction map}.
\end{definition*}
This is a special case of a Lipschitz map, with Lipschitz constant less
than (for strict contractions) or equal to (for contractions) $1$.

\begin{theorem*}[Banach fixed-point theorem] \label{thm:banach}
    Let $X$ be a complete metric space and $f\colon X \to X$ be a
    strict contraction map.
    Then $f$ has a unique fixed point $z$.

    Furthermore, for any $x_0 \in X$, the sequence $x_{n+1} = f(x_n)$
    converges to $z$.
\end{theorem*}
\begin{proof}
    Let $k \in [0, 1)$ be the contraction factor, and let $x_0 \in X$.
    Then $d(x_{n+1}, x_{n+2}) \le k d(x_n, x_{n+1})$.
    By induction, $d(x_n, x_{n+1}) \le k^n d(x_0, x_1)$.
    Thus $(x_n)_n$ is Cauchy, and since $X$ is complete, it converges to
    some $z \in X$.

    Since $f$ is Lipschitz, \[
        f(z) = f\ab(\lim_{n \to \infty} x_n)
            = \lim_{n \to \infty} f(x_n)
            = \lim_{n \to \infty} x_{n+1}
            = z.
    \] Suppose there were another fixed point $z' \ne z$.
    Then \[
        d(z, z') = d(f(z), f(z')) \le k d(z, z'),
    \] a contradiction.
\end{proof}

\begin{examples}
    \item $x \mapsto x/2$ is a contraction map on $(0, 1)$, but has no fixed
    point.
    This does not contradict the theorem, since $(0, 1)$ is incomplete.
    \item $\cos$ on $[0, \pi/2]$ is a (weak) contraction map,
    but $\cos$ on $[0, \pi/3]$ is a strict contraction map.
    \item $f\colon \R \to \R$ given by $x \mapsto \sqrt{x^2 + 1}$ is a
    (weak) contraction map but not a strict one. \[
        f'(x) = \frac{x}{\sqrt{x^2 + 1}} \implies \abs{f'(x)} < 1
    \] It is not a strict contraction since there are no fixed points.
\end{examples}

Tao states a consequence immediately after the Banach fixed-point theorem,
which we cover as part of the inverse function theorem.
\begin{lemma}
    Let $B(0; r)$ be a ball in $\R^n$ centered at the origin, and let
    $g\colon B(0; r) \to \R^n$ be a map such that $g(0) = 0$ and \[
        \norm{g(x) - g(y)} \le k \norm{x - y}
    \] for all $x, y \in B(0; r)$, for some $k < 1$.
    Then the function $f\colon B(0; r) \to \R^n$ defined by
    $f(x) = x + g(x)$ is injective, and furthermore the image $f(B(0; r))$
    contains the ball $B(0; r - kr)$.
\end{lemma}
\begin{proof}
    For $x, y \in B(0; r)$, $x + g(x) = y + g(y)$ implies
    $\norm{x - y} = \norm{g(y) - g(x)} \le k \norm{x - y}$, so that
    $x - y = 0$.

    Fix a $y \in B(0; r - kr)$.
    We need an $x$ such that $f(x) = y$, or $x = y - g(x)$.
    Thus, we need a fixed point of the map $x \mapsto y - g(x)$.
    Denote this map by $\phi_y$.
    Then for $x \in B(0; r)$, \begin{equation}
        \norm{\phi_y(x)} \le \norm{y} + \norm{g(x)}
            \le r - kr + kr = r. \label{eq:phi-y}
    \end{equation} Thus $\phi_y$ maps $B(0; r)$ to itself.
    However, this is not complete, so we cannot apply the Banach fixed-point
    theorem directly.
    We must first restrict the domain to a smaller closed ball.
    In fact, this also follows from \cref{eq:phi-y}, where
    $\norm{\phi_y(x)} \le kr + \norm y < r$.

    Thus $\phi_y$ is a strict contraction map on $B(0; kr + \norm y)$.
    By the Banach fixed-point theorem, there is a unique fixed point $x$
    such that $x = \phi_y(x)$.
    This $x$ is the desired preimage of $y$.
\end{proof}
