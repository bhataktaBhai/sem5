\chapter*{The course} \label{chp:course}
\lecture{2024-08-02}{}

\chapter{Data science / AI} \label{chp:aids}
How does Siri process the query
    ``Show me a good breakfast restaurant near me''?
\begin{enumerate}
    \item Convert speech to text.
    \item Understand the semantics of the query (for example,
        understand keywords like ``breakfast'', ``restaurant'') and
        formulate a structured query
        (place type = restaurant, meal type = breakfast, rating 3--5,
        distance 0--3 km).
        \begin{itemize}
            \item Also needs your current location!
        \end{itemize}
    \item Search for restaurants filtered by the structure above and rank
        based on some metric
        \begin{itemize}
            \item star rating,
            \item go through all reviews and understand the sentiments
            \item check your own past history, and try to give you a
                personalized recommendation
                \begin{itemize}
                    \item either recommend similar restaurants,
                    \item or recommend something completely different
                        with a message like ``would you like to try
                        something new?''
                \end{itemize}
        \end{itemize}
        or some combination of these (by another model).
\end{enumerate}

\section*{An AI system for predicting cricket} \label{sec:cricket}

The data comes from
\begin{itemize}
    \item past performance of the player;
    \item weather, pitch conditions;
    \item other \emph{factors}.
\end{itemize}

Data is usually stored in a tabular format representing
\begin{itemize}
    \item each factor as a column,
    \item each data point as a row.
\end{itemize}

Another piece of data one could analyze is the net practice \emph{videos}.
An easier version of this is, read the \emph{text} commentary from some
source.
Alternatively, record the \emph{audio} commentary and analyze that.
Could also look at the final pose of a batter once they have completed
a shot as a single \emph{image}.

Thus we can categorize data as
\begin{itemize}
    \item tabular,
    \item timeseries (collection of tabular data),
    \item image (table of pixel values),
    \item video (time series of images),
    \item text (list of tokens),
    \item audio (time series of sound waves, or a
        \href{https://en.wikipedia.org/wiki/Spectrogram}{spectrogram}).
\end{itemize}
