\lecture{2024-08-09}{}

\begin{lemma}
    The standard topology on $\R$ has a countable basis.
\end{lemma}
\begin{proof}
    \[
        \mcB \coloneq \set{(p, q) \in \Q^2 \mid p < q}.
    \] For any interval $(a, b) \subseteq \R$, with $a < b$, there exists
    a decreasing sequence $(p_n)_n \subseteq \Q$ and an increasing sequence
    $(q_n)_n \subseteq \Q$ such that $p_n \downto a$ and $q_n \upto b$.
    Then \[
        (a, b) = \bigcup_{n \in \N} (p_n, q_n). \qedhere
    \]
\end{proof}

\begin{lemma}
    Any basis of $\T'$ is uncountable.
\end{lemma}
\begin{proof}
    For each $x \in \R$ consider the open set $[x, x + 1) \in \T'$.
    Since $\mcB'$ is a basis, there is a $B_x \in \mcB'$ such that
    $x \in B_x \subseteq [x, x + 1)$.
    Note that $\inf B_x = x$.
    Thus $x \mapsto B_x$ is injective ($B_x$ and $B_y$ have different
    infimums for $x \ne y$).
    This shows that $\abs{\mcB'} \ge \abs{\R}$.
\end{proof}
\begin{corollary}
    $\T'$ is strictly finer than $\T$.
\end{corollary}
We will soon see that $\T'$ is not even metrizable.

\begin{definition}[Hausdorff] \label{def:hausdorff}
    A topological space $(X, \T)$ is called \emph{Hausdorff} if for
    any distinct $x, y \in X$, there exist disjoint open sets
    $U, V \in \T$ such that $U \ni x$ and $V \ni y$.
\end{definition}

\begin{lemma}
    Every metric space is Hausdorff.
\end{lemma}
\begin{proof}
    Separate two points by open balls of radius half the distance
    between them.
\end{proof}

\begin{definition}[closed] \label{def:closed}
    A subset $A \subseteq X$ is called \emph{closed} if its
    complement $X \setminus A$ is open.
\end{definition}
\begin{examples}
    \item $\R \setminus (0, 1) = (-\infty, 0] \cup [1, \infty)$ is closed.
    \item For any topological space $(X, \T)$,
        $\O$ and $X$ are clopen.
    \item For the discrete topology, every subset is clopen.
\end{examples}

\begin{corollary}
    Closed sets are closed under arbitrary intersections and finite unions.
\end{corollary}
\begin{proof}
    De Morgan on \cref{def:topo}.
\end{proof}

\begin{definition}
    Let $A \subseteq (X, \T)$.
    \begin{itemize}
        \item The \emph{interior} of $A$, denoted $A^\circ$ is the union
            of all open sets contained in $A$.
        \item The \emph{closure} of $A$, denoted $\wbar{A}$, is the
            intersection of all closed sets containing $A$.
    \end{itemize}
\end{definition}
\begin{remarks}
    \item $A^\circ \subseteq A \subseteq \wbar{A}$.
    \item $A^\circ$ is open and $\wbar{A}$ is closed.
    \item If $A$ is open, $A^\circ = A$.
    If $A$ is closed, $\wbar{A} = A$.
\end{remarks}

\begin{lemma}
    $x \in \wbar{A}$ iff any open set containg $x$ intersects $A$.
\end{lemma}
\begin{proof}
    \begin{align*}
        x \in \wbar{A}
            &\iff \forall U \in \T (A \subseteq X \setminus U
                \rightarrow x \in X \setminus U) \\
            &\iff \forall U \in \T (x \in U \rightarrow A \cap U \ne \O)
    \end{align*}
    More verbosely, \begin{align*}
        x \in \wbar{A}
            &\iff \forall V \in 2^X (V \text{ is closed }
                \land A \subseteq V \rightarrow x \in V) \\
            &\iff \forall V \in 2^X (X \setminus V \text{ is open} \land
                X \setminus V \subseteq X \setminus A \rightarrow x \notin (X \setminus V)) \\
            &\iff \forall U \in \T (U \subseteq X \setminus A \rightarrow
                x \notin U) \\
            &\iff \forall U \in \T (U \cap A = \O \rightarrow x \notin U) \\
            &\iff \forall U \in \T (x \in U \rightarrow U \cap A \ne \O)
                \qedhere
    \end{align*}
\end{proof}

\begin{definition}
    A point $x \in X$ is a \emph{limit point} of $A$ is any open set
    containing $x$ intersects $A \setminus \set{x}$.
\end{definition}

\begin{lemma} \label{thm:closure}
    $\wbar{A} = A \cup A'$,
    where $A'$ is the set of all limit points of $A$.
\end{lemma}
\begin{proof}
    \begin{align*}
        x \in \wbar{A}
            &\iff \forall U \in \T (x \in U \rightarrow U \cap A \ne \O) \\
            &\iff \forall U \in \T (x \in U \rightarrow (x \in A \lor U \cap (A \setminus \set{x}) \ne \O)) \\
    \end{align*}
\end{proof}
