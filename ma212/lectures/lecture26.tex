\lecture{2024-10-16}{}
\begin{examples}
    \item Let $A$ be a commutative ring and $S = A^\times$.
    Then $S^{-1}A \cong A$.
    \item Let $A$ be a (commutative) integral domain.
    \item
    \begin{definition}[local ring] \label{def:local-ring}
        A commutative ring $A$ is a local ring if it has a unique maximal
        ideal $M$.
    \end{definition}
    Let $P \subseteq A$ be a prime ideal.
    Consider $S = A \setminus P$.
    Then $S$ is multiplicatively closed and contains $1$,
    so we can construct $S^{-1}A \eqcolon A_P$.
    \begin{claim}
        $A_P$ is a local ring with unique maximal ideal $= ??$.
    \end{claim}
    \begin{proof}
        Find all units in $A_P$.
        Say $\frac as \in A_P^\times$.
        That is, there exist $a', s'$ so that $\frac as \frac{a'}{s'} = 1
        \iff t(aa' - ss') = 0$ for some $a' \in A, s', t \notin P$.
    \end{proof}
    \item With $A = \Z$ and $P = p\Z$,
    $A_P = \sset{\frac as}{a \in \Z, (s, p) = 1}$.
    \begin{remark}
        Don't confuse with the subring $\Z[\frac1p] = \Z[x] / (px-1)$,
        the ring of polynomials in $\frac1p$.
        That is, rational numbers power-of-$p$ denominators.
    \end{remark}
\end{examples}

The fraction field of an integral domain $A$ is in some sence the smallest
field containing $A$.
\begin{proposition}
    Let $F$ be any field which contains some ring $R' \cong R$.
    Then $Q' \cong Q$, where $Q$ is the fraction field of $R$ and $Q'$
    is the subfield of $F$ generated by $R'$.
\end{proposition}
Equivalently, $Q'$ is the smallest subfield of $F$ containing $R'$,
and is the fraction field of $R'$.
\begin{proof}
    Define $\varphi\colon R \xrightarrow{\cong} R' \hookrightarrow F$.
    Then by the universal property, there exists a unique
    $\Phi\colon Q \to F$ such that $\Phi(r/1) = \varphi(r)$.
    Then $\ker \Phi$ is an ideal of $Q$.
    Since $Q$ is a field, $\ker \Phi = 0$ or $Q$.
    But since $\phi$ is injective; $\ker \Phi \ne Q$, so $\Phi$ is
    injective.
    Thus $Q \cong \Phi(Q) \subseteq F$.
    It is easy to see that \[
        \Phi(Q) = \sset*{\frac{a'}{b'}}
                {a' \in R', b' \in R' \setminus \set 0}. \qedhere
    \]
\end{proof}
\begin{definition}[characteristic] \label{def:char}
    Given any integral domain $R$, let $\chr R$ be the smallest positive
    integer $n$ such that \[
        n \cdot 1_R \coloneq
            \underbrace{1_R + \dots + 1_R}_{n \text{ times}} = 0.
    \] If such an $n$ does not exist, define $\chr R = 0$.
\end{definition}
\begin{proposition}
    The characteristic of any integral domain is prime.
\end{proposition}
\begin{proof}
    Let $n = n_1n_2$.
    Then $0 = n1_R = (n_11_R)(n_21_R)$,
    so WLOG $n_11_R = 0$.
    Since $n$ is the smallest such number, $n_1 = n$, so $n$ is prime.
\end{proof}
\begin{proposition}
    Let $F$ be any field.
    \begin{enumerate}
        \item If $\chr F = p > 0$, then $\F_p \subseteq F$.
        \item If $\chr F = 0$, then $\Q \subseteq F$.
    \end{enumerate}
\end{proposition}
\begin{proof} \leavevmode
    \begin{enumerate}
        \item Let $\bar{n}$ denote $n1_F$.
        Then $\bar 1, \bar 2, \dots, \wbar{p-1}$ are distinct, and
        $\bar p = 0$.
        It is easy to see that $\F_p \cong \sset{\bar n}{n \in [p-1]}$.
        \item Obviously $\Z \subseteq F$, so $\Q \subseteq F$. \qedhere
    \end{enumerate}
\end{proof}

\section{Domains} \label{sec:domains}
\begin{definition*}
    Let $A$ be a (commutative) integral domain and
    $a \in A \setminus \set 0$.
    \begin{enumerate}
        \item $a \notin A^\times$ is \emph{irreducible} if $a = bc$ implies
        that either $b$ or $c$ is a unit.
        \item $a \notin A^\times$ is \emph{prime} if
        $a \mid bc \implies a \mid b$ or $a \mid c$.
        \item $a, b \in A$ are \emph{associates} if $a = ub$ for some unit
        $u$.
        \item $a$ has a \emph{unique factorization} (into irreducibles) if
        $a = u \prod_{i=1}^m p_i$ where $u$ is a unit and $p_i$ are
        irreducible,
        and if $a = v \prod_{i=1}^n q_i$ where $v$ is a unit and $q_i$ are
        irreducible, then $n = m$ and $p_i$, $q_i$ are associates under
        a suitable reordering.
        \item Let $a, b \in A$.
        The $d = \gcd(a, b)$ if $d \mid a$, $d \mid b$ and whenever
        $d' \mid a$ and $d' \mid b$, $d' \mid d$.
    \end{enumerate}
    If every $a \in A \setminus \set 0$ has a unique factorization, then
    $A$ is a \emph{unique factorization domain} (UFD).
\end{definition*}

\begin{proposition}
    \begin{enumerate}
        \item Every prime is irreducible.
        \item In a PID, every irreducible is prime.
    \end{enumerate}
\end{proposition}
\begin{proof}
    Let $p$ be a prime and $p = bc$.
    WLOG assume $p \mid b \iff b = pd$.
    The $p = pdc \implies 1 = dc$, so $c$ is a unit.

    Let $\pi$ be irreducible in a PID $A$ and $\pi \mid bc$.
    Assume $\pi \nmid b$.
    Then $(\pi, b) = (1)$, since the divisors of $\pi$ are units or
    associates of $\pi$.
    Associates of $\pi$ cannot divide $b$, so $(\pi, b) = (1)$.
    Thus $\pi x + b y = 1$ for some $x, y \in A$, so $\pi \mid c$.
\end{proof}

\begin{proposition}
    Let $A$ be a PID.
    Then $d = \gcd(a, b) \iff (d) = (a, b)$.
\end{proposition}

\begin{theorem}
    Every PID is a UFD.
\end{theorem}
\begin{proof}
    Let $0 \ne a \in A$.
    Define
    \[
        S = \sset{(b)}{b \ne 0, b \text{ does not have a factorization}}.
    \]
    Let $(a_1) \in S$.
    Consider an infinite ascending chain of ideals \[
        (a_1) \subseteq (a_2) \subseteq \dots
    \] from $S$.
    \begin{claim}
        Such a chain must stabilize.
    \end{claim}
    \begin{subproof}[Proof of claim]
        $\bigcup_i (a_i)$ is an ideal, so it is generated by some $c$.
        But $c \in (a_n)$ for some $n$, so $(c) \subseteq (a_n)$.
        Thus $(c) = (a_n) = (a_{n+1}) = \dots$.
    \end{subproof}
    \begin{claim}
        $a_n$ is irreducible.
    \end{claim}
    \begin{subproof}[Proof of claim]
        Assume $a_n = \alpha \beta$ where neither is a unit.
        Then $(a_n) \subsetneq (\alpha)$, since $\beta \notin (a_n)$.
    \end{subproof}

    \begin{claim}
        In a UFD, every irreducible is prime.
    \end{claim}
    \begin{subproof}[Proof of claim]
    \end{subproof}

    Let $a = u \prod_1^m p_i = v \prod_1^n q_i$ be two factorizations.
    Since $p_1 \mid \prod q_i$, $p_1 \mid q_i$ for some $i$.
    But then $p_1 = \text{unit} \cdot q_i$, since $q_i$ is irreducible.
    Remove $p_1$ and $q_i$ from the factorizations and continue.
    When this terminates, we must not have any irreducibles
    on either side.
    Thus $m = n$ and the factors are associates upto reordering.
\end{proof}

\begin{examples}
    \item $\Z[\sqrt{-5}] \subseteq \C$ is an integral domain.
    Then $6 = 2 \cdot 3 = (1 + \sqrt{-5})(1 - \sqrt{-5})$.
    Let \[
        N(x) = x \bar x = a^2 + 5b^2.
    \] Then $x$ is a unit iff $N(x) = 1$.
    \begin{itemize}
        \item If $N(x) > 1$, then $N(xy) = N(x)N(y) > 1$ for all $y \ne 0$.
        \item If $N(x) = 1$, then $x \bar x = 1$.
    \end{itemize}

    Moreover, $2$ is not prime.
\end{examples}
