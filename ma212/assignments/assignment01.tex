\documentclass[12pt]{article}
% \usepackage{parskip}
\usepackage{lmodern} % https://tex.stackexchange.com/a/58088/295755
\renewcommand\bfdefault{b}
\usepackage{microtype}

\usepackage{amsmath}
\newcommand\yesnumber{\stepcounter{equation}\tag{\theequation}}

\ifundef{\chapter}{}{\providecommand\thechapter{\Roman{chapter}}}

\usepackage{marginnote}
\usepackage[en-GB,calc]{datetime2}
\usepackage{calc}
\usepackage{xifthen}
\usepackage{tocloft}

% https://tex.stackexchange.com/a/454168
\newcommand\monthday[1]{\DTMmonthname{\DTMfetchmonth{#1}}~\number\DTMfetchday{#1}}

\newlistof{lecture}{lec}{Lectures} % what is this file extension business?
\makeatletter
\setlength\marginparwidth{1in}
\newcommand*{\lecture}[3][]{
    \ifthenelse{\isempty{#1}}{%
        \refstepcounter{lecture}%
    }{%
        \setcounter{lecture}{#1}%
    }%
    \DTMsavedate{lecdate}{#2}%
    \def\lecdow{\DTMweekdayname{\DTMfetchdow{lecdate}}}
    \def\lecshortdow{\DTMshortweekdayname{\DTMfetchdow{lecdate}}}
    \def\lecmonth{\DTMmonthname{\DTMfetchmonth{lecdate}}}
    \def\lecday{\number\DTMfetchday{lecdate}}
    % \marginpar{\raggedright\small \textsf{\textbf{Lecture \thelecture.}%
    %             \footnotesize\DTMusedate{lecdate}}}%
    \marginnote{\raggedright\small%
        \textsf{{\textbf{Lecture \thelecture.}} \\
        \footnotesize\lecdow\\\lecmonth\ \lecday}}%
    \ifthenelse{\isempty{#3}}{%
        \addcontentsline{lec}{lecture}{\protect\numberline{\thelecture}%
        \lecshortdow, \lecmonth\ \lecday}%
        \def\@lecture{Lecture \thelecture}%
    }{%
        \addcontentsline{lec}{lecture}{\protect\numberline{\thelecture}%
        \makebox[\widthof{Mon,}][l]{\lecshortdow,}\ \makebox[\widthof{September 00}][l]{\lecmonth\ \lecday} #3}%
        \def\@lecture{Lecture \thelecture: #3}%
    }%
    \par%
}
\g@addto@macro\normalsize{%
  \setlength\abovedisplayskip{7pt}%
  \setlength\belowdisplayskip{7pt}%
  \setlength\abovedisplayshortskip{1pt}%
  \setlength\belowdisplayshortskip{1pt}%
}
\makeatother

\usepackage[twoside]{fancyhdr}
\setlength{\headheight}{15pt}
\pagestyle{fancy}
\fancyhf{}
% \fancyhead[r]{\thepage}
\makeatletter
\fancyhead[LE,RO]{\thepage}
\fancyhead[RE]{\textbf{\nouppercase\leftmark}}
\fancyhead[LO]{\nouppercase\rightmark}
\providecommand\@lecture{}
\fancyfoot[R]{\small\@lecture}
\makeatother

% homeworks
\newlistof{hw}{hw}{Assignments} % counter `assignment' already defined
\makeatletter
\newcommand*{\assignment}[5][]{%[number]{file}{date posted}{date due}{date quiz}
    \ifthenelse{\isempty{#1}}{%
        \refstepcounter{hw}%
        \stepcounter{assignment}%
    }{%
        \setcounter{hw}{#1}%
        \setcounter{assignment}{#1}%
    }%
    \pagebreak
    \ifthenelse{\isempty{#3}}{}{\DTMsavedate{posted}{#3}}%
    \ifthenelse{\isempty{#4}}{}{\DTMsavedate{due}{#4}}%
    \ifthenelse{\isempty{#5}}{}{\DTMsavedate{quiz}{#5}}%
    \section*{Assignment \thehw}
    \ifthenelse{\isempty{#4}}{
        \ifthenelse{\isempty{#5}}{
            \ifthenelse{\isempty{#3}}{
                \def\hw@toc{}
            }{
                \def\hw@toc{posted \monthday{up}}
            }
        }{
            \def\hw@toc{quiz \monthday{quiz}}
        }
    }{
        \def\hw@toc{due \monthday{due}}
    }
    \addcontentsline{hw}{hw}{\protect\numberline{\thehw}\hw@toc}\par
    \marginpar{\raggedright\footnotesize\textsf{%
        \ifthenelse{\isempty{#3}}{}{\makebox[\widthof{quiz}][l]{up} \monthday{posted} \\}%
        \ifthenelse{\isempty{#4}}{}{\makebox[\widthof{quiz}][l]{due} \monthday{due} \\}%
        \ifthenelse{\isempty{#5}}{}{quiz \monthday{quiz}}%
    }}
    \def\@lecture{Assignment \thehw\ifx\hw@toc\empty{}\else\ --- \hw@toc\fi}
    \input{#2}
    \newpage
}
\makeatother

\usepackage{amsmath}
\usepackage{amsthm}
\usepackage[dvipsnames]{xcolor}
\colorlet{exercise}{cyan!70!black}
\colorlet{solved}{green!40!black}
\colorlet{self_proof}{blue!70!black}
\colorlet{Red}{red!80!black}

% Gilles Castel's theorems
\newtheoremstyle{mddefinition}% <name>
  {-.25\topsep}%                 <space above>
  {-.25\topsep}%                         <space below>
  {\normalfont}%              <body font>
  {}%                         <indent amount>
  {\bfseries}%                <theorem head font>
  {.}%                        <punctuation after theorem head>
  {.5em}%                     <space after theorem head>
  {}%                         <theorem head spec>
\newtheoremstyle{mdplain}% <name>
  {-.25\topsep}%                 <space above>
  {-.25\topsep}%                         <space below>
  {\itshape}%                 <body font>
  {}%                         <indent amount>
  {\bfseries}%                <theorem head font>
  {.}%                        <punctuation after theorem head>
  {.5em}%                     <space after theorem head>
  {}%                         <theorem head spec>

\usepackage[framemethod=Tikz]{mdframed}
\mdfsetup{skipbelow=0pt}
\mdfdefinestyle{axiomstyle}{
    outerlinewidth = 1.5,
    roundcorner = 10,
    leftmargin = 15,
    rightmargin = 15,
    backgroundcolor = yellow!7
}
\mdfdefinestyle{defstyle}{
    outerlinewidth = 1,
    roundcorner = 2,
    leftmargin = 7,
    rightmargin = 7,
    backgroundcolor = green!5
}
\mdfdefinestyle{thmstyle}{
    outerlinewidth = 1,
    roundcorner = 8,
    leftmargin = 7,
    rightmargin = 7,
    backgroundcolor = cyan!5
}
\mdfdefinestyle{lemmastyle}{
    outerlinewidth = 1.5,
    roundcorner = 10,
    leftmargin = 7,
    rightmargin = 7,
    backgroundcolor = yellow!10
}
\ifundef\chapter{%
    \providecommand\theoremnumberwithin{section}
    \theoremstyle{mddefinition}
    \newmdtheoremenv[nobreak=true, style=axiomstyle]{axiom}{Axiom}[section]
    \theoremstyle{plain}
    \newtheorem{theorem}{Theorem}[\theoremnumberwithin]

    \newcounter{assignment}
    \theoremstyle{plain}
    \newtheorem{problem}{Problem}
    \theoremstyle{mddefinition}
    \newmdtheoremenv[nobreak=true, outerlinewidth=0.7]{problem*}[problem]{Problem}
}{
    \providecommand\theoremnumberwithin{chapter}
    \theoremstyle{mddefinition}
    \newmdtheoremenv[nobreak=true, style=axiomstyle]{axiom}{Axiom}[chapter]
    \theoremstyle{plain}
    \newtheorem{theorem}{Theorem}[\theoremnumberwithin]

    \newcounter{assignment}
    \theoremstyle{plain}
    \newtheorem{problem}{Problem}[assignment]
    \theoremstyle{mddefinition}
    \newmdtheoremenv[nobreak=true, outerlinewidth=0.7]{problem*}[problem]{Problem}
}
\theoremstyle{mddefinition}
\newmdtheoremenv[nobreak=true, style=defstyle]{definition*}[theorem]{Definition}

\theoremstyle{mdplain}
\newmdtheoremenv[nobreak=true, style=thmstyle]{theorem*}[theorem]{Theorem}
\newmdtheoremenv[nobreak=true]{proposition*}[theorem]{Proposition}
\newmdtheoremenv[nobreak=true]{lemma*}[theorem]{Lemma}
\newmdtheoremenv[nobreak=true]{corollary*}[theorem]{Corollary}
\newmdtheoremenv[nobreak=true, style=thmstyle]{fact*}[theorem]{Fact}
\newmdtheoremenv[nobreak=true]{exercise*}[theorem]{Exercise}
\newmdtheoremenv[nobreak=true]{question*}[theorem]{Question}

\theoremstyle{definition}
\newtheorem{definition}[theorem]{Definition}

\theoremstyle{plain}
\newtheorem{proposition}[theorem]{Proposition}
\newtheorem{lemma}[theorem]{Lemma}
\newtheorem{corollary}[theorem]{Corollary}
\newtheorem{fact}[theorem]{Fact}
\newtheorem{exercise}[theorem]{Exercise}
\newtheorem{question}[theorem]{Question}

\theoremstyle{remark}
\newtheorem*{remark}{Remark}
\newtheorem*{remarkx}{Remarks}
\newtheorem*{example}{Example}
\newtheorem*{examplex}{Examples}
\newtheorem*{idea}{Idea}
%%%% HAXXXXXX %%%%
% \def\innerqed{\qedsymbol}
% \def\outerqed{$\blacksquare$}
\let\oldproof\proof
\let\endoldproof\endproof
\newenvironment{solution}[1][]
  {\renewcommand\qedsymbol{$\blacksquare$}%
  \begin{oldproof}[Solution\ifx&#1&\else\ (#1)\fi]}
  {\end{oldproof}}
\newenvironment{answer}
  {\renewcommand\qedsymbol{$\blacksquare$}\begin{oldproof}[Answer]}
  {\end{oldproof}}
\renewenvironment{proof}
  {\renewcommand\qedsymbol{$\blacksquare$}\begin{oldproof}}
  {\end{oldproof}}
\newenvironment{subproof}[1][Subproof]{%
  \renewcommand{\qedsymbol}{$\square$}\begin{oldproof}[#1]}
  {\end{oldproof}}
\newtheorem*{notation}{Notation}
\newtheorem*{claim}{Claim}

\usepackage{hyperref}
\usepackage[noabbrev]{cleveref}

% <cref>
\crefname{theorem}{theorem}{theorems}
\crefname{proposition}{proposition}{propositions}
\crefname{lemma}{lemma}{lemmas}
\crefname{corollary}{corollary}{corollaries}
\crefname{axiom}{axiom}{axioms}
\crefname{definition}{definition}{definitions}
\crefname{problem}{problem}{problems}
\crefname{exercise}{exercise}{exercises}
\crefname{fact}{fact}{facts}
\crefname{question}{question}{questions}
\crefname{remark}{remark}{remarks}
\crefname{example}{example}{example}
\crefname{notation}{notation}{notations}
\crefname{claim}{claim}{claims}
% \crefname{section}{\S}{\S\S}
\crefname{theorem*}{theorem}{theorems}
\crefname{proposition*}{proposition}{propositions}
\crefname{lemma*}{lemma}{lemmas}
\crefname{corollary*}{corollary}{corollaries}
\crefname{definition*}{definition}{definitions}
\crefname{problem*}{problem}{problems}
\crefname{exercise*}{exercise}{exercises}
\crefname{fact*}{fact}{facts}
\crefname{question*}{question}{questions}
% </cref>

% <hyperlinks>
\hypersetup{colorlinks,
    linkcolor={blue},
    citecolor={blue!50!black},
    urlcolor={blue!80!black}}
% </hyperlinks>

\usepackage[shortlabels]{enumitem}
% change default label for enumerate, and fix long labels popping out
\setenumerate{label*=(\roman*),ref=(\roman*),leftmargin=*}
% casework list using https://tex.stackexchange.com/a/30035
\newcounter{casecount}
\newlist{casework}{description}{1}
\setlist[casework]{%
  before={\setcounter{casecount}{0}%
      \renewcommand*\thecasecount{\arabic{casecount}}}%
  ,font=\bfseries Case \stepcounter{casecount}\thecasecount:
}

\newenvironment{examples}[1][]
{\begin{examplex}[#1]\leavevmode\begin{itemize}}{\end{itemize}\end{examplex}}
\newenvironment{remarks}[1][]
{\begin{remarkx}[#1]\leavevmode\begin{itemize}}{\end{itemize}\end{remarkx}}

% omg this is so HaXy
% \renewenvironment{proof}[1][\proofname]{{\it\bfseries #1. }}{\qed}
% \providecommand{\qedsymbol}{\openbox}
% \makeatletter
% \renewenvironment{proof}[1][\proofname]{\par
%   \pushQED{\qed}%
%   \normalfont \topsep6\p@\@plus6\p@\relax
%   \trivlist
%   \item[\hskip\labelsep
%         \itshape\bfseries%this is the change (boldface instead of italics)
%         % \fontseries{bx}\selectfont
%     #1\@addpunct{.}]\ignorespaces
% }{%
%   \popQED\endtrivlist\@endpefalse
% }
\makeatother
\crefname{enumi}{part}{parts}
\crefname{enumii}{part}{parts}
\crefname{enumiii}{part}{parts}
% \setlist[itemize]{itemsep=2pt}
\newcounter{dummy}
\makeatletter
\newcommand\myitem[1][]{\item[#1]\refstepcounter{dummy}\def\@currentlabel{#1}}
\makeatother

\makeatletter
\newcommand*{\refifdef}[3]{%label,command,fallback
    \@ifundefined{r@#1}{#3}{#2{#1}}%
}
\makeatother

\newcommand\ie{\textit{i.e.}}
\newcommand\eg{\textit{e.g.}}
\usepackage{physics}

\usepackage{amsmath}
\usepackage{amssymb}
\usepackage{mathrsfs} % for \mathscr
\usepackage{bm} % for \bm
\usepackage{booktabs}

% undefine \abs and \norm
\let\abs\relax
\let\norm\relax

\usepackage{mathtools} % for delimiters and \coloneqq
\DeclarePairedDelimiter{\paren}{(}{)}
\DeclarePairedDelimiter{\brk}{[}{]}
\DeclarePairedDelimiter{\set}{\{}{\}}
\DeclarePairedDelimiter{\abs}{\lvert}{\rvert}
\DeclarePairedDelimiter{\norm}{\lVert}{\rVert}
\DeclarePairedDelimiter{\floor}{\lfloor}{\rfloor}
\DeclarePairedDelimiter{\ceil}{\lceil}{\rceil}
\DeclarePairedDelimiter{\angled}{\langle}{\rangle}
% \DeclarePairedDelimiterX{\innerp}[2]{\langle}{\rangle}{#1,\,#2}
% \DeclarePairedDelimiterX{\outerp}[2]{\langle}{\rangle}{#1\otimes#2}
% \DeclarePairedDelimiterX{\braket}[3]{\langle}{\rangle}%
% {#1\,\delimsize\vert\,\mathopen{}#2\,\delimsize\vert\,\mathopen{}#3}
\DeclarePairedDelimiterX{\innerp}[2]{\langle}{\rangle}{#1,\,#2}
% \DeclarePairedDelimiterX{\outerp}[2]{\langle}{\rangle}{#1\otimes#2}
\DeclarePairedDelimiterX{\outerp}[2]{\vert}{\vert}%
{#1\delimsize\rangle\delimsize\langle\mathopen{}#2}
\let\braket\relax
\DeclarePairedDelimiterX{\braket}[3]{\langle}{\rangle}%
{#1\,\delimsize\vert\,\mathopen{}#2\,\delimsize\vert\,\mathopen{}#3}

\renewcommand\O{\ensuremath{\varnothing}}
\newcommand\N{\ensuremath{\mathbb{N}}}
\newcommand\Z{\ensuremath{\mathbb{Z}}}
\newcommand\Q{\ensuremath{\mathbb{Q}}}
\newcommand\R{\ensuremath{\mathbb{R}}}
\newcommand\C{\ensuremath{\mathbb{C}}}
% \renewcommand\P{\ensuremath{\mathbb{P}}}

% fix spacing for \forall and \exists
% \let\oldforall\forall
% \renewcommand{\forall}{\oldforall \, }
% \let\oldexist\exists
% \renewcommand{\exists}{\oldexist \: }
\newcommand\unique{\exists!}
\newcommand\lxor{\oplus}

\providecommand{\dd}{\,\mathrm{d}}

\newcommand\mcA{\ensuremath{\mathcal{A}}}
\newcommand\mcB{\ensuremath{\mathcal{B}}}
\newcommand\mcC{\ensuremath{\mathcal{C}}}
\newcommand\mcD{\ensuremath{\mathcal{D}}}
\newcommand\mcE{\ensuremath{\mathcal{E}}}
\newcommand\mcF{\ensuremath{\mathcal{F}}}
\newcommand\mcG{\ensuremath{\mathcal{G}}}
\newcommand\mcH{\ensuremath{\mathcal{H}}}
\newcommand\mcI{\ensuremath{\mathcal{I}}}
\newcommand\mcJ{\ensuremath{\mathcal{J}}}
\newcommand\mcK{\ensuremath{\mathcal{K}}}
\newcommand\mcL{\ensuremath{\mathcal{L}}}
\newcommand\mcM{\ensuremath{\mathcal{M}}}
\newcommand\mcN{\ensuremath{\mathcal{N}}}
\newcommand\mcO{\ensuremath{\mathcal{O}}}
\newcommand\mcP{\ensuremath{\mathcal{P}}}
\newcommand\mcQ{\ensuremath{\mathcal{Q}}}
\newcommand\mcR{\ensuremath{\mathcal{R}}}
\newcommand\mcS{\ensuremath{\mathcal{S}}}
\newcommand\mcT{\ensuremath{\mathcal{T}}}
\newcommand\mcU{\ensuremath{\mathcal{U}}}
\newcommand\mcV{\ensuremath{\mathcal{V}}}
\newcommand\mcW{\ensuremath{\mathcal{W}}}
\newcommand\mcX{\ensuremath{\mathcal{X}}}
\newcommand\mcY{\ensuremath{\mathcal{Y}}}
\newcommand\mcZ{\ensuremath{\mathcal{Z}}}

%%%% WIDE BAR THAT IS JUST THE RIGHT LENGTH %%%%
%% FROM https://tex.stackexchange.com/a/60253 %%
\makeatletter
\let\save@mathaccent\mathaccent
\newcommand*\if@single[3]{%
  \setbox0\hbox{${\mathaccent"0362{#1}}^H$}%
  \setbox2\hbox{${\mathaccent"0362{\kern0pt#1}}^H$}%
  \ifdim\ht0=\ht2 #3\else #2\fi
  }
%The bar will be moved to the right by a half of \macc@kerna, which is computed by amsmath:
\newcommand*\rel@kern[1]{\kern#1\dimexpr\macc@kerna}
%If there's a superscript following the bar, then no negative kern may follow the bar;
%an additional {} makes sure that the superscript is high enough in this case:
\newcommand*\widebar[1]{\@ifnextchar^{{\wide@bar{#1}{0}}}{\wide@bar{#1}{1}}}
%Use a separate algorithm for single symbols:
\newcommand*\wide@bar[2]{\if@single{#1}{\wide@bar@{#1}{#2}{1}}{\wide@bar@{#1}{#2}{2}}}
\newcommand*\wide@bar@[3]{%
  \begingroup
  \def\mathaccent##1##2{%
%Enable nesting of accents:
    \let\mathaccent\save@mathaccent
%If there's more than a single symbol, use the first character instead (see below):
    \if#32 \let\macc@nucleus\first@char \fi
%Determine the italic correction:
    \setbox\z@\hbox{$\macc@style{\macc@nucleus}_{}$}%
    \setbox\tw@\hbox{$\macc@style{\macc@nucleus}{}_{}$}%
    \dimen@\wd\tw@
    \advance\dimen@-\wd\z@
%Now \dimen@ is the italic correction of the symbol.
    \divide\dimen@ 3
    \@tempdima\wd\tw@
    \advance\@tempdima-\scriptspace
%Now \@tempdima is the width of the symbol.
    \divide\@tempdima 10
    \advance\dimen@-\@tempdima
%Now \dimen@ = (italic correction / 3) - (Breite / 10)
    \ifdim\dimen@>\z@ \dimen@0pt\fi
%The bar will be shortened in the case \dimen@<0 !
    \rel@kern{0.6}\kern-\dimen@
    \if#31
      \overline{\rel@kern{-0.6}\kern\dimen@\macc@nucleus\rel@kern{0.4}\kern\dimen@}%
      \advance\dimen@0.4\dimexpr\macc@kerna
%Place the combined final kern (-\dimen@) if it is >0 or if a superscript follows:
      \let\final@kern#2%
      \ifdim\dimen@<\z@ \let\final@kern1\fi
      \if\final@kern1 \kern-\dimen@\fi
    \else
      \overline{\rel@kern{-0.6}\kern\dimen@#1}%
    \fi
  }%
  \macc@depth\@ne
  \let\math@bgroup\@empty \let\math@egroup\macc@set@skewchar
  \mathsurround\z@ \frozen@everymath{\mathgroup\macc@group\relax}%
  \macc@set@skewchar\relax
  \let\mathaccentV\macc@nested@a
%The following initialises \macc@kerna and calls \mathaccent:
  \if#31
    \macc@nested@a\relax111{#1}%
  \else
%If the argument consists of more than one symbol, and if the first token is
%a letter, use that letter for the computations:
    \def\gobble@till@marker##1\endmarker{}%
    \futurelet\first@char\gobble@till@marker#1\endmarker
    \ifcat\noexpand\first@char A\else
      \def\first@char{}%
    \fi
    \macc@nested@a\relax111{\first@char}%
  \fi
  \endgroup
}
\makeatother
\let\what\widehat
\let\wtld\widetilde
\let\wbar\widebar
\let\ubar\underline

\DeclareMathOperator\sgn{sgn}

\let\oldleft\left
\let\oldright\right
\renewcommand{\left}{\mathopen{}\mathclose\bgroup\oldleft}
\renewcommand{\right}{\aftergroup\egroup\oldright}



\title{Assignment 1}
\author{Naman Mishra}
\date{21 August, 2024}

\begin{document}
\maketitle

% Problem 1
\begin{problem}
    Let $n \ge 3$.
    Let $r, s$ be the usual generators of $D_{2n}$.
    Prove that the map $\varphi\colon D_{2n} \to \GL_2(\R)$ defined by
    \[
        \varphi(r) = \begin{pmatrix}
            \cos \theta & -\sin \theta \\
            \sin \theta & \cos \theta
        \end{pmatrix} \qquad \varphi(s) = \begin{pmatrix}
            1 & 0 \\
            0 & -1
        \end{pmatrix}
    \] where $\theta = \frac{2\pi}{n}$ extends to an injective
    homomorphism.
\end{problem}

\begin{problem}
    Show that $(\Z, +) \ncong (\Q, +)$,
    $S_m \ncong S_n$ for $m \ne n$ and $(\R, +) \ncong (\Q, +)$.
\end{problem}
\begin{solution}
    Let $\varphi\colon (\Z, +) \to (\Q, +)$ be a homomorphism.
    Let $\varphi(1) = a$.
    Then $\varphi(n) = na$ for $n \in \N$.
    $\varphi$ cannot be surjective, as no integer multiple of $a$ can
    equal $a/2$.

    For $m \ne n$, $S_m$ and $S_n$ have different cardinalities.
    So do $\R$ and \Q.
\end{solution}

\begin{problem}
    For a commutative ring $A$, let \[
        A^\times = \set{a \in A \mid ab = 1 \text{ for some } b \in A}
    \] be the unit group of $A$.
    Show that $A^\times$ is a group (under multiplication).

    Are $\Z[x]^\times$ and $\Q[x]^\times$ isomorphic?
    Are $(\Z[x], +)$ and $(\Q[x], +)$ isomorphic?
\end{problem}
\begin{solution}
    $1 \in A^\times$ is the identity.
    If $a_1, a_2 \in A^\times$, then $a_1 a_2 \in A^\times$,
    since there exist $b_1, b_2 \in A$ such that $(a_1 a_2) (b_2 b_1) =
    a_1 b_1 = 1$.
    Associativity is borrowed from $A$.
    Inverses exist by definition.

    $\Z[x]^\times = \set{1, -1}$,
    while $\Q[x]^\times = \Q \setminus \set{0}$.
    They are not isomorphic.

    Assume $\varphi\colon \Q[x] \to \Z[x]$ is an homomorphism.
    Then $\varphi(p) \ne 0$ for each non-zero polynomial $p$.

    Let $\varphi(1) = a_0 + a_1 x + \dots + a_n x^n$, with $a_n \ne 0$.
    Now \[
        \varphi(1) = \varphi\ab(k \cdot \frac1k)
            = \varphi(k) \varphi\ab(\frac1k)
            = k \varphi(1) \varphi\ab(\frac1k)
    \] for each $k \in \Z$.
    Thus $a_n$ is a multiple of $k$ for each $k \in \Z$.
    Absurd.
\end{solution}

\begin{problem}
    Prove that a finitely generated subgroup of $(\Q, +)$ is cyclic.
\end{problem}
\begin{solution}
    Let $H = \angled*{\frac{m_1}{n_1}, \dots, \frac{m_k}{n_k}}$.
    Write this as $\angled*{\frac{m'_1}{N}, \dots, \frac{m'_k}{N}}$,
    where $N = \lcm(n_1, \dots, n_k)$.
    Then $H = \sum_{i=1}^k a_i \frac{m'_i}{N}$ for some $a_i \in \Z$.
    Thus \[
        H = \frac1N \angled*{m'_1, \dots, m'_k}
        = \frac1N \angled*{\gcd(m'_1, \dots, m'_k)}. \qedhere
    \]
\end{solution}

\begin{problem}
    Show that $(\Z^m, +) \ncong (\Z^n, +)$ for $m \ne n$.
\end{problem}

\begin{problem}
    If $\tau_1 = (a_1\ b_1)$ and $\tau_2 = (a_2\ b_2)$ are transpositions
    in $S_n$ with $\set{a_1, b_1} \cap \set{a_2, b_2} = \O$,
    show that $\tau_1 \tau_2 = \tau_2 \tau_1$.
\end{problem}
\begin{solution}
    We have \[
        \tau_1(x) = \begin{cases}
            b_1 & \text{if } x = a_1, \\
            a_1 & \text{if } x = b_1, \\
            x & \text{otherwise}
        \end{cases}
        \qquad \text{and} \qquad
        \tau_2(x) = \begin{cases}
            b_2 & \text{if } x = a_2, \\
            a_2 & \text{if } x = b_2, \\
            x & \text{otherwise.}
        \end{cases}
    \] Now \[
        \tau_1 \tau_2(x) = \begin{cases}
            b_2 & \text{if } x = a_2, \\
            a_2 & \text{if } x = b_2, \\
            b_1 & \text{if } x = a_1, \\
            a_1 & \text{if } x = b_1, \\
            x & \text{otherwise}
        \end{cases}
        \qquad \text{and} \qquad
        \tau_2 \tau_1(x) = \begin{cases}
            b_1 & \text{if } x = a_1, \\
            a_1 & \text{if } x = b_1, \\
            b_2 & \text{if } x = a_2, \\
            a_2 & \text{if } x = b_2, \\
            x & \text{otherwise.}
        \end{cases}
    \] Thus $\tau_1 \tau_2 = \tau_2 \tau_1 = (a_1\ b_1) (a_2\ b_2)$.
\end{solution}

\begin{problem}
    Let $X_{2n} = \angled{x, y \mid x^n = y^2 = 1, xy = yx^2}$.
    What is $X_{2n}$?
\end{problem}
\begin{solution}
    \[
        x^4 = y^2 x^4 = y(yx^2)x^2 = yx(yx^2) = (yx x)y = xyy = x.
    \] Thus $x^3 = 1$.
    If $3 \nmid n$, then $x = 1$ and $X_{2n} = \set{1, y} \cong (\Z_2, +)$.

    If $3 \mid n$, then $X_{2n} = D_6$, since we can simplify the
    relations to \[
        X_{2n} = \angled{x, y \mid x^3 = y^2 = 1, xy = yx^{-1}}.
    \] This is the presentation of $D_6$ with
    $r \mapsto x$ and $s \mapsto y$.
\end{solution}

\begin{problem}
    Let $G = \set*{\begin{pmatrix}
        \pm 1 & c \\
        0 & 1
    \end{pmatrix} \mid c \in \Z/n\Z}$.
    Show that $G \cong D_{2n}$ ($n \ge 3$).
\end{problem}
\begin{solution}
    Put $R = \begin{pmatrix}
        1 & 1 \\
        0 & 1
    \end{pmatrix}$ and $S = \begin{pmatrix}
        -1 & 0 \\
        0 & 1
    \end{pmatrix}$.
    Then \[
        \begin{pmatrix}
            1 & c \\
            0 & 1
        \end{pmatrix} = R^c
        \qquad \text{and} \qquad
        \begin{pmatrix}
            -1 & c \\
            0 & 1
        \end{pmatrix} = S R^c.
    \] Note that $R^n = 1$, $S^2 = 1$ and \[
        SRS = \begin{pmatrix}
            -1 & 0 \\
            0 & 1
        \end{pmatrix} \begin{pmatrix}
            -1 & 1 \\
            0 & 1
        \end{pmatrix}
        = \begin{pmatrix}
            1 & -1 \\
            0 & 1
        \end{pmatrix}
        = R^{-1}.
    \] Thus $\varphi\colon D_{2n} \to G$ defined by \[
        \varphi(r) = R \quad \text{and} \quad \varphi(s) = S
    \] extends to an isomorphism.
\end{solution}

\begin{problem}
    Let $H_\R = \set*{\begin{pmatrix}
        1 & a & b \\
        0 & 1 & c \\
        0 & 0 & 1
    \end{pmatrix} \bigm\mid a, b, c \in \R}$.
    Show that $H_\R$ is a group and that every non-identity element
    has infinite order.
\end{problem}
\begin{solution}
    Let $X_1 = \begin{pmatrix}
        1 & a_1 & b_1 \\
        0 & 1 & c_1 \\
        0 & 0 & 1
    \end{pmatrix}$ and $X_2 = \begin{pmatrix}
        1 & a_2 & b_2 \\
        0 & 1 & c_2 \\
        0 & 0 & 1
    \end{pmatrix}$ be elements of $H_\R$.
    Then \[
        X_1 X_2 = \begin{pmatrix}
            1 & a_1 + a_2 & b_1 + b_2 + a_1 c_2 \\
            0 & 1 & c_1 + c_2 \\
            0 & 0 & 1
        \end{pmatrix} \in H_\R.
    \] Note that the identity matrix is in $H_\R$.
    Associativity is inherited from $M_3(\R)$.

    $a_2 = -a_1$, $c_2 = -c_1$ and $b_2 = -b_1 - a_1 c_2 = a_1 c_1 - b_1$
    gives $X_1 X_2 = I$.

    Let $X = \begin{pmatrix}
        1 & a & b \\
        0 & 1 & c \\
        0 & 0 & 1
    \end{pmatrix}$.
    It is easy to observe that \[
        X^n = \begin{pmatrix}
            1 & n a & * \\
            0 & 1 & n c \\
            0 & 0 & 1
        \end{pmatrix}
    \] If either of $a$ or $c$ is non-zero, then $X^n$ is not the
    identity matrix.
    If $a = c = 0$, then \[
        X^n = \begin{pmatrix}
            1 & 0 & n b \\
            0 & 1 & 0 \\
            0 & 0 & 1
        \end{pmatrix}.
    \] Thus $X^n = I$ iff $X = I$.
\end{solution}

\begin{problem}
    Define the quaternion group $Q_8 = \set{\pm 1, \pm i, \pm j, \pm k}$
    with the multiplication laws
    $i^2 = j^2 = k^2 = -1$, $ij = (-j)i = k$, $jk = (-k)j = i$
    and $ki = (-i)k = j$.
    Also assume that $-1$ commutes with and flips the sign of each element.
    Show that $Q_8$ is a group.
    Show that every subgroup of $Q_8$ is normal, yet it is not abelian.
\end{problem}
\begin{solution}
    % \begin{table}
    %     \centering
    %     $\begin{tblr}{
    %         colspec = {r *{8}{r}},
    %         vline{3,4,5,6,7,8,9} = {dotted,.2pt},
    %         hline{3,4,5,6,7,8,9} = {dotted,.2pt},
    %         vline{2} = {.5pt},
    %         hline{2} = {.5pt},
    %     }
    %            &  1 & -1 &  i & -i &  j & -j &  k & -k \\
    %          1 &  1 & -1 &  i & -i &  j & -j &  k & -k \\
    %         -1 & -1 &  1 & -i &  i & -j &  j & -k &  k \\
    %          i &  i & -i & -1 &    &  k &    &    &    \\
    %         -i & -i &  i &    &    &    &    &  j &    \\
    %          j &  j & -j &    &    & -1 &    &  i &    \\
    %         -j & -j &  j &  k &    &    &    &    &    \\
    %          k &  k & -k &  j &    &    &    & -1 &    \\
    %         -k & -k &  k &    &    &  i &    &    &    \\
    %     \end{tblr}$
    % \end{table}
    I refuse to show it is a group.
    Let's show that every subgroup is normal.
    A subgroup $N \le Q_8$ is normal iff
    $g n g^{-1} \in N$ for all $g \in Q_8$ and $n \in N$.

    $1$ and $-1$ commute with all elements,
    so we only need to check $n, g \in \set{\pm i, \pm j, \pm k}$.

    WLOG let $n = i$.
    $n$ commutes with $\pm i$, so $gng^{-1} = n \in N$ if $g = \pm i$.
    $n$ anti-commutes with $\pm j$ and $\pm k$.
    So if $g \in \set{\pm j, \pm k}$,
    then $g n g^{-1} = -n g g^{-1} = -n = n^{-1} \in N$.
\end{solution}

\end{document}
