\chapter*{The course} \label{chp:course}
\lecture{2024-08-02}{Norms and equivalence}

\section*{Grading} \label{sec:grading}
\begin{itemize}
    \item Homework: 20\%
    \item Quizzes: 20\%
    \item Midterm: 20\%
    \item Final: 40\%
\end{itemize}

\section*{Textbooks} \label{sec:books}
\begin{itemize}
    \item 
\end{itemize}

\chapter{Linear algebra} \label{chp:linalg}

\begin{definition}[homogeneous function] \label{def:home}
    Let $V$ be a vector space over $\R$.
    A function $f\colon V \setminus \set{0} \to \R$ is called
    a \emph{homogeneous function} of degree $k$ if \[
        f(r x) = r^k f(x)
    \] for each $x \in V \setminus \set{0}$ and $r > 0$.
\end{definition}
\begin{remarks}
    \item If $f$ and $g$ are homogeneous functions of degree $k$ and $l$
    respectively, then $f \cdot g$ is homogeneous of degree $k + l$
    and $f / g$ is homogeneous of degree $k - l$
    (provided $g$ is never zero).
    \item $f \equiv 0$ is homogeneous of any degree.
\end{remarks}

\begin{definition}[norm] \label{def:norm}
    Let $V$ be a vector space over $\R$.
    A norm $\norm{\cdot}$ on $V$ is a function from $V$ to $\R$ that
    satisfies
    \begin{enumerate}[label=\small(N\arabic*)]
        \item (Positivity) $\norm{x} \ge 0$ for any $x \in V$.
            \label{def:norm:pos}
        \item (Definiteness) $\norm{x} = 0$ iff $x = 0$.
            \label{def:norm:def}
        \item (Homogeneity) $\norm{r x} = \abs{r} \norm{x}$
            for any $x \in V$ and $r \in \R$.
            \label{def:norm:hom}
        \item (Triangle inequality) $\norm{x + y} \le \norm{x} + \norm{y}$
            for any $x, y \in V$.
            \label{def:norm:tri}
    \end{enumerate}
\end{definition}

\begin{definition}[normed linear space] \label{def:nls}
    A vector space $V$ equipped with a norm $\norm{\cdot}$ is called
    a \emph{normed linear space}.
\end{definition}

\begin{remark}
    Any normed linear space $(V, \norm{\cdot})$ can be given a metric
    space structure by defining the distance $d(x, y)$ between $x, y \in V$
    as $\norm{x - y}$.

    The set $B(x, r) \coloneq \set{y \in V \mid \norm{x - y} < r}$ is called
    the open ball of radius $r$ centered at $x$.

    The set $S(x, r) \coloneq \set{y \in V \mid \norm{x - y} = r}$ is called
    the sphere of radius $r$ centered at $x$.
\end{remark}

\begin{exercise}[reverse triangle inequality] \label{thm:rti}
    Let $V$ be a normed linear space.
    Show that \begin{equation}
        \abs[\big]{\norm{x} - \norm{y}} \le \norm{x - y}
            \label{eq:rti}
    \end{equation} for any $x, y \in V$.
\end{exercise}
\begin{proof}
    First observe from homogeneity \labelcref{def:norm:hom} that
    $\norm{x} = \norm{-x}$ for any $x \in V$.
    Next, from the triangle inequality \labelcref{def:norm:tri} we have \[
        \norm{x} \le \norm{x - y} + \norm{y}
    \] so that \[
        \norm{x} - \norm{y} \le \norm{x - y}.
    \] Similarly, \[
        \norm{y} \le \norm{y - x} + \norm{x}
    \] so that \[
        -\norm{x - y} \le \norm{x} - \norm{y}.
    \] Combining these gives the result.
\end{proof}

This shows that $f = x \mapsto \norm{x}$ is a (Lipschitz) continuous
function on $V$.

\begin{definition}[metric space] \label{def:metric}
    A \emph{metric space} is a set $X$ equipped with a function
    $d\colon X \times X \to \R$ called a \emph{metric} that satisfies
    the following properties:
    \begin{enumerate}[label=\small(M\arabic*)]
        \item $d(x, y) \ge 0$ for any $x, y \in X$.
            \label{def:metric:pos}
        \item $d(x, y) = 0$ iff $x = y$.
            \label{def:metric:def}
        \item $d(x, y) = d(y, x)$ for any $x, y \in X$.
            \label{def:metric:sym}
        \item $d(x, z) \le d(x, y) + d(y, z)$ for any $x, y, z \in X$.
            \label{def:metric:tri}
    \end{enumerate}
\end{definition}

\begin{exercise}[self]
    Show that any normed linear space $(V, \norm{\cdot})$ is a metric space
    under the distance $d(x, y) = \norm{x - y}$.
\end{exercise}
\begin{proof}
    \labelcref{def:metric:pos} and \labelcref{def:metric:def} are immediate
    from \labelcref{def:norm:pos} and \labelcref{def:norm:def}.
    \labelcref{def:norm:hom} implies \labelcref{def:metric:sym} by
    scaling by $-1$.
    Triangle implies triangle.
\end{proof}

\begin{definition}[continuity] \label{def:continuity}
    Let $(X, d)$ and $(Y, \rho)$ be metric spaces.
    A function $f\colon X \to Y$ is called \emph{continuous} at $a \in X$
    iff \begin{align*}
        x_n \to a &\implies f(x_n) \to f(a), \text{ or} \\
        d(x_n, a) \to 0 &\implies \rho(f(x_n), f(a)) \to 0
    \end{align*}
\end{definition}

\begin{exercise}[product metric spaces]
    Let $(X_1, d_1)$ and $(X_2, d_2)$ be metric spaces.
    Let $d\colon X_1 \times X_2 \to \R$ be defined by \[
        d((x_1, x_2), (y_1, y_2)) \coloneq d_1(x_1, y_1) + d_2(x_2, y_2).
    \] Show that $d$ is a metric on $X_1 \times X_2$.

    Let $(z_n)_{n \in \N} = ((x_n, y_n))_{n \in \N}$ be a sequence in
    $X_1 \times X_2$.
    Show that $z_n \to (x, y)$ iff $x_n \to x$ and $y_n \to y$.
\end{exercise}
\begin{proof}
    Suppose $x_n \to x$ and $y_n \to y$.
    That is, $d_1(x_n, x) \to 0$ and $d_2(y_n, y) \to 0$.
    Thus $d_1(x_n, x) + d_2(y_n, y) \to 0$.

    Conversely if $d_1(x_n, x) + d_2(y_n, y) \to 0$ and each is nonnegative,
    then $d_1(x_n, x) \to 0$ and $d_2(y_n, y) \to 0$.
\end{proof}

\begin{remark}
    $\wtld{d}$ given by \[
        \wtld{d}((x_1, x_2), (y_1, y_2)) \coloneq
            \min\set{d_1(x_1, y_1), d_2(x_2, y_2)}
    \] is \emph{not} a metric on $X_1 \times X_2$ as it fails definiteness.

    However, $\max\set{d_1, d_2}$ \emph{is} a metric.
\end{remark}

\begin{exercise}
    Let $(V, \norm{\cdot})$ be a normed linear space.
    \begin{itemize}
        \item The addition map $(x, y) \mapsto x + y$ is a continuous map
        from $V \times V$ to $V$.
        \item The scalar multiplication map $(\alpha, x) \mapsto \alpha x$
        is continuous from $\R \times V$ to $V$.
    \end{itemize}
\end{exercise}
\begin{solution} \leavevmode
    \begin{itemize}
        \item $\norm{x' + y' - (x + y)} \le \norm{x' - x} + \norm{y' - y}
        = \norm{(x', y') - (x, y)}$.
        \item $\norm{\alpha' x' - \alpha x} \le \norm{\alpha' x' - \alpha x'}
        + \norm{\alpha x' - \alpha x} = \abs{\alpha' - \alpha} \norm{x'}
        + \abs{\alpha} \norm{x' - x}$.

        Thus choosing $\delta = \varepsilon / \max\set{\abs{\alpha}, \norm{x}}$
        gives \[
            \norm{\alpha' x' - \alpha x} \le \max\set{\abs{\alpha}, \norm{x}}
                (\abs{\alpha' - \alpha} + \norm{x' - x}) < \varepsilon
        \] whenever $\abs{\alpha' - \alpha} + \norm{x' - x} < \delta$.
    \end{itemize}
    Repeated in \refifdef{prb:cont}{\cref}{assignment 1, problem 1}.
\end{solution}

\begin{examples}
    \item ($\ell^p$ norm) $\R^n$ with $p \in [1, \infty]$ and \begin{align*}
        {\norm x}_p &\coloneq
            \paren[\Big]{\abs{x_1}^p + \cdots + \abs{x_n}^p}^{1/p}
        \shortintertext{where}
        {\norm x}_\infty &\coloneq \max\set{\abs{x_1}, \ldots, \abs{x_n}}
    \end{align*} is the limit of the $l^p$ norms as $p \to \infty$.
\end{examples}

\begin{exercise}
    See \refifdef{prb:lp}{\cref}{assignment 1, problem 5}.
\end{exercise}

\begin{definition}[norm equivalence] \label{def:norm_equivalence}
    Let ${\norm\cdot}_a$ and ${\norm\cdot}_b$ be two norms on $V$.
    We say that $\norm{\cdot}_a$ and $\norm{\cdot}_b$ are \emph{equivalent}
    if these exist $c_1, c_2 > 0$ such that \[
        c_1 {\norm x}_a \le {\norm x}_b \le c_2 {\norm x}_a
    \] for all $x \in V$.
    We write ${\norm\cdot}_a \sim {\norm\cdot}_b$.
\end{definition}
\begin{exercise}
    Check that $\sim$ is an equivalence relation.
\end{exercise}
\begin{solution}
    Reflexivity is obvious.
    Symmetry is since \[
        c_1 {\norm x}_a \le {\norm x}_b \le c_2 {\norm x}_a \implies
        \frac{1}{c_2} {\norm x}_b \le {\norm x}_a \le \frac{1}{c_1} {\norm x}_b.
    \]
    For transitivity, let \begin{align*}
        c_1 {\norm x}_a &\le {\norm x}_b \le c_2 {\norm x}_a, \\
        c_3 {\norm x}_b &\le {\norm x}_c \le c_4 {\norm x}_b.
    \end{align*}
    Then \[
        c_1 c_3 {\norm x}_a \le {\norm x}_c \le c_2 c_4 {\norm x}_a. \qedhere
    \]
\end{solution}
