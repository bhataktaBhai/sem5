\chapter{Machine learning} \label{chp:ml}
\lecture{2024-08-05}{}

\begin{definition}[Training] \label{def:ml:training}
    Given a family of functions $\mcH = \set{h_\theta}_{\theta \in \Theta}$
    from $X$ to $Y$ and a loss function $L\colon \mcH \to \R$,
    \emph{training} is the process of finding a $\theta \in \Theta$
    that minimizes the loss $L(h_\theta)$.
\end{definition}

\section{Continuous \& categorical data} \label{sec:ml:cont-cat}
The amount of rainfall tomorrow in millilitres is a continuous variable.
Whether or not it will rain is considered a categorical variable.

Catergorical data can further be divided into ordinal and nominal data.
If $X$ takes values in $\set{1, 2, 3}$, it is an ordinal variable.
If it takes values in $\set{\text{red}, \text{green}, \text{blue}}$,
it is a nominal variable.

Predicting continuous data is called \emph{regression}.
Predicting categorical data is called \emph{classification}.

\section{Types of models} \label{sec:ml:models}
\begin{itemize}
    \item Linear regression
    \item Gradient boosted tree model, best for tabular and time series data
    \begin{itemize}
        \item Extreme gradient boosting (XGBoost)
        \item Light gradient boosting (LightGBM)
        \item Catetgorical gradient boosting (CatBoost)
    \end{itemize}
    \item Neural networks, best for image, video, text and audio.
    \begin{itemize}
        \item Convolutional neural networks (CNN) for image and video
        \item Recurrent neural networks (RNN)
        \item Transformers
    \end{itemize}
\end{itemize}

\subsection{Predictive \& generative models} \label{sec:pred-gen}
\subsubsection*{Predictive AI}
\begin{itemize}
    \item Input: Any data modality
    \item Output: Continuous or categorical data
\end{itemize}
\subsubsection*{Generative AI}
\begin{itemize}
    \item Input: Any data modality
    \item Output: Text, image, video, audio
\end{itemize}

\section{Foundational model} \label{sec:found}
A foundational model is one that has a broad knowledge and capability.
We interact with it using prompts.
Can be fine-tuned to specialize in a particular domain.

\textbf{Examples:} GPT, BERT, T5
