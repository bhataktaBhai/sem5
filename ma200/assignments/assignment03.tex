\documentclass[12pt]{article}
% \usepackage{parskip}
\usepackage{lmodern} % https://tex.stackexchange.com/a/58088/295755
\renewcommand\bfdefault{b}
\usepackage{microtype}

\usepackage{amsmath}
\newcommand\yesnumber{\stepcounter{equation}\tag{\theequation}}

\ifundef{\chapter}{}{\providecommand\thechapter{\Roman{chapter}}}

\usepackage{marginnote}
\usepackage[en-GB,calc]{datetime2}
\usepackage{calc}
\usepackage{xifthen}
\usepackage{tocloft}

% https://tex.stackexchange.com/a/454168
\newcommand\monthday[1]{\DTMmonthname{\DTMfetchmonth{#1}}~\number\DTMfetchday{#1}}

\newlistof{lecture}{lec}{Lectures} % what is this file extension business?
\makeatletter
\setlength\marginparwidth{1in}
\newcommand*{\lecture}[3][]{
    \ifthenelse{\isempty{#1}}{%
        \refstepcounter{lecture}%
    }{%
        \setcounter{lecture}{#1}%
    }%
    \DTMsavedate{lecdate}{#2}%
    \def\lecdow{\DTMweekdayname{\DTMfetchdow{lecdate}}}
    \def\lecshortdow{\DTMshortweekdayname{\DTMfetchdow{lecdate}}}
    \def\lecmonth{\DTMmonthname{\DTMfetchmonth{lecdate}}}
    \def\lecday{\number\DTMfetchday{lecdate}}
    % \marginpar{\raggedright\small \textsf{\textbf{Lecture \thelecture.}%
    %             \footnotesize\DTMusedate{lecdate}}}%
    \marginnote{\raggedright\small%
        \textsf{{\textbf{Lecture \thelecture.}} \\
        \footnotesize\lecdow\\\lecmonth\ \lecday}}%
    \ifthenelse{\isempty{#3}}{%
        \addcontentsline{lec}{lecture}{\protect\numberline{\thelecture}%
        \lecshortdow, \lecmonth\ \lecday}%
        \def\@lecture{Lecture \thelecture}%
    }{%
        \addcontentsline{lec}{lecture}{\protect\numberline{\thelecture}%
        \makebox[\widthof{Mon,}][l]{\lecshortdow,}\ \makebox[\widthof{September 00}][l]{\lecmonth\ \lecday} #3}%
        \def\@lecture{Lecture \thelecture: #3}%
    }%
    \par%
}
\g@addto@macro\normalsize{%
  \setlength\abovedisplayskip{7pt}%
  \setlength\belowdisplayskip{7pt}%
  \setlength\abovedisplayshortskip{1pt}%
  \setlength\belowdisplayshortskip{1pt}%
}
\makeatother

\usepackage[twoside]{fancyhdr}
\setlength{\headheight}{15pt}
\pagestyle{fancy}
\fancyhf{}
% \fancyhead[r]{\thepage}
\makeatletter
\fancyhead[LE,RO]{\thepage}
\fancyhead[RE]{\textbf{\nouppercase\leftmark}}
\fancyhead[LO]{\nouppercase\rightmark}
\providecommand\@lecture{}
\fancyfoot[R]{\small\@lecture}
\makeatother

% homeworks
\newlistof{hw}{hw}{Assignments} % counter `assignment' already defined
\makeatletter
\newcommand*{\assignment}[5][]{%[number]{file}{date posted}{date due}{date quiz}
    \ifthenelse{\isempty{#1}}{%
        \refstepcounter{hw}%
        \stepcounter{assignment}%
    }{%
        \setcounter{hw}{#1}%
        \setcounter{assignment}{#1}%
    }%
    \pagebreak
    \ifthenelse{\isempty{#3}}{}{\DTMsavedate{posted}{#3}}%
    \ifthenelse{\isempty{#4}}{}{\DTMsavedate{due}{#4}}%
    \ifthenelse{\isempty{#5}}{}{\DTMsavedate{quiz}{#5}}%
    \section*{Assignment \thehw}
    \ifthenelse{\isempty{#4}}{
        \ifthenelse{\isempty{#5}}{
            \ifthenelse{\isempty{#3}}{
                \def\hw@toc{}
            }{
                \def\hw@toc{posted \monthday{up}}
            }
        }{
            \def\hw@toc{quiz \monthday{quiz}}
        }
    }{
        \def\hw@toc{due \monthday{due}}
    }
    \addcontentsline{hw}{hw}{\protect\numberline{\thehw}\hw@toc}\par
    \marginpar{\raggedright\footnotesize\textsf{%
        \ifthenelse{\isempty{#3}}{}{\makebox[\widthof{quiz}][l]{up} \monthday{posted} \\}%
        \ifthenelse{\isempty{#4}}{}{\makebox[\widthof{quiz}][l]{due} \monthday{due} \\}%
        \ifthenelse{\isempty{#5}}{}{quiz \monthday{quiz}}%
    }}
    \def\@lecture{Assignment \thehw\ifx\hw@toc\empty{}\else\ --- \hw@toc\fi}
    \input{#2}
    \newpage
}
\makeatother

\usepackage{amsmath}
\usepackage{amsthm}
\usepackage[dvipsnames]{xcolor}
\colorlet{exercise}{cyan!70!black}
\colorlet{solved}{green!40!black}
\colorlet{self_proof}{blue!70!black}
\colorlet{Red}{red!80!black}

% Gilles Castel's theorems
\newtheoremstyle{mddefinition}% <name>
  {-.25\topsep}%                 <space above>
  {-.25\topsep}%                         <space below>
  {\normalfont}%              <body font>
  {}%                         <indent amount>
  {\bfseries}%                <theorem head font>
  {.}%                        <punctuation after theorem head>
  {.5em}%                     <space after theorem head>
  {}%                         <theorem head spec>
\newtheoremstyle{mdplain}% <name>
  {-.25\topsep}%                 <space above>
  {-.25\topsep}%                         <space below>
  {\itshape}%                 <body font>
  {}%                         <indent amount>
  {\bfseries}%                <theorem head font>
  {.}%                        <punctuation after theorem head>
  {.5em}%                     <space after theorem head>
  {}%                         <theorem head spec>

\usepackage[framemethod=Tikz]{mdframed}
\mdfsetup{skipbelow=0pt}
\mdfdefinestyle{axiomstyle}{
    outerlinewidth = 1.5,
    roundcorner = 10,
    leftmargin = 15,
    rightmargin = 15,
    backgroundcolor = yellow!7
}
\mdfdefinestyle{defstyle}{
    outerlinewidth = 1,
    roundcorner = 2,
    leftmargin = 7,
    rightmargin = 7,
    backgroundcolor = green!5
}
\mdfdefinestyle{thmstyle}{
    outerlinewidth = 1,
    roundcorner = 8,
    leftmargin = 7,
    rightmargin = 7,
    backgroundcolor = cyan!5
}
\mdfdefinestyle{lemmastyle}{
    outerlinewidth = 1.5,
    roundcorner = 10,
    leftmargin = 7,
    rightmargin = 7,
    backgroundcolor = yellow!10
}
\ifundef\chapter{%
    \providecommand\theoremnumberwithin{section}
    \theoremstyle{mddefinition}
    \newmdtheoremenv[nobreak=true, style=axiomstyle]{axiom}{Axiom}[section]
    \theoremstyle{plain}
    \newtheorem{theorem}{Theorem}[\theoremnumberwithin]

    \newcounter{assignment}
    \theoremstyle{plain}
    \newtheorem{problem}{Problem}
    \theoremstyle{mddefinition}
    \newmdtheoremenv[nobreak=true, outerlinewidth=0.7]{problem*}[problem]{Problem}
}{
    \providecommand\theoremnumberwithin{chapter}
    \theoremstyle{mddefinition}
    \newmdtheoremenv[nobreak=true, style=axiomstyle]{axiom}{Axiom}[chapter]
    \theoremstyle{plain}
    \newtheorem{theorem}{Theorem}[\theoremnumberwithin]

    \newcounter{assignment}
    \theoremstyle{plain}
    \newtheorem{problem}{Problem}[assignment]
    \theoremstyle{mddefinition}
    \newmdtheoremenv[nobreak=true, outerlinewidth=0.7]{problem*}[problem]{Problem}
}
\theoremstyle{mddefinition}
\newmdtheoremenv[nobreak=true, style=defstyle]{definition*}[theorem]{Definition}

\theoremstyle{mdplain}
\newmdtheoremenv[nobreak=true, style=thmstyle]{theorem*}[theorem]{Theorem}
\newmdtheoremenv[nobreak=true]{proposition*}[theorem]{Proposition}
\newmdtheoremenv[nobreak=true]{lemma*}[theorem]{Lemma}
\newmdtheoremenv[nobreak=true]{corollary*}[theorem]{Corollary}
\newmdtheoremenv[nobreak=true, style=thmstyle]{fact*}[theorem]{Fact}
\newmdtheoremenv[nobreak=true]{exercise*}[theorem]{Exercise}
\newmdtheoremenv[nobreak=true]{question*}[theorem]{Question}

\theoremstyle{definition}
\newtheorem{definition}[theorem]{Definition}

\theoremstyle{plain}
\newtheorem{proposition}[theorem]{Proposition}
\newtheorem{lemma}[theorem]{Lemma}
\newtheorem{corollary}[theorem]{Corollary}
\newtheorem{fact}[theorem]{Fact}
\newtheorem{exercise}[theorem]{Exercise}
\newtheorem{question}[theorem]{Question}

\theoremstyle{remark}
\newtheorem*{remark}{Remark}
\newtheorem*{remarkx}{Remarks}
\newtheorem*{example}{Example}
\newtheorem*{examplex}{Examples}
\newtheorem*{idea}{Idea}
%%%% HAXXXXXX %%%%
% \def\innerqed{\qedsymbol}
% \def\outerqed{$\blacksquare$}
\let\oldproof\proof
\let\endoldproof\endproof
\newenvironment{solution}[1][]
  {\renewcommand\qedsymbol{$\blacksquare$}%
  \begin{oldproof}[Solution\ifx&#1&\else\ (#1)\fi]}
  {\end{oldproof}}
\newenvironment{answer}
  {\renewcommand\qedsymbol{$\blacksquare$}\begin{oldproof}[Answer]}
  {\end{oldproof}}
\renewenvironment{proof}
  {\renewcommand\qedsymbol{$\blacksquare$}\begin{oldproof}}
  {\end{oldproof}}
\newenvironment{subproof}[1][Subproof]{%
  \renewcommand{\qedsymbol}{$\square$}\begin{oldproof}[#1]}
  {\end{oldproof}}
\newtheorem*{notation}{Notation}
\newtheorem*{claim}{Claim}

\usepackage{hyperref}
\usepackage[noabbrev]{cleveref}

% <cref>
\crefname{theorem}{theorem}{theorems}
\crefname{proposition}{proposition}{propositions}
\crefname{lemma}{lemma}{lemmas}
\crefname{corollary}{corollary}{corollaries}
\crefname{axiom}{axiom}{axioms}
\crefname{definition}{definition}{definitions}
\crefname{problem}{problem}{problems}
\crefname{exercise}{exercise}{exercises}
\crefname{fact}{fact}{facts}
\crefname{question}{question}{questions}
\crefname{remark}{remark}{remarks}
\crefname{example}{example}{example}
\crefname{notation}{notation}{notations}
\crefname{claim}{claim}{claims}
% \crefname{section}{\S}{\S\S}
\crefname{theorem*}{theorem}{theorems}
\crefname{proposition*}{proposition}{propositions}
\crefname{lemma*}{lemma}{lemmas}
\crefname{corollary*}{corollary}{corollaries}
\crefname{definition*}{definition}{definitions}
\crefname{problem*}{problem}{problems}
\crefname{exercise*}{exercise}{exercises}
\crefname{fact*}{fact}{facts}
\crefname{question*}{question}{questions}
% </cref>

% <hyperlinks>
\hypersetup{colorlinks,
    linkcolor={blue},
    citecolor={blue!50!black},
    urlcolor={blue!80!black}}
% </hyperlinks>

\usepackage[shortlabels]{enumitem}
% change default label for enumerate, and fix long labels popping out
\setenumerate{label*=(\roman*),ref=(\roman*),leftmargin=*}
% casework list using https://tex.stackexchange.com/a/30035
\newcounter{casecount}
\newlist{casework}{description}{1}
\setlist[casework]{%
  before={\setcounter{casecount}{0}%
      \renewcommand*\thecasecount{\arabic{casecount}}}%
  ,font=\bfseries Case \stepcounter{casecount}\thecasecount:
}

\newenvironment{examples}[1][]
{\begin{examplex}[#1]\leavevmode\begin{itemize}}{\end{itemize}\end{examplex}}
\newenvironment{remarks}[1][]
{\begin{remarkx}[#1]\leavevmode\begin{itemize}}{\end{itemize}\end{remarkx}}

% omg this is so HaXy
% \renewenvironment{proof}[1][\proofname]{{\it\bfseries #1. }}{\qed}
% \providecommand{\qedsymbol}{\openbox}
% \makeatletter
% \renewenvironment{proof}[1][\proofname]{\par
%   \pushQED{\qed}%
%   \normalfont \topsep6\p@\@plus6\p@\relax
%   \trivlist
%   \item[\hskip\labelsep
%         \itshape\bfseries%this is the change (boldface instead of italics)
%         % \fontseries{bx}\selectfont
%     #1\@addpunct{.}]\ignorespaces
% }{%
%   \popQED\endtrivlist\@endpefalse
% }
\makeatother
\crefname{enumi}{part}{parts}
\crefname{enumii}{part}{parts}
\crefname{enumiii}{part}{parts}
% \setlist[itemize]{itemsep=2pt}
\newcounter{dummy}
\makeatletter
\newcommand\myitem[1][]{\item[#1]\refstepcounter{dummy}\def\@currentlabel{#1}}
\makeatother

\makeatletter
\newcommand*{\refifdef}[3]{%label,command,fallback
    \@ifundefined{r@#1}{#3}{#2{#1}}%
}
\makeatother

\newcommand\ie{\textit{i.e.}}
\newcommand\eg{\textit{e.g.}}
\usepackage{physics}

\usepackage{amsmath}
\usepackage{amssymb}
\usepackage{mathrsfs} % for \mathscr
\usepackage{bm} % for \bm
\usepackage{booktabs}

% undefine \abs and \norm
\let\abs\relax
\let\norm\relax

\usepackage{mathtools} % for delimiters and \coloneqq
\DeclarePairedDelimiter{\paren}{(}{)}
\DeclarePairedDelimiter{\brk}{[}{]}
\DeclarePairedDelimiter{\set}{\{}{\}}
\DeclarePairedDelimiter{\abs}{\lvert}{\rvert}
\DeclarePairedDelimiter{\norm}{\lVert}{\rVert}
\DeclarePairedDelimiter{\floor}{\lfloor}{\rfloor}
\DeclarePairedDelimiter{\ceil}{\lceil}{\rceil}
\DeclarePairedDelimiter{\angled}{\langle}{\rangle}
% \DeclarePairedDelimiterX{\innerp}[2]{\langle}{\rangle}{#1,\,#2}
% \DeclarePairedDelimiterX{\outerp}[2]{\langle}{\rangle}{#1\otimes#2}
% \DeclarePairedDelimiterX{\braket}[3]{\langle}{\rangle}%
% {#1\,\delimsize\vert\,\mathopen{}#2\,\delimsize\vert\,\mathopen{}#3}
\DeclarePairedDelimiterX{\innerp}[2]{\langle}{\rangle}{#1,\,#2}
% \DeclarePairedDelimiterX{\outerp}[2]{\langle}{\rangle}{#1\otimes#2}
\DeclarePairedDelimiterX{\outerp}[2]{\vert}{\vert}%
{#1\delimsize\rangle\delimsize\langle\mathopen{}#2}
\let\braket\relax
\DeclarePairedDelimiterX{\braket}[3]{\langle}{\rangle}%
{#1\,\delimsize\vert\,\mathopen{}#2\,\delimsize\vert\,\mathopen{}#3}

\renewcommand\O{\ensuremath{\varnothing}}
\newcommand\N{\ensuremath{\mathbb{N}}}
\newcommand\Z{\ensuremath{\mathbb{Z}}}
\newcommand\Q{\ensuremath{\mathbb{Q}}}
\newcommand\R{\ensuremath{\mathbb{R}}}
\newcommand\C{\ensuremath{\mathbb{C}}}
% \renewcommand\P{\ensuremath{\mathbb{P}}}

% fix spacing for \forall and \exists
% \let\oldforall\forall
% \renewcommand{\forall}{\oldforall \, }
% \let\oldexist\exists
% \renewcommand{\exists}{\oldexist \: }
\newcommand\unique{\exists!}
\newcommand\lxor{\oplus}

\providecommand{\dd}{\,\mathrm{d}}

\newcommand\mcA{\ensuremath{\mathcal{A}}}
\newcommand\mcB{\ensuremath{\mathcal{B}}}
\newcommand\mcC{\ensuremath{\mathcal{C}}}
\newcommand\mcD{\ensuremath{\mathcal{D}}}
\newcommand\mcE{\ensuremath{\mathcal{E}}}
\newcommand\mcF{\ensuremath{\mathcal{F}}}
\newcommand\mcG{\ensuremath{\mathcal{G}}}
\newcommand\mcH{\ensuremath{\mathcal{H}}}
\newcommand\mcI{\ensuremath{\mathcal{I}}}
\newcommand\mcJ{\ensuremath{\mathcal{J}}}
\newcommand\mcK{\ensuremath{\mathcal{K}}}
\newcommand\mcL{\ensuremath{\mathcal{L}}}
\newcommand\mcM{\ensuremath{\mathcal{M}}}
\newcommand\mcN{\ensuremath{\mathcal{N}}}
\newcommand\mcO{\ensuremath{\mathcal{O}}}
\newcommand\mcP{\ensuremath{\mathcal{P}}}
\newcommand\mcQ{\ensuremath{\mathcal{Q}}}
\newcommand\mcR{\ensuremath{\mathcal{R}}}
\newcommand\mcS{\ensuremath{\mathcal{S}}}
\newcommand\mcT{\ensuremath{\mathcal{T}}}
\newcommand\mcU{\ensuremath{\mathcal{U}}}
\newcommand\mcV{\ensuremath{\mathcal{V}}}
\newcommand\mcW{\ensuremath{\mathcal{W}}}
\newcommand\mcX{\ensuremath{\mathcal{X}}}
\newcommand\mcY{\ensuremath{\mathcal{Y}}}
\newcommand\mcZ{\ensuremath{\mathcal{Z}}}

%%%% WIDE BAR THAT IS JUST THE RIGHT LENGTH %%%%
%% FROM https://tex.stackexchange.com/a/60253 %%
\makeatletter
\let\save@mathaccent\mathaccent
\newcommand*\if@single[3]{%
  \setbox0\hbox{${\mathaccent"0362{#1}}^H$}%
  \setbox2\hbox{${\mathaccent"0362{\kern0pt#1}}^H$}%
  \ifdim\ht0=\ht2 #3\else #2\fi
  }
%The bar will be moved to the right by a half of \macc@kerna, which is computed by amsmath:
\newcommand*\rel@kern[1]{\kern#1\dimexpr\macc@kerna}
%If there's a superscript following the bar, then no negative kern may follow the bar;
%an additional {} makes sure that the superscript is high enough in this case:
\newcommand*\widebar[1]{\@ifnextchar^{{\wide@bar{#1}{0}}}{\wide@bar{#1}{1}}}
%Use a separate algorithm for single symbols:
\newcommand*\wide@bar[2]{\if@single{#1}{\wide@bar@{#1}{#2}{1}}{\wide@bar@{#1}{#2}{2}}}
\newcommand*\wide@bar@[3]{%
  \begingroup
  \def\mathaccent##1##2{%
%Enable nesting of accents:
    \let\mathaccent\save@mathaccent
%If there's more than a single symbol, use the first character instead (see below):
    \if#32 \let\macc@nucleus\first@char \fi
%Determine the italic correction:
    \setbox\z@\hbox{$\macc@style{\macc@nucleus}_{}$}%
    \setbox\tw@\hbox{$\macc@style{\macc@nucleus}{}_{}$}%
    \dimen@\wd\tw@
    \advance\dimen@-\wd\z@
%Now \dimen@ is the italic correction of the symbol.
    \divide\dimen@ 3
    \@tempdima\wd\tw@
    \advance\@tempdima-\scriptspace
%Now \@tempdima is the width of the symbol.
    \divide\@tempdima 10
    \advance\dimen@-\@tempdima
%Now \dimen@ = (italic correction / 3) - (Breite / 10)
    \ifdim\dimen@>\z@ \dimen@0pt\fi
%The bar will be shortened in the case \dimen@<0 !
    \rel@kern{0.6}\kern-\dimen@
    \if#31
      \overline{\rel@kern{-0.6}\kern\dimen@\macc@nucleus\rel@kern{0.4}\kern\dimen@}%
      \advance\dimen@0.4\dimexpr\macc@kerna
%Place the combined final kern (-\dimen@) if it is >0 or if a superscript follows:
      \let\final@kern#2%
      \ifdim\dimen@<\z@ \let\final@kern1\fi
      \if\final@kern1 \kern-\dimen@\fi
    \else
      \overline{\rel@kern{-0.6}\kern\dimen@#1}%
    \fi
  }%
  \macc@depth\@ne
  \let\math@bgroup\@empty \let\math@egroup\macc@set@skewchar
  \mathsurround\z@ \frozen@everymath{\mathgroup\macc@group\relax}%
  \macc@set@skewchar\relax
  \let\mathaccentV\macc@nested@a
%The following initialises \macc@kerna and calls \mathaccent:
  \if#31
    \macc@nested@a\relax111{#1}%
  \else
%If the argument consists of more than one symbol, and if the first token is
%a letter, use that letter for the computations:
    \def\gobble@till@marker##1\endmarker{}%
    \futurelet\first@char\gobble@till@marker#1\endmarker
    \ifcat\noexpand\first@char A\else
      \def\first@char{}%
    \fi
    \macc@nested@a\relax111{\first@char}%
  \fi
  \endgroup
}
\makeatother
\let\what\widehat
\let\wtld\widetilde
\let\wbar\widebar
\let\ubar\underline

\DeclareMathOperator\sgn{sgn}

\let\oldleft\left
\let\oldright\right
\renewcommand{\left}{\mathopen{}\mathclose\bgroup\oldleft}
\renewcommand{\right}{\aftergroup\egroup\oldright}


\setlist[enumerate,1]{label=(\alph*)}

\title{Assignment 3}
\author{Naman Mishra}
\date{11 September, 2024}

\begin{document}
\maketitle

% Problem 5
\setcounter{problem}{4}
\begin{problem}
%     Let f : R2 → R be given by
% ( x3 , (x,y)̸=(0,0), f(x,y) = x2+y2
% 0, (x, y) = (0, 0).
% (a) Show directly that f is continuous.
% (b) Prove that D1f and D2f are bounded functions in R2. (Hence f is
% continuous by Question 4).
% (c) Let u be any unit vector in R2. Show that the directional derivative
% Duf(0,0) exists, and that its absolute value is at most one.
% (d) Show that f is not differentiable at (0, 0).
    Let $f\colon \R^2 \to \R$ be given by \[
        f(x, y) = \begin{cases}
            \frac{x^3}{x^2 + y^2} & \text{if } (x, y) \ne (0, 0), \\
            0 & \text{if } (x, y) = (0, 0).
        \end{cases}
    \]
    \begin{enumerate}
        \item Show directly that $f$ is continuous.
        \item Prove that $D_1f$ and $D_2f$ are bounded functions in
        $\R^2$ (hence $f$ is continuous by
        \crefifdef{prb:bound-cont}{problem 4})
        \item Let $u$ be any vector in $\R^2$.
        Show that the directional derivative $D_uf(0, 0)$ exists, and that
        its absolute value is at most one.
        \item Show that $f$ is not differentiable at $(0, 0)$.
    \end{enumerate}
\end{problem}
\begin{solution} \leavevmode
    \begin{enumerate}
        \item Let $(x, y) \in \R^2 \setminus \set{(0, 0)}$ and
        let $(h, k) \in \R^2$ be such that $\norm{(h, k)} < \norm{(x, y)}$.
        Then \[
            f(x + h, y + k) = \frac{(x + h)^3}{(x + h)^2 + (y + k)^2}.
        \] As $(h, k) \to (0, 0)$, the numerator goes to $x^3$ and the
        denominator goes to $x^2 + y^2 > 0$.
        Therefore $f(x + h, y + k) \to f(x, y)$ as $(h, k) \to (0, 0)$.

        For continuity at $(0, 0)$, note that \[
            f(h, k) = \frac{h^3}{h^2 + k^2} \le \frac{h^3}{h^2} = h,
        \] so that $f(h, k) \to 0$ as $(h, k) \to (0, 0)$.
        \item For each $y \in \R$, let $g_y = x \mapsto f(x, y)$, and for
        each $x \in \R$, let $h_x = y \mapsto f(x, y)$.
        Then for $(x, y) \ne (0, 0)$, we have \begin{align*}
            D_1f(x, y) &= g'_y(x)
                = \frac{3x^2(x^2 + y^2) - x^3(2x)}{(x^2 + y^2)^2}
                = \frac{x^4 + 3x^2y^2}{(x^2 + y^2)^2} \\
            D_2f(x, y) &= h'_x(y)
                = \frac{-2x^3y}{(x^2 + y^2)^2}.
        \end{align*}
        Write \begin{align*}
            \abs{D_1f(x, y)} &= \frac{x^2}{x^2 + y^2} \frac{x^2 + 3y^2}{x^2 + y^2}
                \le 1 \\
            \abs{D_2f(x, y)} &= \frac{x^2}{x^2 + y^2} \frac{2\abs{xy}}{x^2 + y^2}
                \le 1 \tag{AM-GM}.
        \end{align*}
        For $(0, 0)$, we have $g_0(x) = x$ and $h_0(y) = 0$.
        Thus \[
            D_1f(0, 0) = 1, \quad D_2f(0, 0) = 0.
        \] Thus $D_1f$ and $D_2f$ are both bounded by $1$ in all of $\R^2$.
        \item Let $u = (u_1, u_2) \ne 0$.
        ($D_{(0, 0)}f$ is trivially $0$.)
        Then \begin{align*}
            D_uf(0, 0) &= \lim_{t \to 0} \frac{f(tu_1, tu_2)}{t} \\
            &= \lim_{t \to 0} \frac{t^3u_1^3}{t^2(u_1^2 + u_2^2) \cdot t}
            &= \frac{u_1^3}{u_1^2 + u_2^2}.
        \end{align*}
        \item If $f$ were differentiable at $(0, 0)$, then $f'(0, 0)$ would
        be the given by the matrix $\begin{bmatrix} 1 & 0 \end{bmatrix}$.
        Then $D_uf(0, 0) = u_1$ for all $u \ne 0$.
        By the previous part, this is not the case (for example,
        when $u = (1, 1)$).
        Thus $f$ is not differentiable at $(0, 0)$. \qedhere
    \end{enumerate}
\end{solution}

% Problem 14
\setcounter{problem}{13}
\begin{problem}
%     Letf:R→Rbegivenby
% f(x) =
%  x+2x2sin1, x̸=0, x
% 0,
% x = 0.
% Show that
% (a) f is differentiable on R and f′(0) = 1.
% (b) f′ is continuous except at 0.
% (c) f′ is bounded on (−1,1).
% (d) on any open interval around 0, there are points x with f′(x) > 0 and
% also points x with f′(x) < 0.
% (e) f is not one-to-one in any neighborhood of 0. (Hint: use question
% 13)
% Thus the continuity of f′ can not be eliminated from the hypothesis of
% inverse function theorem even for the case n = 1.
    Let $f\colon \R \to \R$ be given by \[
        f(x) = \begin{cases}
            x + 2x^2\sin\frac1x & \text{if } x \ne 0, \\
            0 & \text{if } x = 0.
        \end{cases}
    \]
    Show that
    \begin{enumerate}
        \item $f$ is differentiable on $\R$ and $f'(0) = 1$.
        \item $f'$ is continuous except at $0$.
        \item $f'$ is bounded on $(-1, 1)$.
        \item On any open interval around $0$, there are points $x$ with
        $f'(x) > 0$ and also points $x$ with $f'(x) < 0$.
        \item $f$ is not one-to-one in any neighborhood of $0$.
    \end{enumerate}
    Thus the continuity of $f'$ cannot be eliminated from the hypothesis of
    the inverse function theorem even for the case $n = 1$.
\end{problem}
\begin{solution} \leavevmode
    \begin{enumerate}
        \item Sum, product and chain rules give that
        $f$ is differentiable everywhere on $\R^\times$.
        We can compute it as \[
            f'(x) = 1 + 4x \sin\frac1x - 2\cos\frac1x.
        \] For differentiability at $0$, we have \[
            f'(0) = \lim_{h \to 0} 1 + 2h\sin\frac1h = 1.
        \]
        \item As the sum, product, composition of continuous functions,
        $f'$ is continuous everywhere on $\R^\times$.
        For continuity at $0$, consider the sequence $x_n = \frac1{2n\pi}$.
        As $n \to \infty$, $x_n \to 0$ but $f'(x_n) \to -1 \ne f(0)$.
        Thus $f'$ is not continuous at $0$.
        \item We know $f'(0) = 1$, and for $\abs{x} < 1$, we have \[
            \abs{f'(x)} \le 1 + 4\abs{x} + 2 \le 7.
        \]
        \item We have already produced a sequence $x_n \to 0$ such that
        $f'(x_n) = -1$ for each $n$.
        Consider the sequence $y_n = \frac1{(2n + 1)\pi}$.
        Then $y_n \to 0$ and $f'(y_n) = 3$ for each $n$.
        Thus there exist $x$ arbitrarily close to $0$ with $f'(x) > 0$,
        and $x$ arbitrarily close to $0$ with $f'(x) < 0$.
        \item Let $U$ be any open neighborhood of $0$.

        $f$ is continuous since it is differentiable.
        By problem 13, $f$ is one-to-one on $U$ only if it is either
        strictly increasing or strictly decreasing on $U$.

        If $f$ is increasing on $U$, then $f'(x) \ge 0$ for all $x \in U$.
        If $f$ is decreasing on $U$, then $f'(x) \le 0$ for all $x \in U$.
        However, from the previous part, we know that there are points
        in $U$ where $f'$ is positive and points where $f'$ is negative.
        Thus neither of these cases can hold, so $f$ is not one-to-one
        in any neighborhood of $0$. \qedhere
    \end{enumerate}
\end{solution}

% Problem 16
\setcounter{problem}{15}
\begin{problem} \label{prb:lin-cont}
    % Suppose X,Y are normed linear spaces over R (need not be of finite di- mensional) and T : X → Y is linear. Prove that the following are equiv- alent.
    % (a) T is continuous at some point of X. (b) T is continuous at 0.
    % (c) T is continuous.
    % (d) There exists a constant C > 0 such that ∥T(x)∥ ≤ C∥x∥ for all x ∈ X.
    % (e) For every bounded subset V of X, T (V ) is bounded in Y.
    % (f) T(U)isaboundedsubsetofY,whereU:={x∈X:∥x∥≤1}.
    % (g) T is uniformly continuous.
    Suppose $X, Y$ are normed linear spaces over $\R$ (need not be of finite
    dimension) and $T\colon X \to Y$ is linear.
    Prove that the following are equivalent.
    \begin{enumerate}
        \item \label{prb:lin-cont:some}
            $T$ is continuous at some point of $X$.
        \item \label{prb:lin-cont:0}
            $T$ is continuous at $0$.
        \item \label{prb:lin-cont:cont}
            $T$ is continuous.
        \item \label{prb:lin-cont:Ox}
            There exists a constant $C > 0$ such that
            $\norm{T(x)} \le C\norm{x}$ for all $x \in X$.
        \item \label{prb:lin-cont:bounded}
            For every bounded subset $V$ of $X$, $T(V)$ is bounded in $Y$.
        \item \label{prb:lin-cont:U}
            $T(U)$ is a bounded subset of $Y$, where
            $U \coloneq \set{x \in X : \norm{x} \le 1}$.
        \item \label{prb:lin-cont:unif}
            $T$ is uniformly continuous.
    \end{enumerate}
\end{problem}
\begin{solution}
    Each implication in \[
        \ref{prb:lin-cont:unif}
        \implies \ref{prb:lin-cont:cont}
        \implies \ref{prb:lin-cont:0}
        \implies \ref{prb:lin-cont:some}
    \] is obvious.
    We prove $\ref{prb:lin-cont:some} \implies \ref{prb:lin-cont:unif}$
    to close this chain.

    \begin{claim}
        If $T$ is continuous at some $x_0 \in X$, then $T$ is uniformly
        continuous.
    \end{claim}
    \begin{subproof}[Proof of claim]
        Let $\eps > 0$ and $\delta > 0$ be such that \[
            \norm{T(x) - T(x_0)} < \eps
            \quad\text{whenever } \norm{x - x_0} < \delta.
        \] Let $x, x' \in X$ be such that $\norm{x - x'} < \delta$.
        Then \begin{align*}
            \norm{T(x) - T(x')}
            &= \norm{T(x_0 + (x - x')) - T(x_0)} \tag{linearity} \\
            &< \eps
        \end{align*} since $(x_0 + (x - x')) - x_0 = x - x'$.
        Thus $T$ is uniformly continuous.
    \end{subproof}

    We now show \[
        \ref{prb:lin-cont:Ox}
        \implies \ref{prb:lin-cont:bounded}
        \implies \ref{prb:lin-cont:U}
        \implies \ref{prb:lin-cont:Ox}.
    \]
    \begin{enumerate}
        \item[($\ref{prb:lin-cont:Ox} \implies \ref{prb:lin-cont:bounded}$)]
        Let $V$ be such that $\norm v \le M$ for all $v \in V$.
        Then for any $v \in V$, we have \[
            \norm{T(v)} \le C\norm v \le CM
        \] so that $T(V)$ is bounded.
        \item[($\ref{prb:lin-cont:bounded} \implies \ref{prb:lin-cont:U}$)]
        $U$ is bounded, so $T(U)$ is bounded.
        \item[($\ref{prb:lin-cont:U} \implies \ref{prb:lin-cont:Ox}$)]
        Let $T(u) \le C$ for all $u \in U$.
        Let $x \in X$ be arbitrary.
        Since $T$ is linear, \begin{align*}
            T(x) &= \norm x T\ab(\frac{x}{\norm x}) \\
            &\le C \norm x
        \end{align*} since $\frac{x}{\norm x} \in U$.
    \end{enumerate}
    We have proven \[
        \ref{prb:lin-cont:unif}
            \iff \ref{prb:lin-cont:cont}
            \iff \ref{prb:lin-cont:0}
            \iff \ref{prb:lin-cont:some}
        \quad\text{and}\quad
        \ref{prb:lin-cont:Ox}
            \iff \ref{prb:lin-cont:bounded}
            \iff \ref{prb:lin-cont:U}.
    \] We prove $\ref{prb:lin-cont:0} \iff \ref{prb:lin-cont:U}$ to prove
    equivalence of all statements.

    Suppose $T$ is continuous at $0$.
    Then there exists $\delta > 0$ such that \[
        \norm{T(x)} < 1
            \quad\text{whenever } \norm x < \delta.
    \] Plugging $x = \delta u$ gives \[
        \norm{T(\delta u)} < 1
            \quad\text{whenever } \norm{\delta u} < \delta.
    \] By linearity of $T$ and absolute homogeneity of the norm, we have \[
        \norm{T(u)} < \frac1{\delta} \quad\text{whenever } \norm u < 1.
    \] Thus $T(U)$ is bounded.

    Conversely, suppose $T(U)$ is bounded.
    Then for any $x \in X$, $\norm{T(x)} \le C\norm x$ for some $C > 0$
    (since $\ref{prb:lin-cont:U} \implies \ref{prb:lin-cont:Ox}$).
    For any $\eps > 0$, choose $\delta = \frac{\eps}{C}$.
    Then if $\norm x < \delta$, we have \[
        \norm{T(x)} \le C\norm x < C\delta = \eps.
    \] Thus $T$ is continuous at $0$.
\end{solution}

\end{document}
