\section{Rectangle stabbing} \label{sec:stab}
\lecture{2024-09-18}{Applications of segment trees}

\begin{question*}
    Given a set $\mcR$ of $n$ axis-aligned cells in $\R^d$, process
    queries of the form $q \in \R^d$, reporting all $R \in \mcR$ that
    contain $q$.
\end{question*}
We will write $\sset{R \in \mcR}{q \in R}$ as $q \stab \mcR$
Consider the 2D case first.
\begin{question*}
    Given a set $\mcR$ of $n$ axis-aligned rectangles in $\R^2$, process
    queries of the form $q \in \R^2$, reporting all $R \in \mcR$ that
    contain $q$.
\end{question*}
\begin{solution}
    
\end{solution}

\section{Vertical ray shooting} \label{sec:vrs}
\begin{question*}
    Given a set $\mcS$ of $n$ interior-disjoint (might have common
    endpoints) segments in $\R^2$, process queries of the form
    $q \in \R^2$, reporting the first segment hit by a vertical ray
    emanating upwards from $q$.
\end{question*}
\begin{solution}
    Define $\mcS_x = \sset{\pi_x(S)}{S \in \mcS}$.
    Build a segment tree $T$ on $\mcS_x$.
    For any node $v \in T$, let $\mcS(v)$ denote the segments stored at $v$.

    For any query $q = (x, y)$, query $T$ with $x$ and let $\Pi$ be the
    search path.
    We can solve the problem at each node $v \in \Pi$ in $O(\log n)$ time
    by binary searching through $\mcS(v)$.
    Since they span the range of $v$, the segments in $\mcS(v)$ are totally
    ordered by their $y$-coordinates.

    Doing this for each node in $\Pi$ and picking the lowest of these gives
    an $O(\log^2 n)$ query time.
\end{solution}
