\lecture{2024-08-28}{}
\begin{remark}
    Suppose $H_1, \dots, H_m \le G$
    and $H_{i+1}$ is orthogonal to $H_1 H_2 \dots H_i$ for all $i$,
    and $h_i h_j = h_j h_i$ for all $h_i \in H_i$ and $h_j \in H_j$.

    Then $H_1 \times \cdots \times H_m \le G$ and is isomorphic to
    $H_1 H_2 \dots H_m$.
\end{remark}

\begin{lemma*} \label{thm:left-right}
    Let $H \le G$.
    Then the left and right cosets of $H$ in $G$ are in bijection.
\end{lemma*}
\begin{proof}
    Let $I \subseteq G$ be a set of representatives for the left cosets of
    $H$ in $G$.
    Then \begin{align*}
        G &= \bigsqcup_{x \in I} x H \\
        \implies G = G^{-1} &= \bigcup_{x \in I} H^{-1} x^{-1} \\
        &= \bigcup_{x \in I} H x^{-1}.
    \end{align*}
    Note that $H x^{-1} = H y^{-1} \iff x^{-1} y \in H \iff xH = yH$.
    Thus the union is still disjoint, and
    $I^{-1} = \sset{x^{-1}}{x \in I}$ is a set of
    representatives for the right cosets.
\end{proof}
\begin{definition*}[index] \label{def:index}
    Let $H \le G$,
    Then $(G : H) = \#(G / H)$ is the \emph{index} of $H$ in $G$
    if $G/H$ is finite.
    Otherwise, we say $(G : H) = \infty$.
\end{definition*}

\begin{proposition}
    Let $K \le H \le G$.
    Let $G = \bigsqcup\limits_{x \in I} xH$ and
    $H = \bigsqcup\limits_{y \in J} yK$
    Then \[
        G = \bigsqcup_{\substack{x \in I \\ y \in J}} xyK.
    \]
\end{proposition}
\begin{proof}
    \[
        G = \bigsqcup_{x \in I} xH
        = \bigsqcup_{x \in I} \ab(x \bigsqcup_{y \in J} yK)
        = \bigcup_{x \in I, y \in J} xyK.
    \] We need to check that the union is disjoint.
    \begin{align*}
        xyK = x_1y_1K &\implies y^{-1}x^{-1}x_1y_1 \in K \\
            &\implies x^{-1}x_1 \in yK y_1^{-1} \subseteq HKH^{-1} \\
            &\implies x^{-1}x_1 \in H \\
            &\implies xH = x_1H.
    \end{align*}
    This contradicts the disjointness of $\set{xH}_{x \in I}$.
    Thus \[
        G = \bigsqcup_{x \in I, y \in J} xyK. \qedhere
    \]
\end{proof}

\begin{corollary*}[multiplicity of the index] \label{thm:index:mult}
    Let $K \le H \le G$.
    Then $(G : K) = (G : H)(H : K)$,
    with the understanding that $\infty$ is absorbing.
\end{corollary*}
\begin{example}
    $6\Z \le 2\Z \le \Z$.
    \[
        2\Z = (0 + 6\Z) \sqcup (2 + 6\Z) \sqcup (4 + 6\Z)
    \] so $(2\Z : 6\Z) = 3$.
    \[
        \Z = (0 + 2\Z) \sqcup (1 + 2\Z)
    \] so $(\Z : 2\Z) = 2$.
    Similarly $(\Z : 6\Z) = 6$
    and sure enough, $(\Z : 6\Z) = (\Z : 2\Z)(2\Z : 6\Z)$.
\end{example}
Lagrange's theorem is a special case of this corollary.
Choosing $K = \set 1$ gives $(G : K) = \abs G$, $(H \colon K) = \abs H$,
and $\abs G = \abs H \cdot (G : H)$.

\section{Isomorphism theorems} \label{sec:iso-thm}
\begin{theorem*}[first isomorphism theorem] \label{thm:iso:1}
    Let $\varphi\colon G \to H$ be a homomorphism.
    Let $K = \ker(\varphi)$.
    Then $\varphi$ induces an isomorphism
    $\varphi_*\colon G/K \to \im(\varphi)$
    given by $\varphi_*(gK) = \varphi(g)$.
\end{theorem*}
\begin{proof}
    This is well-defined since
    $gK = g'K \iff g^{-1}g' \in K \iff \varphi(g) = \varphi(g')$.
    This also shows injectivity.

    $\varphi_*$ is clearly a homomorphism.
    It is surjective since for any $y = \varphi(g) \in \im(\varphi)$,
    $y = \varphi_*(gK)$.
\end{proof}

\begin{theorem*}[second isomorphism theorem] \label{thm:iso:2}
    Let $A, B \le G$.
    Suppose $A \subseteq N_G(B)$.
    Then $A \cap B \nsub A$ and \[
        AB/B \cong A/(A \cap B).
    \]
\end{theorem*}
\begin{proof}
    First note that $AB$ is a group by \cref{thm:products}.
    Moreover, $B \nsub AB$ since $abB = aBb = Bab$ for each $ab \in AB$.
    This will also follow independently from the later half of this proof.

    Moreover, $A \cap B \nsub A$ since for each $a \in A$,
    $a(A \cap B)a^{-1} = aAa^{-1} \cap aBa^{-1} = A \cap B$.
    Define \begin{align*}
        \varphi\colon AB &\to A/(A \cap B) \\
        ab &\mapsto a(A \cap B)
    \end{align*}
    This is well-defined since $ab = a'b' \implies a^{-1}a' \in A \cap B$,
    so $a(A \cap B) = a'(A \cap B)$.
    It is a homomorphism since \[
        a_1b_1a_2b_2 = a_1(a_2b')b_2 = a_1 a_2 b' b_2.
    \] for some $b' \in B$, so \[
        \varphi(a_1b_1 a_2b_2) = a_1a_2 (A \cap B)
        = a_1(A \cap B) \cdot a_2(A \cap B)
        = \varphi(a_1b_1) \varphi(a_2b_2).
    \]
    \begin{itemize}
        \item The kernel is $(A \cap B) B = B$.
        (Note that this proves $B \nsub AB$.)
        \item The image is all of $A/(A \cap B)$.
    \end{itemize}
    By the first isomorphism theorem, $AB/B \cong A/(A \cap B)$.
\end{proof}
