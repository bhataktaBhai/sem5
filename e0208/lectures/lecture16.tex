\lecture{2024-10-09}{}
\begin{solution}[Graham's variant]
    Sort the points by $x$-coordinate.
    Perform a Graham's scan from left to right.
    This gives the lower hull.
    Similarly compute the upper hull
    (either use $\neg \Callp{CCW}$ in place of $\Callp{CCW}$, or
    scan from right to left).
    Concatenate the two hulls to get the convex hull.

    Again, the time is $O(n \log n)$ for sorting and $O(n)$ for scanning.
\end{solution}

\begin{proof}[Proof of correctness -- sketch]
    Sorting the points as $p_1, p_2, \dots, p_n$, we start with the initial
    curve joining each $p_i$ to $p_{i+1}$.
    The invariant is that each point lies on or above this curve.
\end{proof}

\begin{solution}[Divide and conquer]
    Partition $P$ into $P_\ell$ and $P_r$ by $x$-coordinate.
    Compute the convex hull of each half recursively.
    Merge them by computing the upper and lower common tengents.
    \TODO[Exercise]
\end{solution}

\begin{solution}[Incremental]
    Sort the points by $x$-coordinate.
    Compute the convex hull for $p_1, \dots, p_i$.
    When encountering the point $p_{i+1}$, compute the upper and lower
    tangents in $O(\log n)$ time (\TODO[exercise]) and remove vertices in
    between.
\end{solution}

\section{Lower time bound} \label{sec:hull:lower-bound}
\begin{theorem}
    The general convex hull problem cannot be solved in time $o(n \log n)$.
\end{theorem}
See \emph{algebraic decision tree}.
\begin{proof}
    Given a set $X \subseteq \R$ of $n$ points, map it under the squaring
    map to place them on a convex curve.
    Compute the hull of these points.
    The points must be in (a cyclic permutation of) their sorted $x$ values.
    We can recover the sorted sequence in $O(n)$ time.
    Since sorting requires $O(n \log n)$ time, computing the convex hull
    cannot be done in less.
\end{proof}

\section{Chan's $O(n \log h)$ algorithm} \label{sec:hull:chan}

Guess $\hat h = 2^{2^k}$
Divide $P$ into $\frac nh$ groups $S_1, S_2, \dots, S_{n/\hat h}$ with
$\hat h$ points
each, completely arbitrarily.
Compute $\CH(S_i)$ for each $i$ using an $O(n \log n)$ algorithm.
This takes $\frac nh \times \hat h \log \hat h = n \log \hat h$ time.

Start with the bottom-most point.
Compute the next point using the incremental algorithm idea:
\begin{itemize}
    \item the next point from a point in $\CH(S_i)$ is either the next point
        in $\CH(S_i)$,
    \item or from some $\CH(S_j)$.
        We can compute the only candidate from each $S_j$ in $\log \hat h$
        time per $S_j$, for total time $\frac{n}{\hat h} \log \hat h$.
\end{itemize}
We determine the best of these candidates in $O(\hat h)$ time.

Doing this for $\hat h$ points gives
$\hat h \cdot \frac{n}{\hat h} \log \hat h = n \log \hat h$ time.
If we return to the starting point, we report the hull.
Otherwise, continue with the guess $\hat h = 2^{2^{k+1}}$.

The total time taken is \[
    \sum_{k=1}^{\log \log h} O(n 2^k) = O(n \log h).
\]

\begin{theorem} \label{thm:hull:chan:lower-bound}
    $O(n \log h)$ is the lower bound.
\end{theorem}
\begin{proof}
    Reduction to multiset problem (read up).
\end{proof}

\chapter{Delaunay triangulation} \label{chp:delaunay}
\begin{align*}
    n - e + f &= 1 \\
    e &= \frac{3f}{2} + \frac h2 \\
    \implies n - \frac f2 - \frac h2 &= 1 \\
    \implies f &= 2n - 2 - h \\
    \implies e &= 3n - 3 - h
\end{align*}

\begin{definition}[Delaunay triangulation] \label{def:delaunay}
    Given a finite point set $P \subseteq \R^2$, a Delaunay triangulation of
    $P$ is such a triangulation such that the circumcircle of any triangle
    contains no point from $P$ in its interior.
\end{definition}
We will assume that
\begin{itemize}
    \item no three points are collinear;
    \item no four points are cocircular.
\end{itemize}

\begin{question*}
    Given a set $P$ on $n$ points in $\R^2$, compute its Delaunay
    triangulation.
\end{question*}

\begin{solution}
    Start with any triangulation.
    Pick any triangle $abc$, and check if any point lies in the interior.
    If no, be happy.
    If yes, call it $d$.
    Split the quadrilateral $abcd$ into $2$ triangles via the \textsc{Flip}
    operation---WLOG let $ac$ be the diagonal of $abcd$.
    Flip this to be $bd$.
    $abcd$ is now partitioned into two triangles $abd$ and $cbd$.
    The circumcircles of either of these don't contain the fourth point.
\end{solution}
