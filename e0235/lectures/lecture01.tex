\chapter*{The course} \label{chp:course}
\lecture{2024-08-06}{Cryptography in ye olden days}

\href{https://www.csa.iisc.ac.in/~cris/e0_235_2020.html}{Course webpage}

\textbf{MS Teams:} % TODO on the part of the professors

\section*{Timeline} \label{sec:timeline}
\textbf{Symmetric/Private/Shared key cryptography} will be covered by
Prof.~Arpita Patra till 18th September, and then resume near the end of
November.
\textbf{Asymmetric/Public key cryptography} will be covered by
Prof.~Sanjit Chatterjee in the middle.

\section*{Evaluation} \label{sec:evaluation}
Both professors may have different evaluation techniques, each with a 50\%
weightage.

\subsubsection{Prof.~Arpita~Patra} \label{subsec:arpita}
TBD

\subsubsection{Prof.~Sanjit~Chatterjee} \label{subsec:sanjit}
\begin{itemize}
    \item Midterm:
    \item Two assignments:
    \item Endterm:
\end{itemize}

\chapter{Introduction} \label{chp:intro}

\section{History} \label{sec:history}
The 1980s mark a transition from classical to modern cryptography.
Classical cryptography dealt only with secure communication,
with ad-hoc codes and ciphers.
When a code is broken, it is fixed, or another one is created.
Creative, intellectually challenging, but not scientifically rigorous.

Modern cryptographic problems:
\begin{itemize}
    \item Authenticated message transmission:
        An adversary tries to tamper with the message.
        The receiver should be able to detect tampering.
    \item Electronic voting and auctions:
        Return the highest bid without revealing the other bids.
        In elections, make the votes anonymous and secret till the end.
    \item Activism with safety:
        \href{https://en.wikipedia.org/wiki/Deniable_encryption}
        {Deniable encryption}
    \item Secure storage, secret sharing, broadcast, zero-knowledge proofs
    \item Secure information retrieval
    \item Secure outsourcing to the cloud
    \item \textbf{Secure computation:} The holy grail of cryptography.
        An abstraction of everything else.
\end{itemize}
\begin{table}
    \centering
    \begin{tabular}{cc}
        \toprule
        \textbf{Year} & \textbf{Technique} \\
        \midrule
        110 BC & Caesar Cipher \\
        WWII   & Enigma \\
        1980s  & Boom! \\
        \bottomrule
    \end{tabular}
    \caption{Timeline of cryptography}
    \label{tab:timeline}
\end{table}

\section{Secure (multiparty) computation} \label{sec:mpc}
\begin{setting*}
    There are $n$ parties, $P_1, P_2, \dots, P_n$,
    who do not trust each other.
    Each party $P_i$ has a private input $x_i$ to a common
    $n$-input function $f$.

    The goal is to compute $f(x_1, x_2, \dots, x_n)$
    while revealing nothing about the inputs except the output.
\end{setting*}
Applications:
\begin{itemize}
    \item Satellite collision avoidance
    \item Secure set intersection
    \item Privacy-preserving everything
\end{itemize}

Modern cryptography follows three principles:
\begin{itemize}
    \item A \emph{formal definition} of security capturing requirement.
    \item \emph{Precise} and \emph{well-studied} assumptions.
    \item Mathematical proof of security.
\end{itemize}

\begin{examples}
    \item Prime factorization
    \item Subset sum: Given a set of integers $S$ and a target $t$,
        is there a subset of $S$ that sums to $t$?
\end{examples}

Our primary objective will be secure communication:
turning insecure channels into secure ones.
We expect (i) privacy, (ii) integrity and (iii) authenticity.

Integrity is the ability to detect tampering.
Authenticity is the ability to detect impersonation.

\section{Classical examples} \label{sec:classical-examples}
In the private key setting, a message $m$ is encrypted with a key $k$
to get a ciphertext $c$.
The ciphertext is decrypted with the same key $k$ to get the message $m$.
We will use the following notation:
\begin{itemize}
    \item The key-generation algorithm $\Gen$ is a randomized algorithm
        that outputs a key $k$ according to some probability distribution.
    \item The encryption algorithm $\Enc_k$ is parameterized by the key $k$,
        and takes a message $m$ to output a ciphertext $c$.
        This may be deterministic or randomized.
    \item The decryption algorithm $\Dec_k$ is parameterized by the key $k$,
        and takes a ciphertext $c$ to output a message $m$.
    \item The key space $\mcK$ is the set of all possible keys.
    \item The message space $\mcM$ is the set of all possible messages.
    \item The cipher-text space $\mcC$ is the set of all possible
        ciphertexts.
\end{itemize}

\begin{example}[Caesar cipher]
    The Caesar cipher has $\mcM = \mcC = \set{0, 1, \dots, 25}$.
    $\Enc(m) = m + 3 \bmod 26$ and $\Dec(c) = c - 3 \bmod 26$.
    It is a keyless cipher.
\end{example}
A keyed cipher needs to be private.

\begin{quotation}
    The security of a cryptographic system should not depend on the
    secrecy of the algorithm, but only on the secrecy of the key.
    -- Kerckhoffs, 1883
\end{quotation}

Here are some arguments for Kerckhoffs' principle:
\begin{itemize}
    \item Maintaining the secrecy of the algorithm is far more difficult
        than maintaining the secrecy of the key.
    \item Replacing the algorithm is infinitely harder than replacing
        the key.
    \item For cryptography to be an everyday tool, secret algorithms
        would need to be devised for every pair of communicating parties.
\end{itemize}

\begin{example}[Shift cipher]
    The obvious generalization of the Caesar cipher using a key $k$.
\end{example}
$26$ is an embarrassing size for a key space.

\begin{example}[Mono-alphabetic substitution]
    The key is a permutation of the alphabet.
    $\Enc_k = k$ and $\Dec_k = k^{-1}$.
\end{example}
The key space has size $26!$.
Brute force is infesible, but \emph{frequency analysis} destroys it.

\begin{example}[Vigenère cipher]
    The key is a random $t$-tuple of shifts.
    The message is divided into blocks of length $t$,
    and the key is applied to each block using the shift cipher.
\end{example}
The key space has size $26^t$, but if $t$ is known,
the key can be broken by frequency analysis.
Consider all the characters at positions $i \pmod t$.
These are all encrypted with the same key, so frequency analysis works.
Repeat to get the entire word.

Even if we take the key to be a tuple of $t$ permutations,
the same method works.
