\chapter*{The course} \label{chp:course}

\section*{Grading} \label{sec:grading}
This is tentative.
\begin{itemize}
    \item Quizzes: 30\%
    \item Midterm: 30\%
    \item Final: 40\%
\end{itemize}

\lecture{2024-08-02}{}

\chapter{Groups} \label{chp:group}

\begin{definition}[Binary operation] \label{def:binop}
    A \emph{binary operation} $\cdot$ on a set $A$ is any map
    from $A \times A \to A$, written $(a, b) \mapsto a \cdot b$.

    We say that $\cdot$ is \emph{associative} if for all $a, b, c \in A$, \[
        (a \cdot b) \cdot c = a \cdot (b \cdot c)
    \]
    and \emph{commutative} if for all $a, b \in A$, \[
        a \cdot b = b \cdot a.
    \]
\end{definition}
\begin{examples}
    \item Addition and multiplication are associative and commutative
        binary operations on $\R$.
    \item Subtraction, division and exponentiation are
        non-associative and non-commutative binary operations.
    \item Composition is an associative but non-commutative
        binary operation on $X^X$.
\end{examples}

\begin{definition}[Group] \label{def:group}
    A \emph{group} is a set $G$ equipped with a binary operation $\cdot$
    satisfying the following properties:
    \begin{enumerate}[label=\small(G\arabic*)]
        \item \textbf{associativity:} $\cdot$ is associative;
        \item \textbf{identity:} there exists an element $1_G = e \in G$
            such that $1_G \cdot x = x \cdot 1_G = x$ for all $x \in G$;
        \item \textbf{inverse:} for every $x \in G$, there exists an element
            $y \in G$ such that $x \cdot y = y \cdot x = 1_G$.
            We write $y$ as $x^{-1}$.
    \end{enumerate}

    If $\cdot$ is also commutative, we say that $G$ is an
    \emph{abelian group}.

    A subset $H \subseteq G$ is a \emph{subgroup} of $G$ if $H$ is a group
    with respect to the same binary operation $\cdot$.
    We write $H \le G$.
\end{definition}

\begin{examples}
    \item $(\Z, +)$, $(\Q, +)$, $(\R, +)$ and $(\C, +)$ are abelian groups.
    \item $(\R^\times, \cdot)$ is a group but $(\R, \cdot)$ is not.
    \item $(\GL_n(\R), \cdot)$ is a non-abelian group, where \[
        \GL_n(\R) = \set{A \in M_n(\R) \mid \det A \neq 0}.
    \]
    \item For any $n \in \N^+$, $(S_n, \circ)$ is a group, where \[
        S_n = \set{\sigma\colon [n] \to [n]
                \mid \sigma \text{ is bijective}}.
    \] \begin{align*}
        S_1 &= \set{1}, \\
        S_2 &= \set{1, (12)}, \\
        S_3 &= \set{1, (12), (13), (23), (123), (132)}.
    \end{align*}
    $S_1$ and $S_2$ are abelian, but $S_3$ is not.
    Let $x = (12)$ and $y = (13)$, then \[
        (x \circ y)(1) = x(3) = 3, \quad
        (x \circ y)(2) = x(2) = 1, \quad
        (x \circ y)(3) = x(1) = 2,
    \] but \[
        (y \circ x)(1) = y(2) = 2, \quad
        (y \circ x)(2) = y(1) = 3, \quad
        (y \circ x)(3) = y(3) = 1.
    \]
    \item Let $H = \set*{\begin{pmatrix}
        1 & x \\
        0 & 1
    \end{pmatrix} \mid x \in \R}$.
    Then $H$ is an abelian subgroup of the non-abelian $\GL_2(\R)$.
\end{examples}

\begin{remarks}[New groups from old]
    \item Let $(A, \cdot)$ and $(B, \ast)$ be groups.
        The cartesian product $A \times B$ is a group with respect to the
        operation \[
            (a_1, b_1) \star (a_2, b_2) = (a_1 \cdot a_2, b_1 \ast b_2).
        \] defined componentwise.
    \item Let $X$ be a set and $S = \R^X$.
        Then $S$ is an abelian group under addition (pointwise).
        In fact, if $(G, \cdot)$ is a group, then $G^X$ is a group
        under the operation \[
            (f \cdot g)(x) = f(x) \cdot g(x).
        \] If $G$ is abelian, then so is $G^X$.
    \item Given any set $A$, we can form the group $S(A)$ of all bijections
        from $A$ to itself, under composition.
\end{remarks}

\begin{proposition}
    Let $(G, \cdot)$ be a group. Then
    \begin{enumerate}
        \item the identity element $1_G$ is unique;
        \item the inverse of each element $x \in G$ is unique;
        \item $(x^{-1})^{-1} = x$ for all $x \in G$;
        \item $(x \cdot y)^{-1} = y^{-1} \cdot x^{-1}$ for all $x, y \in G$;
        \item The product $a_1 a_2 \dots a_n$ does not depend on bracketing.
    \end{enumerate}
\end{proposition}
\begin{proof} \leavevmode
    \begin{enumerate}
        \item Suppose $e$ and $f$ are both identities of $G$.
            Then \[
                e = e \cdot f = f.
            \]
        \item Suppose $y$ and $y'$ are both inverses of $x$.
            Then \[
                xy = 1_G \implies y'xy = y' \implies y' = y.
            \]
        \item We have \[
            x \cdot x^{-1} = 1_G = x^{-1} \cdot x.
        \] reinterpreted in the context of $x^{-1}$.
        \item Checking \[
            (xy)(y^{-1}x^{-1}) = x(y y^{-1})x^{-1} = xx^{-1} = 1_G.
        \] Alternatively, let $z = (xy)^{-1}$.
        Then \begin{align*}
            (xy)z &= 1_G \\
            (x^{-1}x)yz &= x^{-1} \\
            yz &= x^{-1} \\
            (y^{-1}y)z &= y^{-1}x^{-1} \\
            z &= y^{-1}x^{-1}.
        \end{align*}
        \item Induct on $n$.
        Look at the rightmost left bracket \[
            a_1 \dots a_n = (a_1 \dots a_k) \cdot (a_{k+1} \dots a_n).
                \qedhere
        \]
    \end{enumerate}
\end{proof}

\begin{corollary}[Cancellation law] \label{thm:group:cancel}
    Let $(G, \cdot)$ be a group.
    If $x, y, z \in G$ and $xy = xz$, then $y = z$.
\end{corollary}
\begin{proof}
    Multiply by $x^{-1}$ on the left.
\end{proof}
