\lecture{2024-08-30}{}

\begin{theorem*}[third isomorphism theorem] \label{thm:iso:3}
    Let $H, K \nsub G$ with $K \subseteq H$.
    Then $H/K \nsub G/K$ and \[
        \frac{G/K}{H/K} \cong G/H.
    \]
\end{theorem*}
\begin{proof}
    Note that $K \nsub H$ implies $H/K$ is a group and
    $H/K \subseteq G/K$, so that $H/K \le G/K$.
    Since $H \nsub G$, we have $H/K \nsub G/K$. \[
        (gK)(hK)(gK)^{-1} = (ghg^{-1})K = h'K \in H/K.
    \]
    Define \begin{align*}
        \varphi\colon \frac{G/K}{H/K} &\to G/H \\
        (gK)(H/K) &\mapsto gH.
    \end{align*}
    This is well-defined, since \begin{align*}
        (gK)(H/K) = (g'K)(H/K) &\iff (g'K)(gK)^{-1} \in H/K \\
            &\iff (g'g^{-1})K \in H/K \\
            &\iff g'g^{-1} \in H \\
            &\iff gH = g'H.
    \end{align*}
    This also proves injectivity.
    For surjectivity, simply notice that for any $gH \in G/H$,
    $(gK)(H/K) \in (G/K)/(H/K)$ maps to $gH$.
\end{proof}
\begin{remark}
    $\nsub$ is not transitive.
    For example, $\angled s \nsub \angled{r^2, s} \nsub D_8$,
    but $\angled s \nnsub D_8$.
\end{remark}

\begin{theorem}[fourth isomorphism theorem] \label{thm:iso:4}
    Let $N \nsub G$.
    Then there is a one-to-one correspondence between the subgroups of
    $G/N$ and the subgroups of $G$ containing $N$.
\end{theorem}

\section{Further tools} \label{sec:tools}
\begin{definition}[conjugate] \label{def:conjugate}
    Let $S \subseteq G$.
    The \emph{conjugate} of $S$ by $g \in G$ is \[
        gSg^{-1} = \sset{gsg^{-1}}{s \in S}.
    \]
\end{definition}
\begin{proposition*}[class equation] \label{thm:tools:class}
    The number of conjugates of a set $S \subseteq G$ is $(G : N_G(S))$.

    In particular, the number of conjugacy classes of
    $\set x$ is $(G : C_G(x))$.
\end{proposition*}
\begin{proof}
    
\end{proof}
