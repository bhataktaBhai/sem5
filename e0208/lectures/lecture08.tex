\lecture{2024-09-02}{Kd-trees}

To recap, we have solved 2D orthogonal range searching using
\begin{itemize}
    \item \textbf{Range trees:} $O(n \log n)$ space and $O(\log^2 n + k)$
    time, which we later improved to $O(\log n + k)$ using fractional
    cascading.
    \item \textbf{The crazy optimal:}
    $O\ab(n \cdot \frac{\log n}{\log \log n})$
    space and $O(\log n + k)$ time.
\end{itemize}

What if we are limited to $O(n)$ space?
\begin{question*}
    Given a set of points in $\R^2$ and $O(n)$ space,
    process queries of the form $[x_1, x_2] \times [y_1, y_2]$.
\end{question*}
\section{Kd-tree} \label{sec:kd}
Given a set of points $P \subseteq \R^2$, split them evenly by a vertical
cut.
Next, split the two halves by a horizontal cut.
Repeat this process recursively.
Create a tree with the cuts as nodes and the points as leaves.

The construction time is $O(n \log n)$. \[
    T(n) = 2T\ab(\frac{n}{2}) + O(n)
\] The space used is $O(n)$. \[
    S(n) = 2S\ab(\frac{n}{2}) + O(1)
\]
We can do this for any number of dimensions.
\begin{algo}
    \Fn{Kd-tree}{$d, P, a$}
        \If{$\abs{P} = 1$}
            \State \Return \Call{leaf}{$P$}
        \EndIf
        \State $m \gets \Call{median}{\sset{p.a}{p \in P}}$
        \State $P_\ell \gets \sset{p \in P}{p.a \le m}$
        \State $P_r \gets \sset{p \in P}{p.a > m}$
        \State $T_\ell \gets \Call{Kd-tree}{d, P_\ell, a + 1 \bmod d}$
        \State $T_r \gets \Call{Kd-tree}{d, P_r, a + 1 \bmod d}$
        \State \Return $\Call{node}{a, m, T_\ell, T_r}$
    \EndFn
\end{algo}
\begin{solution} \label{sol:kd}
    Store the points in a Kd-tree.
    To each internal node $v$, we have an associated rectangle $R(v)$
    containing all points in the subtree rooted at that node.

    For a query rectangle $Q = [x_1, x_2] \times [y_1, y_2]$,
    we traverse the tree from the root, while keeping track of the
    rectangles $R(v)$.
    \begin{itemize}
        \item At a leaf node, determine if the point is in the rectangle.
        \item At an internal node $v$
        \begin{enumerate}
            \item \label{sol:kd:ignore}
                If $Q \cap R(v) = \O$, ignore the subtree.
            \item \label{sol:kd:report}
                If $Q \supseteq R(v)$, report all points in the subtree.
            \item \label{sol:kd:recurse}
                If only part of $R(v)$ intersects $Q$,
                recurse on both children.
                Equivalently, the boundary of $Q$ intersects $R(v)$.
        \end{enumerate}
    \end{itemize}
    \begin{algo}
        \Fn{Kd-query}{$v, Q, R=[x_1, x_2] \times [y_1, y_2]$}
            \If{$v$ is a leaf}
                \State report $v$ if $v \in Q$ and \Return
            \EndIf
            \If{$Q \cap R = \O$}
                \State \Return
            \EndIf
            \If{$v$ splits by $x$-coordinate}
                \State $R_\ell \gets [x_1, v_x] \times [y_1, y_2]$
                \State $R_r \gets [v_x, x_2] \times [y_1, y_2]$
            \Else
                \State $R_\ell \gets [x_1, x_2] \times [y_1, v_y]$
                \State $R_r \gets [x_1, x_2] \times [v_y, y_2]$
            \EndIf
            \State \Call{Kd-query}{$v_\ell, Q, R_\ell$}
            \State \Call{Kd-query}{$v_r, Q, R_r$}
        \EndFn
    \end{algo}
\end{solution}
\begin{remark}
    Cases \ref{sol:kd:report} and \ref{sol:kd:recurse} are identical from
    the algorithm's perspective.
    Both involve recursing on both children.
    They are distinguished here for the purpose of analysis.
\end{remark}
\begin{proof}[Analysis]
    We take $O(k)$ time across all nodes of type \ref*{sol:kd:report}.

    The parent of every node visited (including leaves) is of type
    \ref*{sol:kd:recurse}, except the subtrees in case \ref*{sol:kd:report}
    already accounted for above.
    Thus the total number of nodes visited can be bounded by $2$ times
    the number of type \ref*{sol:kd:recurse} nodes.
    We will bound this number.

    How many nodes can intersect a line $x = L$?
    Fix a node $v$.
    If $v$ is split by $x$-coordinate, at most one child intersects $x = L$.
    If $v$ is split by $y$-coordinate, both children may intersect.
    Thus the number of intersecting nodes at most doubles every $2$ levels.
    This gives a total of $O(2^{\frac{\log n}{2}}) = O(\sqrt n)$ nodes.
    More formally, \[
        I(n) \le 3 + 2I\ab(\frac n4),
    \] which has solution $I(n) = O(\sqrt n)$.
    The $3$ is for the root and possibly both children,
    but only two of the grandchildren can intersect.

    Thus there are at most $O(\sqrt n)$ nodes of type \ref*{sol:kd:recurse}.
    This gives total time $O(\sqrt n + k)$.
\end{proof}
In the 3D case, the number of nodes doubles twice every $3$ levels,
giving $O(n^{2/3})$ nodes intersecting an axis-perpendicular plane,
and a time complexity of $O(n^{2/3} + k)$.
In $d$ dimensions, we have $O(n^{1 - 1/d} + k)$ time. \[
    I(n) \le (2^d - 1) + 2^{d-1} I(n/2^d)
        \implies I(n) = O(n^{\frac{d-1}{d}})
\]

This is because whenever a cut is parallel to the place,
there is no increase in the number of intersecting nodes.
