\lecture{2024-08-08}{}
\subsubsection{Learnings from classical SKE} \label{sec:classical:learning}
\begin{itemize}
    \item Algorithms of secure key encryption (SKE), and in all of
        cryptography, must be public.
    \item The key space must be huge.
    \item Definitions and proofs.
\end{itemize}

Today, we will formulate a formal definition (the threat and break model)
and 

\chapter{Secret Key Encryption} \label{chp:ske}
\begin{definition*} \label{def:ske}
    A \emph{secret key encryption scheme} is a tuple of algorithms
    $(\Gen, \Enc, \Dec)$ together with three sets $\mcK, \mcM, \mcC$
    called the \emph{key space}, \emph{message space} and
    \emph{ciphertext space} where:
    \begin{itemize}
        \item $\Gen\colon 1 \to \mcK$ is a probabilistic algorithm;
        \item $\Enc\colon \mcK \times \mcM \to \mcC$ and
            $\Dec\colon \mcK \times \mcC \to \mcM$
            are deterministic algorithms
    \end{itemize}
    such that \[
        \forall m \in \mcM(\Dec_k (\Enc_k(m))) = m.
    \]
    We write $\Enc_k$ and $\Dec_k$ to denote $\Enc$ and $\Dec$
    partially applied to the key $k$.
\end{definition*}
\begin{remark}
    Notice that this implies $\Enc_k$ is injective for each $k \in \mcK$.
\end{remark}

\section{A formal definition of security} \label{sec:defining-security}

We consider the computationally omnipotent adversary $\mcR$ (for Ravana).
Suppose they have a capability to listen to all ciphertexts.
This is called \emph{ciphertext only attack} (COA).

We will also assume that $\mcR$ can use randomness.
Encryption schemes often use randomness, so why not the adversary?

Thus, the adversary has the following characteristics:
\begin{itemize}
    \item Randomized
    \item All-powerful
    \item Ciphertext only
\end{itemize}

Let us attempt to define ``security''.
Here are some possibilities:
\begin{itemize}
    \item A scheme is secure if it does not leak the secret key.
        Nope, $m \mapsto m$ would be considered secure.
    \item A scheme is secure if it does not leak the \emph{entire} message.
        Nope, revealing even a single bit is bad.
    \item A scheme is secure if it does not leak \emph{any} bit.
        Nope. It could be that the scheme reveals the parity of the message
        modulo $3$.
    \item A scheme is secure if it does not leak \emph{any} digit in
        \emph{any} base.
        Hmm, interesting.
    \item A scheme is secure if it does not leak \emph{any} reasonable
        information about the message.
        How do we formalize this?
\end{itemize}

\subsection{Probability review} \label{sec:prob-review}
\begin{theorem}[Law of total probability] \label{thm:prob-review:lotp}
    Let $E_1, E_2, \dots, E_n$ be a partition of the sample space.
    Then for any event $A$, \[
        \Pr(A) = \sum_{i=1}^n \Pr(A \mid E_i) \Pr(E_i).
    \]
\end{theorem}

\begin{theorem}[Bayes' theorem] \label{thm:prob-review:bayes}
    Let $A$ and $B$ be events with $\Pr(B) > 0$.
    Then \[
        \Pr(A \mid B) = \frac{\Pr(B \mid A) \Pr(A)}{\Pr(B)}.
    \]
\end{theorem}

\section{Formalizing SKE} \label{sec:formalizing-ske}
% \begin{table}
%     \centering
%     \begin{tabular}{|c|L|c|L|}
%         % \toprule
%         \hline
%         & $\mcK$ & $\mcM$ & $\mcC$ \\
%         \hline
%         \textbf{Random variable} & $K$ & $M$ & $C$ \\
%         \textbf{Distribution} & Depends on $\Gen$ & Depends on the use-case
%         & Depends on $\Enc$ and the distribution of $M$ and $K$. \\
%         \hline
%     \end{tabular}
% \end{table}

\begin{exercise}
    Consider the encryption scheme in \cref{tab:enc1}
    with distributions \[
        K \sim \begin{pmatrix}
            0 & 1 & 2 \\
            1/2 & 1/4 & 1/4
        \end{pmatrix} \qquad
        M \sim \begin{pmatrix}
            a & b & c & d \\
             &  &  & 
        \end{pmatrix}
    \]
\end{exercise}
\begin{table}
    \centering
    \begin{tabular}{ccccc}
        \toprule
        & $a$ & $b$ & $c$ & $d$ \\
        \midrule
        $0$ & \\
        $1$ & \\
        $2$ & \\
        \bottomrule
    \end{tabular}
    \caption{Example encryption scheme}
    \label{tab:enc1}
\end{table}

\begin{definition*}[Perfect security] \label{def:ske:perfect}
    \;\\
    An encryption scheme $(\Gen, \Enc, \Dec)$ is \emph{perfectly secure}
    if for every random variable $M$ over $\mcM$ and
    every $m \in \mcM$, $c \in \mcC$, \[
        \Pr[M = m \mid C = c] = \Pr[M = m].
    \]
\end{definition*}
\begin{remark}
    This is probabily the first formal definition of security,
    by Claude~E.~Shannon in
    Bell Systems Technical Journal, 28(4): 656--715, 1949.
\end{remark}

\begin{theorem*}[Vernam cipher] \label{thm:vernem}
    Let $\mcK = \mcM = \mcC = \set{0, 1}^n$.
    Let $\Gen$ draw uniformly at random from $\mcK$, and
    let $\Enc = \Dec = \oplus$ (bitwise XOR).

    Then $(\Gen, \Enc, \Dec)$ is perfectly secure.
\end{theorem*}
\begin{proof}
    It is easy to see that $\Dec_k \circ \Enc_k = \id$.

    Now for any $c \in \mcC, m \in \mcM$, we have \[
        \Pr[C = c \mid M = m] = \Pr[K = c \oplus m] = 2^{-n}.
    \] Thus \[
        \Pr[C = c] = \sum_{m \in \mcM} \Pr[C = c \mid M = m] \Pr[M = m]
                   = 2^{-n}.
    \]
    This gives $\Pr[C = c \mid M = m] = \Pr[C = c]$
    and so $C$ and $M$ are independent.
\end{proof}

\noindent\textbf{Problems with the Vernam cipher:}
\begin{itemize}
    \item Can we reuse the key for multiple messages?
        Nope. We can XOR two consecutive messages to get the XOR of the
        two messages.
    \item The key is as long as the message.
    \item Coin tossing does not scale well.
\end{itemize}
\begin{fact}
    Key length and key reusability is an issue in
    \emph{any} perfectly secure encryption scheme.
\end{fact}

\begin{exercise} \label{thm:ske:perfect}
    Prove that \cref{def:ske:perfect} is equivalent to the following:
    \[
        \Pr[C = c \mid M = m] = \Pr[C = c \mid M = m'] \quad
        \forall\; m, m' \in \mcM, c \in \mcC.
    \]
\end{exercise}
\begin{solution}
    Suppose \cref{def:ske:perfect} holds.
    Then
    \[
        \Pr[C = c \mid M = m]
        = \frac{\Pr[M = m \mid C = c] \Pr[C = c]}{\Pr[M = m]}
        = \Pr[C = c]
    \] Thus the given condition holds.
\end{solution}
