\lecture{2024-10-09}{}
\begin{definition}[multiindex] \label{def:multiindex}
    Let $n \ge 1$ be fixed.
    $\alpha \in \N^n$ is called a \emph{multiindex}.
    For $\alpha = (\alpha_1, \alpha_2, \dots, \alpha_n)$, we define
    \begin{align*}
        \alpha! &\coloneqq \alpha_1! \alpha_2! \dots \alpha_n! \\
        \abs\alpha &\coloneqq \alpha_1 + \alpha_2 + \dots + \alpha_n \\
        h^\alpha &\coloneqq h_1^{\alpha_1} h_2^{\alpha_2} \dots h_n^{\alpha_n} \\
        D^\alpha &\coloneqq D_1^{\alpha_1} D_2^{\alpha_2} \dots D_n^{\alpha_n}
    \end{align*}
\end{definition}
\begin{example}
    Let $n = 5$ and $\alpha = (1, 2, 0, 1, 2)$.
    Then \begin{align*}
        \alpha! &= 1! 2! 0! 1! 2! = 2 \\
        \abs\alpha &= 1 + 2 + 0 + 1 + 2 = 6 \\
        h^\alpha &= h_1 h_2^2 h_4 h_5^2 \\
        D^\alpha f &= D_1 D_2^2 D_4 D_5^2 f \\
        &= \pdv[order={1,2,1,2}]{f}{x_1, x_2, x_4, x_5}
    \end{align*}
\end{example}

If $f$ is smooth, \[
    D_{i_1 \dots i_k} f = D_{i_{\sigma(1)}} \dots D_{i_{\sigma(k)}} f
\] for any $\sigma \in S_n$.
Thus we can rewrite Taylor's theorem as \begin{equation*}
    \boxed{
        f(a + h) = \sum_{k=0}^{m} \frac1{k!} \sum_{\abs\alpha = k}
            \frac{k!}{\alpha!} D^\alpha f(a) h^\alpha + r(h)
        = \sum_{\abs\alpha \le m}
            \frac{1}{\alpha!} D^\alpha f(a) h^\alpha + r(h)
    }
\end{equation*}
since the number of sequences $(i_1, \dots, i_k)$ with
$\alpha_1$ occurrences of $1$, $\alpha_2$ occurrences of $2$, etc.
is $\binom{k}{\alpha_1, \alpha_2, \dots, \alpha_n} = \frac{k!}{\alpha!}$.
We can write $r(h)$ as \[
    r(h) = \sum_{\abs\alpha = m} \frac{1}{\alpha!}
        (D^\alpha f(c) - D^\alpha f(a)) h^\alpha.
\]

\begin{definition}[definiteness] \label{def:definite}
    Let $A \in M_n(\R)$ be a symmetric matrix.
    We say that $A$ or $q_A$ is
    \begin{enumerate}
        \item \emph{positive} (resp. \emph{negative}) \emph{definite}
            if $q_A(x) > 0$ (resp. $q_A(x) < 0$) for all
            $x \in \R^n \setminus \set 0$;
        \item \emph{positive} (resp. \emph{negative}) \emph{semidefinite}
            if it is \emph{not} definite and $q_A(x) \ge 0$
            (resp. $q_A(x) \le 0$) for all $x \in \R^n$;
        \item \emph{indefinite} if it is neither definite not semidefinite.
    \end{enumerate}
\end{definition}
\begin{remark}
    $A$ is positive (semi)definite iff $-A$ is negative (semi)definite.
\end{remark}

\begin{proposition}
    If $A$ is a positive definite matrix, then there exist $c_1, c_2 > 0$
    such that \[
        c_1 \norm{x}^2 \le q_A(x) \le c_2 \norm{x}^2
    \] for all $x \in \R^n$.
\end{proposition}
\begin{proof}
    $q_A$ is a continuous positive function on the compact set $S^{n-1}$.
    Let $c_1$ and $c_2$ be the minimum and maximum values of $q_A$
    on $S^{n-1}$.
    For any $x \in \R^n$, $q_A(x) = \norm x^2 q_A(\what{x})$, so \[
        c_1 \norm x^2 \le q_A(x) \le c_2 \norm x^2.
    \]

    Alternatively, note that $q_A$ is a norm on $\R^n$, and any two norms
    are equivalent.
\end{proof}
