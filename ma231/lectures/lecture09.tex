\lecture{2024-08-23}{}
\textbf{Next quiz:} Monday, 2024-08-26

\begin{remark}[A subtlety]
    There could be a strictly finer topology on a set $X$ which is
    nevertheless homeomorphic.

    Let $\T$ and $\T'$ be topologies on $X$.
    Then $\T$ is finer that $\T'$ iff $\T \supseteq \T'$ iff
    $\id\colon (X, \T) \to (X, \T')$ is continuous.

    Thus in a case as above, the \emph{identity map} is not a homeomorphism,
    but there is some other homeomorphism.
\end{remark}
\begin{example}
    Consider $\R$ with the topologies
    $\T = \set{(-n, n) \mid n \in \N} \cup \set \R$ and
    $\T' = \set{(-2n, 2n) \mid n \in \N} \cup \set \R$.

    But $x \mapsto 2x$ is a homeomorphism from
    $(\R, \T)$ to $(\R, \T')$.
\end{example}

\begin{theorem}
    Let $X = \bigcup_\alpha U_\alpha$
    where each $U_\alpha$ is open.
    A map $f\colon X \to Y$ is continuous iff $f\vert_{U_\alpha}$
    is continuous for each $\alpha$,
    where we consider the subspace topology on each $U_\alpha$.
\end{theorem}
\begin{proof}
    We have already seen that if $f$ is continuous, then
    all restrictions are continuous.

    Suppose the restrictions are continuous.
    Let $V \subseteq Y$ be open.
    Then \[
        f^{-1}(V) = \bigcup_\alpha (f^{-1}(V) \cap U_\alpha)
            = \bigcup_\alpha f\vert_{U_\alpha}^{-1}(V)
    \] is open, since $f\vert_{U_\alpha}^{-1}(V)$ is open in $U_\alpha$
    and hence in $X$.
\end{proof}

We say that continuity is a ``local'' property.
If $f$ is continuous in neighborhoods of each point,
then $f$ is continuous by the above theorem.

\begin{theorem}[Pasting lemma] \label{thm:pasting-lemma}
    Suppose $X = A \cup B$ where $A$ and $B$ are closed,
    and $f\colon A \to Y$ and $g\colon B \to Y$ are continuous
    with $f = g$ on $A \cap B$.
    Then $h\colon X \to Y$ defined by \[
        h(x) = (f \cup g)(x) = \begin{cases}
            f(x) & \text{if } x \in A, \\
            g(x) & \text{if } x \in B
        \end{cases}
    \] is continuous.
\end{theorem}
\begin{proof}
    Let $C \in Y$ be closed
    Then \[
        (f \cup g)^{-1}(C) = f^{-1}(C) \cup g^{-1}(C)
    \] is closed, since $f^{-1}(C)$ and $g^{-1}(C)$ are closed
    in $A$ and $B$ respectively, and hence in $X$.
\end{proof}

\section{Connectedness} \label{sec:connect}
\begin{definition}[Connectedness] \label{def:connect}
    $X$ is \emph{connected} if $\O$ and $X$ are the only clopen sets in $X$.
\end{definition}
This is an awesome definition.
The motivation is the following:
\begin{definition}[Separation] \label{def:connect:separation}
    A \emph{separation} of $X$ is a pair of disjoint nonempty open sets
    $U$ and $V$ such that $X = U \cup V$.
\end{definition}

\begin{proposition}
    $X$ is connected iff there is no separation of $X$.
\end{proposition}
\begin{proof}
    Suppose there is a separation $U$ and $V$.
    Then $U$ and $V$ are clopen, but neither is $\O$ or $X$.
    Thus $X$ is not connected.

    Conversely, suppose there is no separation of $X$.
    Let $U \ne \O, X$ be open.
    Then $V = X \setminus U$ cannot be open, since otherwise
    $U$ and $V$ would separate $X$.
    Thus $U$ is not closed.
\end{proof}

\begin{proposition}
    $[0, 1]$ is connected.
\end{proposition}
\begin{proof}
    Assume $U$ and $V$ separate $[0, 1]$.
    WLOG suppose $0 \in U$.

    Let $a = \inf V$.
    Since $V$ is closed, $a \in V$.
    But $[0, a) \subseteq U$, so $a \in U$ since $U$ is closed.
\end{proof}
