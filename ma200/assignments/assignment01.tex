\documentclass[12pt]{article}
% \usepackage{parskip}
\usepackage{lmodern} % https://tex.stackexchange.com/a/58088/295755
\renewcommand\bfdefault{b}
\usepackage{microtype}

\usepackage{amsmath}
\newcommand\yesnumber{\stepcounter{equation}\tag{\theequation}}

\ifundef{\chapter}{}{\providecommand\thechapter{\Roman{chapter}}}

\usepackage{marginnote}
\usepackage[en-GB,calc]{datetime2}
\usepackage{calc}
\usepackage{xifthen}
\usepackage{tocloft}

% https://tex.stackexchange.com/a/454168
\newcommand\monthday[1]{\DTMmonthname{\DTMfetchmonth{#1}}~\number\DTMfetchday{#1}}

\newlistof{lecture}{lec}{Lectures} % what is this file extension business?
\makeatletter
\setlength\marginparwidth{1in}
\newcommand*{\lecture}[3][]{
    \ifthenelse{\isempty{#1}}{%
        \refstepcounter{lecture}%
    }{%
        \setcounter{lecture}{#1}%
    }%
    \DTMsavedate{lecdate}{#2}%
    \def\lecdow{\DTMweekdayname{\DTMfetchdow{lecdate}}}
    \def\lecshortdow{\DTMshortweekdayname{\DTMfetchdow{lecdate}}}
    \def\lecmonth{\DTMmonthname{\DTMfetchmonth{lecdate}}}
    \def\lecday{\number\DTMfetchday{lecdate}}
    % \marginpar{\raggedright\small \textsf{\textbf{Lecture \thelecture.}%
    %             \footnotesize\DTMusedate{lecdate}}}%
    \marginnote{\raggedright\small%
        \textsf{{\textbf{Lecture \thelecture.}} \\
        \footnotesize\lecdow\\\lecmonth\ \lecday}}%
    \ifthenelse{\isempty{#3}}{%
        \addcontentsline{lec}{lecture}{\protect\numberline{\thelecture}%
        \lecshortdow, \lecmonth\ \lecday}%
        \def\@lecture{Lecture \thelecture}%
    }{%
        \addcontentsline{lec}{lecture}{\protect\numberline{\thelecture}%
        \makebox[\widthof{Mon,}][l]{\lecshortdow,}\ \makebox[\widthof{September 00}][l]{\lecmonth\ \lecday} #3}%
        \def\@lecture{Lecture \thelecture: #3}%
    }%
    \par%
}
\g@addto@macro\normalsize{%
  \setlength\abovedisplayskip{7pt}%
  \setlength\belowdisplayskip{7pt}%
  \setlength\abovedisplayshortskip{1pt}%
  \setlength\belowdisplayshortskip{1pt}%
}
\makeatother

\usepackage[twoside]{fancyhdr}
\setlength{\headheight}{15pt}
\pagestyle{fancy}
\fancyhf{}
% \fancyhead[r]{\thepage}
\makeatletter
\fancyhead[LE,RO]{\thepage}
\fancyhead[RE]{\textbf{\nouppercase\leftmark}}
\fancyhead[LO]{\nouppercase\rightmark}
\providecommand\@lecture{}
\fancyfoot[R]{\small\@lecture}
\makeatother

% homeworks
\newlistof{hw}{hw}{Assignments} % counter `assignment' already defined
\makeatletter
\newcommand*{\assignment}[5][]{%[number]{file}{date posted}{date due}{date quiz}
    \ifthenelse{\isempty{#1}}{%
        \refstepcounter{hw}%
        \stepcounter{assignment}%
    }{%
        \setcounter{hw}{#1}%
        \setcounter{assignment}{#1}%
    }%
    \pagebreak
    \ifthenelse{\isempty{#3}}{}{\DTMsavedate{posted}{#3}}%
    \ifthenelse{\isempty{#4}}{}{\DTMsavedate{due}{#4}}%
    \ifthenelse{\isempty{#5}}{}{\DTMsavedate{quiz}{#5}}%
    \section*{Assignment \thehw}
    \ifthenelse{\isempty{#4}}{
        \ifthenelse{\isempty{#5}}{
            \ifthenelse{\isempty{#3}}{
                \def\hw@toc{}
            }{
                \def\hw@toc{posted \monthday{up}}
            }
        }{
            \def\hw@toc{quiz \monthday{quiz}}
        }
    }{
        \def\hw@toc{due \monthday{due}}
    }
    \addcontentsline{hw}{hw}{\protect\numberline{\thehw}\hw@toc}\par
    \marginpar{\raggedright\footnotesize\textsf{%
        \ifthenelse{\isempty{#3}}{}{\makebox[\widthof{quiz}][l]{up} \monthday{posted} \\}%
        \ifthenelse{\isempty{#4}}{}{\makebox[\widthof{quiz}][l]{due} \monthday{due} \\}%
        \ifthenelse{\isempty{#5}}{}{quiz \monthday{quiz}}%
    }}
    \def\@lecture{Assignment \thehw\ifx\hw@toc\empty{}\else\ --- \hw@toc\fi}
    \input{#2}
    \newpage
}
\makeatother

\usepackage{amsmath}
\usepackage{amsthm}
\usepackage[dvipsnames]{xcolor}
\colorlet{exercise}{cyan!70!black}
\colorlet{solved}{green!40!black}
\colorlet{self_proof}{blue!70!black}
\colorlet{Red}{red!80!black}

% Gilles Castel's theorems
\newtheoremstyle{mddefinition}% <name>
  {-.25\topsep}%                 <space above>
  {-.25\topsep}%                         <space below>
  {\normalfont}%              <body font>
  {}%                         <indent amount>
  {\bfseries}%                <theorem head font>
  {.}%                        <punctuation after theorem head>
  {.5em}%                     <space after theorem head>
  {}%                         <theorem head spec>
\newtheoremstyle{mdplain}% <name>
  {-.25\topsep}%                 <space above>
  {-.25\topsep}%                         <space below>
  {\itshape}%                 <body font>
  {}%                         <indent amount>
  {\bfseries}%                <theorem head font>
  {.}%                        <punctuation after theorem head>
  {.5em}%                     <space after theorem head>
  {}%                         <theorem head spec>

\usepackage[framemethod=Tikz]{mdframed}
\mdfsetup{skipbelow=0pt}
\mdfdefinestyle{axiomstyle}{
    outerlinewidth = 1.5,
    roundcorner = 10,
    leftmargin = 15,
    rightmargin = 15,
    backgroundcolor = yellow!7
}
\mdfdefinestyle{defstyle}{
    outerlinewidth = 1,
    roundcorner = 2,
    leftmargin = 7,
    rightmargin = 7,
    backgroundcolor = green!5
}
\mdfdefinestyle{thmstyle}{
    outerlinewidth = 1,
    roundcorner = 8,
    leftmargin = 7,
    rightmargin = 7,
    backgroundcolor = cyan!5
}
\mdfdefinestyle{lemmastyle}{
    outerlinewidth = 1.5,
    roundcorner = 10,
    leftmargin = 7,
    rightmargin = 7,
    backgroundcolor = yellow!10
}
\ifundef\chapter{%
    \providecommand\theoremnumberwithin{section}
    \theoremstyle{mddefinition}
    \newmdtheoremenv[nobreak=true, style=axiomstyle]{axiom}{Axiom}[section]
    \theoremstyle{plain}
    \newtheorem{theorem}{Theorem}[\theoremnumberwithin]

    \newcounter{assignment}
    \theoremstyle{plain}
    \newtheorem{problem}{Problem}
    \theoremstyle{mddefinition}
    \newmdtheoremenv[nobreak=true, outerlinewidth=0.7]{problem*}[problem]{Problem}
}{
    \providecommand\theoremnumberwithin{chapter}
    \theoremstyle{mddefinition}
    \newmdtheoremenv[nobreak=true, style=axiomstyle]{axiom}{Axiom}[chapter]
    \theoremstyle{plain}
    \newtheorem{theorem}{Theorem}[\theoremnumberwithin]

    \newcounter{assignment}
    \theoremstyle{plain}
    \newtheorem{problem}{Problem}[assignment]
    \theoremstyle{mddefinition}
    \newmdtheoremenv[nobreak=true, outerlinewidth=0.7]{problem*}[problem]{Problem}
}
\theoremstyle{mddefinition}
\newmdtheoremenv[nobreak=true, style=defstyle]{definition*}[theorem]{Definition}

\theoremstyle{mdplain}
\newmdtheoremenv[nobreak=true, style=thmstyle]{theorem*}[theorem]{Theorem}
\newmdtheoremenv[nobreak=true]{proposition*}[theorem]{Proposition}
\newmdtheoremenv[nobreak=true]{lemma*}[theorem]{Lemma}
\newmdtheoremenv[nobreak=true]{corollary*}[theorem]{Corollary}
\newmdtheoremenv[nobreak=true, style=thmstyle]{fact*}[theorem]{Fact}
\newmdtheoremenv[nobreak=true]{exercise*}[theorem]{Exercise}
\newmdtheoremenv[nobreak=true]{question*}[theorem]{Question}

\theoremstyle{definition}
\newtheorem{definition}[theorem]{Definition}

\theoremstyle{plain}
\newtheorem{proposition}[theorem]{Proposition}
\newtheorem{lemma}[theorem]{Lemma}
\newtheorem{corollary}[theorem]{Corollary}
\newtheorem{fact}[theorem]{Fact}
\newtheorem{exercise}[theorem]{Exercise}
\newtheorem{question}[theorem]{Question}

\theoremstyle{remark}
\newtheorem*{remark}{Remark}
\newtheorem*{remarkx}{Remarks}
\newtheorem*{example}{Example}
\newtheorem*{examplex}{Examples}
\newtheorem*{idea}{Idea}
%%%% HAXXXXXX %%%%
% \def\innerqed{\qedsymbol}
% \def\outerqed{$\blacksquare$}
\let\oldproof\proof
\let\endoldproof\endproof
\newenvironment{solution}[1][]
  {\renewcommand\qedsymbol{$\blacksquare$}%
  \begin{oldproof}[Solution\ifx&#1&\else\ (#1)\fi]}
  {\end{oldproof}}
\newenvironment{answer}
  {\renewcommand\qedsymbol{$\blacksquare$}\begin{oldproof}[Answer]}
  {\end{oldproof}}
\renewenvironment{proof}
  {\renewcommand\qedsymbol{$\blacksquare$}\begin{oldproof}}
  {\end{oldproof}}
\newenvironment{subproof}[1][Subproof]{%
  \renewcommand{\qedsymbol}{$\square$}\begin{oldproof}[#1]}
  {\end{oldproof}}
\newtheorem*{notation}{Notation}
\newtheorem*{claim}{Claim}

\usepackage{hyperref}
\usepackage[noabbrev]{cleveref}

% <cref>
\crefname{theorem}{theorem}{theorems}
\crefname{proposition}{proposition}{propositions}
\crefname{lemma}{lemma}{lemmas}
\crefname{corollary}{corollary}{corollaries}
\crefname{axiom}{axiom}{axioms}
\crefname{definition}{definition}{definitions}
\crefname{problem}{problem}{problems}
\crefname{exercise}{exercise}{exercises}
\crefname{fact}{fact}{facts}
\crefname{question}{question}{questions}
\crefname{remark}{remark}{remarks}
\crefname{example}{example}{example}
\crefname{notation}{notation}{notations}
\crefname{claim}{claim}{claims}
% \crefname{section}{\S}{\S\S}
\crefname{theorem*}{theorem}{theorems}
\crefname{proposition*}{proposition}{propositions}
\crefname{lemma*}{lemma}{lemmas}
\crefname{corollary*}{corollary}{corollaries}
\crefname{definition*}{definition}{definitions}
\crefname{problem*}{problem}{problems}
\crefname{exercise*}{exercise}{exercises}
\crefname{fact*}{fact}{facts}
\crefname{question*}{question}{questions}
% </cref>

% <hyperlinks>
\hypersetup{colorlinks,
    linkcolor={blue},
    citecolor={blue!50!black},
    urlcolor={blue!80!black}}
% </hyperlinks>

\usepackage[shortlabels]{enumitem}
% change default label for enumerate, and fix long labels popping out
\setenumerate{label*=(\roman*),ref=(\roman*),leftmargin=*}
% casework list using https://tex.stackexchange.com/a/30035
\newcounter{casecount}
\newlist{casework}{description}{1}
\setlist[casework]{%
  before={\setcounter{casecount}{0}%
      \renewcommand*\thecasecount{\arabic{casecount}}}%
  ,font=\bfseries Case \stepcounter{casecount}\thecasecount:
}

\newenvironment{examples}[1][]
{\begin{examplex}[#1]\leavevmode\begin{itemize}}{\end{itemize}\end{examplex}}
\newenvironment{remarks}[1][]
{\begin{remarkx}[#1]\leavevmode\begin{itemize}}{\end{itemize}\end{remarkx}}

% omg this is so HaXy
% \renewenvironment{proof}[1][\proofname]{{\it\bfseries #1. }}{\qed}
% \providecommand{\qedsymbol}{\openbox}
% \makeatletter
% \renewenvironment{proof}[1][\proofname]{\par
%   \pushQED{\qed}%
%   \normalfont \topsep6\p@\@plus6\p@\relax
%   \trivlist
%   \item[\hskip\labelsep
%         \itshape\bfseries%this is the change (boldface instead of italics)
%         % \fontseries{bx}\selectfont
%     #1\@addpunct{.}]\ignorespaces
% }{%
%   \popQED\endtrivlist\@endpefalse
% }
\makeatother
\crefname{enumi}{part}{parts}
\crefname{enumii}{part}{parts}
\crefname{enumiii}{part}{parts}
% \setlist[itemize]{itemsep=2pt}
\newcounter{dummy}
\makeatletter
\newcommand\myitem[1][]{\item[#1]\refstepcounter{dummy}\def\@currentlabel{#1}}
\makeatother

\makeatletter
\newcommand*{\refifdef}[3]{%label,command,fallback
    \@ifundefined{r@#1}{#3}{#2{#1}}%
}
\makeatother

\newcommand\ie{\textit{i.e.}}
\newcommand\eg{\textit{e.g.}}
\usepackage{physics}

\usepackage{amsmath}
\usepackage{amssymb}
\usepackage{mathrsfs} % for \mathscr
\usepackage{bm} % for \bm
\usepackage{booktabs}

% undefine \abs and \norm
\let\abs\relax
\let\norm\relax

\usepackage{mathtools} % for delimiters and \coloneqq
\DeclarePairedDelimiter{\paren}{(}{)}
\DeclarePairedDelimiter{\brk}{[}{]}
\DeclarePairedDelimiter{\set}{\{}{\}}
\DeclarePairedDelimiter{\abs}{\lvert}{\rvert}
\DeclarePairedDelimiter{\norm}{\lVert}{\rVert}
\DeclarePairedDelimiter{\floor}{\lfloor}{\rfloor}
\DeclarePairedDelimiter{\ceil}{\lceil}{\rceil}
\DeclarePairedDelimiter{\angled}{\langle}{\rangle}
% \DeclarePairedDelimiterX{\innerp}[2]{\langle}{\rangle}{#1,\,#2}
% \DeclarePairedDelimiterX{\outerp}[2]{\langle}{\rangle}{#1\otimes#2}
% \DeclarePairedDelimiterX{\braket}[3]{\langle}{\rangle}%
% {#1\,\delimsize\vert\,\mathopen{}#2\,\delimsize\vert\,\mathopen{}#3}
\DeclarePairedDelimiterX{\innerp}[2]{\langle}{\rangle}{#1,\,#2}
% \DeclarePairedDelimiterX{\outerp}[2]{\langle}{\rangle}{#1\otimes#2}
\DeclarePairedDelimiterX{\outerp}[2]{\vert}{\vert}%
{#1\delimsize\rangle\delimsize\langle\mathopen{}#2}
\let\braket\relax
\DeclarePairedDelimiterX{\braket}[3]{\langle}{\rangle}%
{#1\,\delimsize\vert\,\mathopen{}#2\,\delimsize\vert\,\mathopen{}#3}

\renewcommand\O{\ensuremath{\varnothing}}
\newcommand\N{\ensuremath{\mathbb{N}}}
\newcommand\Z{\ensuremath{\mathbb{Z}}}
\newcommand\Q{\ensuremath{\mathbb{Q}}}
\newcommand\R{\ensuremath{\mathbb{R}}}
\newcommand\C{\ensuremath{\mathbb{C}}}
% \renewcommand\P{\ensuremath{\mathbb{P}}}

% fix spacing for \forall and \exists
% \let\oldforall\forall
% \renewcommand{\forall}{\oldforall \, }
% \let\oldexist\exists
% \renewcommand{\exists}{\oldexist \: }
\newcommand\unique{\exists!}
\newcommand\lxor{\oplus}

\providecommand{\dd}{\,\mathrm{d}}

\newcommand\mcA{\ensuremath{\mathcal{A}}}
\newcommand\mcB{\ensuremath{\mathcal{B}}}
\newcommand\mcC{\ensuremath{\mathcal{C}}}
\newcommand\mcD{\ensuremath{\mathcal{D}}}
\newcommand\mcE{\ensuremath{\mathcal{E}}}
\newcommand\mcF{\ensuremath{\mathcal{F}}}
\newcommand\mcG{\ensuremath{\mathcal{G}}}
\newcommand\mcH{\ensuremath{\mathcal{H}}}
\newcommand\mcI{\ensuremath{\mathcal{I}}}
\newcommand\mcJ{\ensuremath{\mathcal{J}}}
\newcommand\mcK{\ensuremath{\mathcal{K}}}
\newcommand\mcL{\ensuremath{\mathcal{L}}}
\newcommand\mcM{\ensuremath{\mathcal{M}}}
\newcommand\mcN{\ensuremath{\mathcal{N}}}
\newcommand\mcO{\ensuremath{\mathcal{O}}}
\newcommand\mcP{\ensuremath{\mathcal{P}}}
\newcommand\mcQ{\ensuremath{\mathcal{Q}}}
\newcommand\mcR{\ensuremath{\mathcal{R}}}
\newcommand\mcS{\ensuremath{\mathcal{S}}}
\newcommand\mcT{\ensuremath{\mathcal{T}}}
\newcommand\mcU{\ensuremath{\mathcal{U}}}
\newcommand\mcV{\ensuremath{\mathcal{V}}}
\newcommand\mcW{\ensuremath{\mathcal{W}}}
\newcommand\mcX{\ensuremath{\mathcal{X}}}
\newcommand\mcY{\ensuremath{\mathcal{Y}}}
\newcommand\mcZ{\ensuremath{\mathcal{Z}}}

%%%% WIDE BAR THAT IS JUST THE RIGHT LENGTH %%%%
%% FROM https://tex.stackexchange.com/a/60253 %%
\makeatletter
\let\save@mathaccent\mathaccent
\newcommand*\if@single[3]{%
  \setbox0\hbox{${\mathaccent"0362{#1}}^H$}%
  \setbox2\hbox{${\mathaccent"0362{\kern0pt#1}}^H$}%
  \ifdim\ht0=\ht2 #3\else #2\fi
  }
%The bar will be moved to the right by a half of \macc@kerna, which is computed by amsmath:
\newcommand*\rel@kern[1]{\kern#1\dimexpr\macc@kerna}
%If there's a superscript following the bar, then no negative kern may follow the bar;
%an additional {} makes sure that the superscript is high enough in this case:
\newcommand*\widebar[1]{\@ifnextchar^{{\wide@bar{#1}{0}}}{\wide@bar{#1}{1}}}
%Use a separate algorithm for single symbols:
\newcommand*\wide@bar[2]{\if@single{#1}{\wide@bar@{#1}{#2}{1}}{\wide@bar@{#1}{#2}{2}}}
\newcommand*\wide@bar@[3]{%
  \begingroup
  \def\mathaccent##1##2{%
%Enable nesting of accents:
    \let\mathaccent\save@mathaccent
%If there's more than a single symbol, use the first character instead (see below):
    \if#32 \let\macc@nucleus\first@char \fi
%Determine the italic correction:
    \setbox\z@\hbox{$\macc@style{\macc@nucleus}_{}$}%
    \setbox\tw@\hbox{$\macc@style{\macc@nucleus}{}_{}$}%
    \dimen@\wd\tw@
    \advance\dimen@-\wd\z@
%Now \dimen@ is the italic correction of the symbol.
    \divide\dimen@ 3
    \@tempdima\wd\tw@
    \advance\@tempdima-\scriptspace
%Now \@tempdima is the width of the symbol.
    \divide\@tempdima 10
    \advance\dimen@-\@tempdima
%Now \dimen@ = (italic correction / 3) - (Breite / 10)
    \ifdim\dimen@>\z@ \dimen@0pt\fi
%The bar will be shortened in the case \dimen@<0 !
    \rel@kern{0.6}\kern-\dimen@
    \if#31
      \overline{\rel@kern{-0.6}\kern\dimen@\macc@nucleus\rel@kern{0.4}\kern\dimen@}%
      \advance\dimen@0.4\dimexpr\macc@kerna
%Place the combined final kern (-\dimen@) if it is >0 or if a superscript follows:
      \let\final@kern#2%
      \ifdim\dimen@<\z@ \let\final@kern1\fi
      \if\final@kern1 \kern-\dimen@\fi
    \else
      \overline{\rel@kern{-0.6}\kern\dimen@#1}%
    \fi
  }%
  \macc@depth\@ne
  \let\math@bgroup\@empty \let\math@egroup\macc@set@skewchar
  \mathsurround\z@ \frozen@everymath{\mathgroup\macc@group\relax}%
  \macc@set@skewchar\relax
  \let\mathaccentV\macc@nested@a
%The following initialises \macc@kerna and calls \mathaccent:
  \if#31
    \macc@nested@a\relax111{#1}%
  \else
%If the argument consists of more than one symbol, and if the first token is
%a letter, use that letter for the computations:
    \def\gobble@till@marker##1\endmarker{}%
    \futurelet\first@char\gobble@till@marker#1\endmarker
    \ifcat\noexpand\first@char A\else
      \def\first@char{}%
    \fi
    \macc@nested@a\relax111{\first@char}%
  \fi
  \endgroup
}
\makeatother
\let\what\widehat
\let\wtld\widetilde
\let\wbar\widebar
\let\ubar\underline

\DeclareMathOperator\sgn{sgn}

\let\oldleft\left
\let\oldright\right
\renewcommand{\left}{\mathopen{}\mathclose\bgroup\oldleft}
\renewcommand{\right}{\aftergroup\egroup\oldright}


\setlist[enumerate,1]{label=(\alph*)}

\title{Assignment 1}
% \subtitle{MA 200: Multivariable Calculus}
\author{Naman Mishra}
\date{3 August, 2024}

\begin{document}
\maketitle

% Problem 1
\begin{problem} \label{prb:add-scale-cont}
    Let $(V, \norm{\cdot})$ be a normed linear space.
    \begin{enumerate}
        \item Show that the addition map $(u, v) \mapsto u + v$
            is continuous.
        \item Show that the scalar multiplication map
            $(\alpha, u) \mapsto \alpha u$ is continuous.
    \end{enumerate}
\end{problem}
\begin{proof} \leavevmode
    \begin{enumerate}
        \item $\norm{u_2 + v_2 - (u_1 + v_1)}
            \le \norm{u_2 - u_1} + \norm{v_2 - v_1}$.
        \item $\norm{\alpha_2 u_2 - \alpha_1 u_1}
            = \norm{\alpha_2 u_2 - \alpha_1 u_2 + \alpha_1 u_2 - \alpha_1 u_1}
            = \norm{(\alpha_2 - \alpha_1) u_2 + \alpha_1(u_2 - u_1)}
            \le \abs{\alpha_2 - \alpha_1} \norm{u_2}
                + \abs{\alpha_1} \norm{u_2 - u_1}$. \qedhere
    \end{enumerate}
\end{proof}

% Problem 2
\begin{problem}
    Let $(V, \norm{\cdot})$ be a normed linear space.
    Prove that \[
        \abs[\big]{\norm x - \norm y} \le \norm{x - y}
    \] for all $x, y \in V$.
    Show that the function $x \mapsto \norm x$ from $V$ to $\R$
    is continuous.
\end{problem}
\begin{proof}
    By the $\triangle$ inequality, \[
        \norm x = \norm{x - y + y} \le \norm{x - y} + \norm y
        \implies \norm x - \norm y \le \norm{x - y}.
    \] Similarly \[
        \norm y = \norm{y - x + x} \le \norm{y - x} + \norm x
        \implies \norm x - \norm y \ge -\norm{x - y}.
    \]
    Thus \[
        \abs[\big]{\norm x - \norm y} \le \norm{x - y}.
    \]
    To show that $\norm\cdot$ is continuous, do what exactly?
    Notice \[
        \abs[\big]{\norm x - \norm y} \le \norm{x - y}? \qedhere
    \]
\end{proof}

% Problem 3
\begin{problem}
    For $x, y \in \R^n$, show that \begin{equation}
        \abs{\innerp xy} \le {\norm x}_2 {\norm y}_2 \label{eq:cs}
    \end{equation}
    Also show that the two sides in \cref{eq:cs} are equal if and only if
    $x$ and $y$ are linearly dependent over $\R$.
\end{problem}
\begin{proof}
    If either of $x$ or $y$ is $0$, both sides are $0$.

    Suppose $x, y \ne 0$.
    Let $\what{x} = \dfrac{x}{{\norm x}_2}$ and
    $\what{y} = \dfrac{y}{{\norm y}_2}$.
    Then proving \cref{eq:cs} amounts to proving \[
        \abs{\innerp{\what{x}}{\what{y}}} \le 1
    \] because of homogeneity of the inner product.
    \begin{align*}
        0 &\le \sum_{i=1}^n (\what{x}_i - \what{y}_i)^2 \\
        0 &\le \sum_{i=1}^n \what{x}_i^2 - 2\what{x}_i\what{y}_i + \what{y}_i^2 \\
        2\sum_{i=1}^n \what{x}_i\what{y}_i &\le \sum_{i=1}^n \what{x}_i^2 + \sum_{i=1}^n \what{y}_i^2 \\
        \innerp{\what{x}}{\what{y}} &\le 1.
    \end{align*}
    Similarly $\innerp{-\what{x}}{\what{y}} \le 1$, which gives
    $\innerp{\what{x}}{\what{y}} \ge -1$.
\end{proof}

% Problem 4
\begin{problem}
    Let $\set{x_k}_{k \in \N} \subseteq \R^n$ and $x \in \R^n$.
    Show that $\set{x_k}_{k \in \N}$ converges to $x$ if and only if
    $\set{\innerp{x_k}{y}}$ converges to $\innerp xy$ for all $y \in \R^n$.
\end{problem}
\begin{proof}
    Suppose $x_k \to x$.
    Let $y \in \R^n$.
    Then
    \[
        \abs{\innerp{x_k}{y} - \innerp xy}
            = \abs{\innerp{x_k - x}{y}}
            \le \norm{x_k - x} \norm y
            \to 0.
    \]
    Conversely, suppose
    $\innerp{x_k}{y} \to \innerp xy$ for all
    $y \in \R^n$.
    Then $\innerp{x_k}{e_i} \to \innerp{x}{e_i}$ for all $i$.
    Thus $x_k \to x$ componentwise.
\end{proof}

% Problem 5
\begin{problem}
    Let $1 < p, q < \infty$ be such that $\dfrac1p + \dfrac1q = 1$.
    Show that for any $a \ge 0$ and $x \ge 0$ the following holds: \begin{equation}
        xa \le \frac{a^p}{p} + \frac{x^q}{q}. \label{eq:holder}
    \end{equation}
    Show that in \cref{eq:holder} equality holds if and only if $x^q = a^p$.
\end{problem}
\begin{proof}
    Let $a \ge 0$ be fixed.
    Define $f(x) = xa - \dfrac{a^p}{p} - \dfrac{x^q}{q}$.
    This is differentiable on $[0, \infty)$ since $q > 0$.
    $f'(x) = a - x^{q-1}$.
    Thus \begin{align*}
        f'(x) \le 0 &\iff x^{q-1} \le a \\
        &\iff x^{q/p} \le a \\
        &\iff x^q \le a^p.
    \end{align*}
    Thus $f$ is decreasing on $[a^{p/q}, \infty)$ and increasing on
    $[0, a^{p/q}]$.
    Thus $f(x) \ge f(a^{p/q}) = 0$.
    Moreover, since $f'(x) \ne 0$ for $x^q \ne a^p$, we have
    $f(x) = 0 \iff x^q = a^p$.

    Thus $xa \le \dfrac{a^p}{p} + \dfrac{x^q}{q}$ with equality
    only if $x^q = a^p$.
\end{proof}

% Problem 6
\begin{problem} \label{prb:lp}
    For $1 \le p \le \infty$ and $x = (x_1, x_2, \dots, x_n)$, we define \[
        {\norm x}_p = \begin{cases}
            \big(\abs{x_1}^p + \abs{x_2}^p + \dots + \abs{x_n}^p\big)^{\frac1p} & 1 \le p < \infty \\
            \max\limits_{1 \le i \le n} \abs{x_i} & p = \infty
        \end{cases}
    \]
    \begin{enumerate}
        \item Let $1 \le q \le \infty$ be such that $\dfrac1p + \dfrac1q = 1$.
            For any $x, y \in \R^n$, show that \begin{equation}
                \abs{\innerp xy} \le {\norm x}_p {\norm y}_q
                \text{ and }
                {\norm {x + y}}_p \le {\norm x}_p + {\norm y}_p.
                \label{eq:holder-norm}
            \end{equation}
        \item Show that ${\norm \cdot}_p$ defines a norm on $\R^n$.
        \item Show that ${\norm x}_\infty = \lim\limits_{p \to \infty} {\norm x}_p$
            for any $x \in \R^n$.
    \end{enumerate}
\end{problem}
\begin{proof} \leavevmode
    We first deal with the case $p = \infty$ for parts (a) and (b).
    \begin{enumerate}
        \item $q = 1$.
        \begin{align*}
            \abs{\innerp xy} &= \abs{x_1 y_1 + x_2 y_2 + \dots + x_n y_n} \\
            &\le \abs{x_1} \abs{y_1} + \abs{x_2} \abs{y_2} + \dots + \abs{x_n} \abs{y_n} \\
            &= \max_{1 \le i \le n} \abs{x_i} (\abs{y_1} + \abs{y_2} + \dots + \abs{y_n}) \\
            &= {\norm x}_\infty {\norm y}_1
        \end{align*} and \begin{align*}
            {\norm {x + y}}_\infty &= \max_{1 \le i \le n} \abs{x_i + y_i} \\
            &\le \max_{1 \le i \le n} (\abs{x_i} + \abs{y_i}) \\
            &\le \max_{1 \le i, j \le n} (\abs{x_i} + \abs{y_j}) \\
            &= \max_{1 \le i \le n} \abs{x_i} + \max_{1 \le j \le n} \abs{y_j} \\
            &= {\norm x}_\infty + {\norm y}_\infty.
        \end{align*}
        \item We have positivity by definition.
        ${\norm x}_p = 0 \iff \max_{1 \le i \le n} \abs{x_i} = 0 \iff
        \abs{x_1} = \abs{x_2} = \dots = \abs{x_n} = 0 \iff x = 0$, so
        definiteness holds.
        Homogeneity is since \[
            {\norm{\alpha x}}_\infty = \max_{1 \le i \le n} \abs{\alpha x_i}
            = \abs\alpha \max_{1 \le i \le n} \abs{x_i}
            = \abs\alpha {\norm x}_\infty.
        \] Triangle inequality is proven above.

        Thus ${\norm \cdot}_\infty$ is a norm.
    \end{enumerate}

    Now we deal with the case $1 \le p < \infty$.
    \begin{enumerate}
        \item For $\abs{\innerp xy} \le {\norm x}_p {\norm y}_q$, we only
        concern ourselves with $1 < p, q < \infty$.
        The case $p = 1$ requires $q = \infty$, which is covered above with
        $p$ and $q$ interchanged.
        We will show that the ratio of the two sides is bounded by $1$.
        \begin{align*}
            \frac{\abs{\innerp xy}}{{\norm x}_p {\norm y}_q}
            &= \abs*{\frac{x_1 y_1 + x_2 y_2 + \dots + x_n y_n}{{\norm x}_p {\norm y}_q}} \\
            &\le \sum_{i=1}^n \frac{\abs{x_i} \abs{y_i}}{{\norm x}_p {\norm y}_q} \\
            &\le \sum_{i=1}^n \ab(\frac1p \frac{\abs{x_i}^p}{{\norm x}_p^p} + \frac1q \frac{\abs{y_i}^q}{{\norm y}_q^q}) \tag{by \cref{eq:holder}} \\
            &= \frac1p \frac{\sum_i \abs{x_i}^p}{{\norm x}_p^p} + \frac1q \frac{\sum_i \abs{y_i}^q}{{\norm y}_q^q} \\
            &= \frac1p + \frac1q \\
            &= 1.
        \end{align*}
        We use this result to prove the triangle inequality.
        (We did this in a UM 204 assignment last semester, with ample
        of hints and time to spare.)
        \begin{align*}
            {\norm {x + y}}_p^p
            &= \sum_{i=1}^n \abs{x_i + y_i}^p \\
            &= \sum_{i=1}^n \abs{x_i + y_i} \abs{x_i + y_i}^{p-1} \\
            &\le \sum_{i=1}^n \abs{x_i} \abs{x_i + y_i}^{p-1}
            + \sum_{i=1}^n \abs{y_i} \abs{x_i + y_i}^{p-1}
        \end{align*}
        Let $X = (\abs{x_1}, \abs{x_2}, \dots, \abs{x_n})$ and
        $Z = (\abs{x_1 + y_1}^{p-1}, \abs{x_2 + y_2}^{p-1}, \dots,
        \abs{x_n + y_n}^{p-1})$.
        Then by \cref{eq:holder}, \begin{align*}
            \sum_{i=1}^n \abs{x_i} \abs{x_i + y_i}^{p-1}
                &= \abs{\innerp XZ} \\
                &\le {\norm X}_p {\norm Z}_q \\
            \intertext{where $q = \frac{p}{p-1}$}
            &\le {\norm x}_p \ab(\abs{x_1 + y_1}^p + \dots + \abs{x_n + y_n}^p)^{\frac{p}{p-1}} \\
            &= {\norm x}_p {\norm {x + y}}_p^{p-1}.
            \intertext{Similarly,}
            \sum_{i=1}^n \abs{y_i} \abs{x_i + y_i}^{p-1}
                &\le {\norm y}_p {\norm {x + y}}_p^{p-1}. \\
            \intertext{This gives}
            {\norm {x + y}}_p^p &\le ({\norm x}_p + {\norm y}_p) {\norm {x + y}}_p^{p-1} \\
            {\norm {x + y}}_p &\le {\norm x}_p + {\norm y}_p.
        \end{align*}
        \item Positivity is again by definition.
        ${\norm x}_p = 0 \iff \abs{x_i}^p = 0$ for all $i$, which
        is iff $x = 0$.
        Homogeneity is trivial to check.
        \begin{align*}
            {\norm {\alpha x}}_p
            &= \ab(\abs{\alpha x_1}^p + \abs{\alpha x_2}^p + \dots + \abs{\alpha x_n}^p)^{\frac1p} \\
            &= \ab(\abs\alpha^p \abs{x_1}^p + \abs\alpha^p \abs{x_2}^p + \dots + \abs\alpha^p \abs{x_n}^p)^{\frac1p} \\
            &= \abs{\alpha} {\norm x}_p.
        \end{align*}
        Triangle inequality is proven above.

        Thus ${\norm \cdot}_p$ is a norm.
    \end{enumerate}

    \noindent We now prove part (c).
    The case $x = 0$ is trivial since ${\norm x}_p = {\norm x}_\infty = 0$
    for any $p$.

    WLOG let ${\norm x}_\infty = \abs{x_1} > 0$.
    Then for $1 \le p < \infty$, \begin{align*}
        {\norm x}_p &= \abs{x_1} \ab(1 + \frac{\abs{x_2}^p}{\abs{x_1}^p} + \dots + \frac{\abs{x_n}^p}{\abs{x_1}^p})^{\frac1p} \\
        &\le \abs{x_1} \cdot n^{\frac1p}
    \end{align*}
    Further, \[
        {\norm x}_p = \ab(\abs{x_1}^p + \abs{x_2}^p + \dots + \abs{x_n}^p)^{\frac1p}
        \ge \ab(\abs{x_1}^p)^{\frac1p} = \abs{x_1}.
    \] Thus \[
        \abs{x_1} \le {\norm x}_p \le n^{\frac1p} \abs{x_1}.
    \] As $p \to \infty$, $n^{\frac1p} \to 1$.
    Thus by the squeeze theorem,
    ${\norm x}_p \to \abs{x_1} = {\norm x}_\infty$.
\end{proof}

% Problem 7
\begin{problem}
    Let $C[a, b]$ be the set of all complex-valued continuous functions
    on $[a, b]$.
    \begin{enumerate}
        \item Let $f \in C[a, b]$ be such that $f$ is non-negative and
            $\int_a^b f(x) \dd x = 0$.
            Show that $f \equiv 0$.
        \item For $f \in C[a, b]$, define \[
            {\norm f}_\infty \coloneq \sup_{x \in [a, b]} \abs{f(x)},
            \qquad
            {\norm f}_1 \coloneq \int_a^b \abs{f(x)} \dd x.
        \] Show that ${\norm \cdot}_\infty$ and ${\norm \cdot}_1$ are norms
        on $C[a, b]$.
        \item Are the above two norms on $C[a, b]$ equivalent?
        Are they comparable?
    \end{enumerate}
\end{problem}
\begin{solution} \leavevmode
    \begin{enumerate}
        \item Suppose $f$ is non-zero at some point $c \in [a, b]$.
            By continuity, $f(x) \ge \frac{f(c)}{2}$ in some neighbourhood
            $[c - \delta, c + \delta]$.
            Then $f$ is lower bounded by the step function \[
                g(x) = \begin{cases}
                    \frac{f(c)}{2} & x \in [c - \delta, c + \delta] \\
                    0 & \text{otherwise}
                \end{cases}
            \] which has positive integral.
            This would force $\int_a^b f(x) \dd x > 0$.
            Contradiction! Such a $c$ cannot exist.
        \item Clearly both are non-negative.
            ${\norm f}_\infty = 0 \iff \abs{f(x)} \le 0$ for all
            $x \in [a, b]$, which is iff $f \equiv 0$.
            Definiteness of ${\norm\cdot}_1$ is by the previous part.
            Homogeneity is obvious.
            Triangle inequality is an extension of the triangle inequality
            for complex numbers.
        \item They are \emph{not} equivalent.
            Consider $[a, b] = [0, 1]$ and $f(x) = e^{-\lambda x}$.
            Then ${\norm f}_\infty = 1$ and
            ${\norm f}_1 = \frac{1 - e^{-\lambda}}{\lambda}$.
            One can choose $\lambda$ to make ${\norm f}_1$ arbitrarily
            close to $0$.
            Thus there are no constants $c_1, c_2 > 0$ such that \[
                c_1 {\norm f}_\infty
                  \le {\norm f}_1
                    \le c_2 {\norm f}_\infty.
            \]

            However, we \emph{can} compare the norms as \[
                {\norm f}_1 \le (b - a) {\norm f}_\infty.
            \] This is simply by noticing that the constant function
            $x \mapsto {\norm f}_\infty$ upper bounds $\abs{f(x)}$ and has
            integral $(b - a) {\norm f}_\infty$ over $[a, b]$. \qedhere
    \end{enumerate}
\end{solution}

\begin{problem}
    For $A \in L(\R^n, \R^m)$, let $\norm A$ denote the operator norm of $A$.
    Show that \[
        \norm A = \inf\set{M : \norm{Ax} \le M \norm x
                    \text{ for all } x \in \R^n}.
    \]
\end{problem}
\begin{proof}
    $\norm{Ax} \le M \norm x$ is trivially true for $x = 0$ no matter
    what $M$ is.
    Thus \begin{align*}
        \inf\set{M : \norm{A&x} \le M \norm x
                    \text{ for all } x \in \R^n} \\
        &= \inf\set{M : \norm{Ax} \le M \norm x
                    \text{ for all } x \in \R^n \setminus \set{0}} \\
        &= \inf\set{M : \norm*{A\frac{x}{\norm x}} \le M
                    \text{ for all } x \in \R^n \setminus \set{0}} \\
        &= \inf\set{M : \norm{Ay} \le M \text{ for all } y \in S^{n-1}} \\
        &= \inf\set{\text{upper bounds of } \set{\norm{Ay} : y \in S^{n-1}}} \\
        &= \sup\set{\norm{Ay} : y \in S^{n-1}} \\
        &= \norm A. \qedhere
    \end{align*}
\end{proof}

\begin{problem} \label{prb:operator-eigen}
    Let $A$ be a real symmetric $n \times n$ matrix.
    \begin{enumerate}
        \item Show that all eigenvalues of $A$ are real.
        \item For $1 \le i \le n$, let $\lambda_i$ denote the eigenvalues of $A$.
            Show that \[
                \norm A = \max_{1 \le i \le n} \abs{\lambda_i}.
            \]
    \end{enumerate}
\end{problem}
\begin{solution} \leavevmode
    \begin{enumerate}
        \item View $A$ as a linear operator on $\C^n$.
            Let $\lambda$ be an eigenvalue of $A$ and $v$ be the
            corresponding eigenvector.
            Then \[
                \lambda \innerp v v
                    = \innerp{Av} v
                    = \innerp v{Av}
                    = \wbar\lambda \innerp v v.
            \] Thus $\lambda = \wbar\lambda$ is real.
        \item (assuming spectral theorem)
            WLOG let $\lambda_1 = \max_{1 \le i \le n} \abs{\lambda_i}$.
            Write any vector $x \in \R^n$ as a linear combination of
            orthonormal eigenvectors $x = \sum_{i=1}^n c_i v_i$,
            where $v_i$ is the eigenvector corresponding to $\lambda_i$.
            Then $Ax = \sum_{i=1}^n c_i \lambda_i v_i$.
            \begin{align*}
                {\norm{Ax}}^2 &= \sum_{i=1}^n c_i^2 \lambda_i^2 \\
                &\le \lambda_1^2 \sum_{i=1}^n c_i^2 \\
                &= \lambda_1^2 {\norm x}^2.
            \end{align*}
            Thus $\norm A \le \lambda_1$.
            Moreover, $\norm{Av_1} = \abs{\lambda_1} \norm{v_1}$.
            Thus $\norm A \ge \lambda_1$. \qedhere
    \end{enumerate}
\end{solution}

\begin{problem} \label{prb:operator-hs}
    Let $A \in L(\R^n, \R^m)$ and $B \in L(\R^k, \R^n)$.
    Show that \[
        \norm A \le \hsn A \le \sqrt n \norm A
        \quad \text{and} \quad
        \hsn{AB} \le \hsn A \hsn B.
    \]
\end{problem}
\begin{proof}
    $\hsn A = \sqrt{\Tr(A^\T A)}$.
    Recall that the trace of a matrix is the sum of its eigenvalues.

    Let $v_1, v_2, \dots, v_n$ be orthonormal eigenvectors of $A^\T A$
    with eigenvalues $\lambda_1 \ge \lambda_2 \ge \dots \ge \lambda_n$
    (spectral theorem).
    Each $\lambda_i$ is non-negative, since
    $\innerp{A^\T Ax}{x} = \innerp{Ax}{Ax} \ge 0$.

    Then for any $x = \sum_{i=1}^n c_i v_i$ with $\norm x = 1$, \[
        \norm{Ax}^2 = \innerp{Ax}{Ax} = \innerp{A^\T Ax}{x}
            = \sum_{i=1}^n c_i^2 \lambda_i
            \le \lambda_1
    \] where the equality holds for $x = v_1$.
    Thus $\norm A = \sqrt{\lambda_1}$.
    Since $\hsn A^2 = \sum_{i=1}^n \lambda_i$, we have
    $\lambda_1 \le \hsn A^2 \le n \lambda_1$.
    This gives $\norm A \le \hsn A \le \sqrt n \norm A$.

    For $1 \le i \le m$ and $1 \le j \le k$ let \[
        a_i = \begin{pmatrix}
            A_{i1} & A_{i2} & \cdots & A_{in}
        \end{pmatrix}^\T, \qquad b_j = \begin{pmatrix}
            B_{1j} \\ B_{2j} \\ \vdots \\ B_{nj}
        \end{pmatrix}.
    \] Then \[
        AB = \begin{pmatrix}
            \innerp{a_1}{b_1} & \innerp{a_1}{b_2} & \cdots & \innerp{a_1}{b_k} \\
            \innerp{a_2}{b_1} & \innerp{a_2}{b_2} & \cdots & \innerp{a_2}{b_k} \\
            \vdots & \vdots & \ddots & \vdots \\
            \innerp{a_m}{b_1} & \innerp{a_m}{b_2} & \cdots & \innerp{a_m}{b_k}
        \end{pmatrix}
    \] so by Cauchy-Schwarz, \begin{align*}
        \hsn{AB}^2 &= \sum_{i=1}^m \sum_{j=1}^k \innerp{a_i}{b_j}^2 \\
        &\le \sum_{i=1}^m \sum_{j=1}^k \norm{a_i}^2 \norm{b_j}^2 \\
        &= \ab(\sum_{i=1}^m \norm{a_i}^2) \ab(\sum_{j=1}^k \norm{b_j}^2) \\
        &= \hsn A^2 \hsn B^2. \qedhere
    \end{align*}
\end{proof}
\begin{remark}
    A far simpler proof that I missed is the following.
    \begin{align*}
        \norm{Ax}^2
        &\le \sum_i \innerp{a_i}{x}^2
            &\hsn A^2 &= \sum_j \sum_i a_{ij}^2 \\
        &\le \sum_i \norm{a_i}^2 \norm x^2
            &&= \sum_j \norm{Ae_j}^2 \\
        &= \hsn A^2 \norm x^2
            &&\le \sum_j \norm A^2 \\
        &
            &&= n \norm A^2.
    \end{align*}
\end{remark}

\section*{Quiz} \label{sec:a1:quiz}
\begin{problem}
    Recall the definition of a \nameref{def:home}.
    Let $f\colon \R^n \setminus \set 0 \to \R$ be a continuous,
    non-vanishing homogenous function of degree $k$
    and $\norm\cdot$ be a fixed norm on $\R^n$.
    Show that there exist positive constants $C_1, C_2 > 0$ such that \[
        C_1 \norm x^k \le \abs{f(x)} \le C_2 \norm x^k,
    \] for every $0 \ne x \in \R^n$.
\end{problem}
\begin{proof}
    Choose $C_1 = \min_{\norm x = 1} \abs{f(x)}$ and
    $C_2 = \max_{\norm x = 1} \abs{f(x)}$.
    They exist by compactness of the unit sphere, and are positive
    since $f$ does not vanish.

    Then for any $x \ne 0$, \[
        \abs{f(x)} = \norm x^k \abs*{f\ab({\frac{x}{\norm x}})}
    \] is bounded between $C_1 \norm x^k$ and $C_2 \norm x^k$.
\end{proof}

\begin{problem}
    Let $V$ be a vector space over \R.
    Let $d$ be the discrete metric on $V$.
    Is $d$ induced by a norm on $V$?
\end{problem}
\begin{solution}
    No.
    Suppose $d(x, y) = \norm{x - y}$ for some norm $\norm\cdot$,
    for all $x, y \in V$.
    Let $x \ne y$.
    Then $d(x, y) = 1 = \norm{x - y}$.
    But $d(2x, 2y) = 1 = \norm{2x - 2y} = 2 \norm{x - y} = 2$.
    Contradiction!
\end{solution}

\begin{problem}
    For $A \in L(\R^n, \R^m)$, show that $\norm A = \norm{A^\T}$.
\end{problem}
\begin{proof}
    Notice by Cauchy-Schwarz that for any vector $v$ in a real inner product
    space, \[
        \norm v = \sup_{\norm w = 1} \innerp wv.
    \] (The supremum is achieved at $v / \norm v$ for $v \ne 0$.)
    Then \begin{align*}
        \norm A &= \sup_{x \in S^{n-1}} \norm{Ax} \\
        &= \sup_{x \in S^{n-1}} \sup_{y \in S^{m-1}} \innerp y{Ax} \\
        &= \sup_{y \in S^{m-1}} \sup_{x \in S^{n-1}} \innerp{A^\T y}x \\
        &= \sup_{y \in S^{m-1}} \norm{A^\T y} \\
        &= \norm{A^\T}. \qedhere
    \end{align*}
\end{proof}

\begin{problem}
    Find maximum of $x + 2y + 3z$ subject to the condition
    $x^2 + y^2 + z^2 = 1$.
\end{problem}
\begin{solution}
    The function is continuous and the constraint is compact.
    Thus a maximum exists.

    Let $r = (x, y, z)$ and $n = (1, 2, 3)$.
    As discussed in the previous problem, \[
        \max_{\norm r = 1} \innerp nr = \norm n.
    \] Thus the maximum is $\sqrt{14}$.
\end{solution}

\begin{problem}
    See \cref{prb:operator-eigen}.
\end{problem}

\end{document}
