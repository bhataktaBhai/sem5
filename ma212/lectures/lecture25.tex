\lecture[25]{2024-10-14}{}
\section{Field of fractions and localization} \label{sec:local}

Let $R$ be a commutative ring.
Let $S \subseteq R$ be closed under multiplication, and $1 \in S$.
We wish to construct a certain ring where the elements of $S$ can be
inverted.

\begin{definition}
    Let $S \subseteq R$ be closed under multiplication, and $1 \in S$.
    Define an equivalence relation on $R \times S$ as \[
        (a, s) \sim (a', s') \iff \exists s_1 \in S \text{ such that }
            s_1(as' - a's) = 0.
    \]
    We denote the set of equivalence classes by $S^{-1}R$, and write
    $\frac as$ for the equivalence class of $(a, s)$.
\end{definition}
\begin{exercise}
    $\sim$ is an equivalence relation.
\end{exercise}
\begin{proof}
    Reflexivity and symmetry are clear.
    For transitivity, suppose $(a, s) \sim (a', s') \sim (a'', s'')$,
    with $s_1, s_2 \in S$ such that $s_1(as' - a's) = 0$ and
    $s_2(a's'' - a''s') = 0$.

    Then \begin{align*}
        s''(as' - a's) + s(a's'' - a''s')
            &= as's'' - a'ss'' + a'ss'' - a''ss' \\
            &= s'(as'' - a''s) \\
        \implies (s_1s_2s') (as'' - a''s) &= 0. \qedhere
    \end{align*}
\end{proof}

\begin{exercise}
    Let $S \subseteq R$ be closed under multiplication, and $1 \in S$.
    Then $0 \in S$ iff $S$ has a nilpotent element iff $S^{-1}R = \set 0$.
\end{exercise}
\begin{proof}
    
\end{proof}

\begin{definition}
    For $\frac as, \frac{a'}{s'} \in S^{-1}R$, \begin{align*}
        \frac as + \frac{a'}{s'} &\coloneq \frac{as' + a's}{ss'}, \\
        \frac as \cdot \frac{a'}{s'} &\coloneq \frac{aa'}{ss'}.
    \end{align*}
\end{definition}
\begin{exercise}
    Addition and multiplication are well-defined,
    and $S^{-1}R$ is a ring under these.
\end{exercise}
\begin{proof}
    Let $\frac as = \frac \alpha\sigma$ and
    $\frac{a'}{s'} = \frac{\alpha'}{\sigma'}$.
    Let $s_1, \sigma_1 \in S$ be such that
    $s_1(as' - a's) = \sigma_1(\alpha\sigma' - \alpha'\sigma) = 0$.
    Then \begin{align*}
        s_1\sigma_1\big(\sigma\sigma'(as' + a's)
        - ss'(\alpha\sigma' + \alpha'\sigma)\big)
            &= s_1\sigma_1\big(
    \end{align*}
\end{proof}

\begin{proposition} \label{prp:inv}
    Every element of $S$ is a unit in $S^{-1}R$.
    That is, if $\varphi_S \colon R \to S^{-1}R$ is a ring homomorphism
    given by $\varphi_S(r) = \frac r1$, then $\varphi_S(s)$ is a unit in
    $S^{-1}R$ for all $s \in S$.
\end{proposition}
\begin{proof}
    $\frac s1 \cdot \frac 1s = \frac 11$ is the identity.
\end{proof}

Let $\mcC$ be the category of objects \[
    \Ob(\mcC) = \sset*{(f\colon R \to B, B)}{\begin{matrix}
        B \text{ is a commutative ring}, \\
        f \text{ is a ring homomorphism}, \\
        \forall s \in S (f(s) \text{ is a unit in } B)
    \end{matrix}}
\]

\begin{proposition}
    Given any ring homomorphism $f\colon R \to B$ such that $f(s)$ is a
    unit in $B$ for each $s \in S$, there exists a unique ring homomorphism
    $g\colon S^{-1}R \to B$ such that $g \circ \varphi_S = f$.
\end{proposition}
\begin{proof}
    Define $g(a/s) \coloneq f(a) f(s)^{-1}$.
    This exists by \cref{prp:inv}.
    To check that $g$ is well-defined, let $s_1(as' - a's) = 0$.
    Then \begin{align*}
        f(s_1)(f(a) f(s') - f(a') f(s)) &= 0 \\
        \implies f(a) f(s') &= f(a') f(s) \\
        \implies f(a) f(s)^{-1} &= f(a') f(s')^{-1}.
    \end{align*}
    Clearly $g(\varphi_S(r)) = g(r/1) = f(r)$.

    Now let $g'\colon S^{-1}R \to B$ be another such homomorphism.
    Then
    \[
        1 = g'(1/s \cdot s/1) = g'(1/s) g'(s/1) = g'(1/s) g'(\varphi_S(s))
        = g'(1/s) f(s).
    \] Thus \[
        g'(r/s) = g'(r/1) g'(1/s) = g(\varphi_S(r)) f(s)^{-1}
        = f(r) f(s)^{-1}. \qedhere
    \]
\end{proof}

\begin{example}
    Let $R = \Z/6\Z$ and $S = \set{1, 3}$.
    Then \begin{align*}
        \frac03 &= \frac01 \\
        \frac13 &= \frac39 &= \frac33 = \frac11 \\
    \end{align*}
\end{example}
